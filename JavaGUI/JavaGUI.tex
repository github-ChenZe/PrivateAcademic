\documentclass{ctexart}

% Mathematics Include

\usepackage{amsmath}
\usepackage{amssymb}
\usepackage{amsthm}
\usepackage{amsfonts}
\usepackage{mathrsfs}
\usepackage{enumitem}
\usepackage{braket}
\usepackage{hyperref}
\usepackage[all, pdf]{xy}
\usepackage{wrapfig}
\usepackage{leftidx}

% Physics Include
\usepackage{amsmath}
\usepackage{physics}
\usepackage{siunitx}
\usepackage[makeroom]{cancel}
\usepackage{pstricks}
\usepackage{pstricks-add}
\psset{algebraic=true}

\usepackage[version=4]{mhchem}
\usepackage{array,booktabs}
\usepackage{longtable}
\usepackage{mathtools}
\usepackage[normalem]{ulem}
\usepackage{multicol}

\usepackage{mdframed}
\usepackage{lipsum}% just to generate text for the example

\newmdenv[
  topline=false,
  bottomline=false,
  skipabove=\topsep,
  skipbelow=\topsep
]{siderules}


% Mathematics Head

\newcommand{\pare}[1]{\left(#1\right)}
\newcommand{\blr}[1]{\left[#1\right)}
\newcommand{\lbr}[1]{\left(#1\right]}
\newcommand{\brac}[1]{\left[#1\right]}
\newcommand{\curb}[1]{\left\{#1\right\}}
% \newcommand{\abs}[1]{\left|\, #1 \,\right|}
\newcommand{\rec}[1]{\frac{1}{#1}}
\newcommand{\N}{\mathbb{N}}
\newcommand{\bC}{\mathbb{C}}
\newcommand{\Q}{\mathbb{Q}}
\newcommand{\Z}{\mathbb{Z}}
\newcommand{\R}{\mathbb{R}}
\newcommand{\unk}{\mathcal{X}}
\newcommand{\bu}[3]{#1_{#2}^{\pare{#3}}}
\newcommand{\dref}[1]{定义\ref{def:#1}}
\newcommand{\tref}[1]{定理\ref{thm:#1}}
\newcommand{\lref}[1]{引理\ref{lem:#1}}
\newcommand{\cref}[1]{推论\ref{coll:#1}}
\newcommand{\pref}[1]{命题\ref{prp:#1}}
\newcommand{\rmref}[1]{附注\ref{rm:#1}}
\newcommand{\eref}[1]{例\ref{ex:#1}}
\newcommand{\dcompare}[1]{\textit{平行于\dref{#1}}}
\newcommand{\tcompare}[1]{\textit{平行于\tref{#1}}}
\newcommand{\lcompare}[1]{\textit{平行于\lref{#1}}}
\newcommand{\ecompare}[1]{\textit{平行于\eref{#1}}}
\newcommand{\ccompare}[1]{\textit{平行于\cref{#1}}}
\newcommand{\func}[3]{#1:\, #2 \rightarrow #3}
\newcommand{\overbar}[1]{\mkern 1.5mu\overline{\mkern-1.5mu#1\mkern-1.5mu}\mkern 1.5mu}
\newcommand{\clo}[1]{\overbar{#1}}
\newcommand{\supi}[2]{\overbar{\int_{#1}^{#2}}}
\newcommand{\infi}[2]{\underbar{\int_{#1}^{#2}}}
\newcommand{\setf}{\mathscr}
\newcommand{\bool}{\mathrm{bool}}
\newcommand{\inc}{++}
\newcommand{\defeq}{:=}
\newcommand{\ntuple}{$n$元组}
\newcommand{\card}[1]{\#\pare{#1}}
\newcommand{\setcond}[2]{\curb{#1 \, \left| \, #2 \right.}}
\newcommand{\setcondl}[2]{\curb{\left. #1 \, \right| \, #2}}
\newcommand{\bv}[1]{\mathbf{#1}}
\newcommand{\bfa}{\bv{a}}
\newcommand{\bfb}{\bv{b}}
\newcommand{\bfx}{\bv{x}}
\newcommand{\bfy}{\bv{y}}
\newcommand{\bfe}{\bv{e}}
\newcommand{\bfF}{\bv{F}}
\newcommand{\bff}{\bv{f}}
\newcommand{\bfG}{\bv{G}}
\newcommand{\bfH}{\bv{H}}
\newcommand{\bfg}{\bv{g}}
\newcommand{\bfh}{\bv{h}}
\newcommand{\bfr}{\bv{r}}
\newcommand{\bfk}{\bv{k}}
\newcommand{\bfu}{\bv{u}}
\newcommand{\bfv}{\bv{v}}
\newcommand{\oo}[1]{o\pare{#1}}
\newcommand{\OO}[1]{O\pare{#1}}
% \newcommand{\norm}[1]{\left\| #1 \right\|}
\newcommand{\DD}{\mathbf{D}}
\newcommand{\comp}{\circ}
\newcommand{\const}{\mathrm{const}}
\newcommand{\dist}[2]{d\pare{#1,#2}}
\newcommand{\len}{\ell}
\newcommand{\siga}{$\sigma$-代数}
\newcommand{\cara}{Carath\'{e}odory}
\newcommand{\Gd}{G_\delta}
\newcommand{\Fs}{F_\sigma}
\newcommand{\mmani}{$m$-维流形}
\newcommand{\open}[1]{\mathcal{#1}}
\newcommand{\half}{\frac{1}{2}}
\newcommand{\maxo}[1]{\text{max}\curb{#1}}
\newcommand{\mino}[1]{\text{min}\curb{#1}}
\newcommand{\epsclo}{$\epsilon$-接近}
\newcommand{\close}[1]{$#1$-接近}
\newcommand{\cinf}{$C^\infty$}
\newcommand{\cuno}{$C^1$}
\newcommand{\Int}{\text{Int}\,}
\newcommand{\Ext}{\text{Ext}\,}
\newcommand{\funcf}{\mathcal}
\newcommand{\DDu}{\overbar{\DD}}
\newcommand{\DDl}{\underbar{\DD}}
\newcommand{\Diff}[1]{\mathrm{Diff}_{#1}\,}
\newcommand{\Av}[1]{\mathrm{Av}_{#1}\,}
\newcommand{\Lip}[1]{Lipschitz-$#1$}
\newcommand{\sgn}{\mathrm{sgn}}
\newcommand{\sgnf}{\mathrm{sgn}\,}
\newcommand{\eset}{\varnothing}
\newcommand{\difn}[1]{^{\pare{#1}}}
%\newcommand{\ca}{\mathcal{a}}
%\newcommand{\cb}{\mathcal{b}}
%\newcommand{\cc}{\mathcal{c}}
%\newcommand{\cd}{\mathcal{d}}
%\newcommand{\ce}{\mathcal{e}}
%\newcommand{\cf}{\mathcal{f}}
%\newcommand{\cg}{\mathcal{g}}
%\newcommand{\ch}{\mathcal{h}}
%\newcommand{\ci}{\mathcal{i}}
%\newcommand{\cj}{\mathcal{j}}
%\newcommand{\ck}{\mathcal{k}}
%\newcommand{\cl}{\mathcal{l}}
%\newcommand{\cm}{\mathcal{m}}
%\newcommand{\cn}{\mathcal{n}}
%\newcommand{\co}{\mathcal{o}}
%\newcommand{\cp}{\mathcal{p}}
%\newcommand{\cq}{\mathcal{q}}
%\newcommand{\cr}{\mathcal{r}}
%\newcommand{\cs}{\mathcal{s}}
%\newcommand{\ct}{\mathcal{t}}
%\newcommand{\cu}{\mathcal{u}}
%\newcommand{\cv}{\mathcal{v}}
%\newcommand{\cw}{\mathcal{w}}
%\newcommand{\cx}{\mathcal{x}}
%\newcommand{\cy}{\mathcal{y}}
%\newcommand{\cz}{\mathcal{z}}
\newcommand{\cA}{\mathcal{A}}
\newcommand{\cB}{\mathcal{B}}
\newcommand{\cC}{\mathcal{C}}
\newcommand{\cD}{\mathcal{D}}
%\newcommand{\cE}{\mathcal{E}}
\newcommand{\cF}{\mathcal{F}}
\newcommand{\cG}{\mathcal{G}}
\newcommand{\cH}{\mathcal{H}}
\newcommand{\cI}{\mathcal{I}}
\newcommand{\cJ}{\mathcal{J}}
\newcommand{\cK}{\mathcal{K}}
\newcommand{\cL}{\mathcal{L}}
\newcommand{\cM}{\mathcal{M}}
\newcommand{\cN}{\mathcal{N}}
\newcommand{\cO}{\mathcal{O}}
\newcommand{\cP}{\mathcal{P}}
\newcommand{\cQ}{\mathcal{Q}}
\newcommand{\cR}{\mathcal{R}}
\newcommand{\cS}{\mathcal{S}}
\newcommand{\cT}{\mathcal{T}}
\newcommand{\cU}{\mathcal{U}}
\newcommand{\cV}{\mathcal{V}}
\newcommand{\cW}{\mathcal{W}}
\newcommand{\cX}{\mathcal{X}}
\newcommand{\cY}{\mathcal{Y}}
\newcommand{\cZ}{\mathcal{Z}}
\newcommand{\inter}[1]{\mathring{#1}}
\newcommand{\forest}[3]{对于{#1},存在{#2},使得{#3}}
\newcommand{\tuno}{$T_1$公理}
\newcommand{\isom}{\overset{\sim}{=}}
\newcommand{\diam}{\mathrm{diam}\,}
\newcommand{\ord}[1]{\abs{#1}}
\newcommand{\sbm}[1]{\overbar{#1}}
\newcommand{\inv}[1]{#1^{-1}}
\newcommand{\restr}[2]{#1|_{#2}}
\newcommand{\divs}{|}
\newcommand{\ndivs}{\nmid}
\newcommand{\modeq}[1]{\overbar{#1}}
\newcommand{\ggen}[1]{\langle#1\rangle}
\newcommand{\ggencond}{\braket}

\newcommand{\hd}{H\"{o}lder}

\renewcommand{\proofname}{证明}

\newenvironment{cenum}{\begin{enumerate}\itemsep0em}{\end{enumerate}}

\newtheorem{definition}{定义}[section]
\newtheorem{lemma}{引理}[section]
\newtheorem{theorem}{定理}[section]
\newtheorem{collary}{推论}[section]
\newtheorem{corollary}{推论}[section]
\newtheorem{proposition}{命题}[section]
\newtheorem{axiom}{公理}[section]
\newtheorem{ass}{假设}[section]
\newtheorem{ex}{例}[section]
\newtheorem{remark}{附注}[section]
%\newtheorem*{remark*}{附注}[section]
\newtheorem{reflection}{反射}[section]
\newcommand{\refl}[1]{\vspace{0.5em}\par\noindent\fbox{%
    \parbox{0.97\textwidth}{%
    \begin{reflection}
        #1
    \end{reflection}
    }%
}\vspace{0.5em}\par}
\newcommand{\rref}[1]{反射\ref{refl:#1}}
\newcommand{\tbref}[1]{表\ref{table:#1}}
\allowdisplaybreaks

\newenvironment{aenum}{\begin{enumerate}[label=\textnormal{(\alph*)}]}{\end{enumerate}}

% Physics Head

\DeclareSIUnit\dyne{dynes}

\newcommand{\ddel}[1]{\frac{\partial}{\partial #1}}
\newcommand{\ddelon}[2]{\frac{\partial #1}{\partial #2}}
\newcommand{\dddel}[1]{\frac{\partial^2}{\partial^2 #1}}
\newcommand{\ddt}{\ddel{t}}
\newcommand{\ddT}{\ddel{T}}
\newcommand{\ddV}{\ddel{V}}
\newcommand{\ddr}{\ddel{r}}
\newcommand{\dds}{\ddel{s}}
\newcommand{\ddron}[1]{\ddelon{#1}{r}}
\newcommand{\ddson}[1]{\ddelon{#1}{s}}
\newcommand{\ddton}[1]{\ddelon{#1}{t}}
\newcommand{\ddxon}[1]{\ddelon{#1}{x}}
\newcommand{\ddyon}[1]{\ddelon{#1}{y}}
\newcommand{\ddzon}[1]{\ddelon{#1}{z}}
\newcommand{\ddthon}[1]{\ddelon{#1}{\theta}}
\newcommand{\ddalon}[1]{\ddelon{#1}{\alpha}}
\newcommand{\ddth}{\ddel{\theta}}
\newcommand{\ddph}{\ddel{\phi}}
\newcommand{\dddt}{\dddel{t}}
\newcommand{\dddr}{\dddel{t}}
\newcommand{\dddth}{\dddel{\theta}}
\newcommand{\dddph}{\dddel{\phi}}
\newcommand{\rd}[1]{\mathrm{d} #1}
\newcommand{\dt}{\rd{t}}
\newcommand{\dy}{\rd{y}}
\newcommand{\dx}{\rd{x}}
\newcommand{\edd}[1]{\frac{\mathrm{d}}{\mathrm{d} #1}}
\newcommand{\eddd}[1]{\frac{\mathrm{d}^2}{\mathrm{d}^2 #1}}
\newcommand{\eddon}[2]{\frac{\mathrm{d} #1}{\mathrm{d} #2}}
\newcommand{\edddon}[2]{\frac{\mathrm{d}^2 #1}{\mathrm{d}^2 #2}}
\newcommand{\edt}{\edd{t}}
\newcommand{\edton}[1]{\eddon{#1}{t}}
\newcommand{\edalon}[1]{\eddon{#1}{\alpha}}
\newcommand{\edT}{\edd{T}}
\newcommand{\edr}{\edd{r}}
\newcommand{\edl}{\edd{l}}
\newcommand{\edx}{\edd{x}}
\newcommand{\edth}{\edd{\theta}}
\newcommand{\eddton}[1]{\edddon{#1}{t}}
\newcommand{\eddzon}[1]{\edddon{#1}{z}}
\newcommand{\vect}[1]{\boldsymbol{#1}}
\newcommand{\alp}{\frac{1}{\sqrt{2}}}
\newcommand{\alpi}{\frac{i}{\sqrt{2}}}
\newcommand{\expc}[1]{\langle#1\rangle}
\newcommand{\bkn}[1]{\bra{#1}\ket{#1}}
\newcommand{\bk}[2]{\bra{#1}\ket{#2}}
\newcommand{\bik}[3]{\bra{#1} #2 \ket{#3}}
\newcommand{\vari}[1]{\sigma_{#1}}
\newcommand{\intc}[2]{\left[#1, #2\right]}
\newcommand{\sch}{Schr\"{o}dinger}
\newcommand{\moment}{\boldsymbol{p}}
\newcommand{\coor}{\boldsymbol{x}}
\newcommand{\lapc}{\nabla^2}
% \newcommand{\rec}[1]{\frac{1}{#1}}
\newcommand{\vva}{\boldsymbol{a}}
\newcommand{\vvb}{\boldsymbol{b}}
\newcommand{\vc}{\boldsymbol{c}}
\newcommand{\vd}{\boldsymbol{d}}
\newcommand{\ve}{\boldsymbol{e}}
\newcommand{\vf}{\boldsymbol{f}}
\newcommand{\vg}{\boldsymbol{g}}
\newcommand{\vh}{\boldsymbol{h}}
\newcommand{\vi}{\boldsymbol{i}}
\newcommand{\vj}{\boldsymbol{j}}
\newcommand{\vk}{\boldsymbol{k}}
\newcommand{\vl}{\boldsymbol{l}}
\newcommand{\vm}{\boldsymbol{m}}
\newcommand{\vn}{\boldsymbol{n}}
\newcommand{\vo}{\boldsymbol{o}}
\newcommand{\vp}{\boldsymbol{p}}
\newcommand{\vq}{\boldsymbol{q}}
\newcommand{\vr}{\boldsymbol{r}}
\newcommand{\vs}{\boldsymbol{s}}
\newcommand{\vt}{\boldsymbol{t}}
\newcommand{\vvu}{\boldsymbol{u}}
\newcommand{\vv}{\boldsymbol{v}}
\newcommand{\vw}{\boldsymbol{w}}
\newcommand{\vx}{\boldsymbol{x}}
\newcommand{\vy}{\boldsymbol{y}}
\newcommand{\vz}{\boldsymbol{z}}
\newcommand{\vA}{\boldsymbol{A}}
\newcommand{\vB}{\boldsymbol{B}}
\newcommand{\vC}{\boldsymbol{C}}
\newcommand{\vD}{\boldsymbol{D}}
\newcommand{\vE}{\boldsymbol{E}}
\newcommand{\vF}{\boldsymbol{F}}
\newcommand{\vG}{\boldsymbol{G}}
\newcommand{\vH}{\boldsymbol{H}}
\newcommand{\vI}{\boldsymbol{I}}
\newcommand{\vJ}{\boldsymbol{J}}
\newcommand{\vK}{\boldsymbol{K}}
\newcommand{\vL}{\boldsymbol{L}}
\newcommand{\vM}{\boldsymbol{M}}
\newcommand{\vN}{\boldsymbol{N}}
\newcommand{\vO}{\boldsymbol{O}}
\newcommand{\vP}{\boldsymbol{P}}
\newcommand{\vQ}{\boldsymbol{Q}}
\newcommand{\vR}{\boldsymbol{R}}
\newcommand{\vS}{\boldsymbol{S}}
\newcommand{\vT}{\boldsymbol{T}}
\newcommand{\vU}{\boldsymbol{U}}
\newcommand{\vV}{\boldsymbol{V}}
\newcommand{\vW}{\boldsymbol{W}}
\newcommand{\vX}{\boldsymbol{X}}
\newcommand{\vY}{\boldsymbol{Y}}
\newcommand{\vZ}{\boldsymbol{Z}}
\newcommand{\vzero}{\boldsymbol{0}}
\newcommand{\vomega}{\boldsymbol{\omega}}
%\newcommand{\half}{\frac{1}{2}}
\newcommand{\thalf}{\frac{3}{2}}
\newcommand{\rot}{\nabla\times}
\newcommand{\divg}{\nabla\cdot}
\newcommand{\cE}{\mathcal{E}}
\newcommand{\conclu}[1]{\vspace{1em}\par\noindent\fbox{\parbox{0.97\linewidth}{#1}}\vspace{1em}}
\newcommand{\subentrynote}{$\bullet$}
\newcommand{\keypoint}[1]{\par\begin{siderules}\subentrynote\quad #1 \end{siderules}\par}
\newcommand{\fconclu}{\boxed}
\newcommand{\pair}[2]{#1 \, #2}
\newcommand{\barbelow}[1]{\underline{#1}}
\newcommand{\intn}[2]{\int #1 \,\mathrm{d} #2}
\newcommand{\intu}[3]{\int_0^{#1} #2 \,\mathrm{d} #3}
\newcommand{\intiu}[3]{\int_{-\infty}^{#1} #2 \, \rd{} #3}
\newcommand{\intui}[2]{\int_0^{\infty} #1 \,\mathrm{d} #2}
\newcommand{\intii}[2]{\int_{-\infty}^{\infty} #1 \,\mathrm{d} #2}
\newcommand{\intt}[2]{\int_0^\infty #1 \, \rd{} #2}
\newcommand{\intr}[2]{\int_{-\infty}^{\infty} #1 \, \rd{} #2}
\newcommand{\intbi}[3]{\int_{#1}^{\infty} #2 \, \rd{} #3}
\newcommand{\intab}[4]{\int_{#1}^{#2} #3 \, \rd{} #4}
\newcommand{\supint}[2]{\overbar{\int_{#1}^{#2}}}
\newcommand{\infint}[2]{\barbelow{\int_{#1}^{#2}}}
\newcommand{\intpostf}[1]{\, \rd{} #1}
\newcommand{\bfactor}[1]{e^{-#1/k_BT}}
\newcommand{\pbfactor}[1]{e^{#1/k_BT}}
\newcommand{\dn}[2]{#1^{\pare{#2}}}
\newcommand{\prodg}[1]{\pare{#1}^\times}
\newcommand{\muc}{S_\Omega}
\newcommand{\otw}{\, || \,}

\newcommand{\notion}{\emph}
\newcommand{\iP}{\mathcal{P}}
\newcommand{\eiP}{e^{-\iP}}
\newcommand{\iF}{\mathcal{F}}
\newcommand{\eiF}{e^{-\iF}}
\newcommand{\iG}{\mathcal{G}}

\newcommand{\rc}{r\cos\theta}
\newcommand{\rs}{r\sin\theta}
\newcommand{\sn}{\mathrm{sn}}
\newcommand{\cn}{\mathrm{cn}}
\newcommand{\rdn}{\mathrm{dn}}

\newcommand{\hankel}{H_p^{\pare{1}}}
\newcommand{\hankell}{H_p^{\pare{2}}}
\newcommand{\hhankel}{H_n^{\pare{1}}}
\newcommand{\hhankell}{H_n^{\pare{2}}}
\newcommand{\ber}{\text{ber}\,}
\newcommand{\bei}{\text{bei}\,}
\newcommand{\kker}{\text{ker}\,}
\newcommand{\kei}{\text{kei}\,}
\newcommand{\Ai}{\text{Ai}}
\newcommand{\Bi}{\text{Bi}}

\newcommand{\re}{\text{Re}\,}

\newcommand{\Fp}{F_\phi}
\newcommand{\Ep}{E_\phi}
\newcommand{\Fx}{F_x}
\newcommand{\FF}{\mathbf{F}}
\newcommand{\Ex}{E_x}

%\newcommand{\erf}{\mathrm{erf}}
\newcommand{\erfi}{\mathrm{erfi}}
\newcommand{\erfc}{\mathrm{erfc}}
\newcommand{\ehxs}[1]{e^{-\frac{#1^2}{2}}}
\newcommand{\dcol}[2]{\[ \left.#1 \hspace{1em}\right\vert\hspace{1em} #2 \]}
\newcommand{\titlegamma}{\texorpdfstring{$\Gamma$}{Gamma}}
\newcommand{\titleB}{\texorpdfstring{$B$}{B}}
\newcommand{\switch}[2]{\brac{#1 | #2}}
\newcommand{\SYSexeref}[1]{(舒幼生#1)}
\newcommand{\warning}[1]{\par\textit{注意:#1}}

% Computer Science Head
\usepackage{listings}
\usepackage{color}

\definecolor{dkgreen}{rgb}{0,0.6,0}
\definecolor{gray}{rgb}{0.5,0.5,0.5}
\definecolor{mauve}{rgb}{0.58,0,0.82}

\lstset{frame=tb,
  language=Java,
  aboveskip=3mm,
  belowskip=3mm,
  showstringspaces=false,
  columns=flexible,
  basicstyle={\small\ttfamily},
  numbers=none,
  numberstyle=\tiny\color{gray},
  keywordstyle=\color{blue},
  commentstyle=\color{dkgreen},
  stringstyle=\color{mauve},
  breaklines=true,
  breakatwhitespace=true,
  tabsize=3
}
\lstset{language=Java}
\newcommand{\snp}[1]{\lstinline!#1!}
\newcommand{\term}[2]{\textbf{#1(#2)}}

\begin{document}
\subsubsection*{图形程序设计}
Swing用于表示“被绘制的”用户界面类,AWT表示事件处理等窗口工具箱的底层机制。所有代码无条件加入
\begin{lstlisting}
import java.awt.*;
import javax.swing.*;
\end{lstlisting}
\par
顶层窗口(未被包含在其他窗口中的窗口)称为\term{框架}{frame}。在Swing中框架类为JFrame,它不绘制在画布上,而是由窗口系统绘制。
\par
类似下图的代码可创建窗口
\begin{lstlisting}
public static void main(String[] args) {
  EventQueue.invokeLater(() -> {
    MyFrame frame = new MyFrame();
    frame.setDefaultCloseOperation(JFrame.EXIT_ON_CLOSE);
    frame.setVisible(true);
  });
}
\end{lstlisting}
\par
注意让框架显示需要调用setVisible方法。可以用setLocation,setBound,setIconImage,setTitle,setResizable方法设置其他属性。
\par
Tookit类包含许多于本地窗口系统打交道的方法,可以借助
\begin{lstlisting}
Toolkit kit = Toolkit.getDefaultToolkit();
Dimension screenSize = kit.getScreenSize();
\end{lstlisting}
获得屏幕大小。
\par
为了向frame中添加组件,需要向frame的\term{内容窗格}{content pane}中添加之,然而实际上可以直接添加为
\begin{lstlisting}
frame.add(c);
\end{lstlisting}
\par
为了定义component类,下列方法是必须的:
\begin{lstlisting}
public void paintComponent(Graphics g) {
  g.drawString("Hello World!", x, y);
}
\end{lstlisting}
paintComponent方法由系统自动调用,无需自行调用。如果需要刷新,应调用repaint方法。Graphics类型的对象g保留绘制图像和文本的设置,所有绘制都依赖于它。
\par
同时还应当重写getPreferredSize方法,返回首选宽度和高度。
\begin{lstlisting}
public Dimension getPreferredSize() {
  return new Dimension(w, h);
}
\end{lstlisting}
\par
最后,不同于上述空框架的setSIze,可以直接用pack方法将frame设置为满意的大小。
\par
Graphics2D提供了比Graphics更强大的绘图功能,可以通过下列方法调用
\begin{lstlisting}
Graphics2D g2 = (Graphics2D) g;
Rectangle2D r = new Rectangle2D.Double(...);
g2.draw(r);
\end{lstlisting}
使用内部类Double是为了使用Double指定坐标。
\par
Rectangle2D和Ellipse2D在很大程度上类似,并且可以通过
\begin{lstlisting}
Rectangle2D r = new Rectangle2D.Double();
r.setFrameFromDIagonal(px, py, qx, qy);
\end{lstlisting}
或者
\begin{lstlisting}
Point2D p = ...; Point2D q = ...;
r.setFrameFromDiagonal(p, q);
\end{lstlisting}
从对角点设定大小。此外还有Line2D可供绘图使用。
\begin{lstlisting}
g2.setPaint(Color.RED);
g2.draw(r);
g2.fill(r);
\end{lstlisting}
\par
可以以给定的颜色绘图或填充。SystemColor.*可获得预设颜色。
\begin{lstlisting}
setBackground(color);
setOpaque(true);
\end{lstlisting}
可以设定背景。
\par
通过
\begin{lstlisting}
String[] fonts = GraphicsEnvironment
                   .getLocalGraphicsEnvironment.
                   .getAvailableFontFamilyNames();
\end{lstlisting}
可以获得字体集。也可以直接使用SansSerif,Serif,Monospaced等直接引用字体。通过
\begin{lstlisting}
// 常规方法
Font f = new Font("SansSerif", Font.BOLD + Font.ITALIC, 14);
// 换字号
Font f2 = f.deriveFont(14.0F);
// 从文件获得
URL url = new URL("Arial.ttf");
InputStream in = url.openStream();
Font f3 = Font.createFont(Font.TRUETYPE_FONT, in);
// 设定字体
g2.setFont(f);
\end{lstlisting}
获得字体并且设定字体。
\par
为了获得渲染字体的大小,需要字符串与绘制设备,执行
\begin{lstlisting}
FontRenderContext context = g2.getFontRenderContext();
Rectangle2D bounds = f.getStringBounds(message, context);
\end{lstlisting}
获得矩形。其坐标横轴为基线,纵轴为左边界,故
\begin{lstlisting}
double width = bounds.getWidth(); // 宽
double height = bounds.getHeight();  // 高
double acsent = -bounds.getY();  // 上坡度

LineMetrics metrics = f.getLineMetrics(message, context);
float ascent = metrics.getAscent(); // 上坡度
float descent = metrics.getDescent(); // 下坡度
float leading = metrics.getLeading(); // 行距
\end{lstlisting}
\par
为了绘制文本,应当获得基线左端点的坐标。
\begin{lstlisting}
g2.drawString(message, x, upperLeftCornerY + ascent);
\end{lstlisting}
\par
通过
\begin{lstlisting}
Image image = new ImageIcon(filename).getImage();
g2.drawImage(image, x, y);
g2.copyArea(upperleftX, upperleftY, bottomrightX, bottomrightY, targetX, targetY);
\end{lstlisting}
获得并绘制、复制图像。
\subsubsection*{事件处理}
AWT处理事件的方式谓,从事件源到事件监听器的事件委托模型。事件源产生事件时,会向事件注册的所有事件监听器对象发送一个通告。
\begin{lstlisting}
class MyListener implements ActionListener {
  public void actionPerformed(ActionEvent event) { ... }
}
ActionListener listener = ...;
button.addActionListener(listener);
// lambda表达式
button.addActionListener(event -> ...);
\end{lstlisting}
\par
为了调整观感,可以借助
\begin{lstlisting}
try {
  UIManager.setLookAndFeel(className);
  SwingUtilities.updateComponentTreeUI(frame);
  pack();
} catch (Exception e) { e.printStackTrace(); }
\end{lstlisting}
其中className的选择范围在
\begin{lstlisting}
UIManager.LookAndFeelInfo[] infos = UIManager.getInstalledLookAndFeels();
\end{lstlisting}
\par
对于有多个事件的Listener接口,可以通过继承其Adapter来获得同样效果,而无需将所有方法均重写。
\begin{lstlisting}
class Terminator extends WindowAdapter {
  public void windowClosing(WindowsEvent e) { ... }
}
frame.addWindowListener(new Terminator());
\end{lstlisting}
\par
可以通过继承Action或AbstractAction以实现将同一动作复用至多个事件源。
\begin{lstlisting}
public class MyAction extends AbstractAction {
  public void ActionPerformed(ActionEvent event) { ... }
}
Action myAction = new MyAction(...);
JButton myButton = new JButton(myAction); // 添加给按钮
JMenuItem myItem = new JMenuItem(myAction); // 添加给菜单项

\end{lstlisting}
还可以通过设定
\begin{lstlisting}
public myAction() {
  putValue(Action.NAME, name);
  putValue(Action.SMALL_ICON, icon);
}
\end{lstlisting}
改变按钮上的文字与图标。
\par
为了获得键盘输入,先通过InputMap将子元素的键盘输入映射到动作键,再通过ActionMap将动作键映射到Action。
\begin{lstlisting}
InputMap imap = panel.getInputMap(JComponent.WHEN_ANCESTOR_OF_FOCUSED_COMPONENT);
imap.put(KeyStroke.getKeyStroke("control Y"), "panel.yellow");
ActionMap amap = panel.getActionMap();
amap.put("panel.yellow", yellowAction);
\end{lstlisting}
与鼠标点击有关的事件定义在MouseAdapter内,与鼠标移动、拖动有关的事件定义在MouseMotionListener内。
\begin{lstlisting}
setCursor(Cursor.getDefaultCursor());
setCursor(Cursor.getPredefinedCursor(Cursor.CROSSHAIR_CURSOR));
\end{lstlisting}
可以修改光标样式。
\par
除了上述各种Listener,还有用于滚动条的AdjustmentListener,用于复选框的ItemListener等。
\subsubsection*{各种控件}
表格中的数据以二维数组的形式储存
\begin{lstlisting}
Object[][] cells = { { "Alice", 92 }, { "Bob", 73 } };
\end{lstlisting}
标题以\snp{String[]}的形式储存,后
\begin{lstlisting}
JTable table = new JTable(cells, titles);
JScrollPane pane = new JScrollPane(table);
\end{lstlisting}
\subsubsection*{XML}
为了读取XML,通过
\begin{lstlisting}
DocumentBuilderFactory factory = DocumentBuilderFactory.newInstance();
DocumentBuilder builder = factory.newDocumentBuilder();
File f = ... ; // 文件方式
Document doc = builder.parse(f);
URL u = ... ; // 指定URL
Document doc = buidler.parse(u);
InputStream in = ... ; // 指定输入流
Document doc = builder.parse(in);
\end{lstlisting}
之后通过
\begin{lstlisting}
Element root = doc.getDocumentElement();
String tagName = root.getTagname();
\end{lstlisting}
获得根节点与标签名。之后枚举
\begin{lstlisting}
NodeList children = root.getChildNodes();
for ... {
  Node child = children.item(i);
  if (child instance of Element) Element childElement = (Element) child;
}
\end{lstlisting}
对于只包含文本的Element,可以
\begin{lstlisting}
Text textNode = (Text) childElement.getFirstChild();
String text = textNode.getData().trim();
\end{lstlisting}
\par
为了获得属性,可以通过
\begin{lstlisting}
NamedNodeMap attributes = element.getAttributes();
Node attribute = attributes.item(0);
String name = attribute.getNodeName();
String value = attribute.getNodeValue();
// 知道属性名可以直接获取
String unit = element.getAttribute("unit");
\end{lstlisting}
\end{document}