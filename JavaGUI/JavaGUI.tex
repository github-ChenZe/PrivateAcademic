\documentclass{ctexart}

\usepackage{listings}
\usepackage{color}

\definecolor{dkgreen}{rgb}{0,0.6,0}
\definecolor{gray}{rgb}{0.5,0.5,0.5}
\definecolor{mauve}{rgb}{0.58,0,0.82}

\lstset{frame=tb,
  language=Java,
  aboveskip=3mm,
  belowskip=3mm,
  showstringspaces=false,
  columns=flexible,
  basicstyle={\small\ttfamily},
  numbers=none,
  numberstyle=\tiny\color{gray},
  keywordstyle=\color{blue},
  commentstyle=\color{dkgreen},
  stringstyle=\color{mauve},
  breaklines=true,
  breakatwhitespace=true,
  tabsize=3
}
\lstset{language=Java}

\newcommand{\term}[2]{\textbf{#1(#2)}}

\begin{document}
Swing用于表示“被绘制的”用户界面类,AWT表示事件处理等窗口工具箱的底层机制。所有代码无条件加入
\begin{lstlisting}
import java.awt.*;
import javax.swing.*;
\end{lstlisting}
\par
顶层窗口(未被包含在其他窗口中的窗口)称为\term{框架}{frame}。在Swing中框架类为JFrame,它不绘制在画布上,而是由窗口系统绘制。
\par
类似下图的代码可创建窗口
\begin{lstlisting}
public static void main(String[] args) {
  EventQueue.invokeLater(() -> {
    MyFrame frame = new MyFrame();
    frame.setDefaultCloseOperation(JFrame.EXIT_ON_CLOSE);
    frame.setVisible(true);
  });
}
\end{lstlisting}
\par
注意让框架显示需要调用setVisible方法。可以用setLocation,setBound,setIconImage,setTitle,setResizable方法设置其他属性。
\par
Tookit类包含许多于本地窗口系统打交道的方法,可以借助
\begin{lstlisting}
Toolkit kit = Toolkit.getDefaultToolkit();
Dimension screenSize = kit.getScreenSize();
\end{lstlisting}
获得屏幕大小。
\par
为了向frame中添加组件,需要向frame的\term{内容窗格}{content pane}中添加之,然而实际上可以直接添加为
\begin{lstlisting}
frame.add(c);
\end{lstlisting}
\par
为了定义component类,下列方法是必须的:
\begin{lstlisting}
public void paintComponent(Graphics g) {
  g.drawString("Hello World!", x, y);
}
\end{lstlisting}
paintComponent方法由系统自动调用,无需自行调用。Graphics类型的对象g保留绘制图像和文本的设置,所有绘制都依赖于它。
\par
同时还应当重写getPreferredSize方法,返回首选宽度和高度。
\begin{lstlisting}
public Dimension getPreferredSize() {
  return new Dimension(w, h);
}
\end{lstlisting}
\par
最后,不同于上述空框架的setSIze,可以直接用pack方法将frame设置为满意的大小。
\par
Graphics2D提供了比Graphics更强大的绘图功能,可以通过下列方法调用
\begin{lstlisting}
Graphics2D g2 = (Graphics2D) g;
Rectangle2D r = new Rectangle2D.Double(...);
g2.draw(r);
\end{lstlisting}
使用内部类Double是为了使用Double指定坐标。
\par
Rectangle2D和Ellipse2D在很大程度上类似,并且可以通过
\begin{lstlisting}
Rectangle2D r = new Rectangle2D.Double();
r.setFrameFromDIagonal(px, py, qx, qy);
\end{lstlisting}
或者
\begin{lstlisting}
Point2D p = ...; Point2D q = ...;
r.setFrameFromDiagonal(p, q);
\end{lstlisting}
从对角点设定大小。此外还有Line2D可供绘图使用。
\begin{lstlisting}
g2.setPaint(Color.RED);
g2.draw(r);
g2.fill(r);
\end{lstlisting}
\par
可以以给定的颜色绘图或填充。SystemColor.*可获得预设颜色。
\end{document}