\documentclass{ctexart}

\usepackage{listings}
\usepackage{color}

\definecolor{dkgreen}{rgb}{0,0.6,0}
\definecolor{gray}{rgb}{0.5,0.5,0.5}
\definecolor{mauve}{rgb}{0.58,0,0.82}

\lstset{frame=tb,
  language=Java,
  aboveskip=3mm,
  belowskip=3mm,
  showstringspaces=false,
  columns=flexible,
  basicstyle={\small\ttfamily},
  numbers=none,
  numberstyle=\tiny\color{gray},
  keywordstyle=\color{blue},
  commentstyle=\color{dkgreen},
  stringstyle=\color{mauve},
  breaklines=true,
  breakatwhitespace=true,
  tabsize=3
}
\lstset{language=Java}

\newcommand{\term}[2]{\textbf{#1(#2)}}

\begin{document}
\subsubsection*{图形程序设计}
Swing用于表示“被绘制的”用户界面类,AWT表示事件处理等窗口工具箱的底层机制。所有代码无条件加入
\begin{lstlisting}
import java.awt.*;
import javax.swing.*;
\end{lstlisting}
\par
顶层窗口(未被包含在其他窗口中的窗口)称为\term{框架}{frame}。在Swing中框架类为JFrame,它不绘制在画布上,而是由窗口系统绘制。
\par
类似下图的代码可创建窗口
\begin{lstlisting}
public static void main(String[] args) {
  EventQueue.invokeLater(() -> {
    MyFrame frame = new MyFrame();
    frame.setDefaultCloseOperation(JFrame.EXIT_ON_CLOSE);
    frame.setVisible(true);
  });
}
\end{lstlisting}
\par
注意让框架显示需要调用setVisible方法。可以用setLocation,setBound,setIconImage,setTitle,setResizable方法设置其他属性。
\par
Tookit类包含许多于本地窗口系统打交道的方法,可以借助
\begin{lstlisting}
Toolkit kit = Toolkit.getDefaultToolkit();
Dimension screenSize = kit.getScreenSize();
\end{lstlisting}
获得屏幕大小。
\par
为了向frame中添加组件,需要向frame的\term{内容窗格}{content pane}中添加之,然而实际上可以直接添加为
\begin{lstlisting}
frame.add(c);
\end{lstlisting}
\par
为了定义component类,下列方法是必须的:
\begin{lstlisting}
public void paintComponent(Graphics g) {
  g.drawString("Hello World!", x, y);
}
\end{lstlisting}
paintComponent方法由系统自动调用,无需自行调用。如果需要刷新,应调用repaint方法。Graphics类型的对象g保留绘制图像和文本的设置,所有绘制都依赖于它。
\par
同时还应当重写getPreferredSize方法,返回首选宽度和高度。
\begin{lstlisting}
public Dimension getPreferredSize() {
  return new Dimension(w, h);
}
\end{lstlisting}
\par
最后,不同于上述空框架的setSIze,可以直接用pack方法将frame设置为满意的大小。
\par
Graphics2D提供了比Graphics更强大的绘图功能,可以通过下列方法调用
\begin{lstlisting}
Graphics2D g2 = (Graphics2D) g;
Rectangle2D r = new Rectangle2D.Double(...);
g2.draw(r);
\end{lstlisting}
使用内部类Double是为了使用Double指定坐标。
\par
Rectangle2D和Ellipse2D在很大程度上类似,并且可以通过
\begin{lstlisting}
Rectangle2D r = new Rectangle2D.Double();
r.setFrameFromDIagonal(px, py, qx, qy);
\end{lstlisting}
或者
\begin{lstlisting}
Point2D p = ...; Point2D q = ...;
r.setFrameFromDiagonal(p, q);
\end{lstlisting}
从对角点设定大小。此外还有Line2D可供绘图使用。
\begin{lstlisting}
g2.setPaint(Color.RED);
g2.draw(r);
g2.fill(r);
\end{lstlisting}
\par
可以以给定的颜色绘图或填充。SystemColor.*可获得预设颜色。
\begin{lstlisting}
setBackground(color);
setOpaque(true);
\end{lstlisting}
可以设定背景。
\par
通过
\begin{lstlisting}
String[] fonts = GraphicsEnvironment
                   .getLocalGraphicsEnvironment.
                   .getAvailableFontFamilyNames();
\end{lstlisting}
可以获得字体集。也可以直接使用SansSerif,Serif,Monospaced等直接引用字体。通过
\begin{lstlisting}
// 常规方法
Font f = new Font("SansSerif", Font.BOLD + Font.ITALIC, 14);
// 换字号
Font f2 = f.deriveFont(14.0F);
// 从文件获得
URL url = new URL("Arial.ttf");
InputStream in = url.openStream();
Font f3 = Font.createFont(Font.TRUETYPE_FONT, in);
// 设定字体
g2.setFont(f);
\end{lstlisting}
获得字体并且设定字体。
\par
为了获得渲染字体的大小,需要字符串与绘制设备,执行
\begin{lstlisting}
FontRenderContext context = g2.getFontRenderContext();
Rectangle2D bounds = f.getStringBounds(message, context);
\end{lstlisting}
获得矩形。其坐标横轴为基线,纵轴为左边界,故
\begin{lstlisting}
double width = bounds.getWidth(); // 宽
double height = bounds.getHeight();  // 高
double acsent = -bounds.getY();  // 上坡度

LineMetrics metrics = f.getLineMetrics(message, context);
float ascent = metrics.getAscent(); // 上坡度
float descent = metrics.getDescent(); // 下坡度
float leading = metrics.getLeading(); // 行距
\end{lstlisting}
\par
为了绘制文本,应当获得基线左端点的坐标。
\begin{lstlisting}
g2.drawString(message, x, upperLeftCornerY + ascent);
\end{lstlisting}
\par
通过
\begin{lstlisting}
Image image = new ImageIcon(filename).getImage();
g2.drawImage(image, x, y);
g2.copyArea(upperleftX, upperleftY, bottomrightX, bottomrightY, targetX, targetY);
\end{lstlisting}
获得并绘制、复制图像。
\subsubsection*{事件处理}
AWT处理事件的方式谓,从事件源到事件监听器的事件委托模型。事件源产生事件时,会向事件注册的所有事件监听器对象发送一个通告。
\begin{lstlisting}
class MyListener implements ActionListener {
  public void actionPerformed(ActionEvent event) { ... }
}
ActionListener listener = ...;
button.addActionListener(listener);
// lambda表达式
button.addActionListener(event -> ...);
\end{lstlisting}
\par
为了调整观感,可以借助
\begin{lstlisting}
try {
  UIManager.setLookAndFeel(className);
  SwingUtilities.updateComponentTreeUI(frame);
  pack();
} catch (Exception e) { e.printStackTrace(); }
\end{lstlisting}
其中className的选择范围在
\begin{lstlisting}
UIManager.LookAndFeelInfo[] infos = UIManager.getInstalledLookAndFeels();
\end{lstlisting}
\par
对于有多个事件的Listener接口,可以通过继承其Adapter来获得同样效果,而无需将所有方法均重写。
\begin{lstlisting}
class Terminator extends WindowAdapter {
  public void windowClosing(WindowsEvent e) { ... }
}
frame.addWindowListener(new Terminator());
\end{lstlisting}
\par
可以通过继承Action或AbstractAction以实现将同一动作复用至多个事件源。
\begin{lstlisting}
public class MyAction extends AbstractAction {
  public void ActionPerformed(ActionEvent event) { ... }
}
Action myAction = new MyAction(...);
JButton myButton = new JButton(myAction); // 添加给按钮
JMenuItem myItem = new JMenuItem(myAction); // 添加给菜单项

\end{lstlisting}
还可以通过设定
\begin{lstlisting}
public myAction() {
  putValue(Action.NAME, name);
  putValue(Action.SMALL_ICON, icon);
}
\end{lstlisting}
改变按钮上的文字与图标。
\par
为了获得键盘输入,先通过InputMap将子元素的键盘输入映射到动作键,再通过ActionMap将动作键映射到Action。
\begin{lstlisting}
InputMap imap = panel.getInputMap(JComponent.WHEN_ANCESTOR_OF_FOCUSED_COMPONENT);
imap.put(KeyStroke.getKeyStroke("control Y"), "panel.yellow");
ActionMap amap = panel.getActionMap();
amap.put("panel.yellow", yellowAction);
\end{lstlisting}
与鼠标点击有关的事件定义在MouseAdapter内,与鼠标移动、拖动有关的事件定义在MouseMotionListener内。
\begin{lstlisting}
setCursor(Cursor.getDefaultCursor());
setCursor(Cursor.getPredefinedCursor(Cursor.CROSSHAIR_CURSOR));
\end{lstlisting}
可以修改光标样式。
\par
除了上述各种Listener,还有用于滚动条的AdjustmentListener,用于复选框的ItemListener等。
\end{document}