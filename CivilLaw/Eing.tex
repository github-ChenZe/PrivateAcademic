% !TEX encoding = UTF-8 Unicode
\documentclass[UTF-8]{ctexrep}
\usepackage[version=4]{mhchem}
\usepackage{array,booktabs}
\usepackage{longtable}
\usepackage{amsmath}
\usepackage{amsthm}
\usepackage[normalem]{ulem}
\newcommand{\article}[1]{民法第#1条}
\newcommand{\articlesub}[2]{民法第#1条第#2项}
\newcommand{\bgb}[1]{BGB第#1条}
\newcommand{\cod}[2]{\vspace{1em}\par\noindent\fbox{\parbox{\textwidth}{\textbf{民法第#1条}\quad\textit{#2}}}\vspace{1em}}
\newcommand{\quot}[1]{“\emph{#1}”}
\newcommand{\cas}[2]{#1年台上字#2号}
\newcommand{\prov}[2]{\par\noindent\rule{\textwidth}{1pt}\par {\itshape #1}\par\mbox{}\hfill——#2\par\noindent\rule{\textwidth}{1pt}}
\newtheorem{ex}{例}
\newcommand{\pA}{{甲}}
\newcommand{\pB}{{乙}}
\newcommand{\pC}{{丙}}
\newcommand{\pD}{{丁}}
\newcommand{\dotss}{{\centerline{... ...}}}
\title{民法 笔记}
\author{C.Z.}
\begin{document}
  \maketitle
  \chapter{不当得利}
  \section{绪论}
  \subsection{机能与体系}
  \cod{179}{無法律上之原因而受利益,致他人受損害者,應返還其利益。雖有法律上
之原因,而其後已不存在者,亦同。}
  \par
  \emph{反射利益}如房屋因邻地兴建商场而增值,利用他人灯塔夜航捕鱼等不属于不当得利。\quot{致他人受損害}是否有法律上之原因,与具体法律领域相关,例如第三人善意取得动产所有权则有法律上之原因。
  \paragraph{不当得利制度的作用}认定财产变动过程中受益者是否具有法律上之原因保有其所得利益。
  \begin{enumerate}
  \item 矫正欠缺法律关系的财货移转:撤销有瑕疵契约的效果。
  \item 保护财货的归属:无权处分得向无权处分人追偿。
  \end{enumerate}
    \prov{按不當得利請求權,係以使得利人返還其所受利得為目的,非以相對人(損失者)所
受損害之填補為目的,故與損害賠償請求權不同。
  \par
  \dotss
  \par
  既係利得之返還,則原審准許被
上訴人請求上訴人塗銷已為之繼承登記,亦不知何所依憑?上訴論旨,指摘原判決違
法,聲明廢棄,非無理由。}{\cas{1984}{3398}}
  \par
  不当得利制度在于去除所受利益,不一定能填补被害人所受损害。受益人之故意或过失,是否具有可非难性,在所不问。随常与侵权行为并存,亦有不存在侵权行为而单独存在不当得利之例。
  \paragraph{不当得利的性质}不当得利为债之发生原因之一,其发生则基于\quot{無法律上之原因而受利益,致他人受損害}之事实,不一定为人之行为,如牲畜误食他人饲料。
  \paragraph{成立要件与法律效果}积极要件参照\article{179},消极要件谓
  \cod{180}{給付,有左列情形之一者,不得請求返還:\\
一、給付係履行道德上之義務者。\\
二、債務人於未到期之債務因清償而為給付者。\\
三、因清償債務而為給付,於給付時明知無給付之義務者。\\
四、因不法之原因而為給付者。但不法之原因僅於受領人一方存在時,不
    在此限。}
  \par
  不当得利引发若干法律效果,由\article{181\textasciitilde 183}规定。
  \cod{181}{不當得利之受領人,除返還其所受之利益外,如本於該利益更有所取得者
,並應返還。但依其利益之性質或其他情形不能返還者,應償還其價額。}
  \cod{182}{不當得利之受領人,不知無法律上之原因,而其所受之利益已不存在者,
免負返還或償還價額之責任。\\
受領人於受領時,知無法律上之原因或其後知之者,應將受領時所得之利
益,或知無法律上之原因時所現存之利益,附加利息,一併償還;如有損
害,並應賠償。}
  \par
  例如\emph{菜单案}中菜已下肚,无法返还,即属此例。
  \cod{183}{不當得利之受領人,以其所受者,無償讓與第三人,而受領人因此免返還
義務者,第三人於其所免返還義務之限度內,負返還責任。}
  \par
  民法尚设有若干准用不当得利之条文,如
  \begin{enumerate}
  \item \article{197}允许侵权行为时效消灭后仍得依不当得利追偿
  \item \article{266}关于给付不能而免于对待给付之不当得利设有规定
  \item \articlesub{419}{2}规定了赠与撤销时发生的不当得利
  \item \article{816}则以不当得利“衡平”动产附和与混合造成的所有权取得
  \end{enumerate}
  上揭条文之准用是否及于构成要件,抑或仅准用法律后果,不无疑问。盖动产之附和造成取得,实际上似乎已经有\article{811}之“法律上之原因”;而\article{266}中倘若契约并未失效时,受领给付似乎具有法律上的原因(参见\ref{par:defect})。
  \par
  各条主旨亦可概括如表\ref{table:article179-183}。
  \begin{table}[!h]
  \centering
  \begin{tabular}{|c|c|}
    \hline
    \article{179} & 要件:无法律上之原因,致他人受损害\\
    \article{180} & 排除(例外):道德,提前,明知,不法\\
    \article{181} & 客体:连本带利及无法返还之例外\\
    \article{182} & 范围(例外与例外):不知且灭失则免,明知且灭失不得免\\
    \article{183} & 第三人返还义务\\
    \hline
  \end{tabular}
  \caption{各条主旨}
  \label{table:article179-183}
  \end{table}
  \begin{ex}
    \pA 售A书予\pB ,误交付A、B二书。\pB 赠与\pC 并交付之。
    \begin{enumerate}
    \item \pB 虽应依\article{181}返还书的所有权及占有的利益,然而此时书已交付,不能返还,应依\article{181}但书偿还其价额;
    \item \pB 不知无法律上之原因,依\article{182}免返还义务;
    \item 依\article{183}\pC 在\pB 免返还义务之限度内负返还责任。
    \end{enumerate}
    故\pA 可向\pC 要求返还书的所有权与占有。
    \end{ex}
  \paragraph{与契约的关系}不当得利补救失败的契约,与\article{259}之解除契约的恢复原状义务同具调整功能,二者之关系有争议。
  \paragraph{与无因管理的关系}明知为他人事物之情形,得构成无因管理而不成立不当得利。设他人之物为自己之物而修缮,或不知租赁契约不成立而整修房屋,则构成支出费用不当得利。
  \paragraph{与侵权行为的关系}不当得利得与侵权行为竞合,即不当得利之侵害权益不当得利,指无法律上之原因取得他人权利的归属内容之不当得利,如消费他人食物。不以损害或过失为要件,可补充侵权行为法。
  \paragraph{与物权行为的关系}\begin{enumerate}
  \item \article{767}之物上请求权与不当得利之关系;
  \item \articlesub{759}{1}、\article{801}、\article{948}、\article{811}、\article{816}之善意取得、添附等致物权消灭时,不当得利得替代物权请求权以保护权利;
  \item \article{951}以下之占有恢复请求权与不当得利请求权之竞合。
  \end{enumerate}
  \subsection{比较法上的观察}
    \begin{table}[!htbp]
  \centering
  \begin{tabular}{|>{\centering}m{2.7cm}| >{\centering}m{4.5cm}| >{\centering\arraybackslash}m{3.5cm}|}
    \hline
    \textbf{\bgb{812}} & 无法律上之原因,或虽有而日后消失;债务承认或免除视为给付 & 相当于\article{179} \\
    \bgb{813} & 虽有永久抗辩而履行得请求返还,提前清偿不得请求返还或中间利息 & \article{180}未规定中间利息 \\
    \bgb{814} & 道德、明知则不得请求 & 相当于\articlesub{180}{2\textasciitilde3}\\
    \bgb{815} & 明知自始不能,或蓄意阻碍,不得请求 & 让人联想起\article{101}\\
    \bgb{816} & 无权处分与善意取得,以及无偿取得 & 未见类似规定,无偿取得之情形类似于\article{183}\\
    \bgb{817} & 给付人欲受领人违反公序良俗或违法,则受领人应返还;但给付人亦违反则仅得请求返还负担而不得请求返还处分 & 相当于\articlesub{183}{4}\\
    \textbf{\bgb{818}} & 孳息,不能返还则折价,不再存在则无义务 & 相当于\article{181}与\article{182}{1}\\
    \textbf{\bgb{819}} & 受益人明知或自知悉时负返还义务;因受领给付违反法律或公序良俗亦同 & 参考\bgb{817},前段相当于\articlesub{182}{2} \\
    \sout{\bgb{820}} & 给付结果不明确时或原因可能消失时,结果未发生或原因消失时负有返还义务 & 未设规定\\
    \bgb{821} & 无法律上的原因承担债务,得拒绝履行 & 未设规定 \\
    \textbf{\bgb{822}} & 无偿让于第三人,则第三人负返还义务,视为第三人直接之不当得利 & 相当于\article{183}\\
    \hline
  \end{tabular}
  \caption{德国民法典与台湾民法之比较}
  \label{table:bgbtotw}
  \end{table}
  \paragraph{罗马法}罗马法无统一的不当得利制度,仅承认个别诉权如非债清偿,目的不达、盗窃、违反公序良俗与不法原因之不当得利。
  \paragraph{法国法}不当得利与无因管理被视为准契约,尚未形成独立制度,不当得利仅就非债清偿有单独规定,并在判例中对其一般原则有所承认。
  \paragraph{德国法}有与\article{179}相似之规定,并有若干个别不当得利与排除之消极要件,同时规定债务承认与债务免除为给付。德国法具有统一说与非统一说的争论,检讨\quot{受利益致他人受损害}的要件。
  \paragraph{瑞士法}瑞士于1882年即将不当得利列入债之发生原因并设有一般规定,其判例学说亦对不当得利采用非统一说。
  \paragraph{日本民法}日本民法有与\article{179}相似之规定,且同样采非统一说为通说。
  \paragraph{大陆民法}采概括原则,《民通意见》亦将返还范围延及孳息。
  \paragraph{英美法}不当得利分散于各种诸如准契约、信托的制度中。美国有草案试图将不当得利分情形纳入准契约、侵权行为、其他(如误偿他人之债)处理。英国法学说有认为准契约或其他请求权皆建立在不当得利的原则上。英国上议院肯定其原则并创设与\article{179}相当之判例,且承认非统一说,然而不当得利之发生则采用个别因素如错误、胁迫。
  \paragraph{私法欧洲化}欧洲民法典试图创设等价于\article{179}的规定。
  \subsection{中国法上的观察}
  《民律草案》的不当得利之规定继受自德国民法,并就无权处分之不当得利有规定,现行民法去除之,滋生疑义。现行民法对德国民法之继受,可比较如表\ref{table:bgbtotw}。

    \par
  现行民法简化德国民法的设定,删除如\bgb{812}后段、\bgb{820}之所求结果未发生的不当得利,继受要件、范围、明知或不知与第三人义务等规定。
  \par
  台湾民法对已继受的德国民法部分得参照德国之解释,未继受之部分应作相同解释或反面推论,应视其能否纳入现行法规范体系而认定。最高法院于\cas{2012}{1722}判决承认不当得利的非统一说。
  \subsection{不当得利的起源:衡平原则}
  \prov{Jure aequum est nemiem cum alterus detrimento et iniuria fieri locupletiorem.\par 损人利己,违反衡平。}{Pomponius}
  \par
  不当得利的思想源自于衡平,即虽财产价值之移动在形式上正当,而实质上不正当时,则应依公平理念调整之。尽管如此,现今之不当得利已成具体法律制度,不应令衡平原则过度干预。最高法院于\cas{1997}{1102}径直引用“公平正义”为判决依据,而绕过\article{179}之成文法,应当避免。此外,最高法院一再坚持使用他人之物需有致他人受“无法使用之损害”之要件,实则应以受益人取得归属于他人之权利为要。借他人房屋悬挂招牌,无论是否导致无法使用,皆应当构成不当得利。
  \par
  关于承揽契约终止而定作人之\article{511}但书损害赔偿请求权罹于时效消灭后,最高法院\cas{1984}{4477}认为承揽契约向将来失效,且承揽所受损害与定作人之撤销契约有\emph{相当因果关系}且令定作人受有利益,故得请求返还不当得利。
  \prov{如上訴人就其承攬工作之全部支出費用,因被上
訴人終止契約而受有損害,並使被上訴人受有利益,此項利益與上訴人所受損害之間
有相當因果關係,即與民法第一百七十九條後段所定:「雖有法律上之原因而其後已
不存在者」之情形相當,上訴人似非不得據以請求被上訴人返還不當得利。原審僅以
被上訴人在承攬契約有效期間內,因上訴人完成工作而受利益,係本於契約而來,並
非無法律上之原因,不備不當得利之要件,據以認定上訴人對於被上訴人不得為返還
不當得利之請求,尤嫌率斷。上訴論旨,執以指摘原判決不當,聲明廢棄,非無理由
。}{\cas{1984}{4477}}
  \par
  然而在给付型不当得利,不应当以相当因果关系作为判据,而应代之以给付关系。致命之处在于,承揽契约既向将来失效,其失效前所受利益既有法律上之原因,应当不构成不当得利。
  \par
  \prov{終止
契約,僅使契約自終止之時起向將來消滅,並無溯及之效力,使契約自始歸於消滅。
故定作人在承攬契約有效期間內,因承攬人所為工作致受利益,乃本於終止前有效之
承攬契約而來,並非無法律上之原因,與不當得利之要件不符。故終止契約後,不論
被上訴人有無受利益,上訴人如受有損害,僅得依民法第五百十一條但書之規定,請
求損害賠償,不生返還不當得利請求權相與競合而得選擇行使之問題。}{\cas{1988}{69}}
  \par
  所幸如上,最高法院在四年之后推翻前述观点,认定不发生不当得利。
  \par
  物的瑕疵担保于不当得利亦存在争议。
  \prov{被上訴人交付之房屋坪數短少,而有溢收價
金之情形,如果屬實,被上訴人對於溢收之房屋價金,是否不能成立不當得利,尚有
疑問}{\cas{69}{677}}
  \par
  \label{par:defect}
  最高法院在此案中认为\quot{對給付原因之欠缺,目的之不能達到,亦屬給付原因欠缺形態之一種},且物之瑕疵存在而溢付价金应当属于目的不达。然而目的不达通常指欲实现某种目的而为给付,事后发现目的未能达成,则此时相对人已不具有保留利益之基础,构成不当得利,即以目的不达作为“无法律上之原因”。如以受清偿为目的开具收据而未获清偿,即预期条件未成就(参考表\ref{table:types})。此处“清偿”之目的已达,难谓有何不当得利可言。
  \par
  买受人于出卖人订立买卖契约,形成两个债权。买受人支付价金,在于对出卖人之债权为清偿。除非契约存在瑕疵而无效或经撤销(视为自始无效),否则此项清偿难以“目的不达”。苟滥用不当得利,则物之瑕疵担保与契约解除殆可以不当得利代之。
  \par
  值得注意的是,于此处若当事人解除契约,则可以依构成不当得利,且与\article{259}发生竞合。然而当事人欲解除契约,亦受\articlesub{365}{1}之六个月时效限制。而一般不当得利请求权消灭时效为15年,故若得绕过\article{256}之“解除契约”而径直适用不当得利,则等价于绕过六个月之时效。
  \prov{解除契約,係指契約當事人之一方,行使解除權而使契約自始歸於消滅者
而言。債之契約既溯及的消滅,則因契約之履行而受益之一方,即欠缺法
律上之原因,形成不當得利,故因履行契約而為給付之一方,固得依民法
第二百五十九條之規定,行使回復原狀請求權,亦得行使不當得利返還請
求權。}{\cas{1993}{1292}}
  \subsubsection{类型化}
  无法律上之原因之判断,有统一说与非统一说。非统一说的Wilburg认为应区分因给付而受利益与因给付以外的行为而受利益分别探求财产变动是否有法律上之原因。
  \begin{table}[!h]
    \centering
    \begin{tabular}{>{\centering}m{6cm} >{\centering\arraybackslash}m{3cm}}
      \hline
      甲售车与乙,交付后乙撤销契约 & 目的自始欠缺 \\
      甲对乙表示考上律师时赠书,乙表示猜题成功并交付,后乙落榜 & 目的不达 \\
      甲男与乙女订婚赠屋,移转所有权,后婚约解除 & 目的消灭 \\
      \hline
      盗取物并使用,或占有而无权处分 & 权益侵害 \\
      误认而为修缮 & 支出费用 \\
      甲向乙保留所有权分期购买车并售予丙,丙向乙支付剩余价金 & 求偿 \\
      \hline
    \end{tabular}
    \caption{类型与例子}
    \label{table:types}
  \end{table}
  \par
  最高法院曾就个别具体情形探究是否成立不当得利,迄未为统一之说明,惟其在\cas{2012}{1722}将不当得利类型化。
  \prov{不當得利依其類型可區分為「給付型之不當得利」
與「非給付型不當得利」,前者係基於受損人有目的及有意識之
給付而發生之不當得利,後者乃由於給付以外之行為(受損人、
受益人、第三人之行為)或法律規定所成立之不當得利。又於「
非給付型之不當得利」中之「權益侵害之不當得利」,凡因侵害
歸屬於他人權益內容而受利益,致他人受損害,即可認為基於同
一原因事實致他人受損害,並欠缺正當性;亦即以侵害行為取得
應歸屬他人權益內容的利益,而不具保有該利益之正當性,即應
構成無法律上之原因,成立不當得利。}{\cas{2012}{1722}}
  \par
  判决强调“给付目的”作为给付型不当得利请求权的要件,亦指出权利侵害不当得利之存在,且权利侵害型不当得利,应以权利归属内容为判断基准。
  \begin{ex}
    \pA 与\pB 为通谋虚伪表示而转让房产予\pB ,\pB 向\pC 抵押房产取得借款。
    \begin{enumerate}
    \item \pC 善意取得抵押权,有法律上之原因,不构成不当得利;
    \item \pA 与\pB 间转让所有权之契约因通谋虚伪表示而无效,\pB 受有房产占有之利益且来自于\pA 之给付,故构成不当得利;
    \item \pB 无权处分之并由\pC 善意取得,且获得借款利益并使\pA 受有房产附抵押权之损害,\pB 所受借款利益应归属于房产所有人,构成不当得利。
    \end{enumerate}
  \end{ex}
  \par
  关于采纳非统一说,有若干理由:
  \paragraph{立法史}民律草案原就二情形有分别说明,且\article{180}所用\quot{给付}之字眼亦意指给付型不当得利。
  \paragraph{统一说缺陷}衡平说有失周密,权利说无法解释因法律规定(物权行为无因性)而取得利益的情形,债权说则无法解释诸如时效取得等非给付而取得利益但不属于不当得利的情形。
  \paragraph{类型化明确要件}给付型不当得利在于调整欠缺给付目的之财产变动,其思想乃凡依当事人意思而增进他人财产者,均有一定目的,倘目的自始不存在、目的不达或目的消灭,给付即欠缺法律上之原因而构成不当得利。给付外的事由,亦得按照受益人、受损人与第三人的行为而分类(参见表\ref{table:types})。
  \paragraph{比较法}德国,瑞士与奥地利等概括立法的国家对不当得利皆采用非统一说且类型化。
\end{document}