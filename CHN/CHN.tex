\documentclass{ctexart}

\usepackage{ulem}
\usepackage{ifthen}
\usepackage[usenames, dvipsnames]{color}
\usepackage{graphicx}
\newcommand\sbullet[1][.5]{\mathbin{\vcenter{\hbox{\scalebox{#1}{$\bullet$}}}}}
\newcommand{\subentrynote}{$\bullet$}
\newcommand{\aaa}{\=a}
\newcommand{\ii}{\=i}
\newcommand{\oo}{\=o}
\newcommand{\ee}{\=e}
\definecolor{gray}{gray}{0.75}
\newcommand{\eliminate}[1]{\textcolor{gray}{#1}}
\newcommand{\notimplemented}{{\color{red} 亟待補充}}

\title{威斯敏斯特戰記\\
\large Commentari\ii\ d\ee\ Bell\oo\ Westminster\ii}
\author{C.Z.}

\newcounter{cvoc}


\begin{document}
\newcommand{\ddd}{$\cdots$}
\newcommand{\ent}{\textsuperscript}
\newcommand{\pI}{\=}
\newcommand{\pII}{\'}
\newcommand{\pIII}{\v}
\newcommand{\pIV}{\`}
\newcommand{\eg}[1]{\quad---\textit{#1}}
\newcommand{\warn}{\par\noindent - \textsl{警告:}}
\newcommand{\prono}[1]{\textsuperscript{\textnormal{\textmd{[\textit{#1}]}}}}
\newcommand{\quot}[2]{\textit{\ifthenelse{\equal{#1}{}}{}{#1 :}\ifthenelse{\equal{#2}{}}{}{「#2」}}}
\newcommand{\centry}[1]{\\ \noindent \subentrynote #1}
\newcommand{\sentry}[1]{\newline \noindent \textit{按}:#1}
\newcommand{\entry}[4]{\par \noindent \stepcounter{cvoc} \textbf{#1} :#2 。 \ifthenelse{\equal{#3}{}\AND \equal{#4}{}}{}{\quad---\quot{#3}{#4}}}
\newcommand{\subentry}[4]{\\ \noindent \subentrynote #1:#2 。 \ifthenelse{\equal{#3}{}\AND \equal{#4}{}}{}{\quad---\quot{#3}{#4}}}
\newcommand{\eentry}[3]{\subentry{\textsl{後起義}}{#1}{#2}{#3}}
\newcommand{\bentry}[3]{\subentry{\textsl{引申義}}{#1}{#2}{#3}}
\newcommand{\aentry}[3]{\subentry{\textsl{使動}}{#1}{#2}{#3}}
\newcommand{\rentry}[3]{\subentry{\textit{另義}}{#1}{#2}{#3}}
\newcommand{\pentry}[4]{\subentry{\textit{#1}}{#2}{#3}{#4}}
\newcommand{\simp}[1]{$*$\textsl{#1}}
\newcommand{\qentry}[2]{\par\noindent\quot{#1}{#2}}
\newcommand{\egqentry}[2]{\quad---\quot{#1}{#2}}
\newcommand{\qref}[3]{\quad---\textit{#1}.p.\-#2.\-#3}
\newcommand{\term}[2]{\par\subentrynote\textit{表述:}#1 謂\textbf{#2}。}
\newcommand{\bio}{せ}
\newcommand{\tek}{テ}

\newcommand{\uncertain}[1]{$*$#1}
\maketitle
\section*{通論}
空缺。
\section{國文}
\subsection*{國文·通論}
應注意分點作答,且不惜以答案流暢性換取清晰度,得借助[*結論:*分析]之樣式。
\subsection{國文·卷I·古詩詞}
\subsubsection*{通論}
回答表達情感的問題,應當首列所表達之情感,後置分析。分析時,應當使用概括性語句,無需直接引用原文。
\par
分析思想感情應當注意可能的變化,即在同一首詩中思想感情不一定完全一致,分析時應闡明變化過程。
\subsubsection*{擬人}
作用為使被描寫對象更形象生動,其判定不必拘泥於物本身的動作,亦可由人的動作推斷擬人化。\egqentry{點絳唇·訪牟存叟南漪釣隱}{卷帘盡放春愁去}
\subsubsection*{用典}
用典之判定,可依常識或註釋為之。
\subsubsection*{虛寫}
虛寫之判定,可依常識或目的為之。\egqentry{訪隱者不遇成二絕}{滄江白石漁樵路,日暮歸來雨滿衣}
\par
虛寫之作用,若為夢境,可謂思極入幻,以夢或幻境表現思念,較直接敘述更深。\egqentry{詩·關雎}{窈窕淑奴,鐘鼓樂之}
\subsubsection*{象徵}
「角」,即鳥的堅硬的嘴,象徵強暴者。\egqentry{詩·行露}{誰謂雀無角,何以穿我屋}
\subsubsection*{意象}
對農耕之描寫視為農村生活圖畫,表現生活之自由、沒有。\egqentry{擊壤歌}{吾日出而作,日落而息}
\par
對動物嬉戲之樂的描寫可作男女情思的啓興。\egqentry{詩·關雎}{關關雎鳩,在河之州}
\par
對江水的描寫可視為以江水之渺茫襯托焦慮心境。\egqentry{詩·漢廣}{漢之廣矣,不可泳思}
\subsubsection*{意象·植物}
「桃花」可渲染明豔色澤與蓬勃生機,以及鄉土氣息。\egqentry{詩·桃夭}{桃之夭夭,灼灼其華}\egqentry{歸園田居}{榆柳蔭後園,桃李羅堂前}
\subsubsection*{懸想}
慾表現思念,寫對方思念自己。\egqentry{詩·卷耳}{我姑酌彼金罍,維以不永懷}\egqentry{九月九日憶山東兄弟}{遙知兄弟登高處,遍插茱萸少一人}
\subsubsection*{唱嘆抒情}
「那堪」等語氣較重的反問可視為唱嘆抒情。\egqentry{溪行逢雨與柳中庸}{那堪兩處宿,共聽一聲猿}
\par
反覆詠唱,一唱三嘆,可以加強情感。\egqentry{詩·桃夭}{桃之夭夭\ddd}\egqentry{詩·漢廣}{漢之廣矣\ddd}
\par
反覆詠唱亦可渲染輕快氣氛。\egqentry{詩·芣莒}{采采芣莒\ddd}
\subsubsection*{說理抒情}
辯證的寫法可視為說理,似亦可作對比。\egqentry{三峽吟}{啼猿不自愁,愁落行人心}
\subsubsection*{點面結合}
同一情形之描寫,概覽者謂「面」,具體之其中一對象者謂「點」。作用謂相互映襯,細節傳神,豐富情感。\egqentry{阮郎歸·西湖暮春}{香紅漸漸稀\ddd 花褪雨,絮沾泥}
\subsubsection*{煉字·頻度}
「常」用於強調時間頻度,強調「常」字後的物。\egqentry{秋暮吟望}{寒山常帶斜陽色}
\subsubsection*{煉字·轉折}
「偏」將自然現象視為有意為之,改變客觀性,凸顯情感。\egqentry{秋暮吟望}{新月偏明葉落時}
\par
「怪」解為「怪罪」,可用以側面凸顯情感。\egqentry{望江怨·送別}{怪人催去早}
\par
「怪」解偉「難怪」,可用以揭示原因或表達讚美。\egqentry{休暇日訪王侍御不遇}{怪來詩思清人骨,門對寒流雪滿山}
\subsubsection*{煉字·動詞}
「鬧」得體現愉悅,可引伸為喜愛。\egqentry{阮郎歸·西湖暮春}{番騰裝束鬧蘇堤}
\par
「不移」表現不捨。\egqentry{阮郎歸·西湖暮春}{凌波寸不移}
\par
「棄」「置」表現心痛苦喪,加強情感。\egqentry{詩·卷耳}{嗟我懷仁,置彼周行}
\par
「茹」表示容納,容忍,反之亦然。\egqentry{詩·柏舟}{我心匪鑑,不可以茹}
\subsection{國文·卷II·古文}
\subsubsection*{「寓山注」序}
\entry{固}{本來}{論語·子罕}{固天縱之將聖,又多能也。}
\entry{畚}{農具}{}{}
\entry{鍤}{農具}{}{}
\entry{楹}{量詞}{唐·陸龜蒙·甫里先生傳}{先生之居,有地數畝,屋三十楹。}
\entry{胼胝}{繭,厚皮}{荀子·子道}{耕耘樹藝,手足胼胝。}
\centry{辭源:call\oo sum:皮糙肉厚的。corpus call\oo sus:胼胝體。}
\entry{榭}{屋子}{左傳·襄公三十一年}{宮室卑庳\ent{庳:低下},無觀臺榭。}
\entry{及}{等到}{論語·季氏}{及其壯也,血氣方剛,戒之在鬭。}
\entry{橐\prono{ㄊㄨ\pII{ㄛ}}}{口袋}{詩·大雅·公劉}{迺\prono{ㄋ\pIII{ㄞ}}\ent{迺:於是。}裹餱糧,于橐于囊。}
\entry{迨\prono{ㄉ\pIV{ㄞ}}}{及,趕上}{}{}
\eentry{等到}{}{}
\entry{牖\prono{丨\pIII{ㄡ}}}{窗}{詩經·召南·采蘋}{于以奠之,宗室牖下。}
\entry{几}{矮桌子}{孟子·公孫丑·下}{隱几而臥。}
\entry{堪}{經得起}{詩經·周頌·小毖}{未堪家多難。}
\bentry{可能}{韓非子·難三}{君令不二,除君之惡,唯恐不堪。}
\warn{堪不得解釋為堪稱。}
\warn{[*a]有[*b]者應當解釋為有[b]的[a]}\eg{「客有吹洞簫者」解釋為「有吹洞簫的客人」。}
\warn{補全句子成分。}\eg{這是開園的經營、構建(的情況)。}
\warn{警惕強加因果。}\eg{\sout{美景}使作者的興趣濃厚。}
\subsubsection*{天下之患}
\entry{棘}{「戟」,兵器}{左傳·隱公十一年}{潁考叔挾車以走,子都拔棘以逐之。}
\entry{矜\prono{ㄑ丨\pII{ㄣ}}}{矛柄}{}{}
\warn{詞語的解釋應當視其語境區分近義詞。}\eg{「此陳涉所以為資也」的「資」應當解釋為「憑藉」而非「資質」。}
\entry{勸}{鼓勵}{尚書·大禹·謨}{勸之以九歌。}
\entry{禽}{獵物}{}{}
\bentry{「擒」,擒獲}{}{}
\entry{間\prono{ㄐ丨ㄢ}}{近來}{文選·三國·魏·嵇康·與山巨源絕交書}{間聞組下遷,惕然不喜。}
\entry{重\prono{ㄓㄨ\pIV{ㄥ}}}{加重,增加}{楚辭·屈原·離騷}{紛吾既有此內美兮,有重之以脩能。}
\entry{數\prono{ㄙ\pIV{ㄨ}}}{道理}{韓非子·孤憤}{夫以疏遠與近愛信爭,其數不勝也。}
\entry{化}{變化}{禮記·月令·季春之月}{以禮樂全天地之化。}
\warn{「化」可直接作名詞。}
\entry{機}{關鍵}{}{}
\entry{廟堂}{朝廷}{}{}
\centry{吐槽:\simp{庙堂}之字形對其本意不甚友善。}
\warn{注意名詞的所有者。}\eg{「雖有強國勁兵」指外敵的「強國勁兵」。}
\entry{私}{私交}{史記·項羽本紀}{項王乃疑范增與漢有私。}
\entry{負}{背靠着}{禮記·孔子間居}{子夏蹶然而起,負牆而立。}
\entry{扆\prono{\pIII{丨}}}{屏風}{荀子·儒效}{周公\ddd 負扆而坐,諸侯趨走堂下。}
\entry{攝}{提起}{論語·鄉黨}{攝齊\prono{ㄗ}\ent{齊,衣的下襬。}升堂,鞠躬如也。}
\warn{詞語的本意無法令句子通順時可以適用引申義。}\eg{「臣聞圖王不成,其敝足以安」之「敝」可譯作「最壞的結果」。}
\subsubsection*{真州東園記}
\warn{詞的翻譯應與詞性相配。}\eg{「敞其中以為清宴之堂」之「敞」應作「開闢」。}
\entry{甍\prono{ㄇ\pII{ㄥ}}}{屋棟,屋脊}{唐·王勃·滕王閣序}{披繡闥,俯雕甍。}
\entry{桷\prono{ㄐㄩ\pII{ㄝ}}}{椽子}{詩經·魯頌·悶宮}{松桷有舄\prono{ㄒ\pIV{丨},集韻ㄊㄨㄛ}\ent{舄:大貌。},路寢孔{\ent:甚。}碩。}
\entry{垣}{圍牆}{尚書·梓材}{若作室家,既勤垣墉。}
\entry{塹}{坑}{}{}
\entry{信}{真實,的確}{論語·憲問}{信乎,夫子不言,不笑,不取乎?}
\entry{力}{盡力}{詩經·大雅·烝民}{古訓是\ent{是:賓語前置。}式\ent{式:榜樣,效法。},威儀是力。}
\warn{「有」在古代漢語可作無義祠頭。}{「吾于是信有力焉」應作「我對此確實是盡力了」。}
\warn{補全句子成分,不拘泥於名詞,亦得為動賓短語。}\eg{「豈獨私吾三人者哉」應作「修建東園,難道僅僅是為了我們三個人嗎」。}
\warn{「以」字可表修飾。}\eg{「草木日益以茂」「余與四人擁火以入」。}
\warn{警惕因果斷章。多處提及同一事的原因,則應判斷何處為真實原因。}
\subsubsection*{呂僧珍傳}
\entry{曹}{輩,同類}{呂氏春秋}{貪得偽詐之曹遠矣。}
\pentry{曹輩}{同輩}{}{}
\entry{祅}{「妖」,反常}{左傳·宣公十五年}{天反時爲災,地反物為祅。}
\pentry{祅賊}{反賊。}{}{}
\entry{當}{承擔,擔當}{孟子·離婁下}{不詳之實,蔽賢者當之。}
\entry{行}{行列}{詩經·大雅·常武}{左右陳行,戒我師旅。}
\pentry{當行}{擔當職務}{}{}
\centry{比較:\quot{}{陳勝、吳廣皆次當行。}}
\entry{由}{從此行走}{論語·雍也}{行不由徑。}
\entry{問}{音訊}{漢書·匈奴傳}{漢不知吉音問。}
\entry{固}{堅決}{尚書·舜典}{禹拜稽首固辭。}
\entry{嘉}{讚美,讚賞}{尚書·大禹·謨}{嘉乃丕績。}
\entry{膂}{脊梁}{尚書·君牙}{今命爾予翼,作股肱心膂。}
\pentry{心膂}{親信,得力助手}{}{}
\warn{翻譯不必拘泥於組詞,必要時可修改使適合語境。}\eg{「武帝甚嘉之」之「嘉」作「讚賞」,不必作「嘉獎」。}
\entry{頗}{程度較高}{史記·河渠書}{其後漕稍多,而渠下之民頗得以溉田矣。}
\centry{對「頗」的翻譯可以較為靈活,作「都」「注重」等均可。}
\entry{按}{巡行}{史記·衛青列傳}{遂西定河南地,按榆谿舊賽。}
\entry{沈\prono{ㄔ\pII{ㄣ}}}{沒於水中}{詩經·小雅·菁菁者莪}{汎汎楊舟,載沈載浮。}
\entry{葺\prono{ㄑ\pIV{丨}}}{用茅草蓋屋}{周禮·考工記·匠人}{葺屋參分,瓦屋四分。}
\entry{辦}{辦成}{左傳·哀公三年}{無備而官辦者,猶拾瀋\prono{ㄕ\pIII{ㄣ}}\ent{瀋:汁水。}也。}
\entry{旋}{歸還}{易經·履}{視履\ent{履:實踐。}考祥\ent{祥:吉凶的徵兆。},其旋元吉。}
\pentry{旋軍}{軍隊凱旋而歸}{}{}
\entry{廨\prono{ㄒ丨\pIV{ㄝ}}}{官署,官舍}{論衡·感虛}{星之在天也,爲日月舍,猶地由郵亭,爲長吏廨也。}
\entry{從\prono{ㄘㄨ\pII{ㄥ},ㄗㄨ\pIV{ㄥ}}}{隨行}{左傳·莊公十年}{戰則請從。}
\aentry{使\ddd 跟從}{史記·項羽本紀}{沛公旦日從百余騎來見項王。}
\qentry{釋文}{從,才用反。}
\centry{吐槽:釋文的注音與課本不同。}
\sentry{\sout{釋文的編者陸德明為蘇州人,察吳語對「從」之發音同ㄗㄨㄥ,釋文之發音可能受編者方言影響。}構擬之上古發音如此。}
\entry{鹵簿}{儀仗隊}{}{}
\entry{輦}{搬運}{淮南子·人間}{一鼓,民被甲括矢,操兵弩而出。再鼓,負輦粟而至。}
\subsubsection*{夏侯詳傳}
\entry{艱}{遭父母喪}{世說新語·德行}{吳道助、附子兄弟居在丹陽郡後,遭母童夫人艱,朝夕哭臨。}
\entry{毀}{居喪過於哀傷而毀壞身體}{韓非子·內儲說上}{宋崇門之巷人,服喪而毀,甚瘠。}
\entry{舉}{攻克}{孟子·梁惠王上}{以萬乘之國,伐萬乘之國,五旬而舉之。}
\entry{奉}{事奉}{左傳·僖公十一年}{鄭伯使許大夫百里奉許叔以居許東偏。}
\entry{款}{真誠,誠懇}{荀子·修身}{愚\ent{愚:老實。}款端慤\prono{ㄑㄩ\pIV{ㄝ}}\ent{慤:忠厚。},則合之以禮樂。}
\entry{敢}{冒昧}{論語·先進}{敢問死。曰:『未知生,焉知死?』}
\entry{布}{陳述}{左傳·昭公十六年}{僑若獻玉,不知所成,敢私布之。}
\entry{腹}{內心的想法}{左傳·宣公十二年}{敢布腹心,君實圖之。}
\entry{雅}{極,甚}{南朝·梁·劉勰·文心雕龍·時序}{觀其時文,雅好慷慨。}
\entry{遇}{對待,禮遇}{韓非子·外儲說右上}{宋人有酤酒者,升慨甚平,遇客甚謹\ent{謹:恭謹,恭敬。}。}
\entry{酬}{酬對,贈答}{漢·張衡·思玄賦}{有無言而不酬兮,又何往而不復。}
\entry{末略}{漫不經心貌}{梁書·夏侯詳傳}{每引詳及鄉人裴叔業日夜與語,詳輒末略不酬。}
\entry{頓}{停留,駐紮}{史記·淮陰侯列傳}{今將軍欲舉倦獘\prono{ㄅ\pIV{ㄧ}}\ent{獘:「斃」,仆倒。}之兵,頓之燕堅城之下,欲戰恐久力不能拔。}
\entry{略}{策略}{漢書·趙充國傳}{臣願馳至金城,圖上方略。}
\pentry{經略}{謀劃策略}{}{}
\entry{延}{延請,接待}{漢書·公孫弘傳}{於是起客館,開東閣,以延賢人。}
\entry{損}{謙抑}{史記·晏嬰列傳}{其後夫自抑損。}
\entry{挹\prono{\pIV{丨}}}{謙退}{文選·漢·朱浮·爲幽州牧與彭寵書}{俠遊謙讓,屢有降挹之言。}
\subsubsection*{送高陽齊皥下第序}
\warn{上下文對偶可有助於推斷詞義。}\eg{與「舉」相對之「去」可解為「不任用」而非離去。}
\entry{怫\prono{ㄅ\pIV{ㄟ}}}{「悖」,違反}{史記·太史公自序}{五家之文怫異。}
\entry{居}{坐下}{論語·陽貨}{居!吾語女\ent{女:「汝」。}。}
\entry{命}{\uncertain{道理}}{}{}
\entry{用}{表原因}{詩經·小旻}{謀夫孔多,是用不集。}
\entry{計}{審核,考察}{管子·八觀}{行其田野,視其耕耘,計其農事,而飢飽之國可以知也。}
\subsubsection*{魏誠甫行狀}
\entry{而}{連接主謂,強調}{左傳·襄公三十年}{子產而死,誰其嗣\ent{嗣:繼承}之?}
\rentry{反問}{論語·顏淵}{為仁由己,而由人乎哉?}
\entry{竟}{最終}{史記·淮陰侯列傳}{(韓)信亦知其意,怒,竟絕而去。}
\entry{諱}{尊長的名}{禮記·王制}{大史典禮,執簡記,奉諱惡。}
\entry{撓}{屈服}{戰國策·魏策四}{秦王色撓,長跪而謝之。}
\aentry{使屈服}{}{}
\entry{恣睢}{放縱暴戾貌}{荀子·非十二子}{縱性情,安恣睢,禽獸行。}
\entry{已}{隨後,旋即}{史記·夏本紀}{迺召湯而囚之夏臺;已而釋之。}
\pentry{已而}{\textit{Ibid.}}{}{}
\entry{砥勵}{鍛鍊,勤勉}{荀子·王制}{砥礪百姓。}
\entry{匡}{糾正}{孝經}{匡救其惡。}
\entry{飭\prono{\pIV{ㄔ}}}{整頓}{詩經·小雅·六月}{戎車既飭。}
\entry{褒(襃)}{衣襟寬大}{漢書·雋不疑傳}{襃衣博帶,盛服至門上謁。}
\entry{貲\prono{ㄗ}}{計算,估量}{後漢書·陳蕃傳}{而采女數千,食肉衣綺,脂油粉黛,不可貲計。}
\entry{忻}{「欣」,喜}{史記·周本紀}{姜原出野,見巨人跡,心忻然悅,欲踐之。}
\entry{率}{大都}{史記·老子韓非列傳}{故其著書十萬餘言,大抵率寓言也。}
\entry{廢}{疲乏不起}{禮記·中庸}{半塗而廢。}
\entry{紛}{雜亂}{\notimplemented}{\notimplemented}
\entry{比}{及,等到}{孟子·梁惠王下}{比其反也,則凍餒\prono{ㄋ\pIII{ㄟ}}\ent{餒:飢餓。}其妻子。}
\entry{畜}{「蓄」,積蓄}{詩經·邶風·谷風}{我有旨畜,亦以御冬。}
\par
\vspace{2cm}
凡\arabic{cvoc}詞。
\subsection{國文·卷III·引用}
\qentry{}{取乎其上,得乎其中;取乎其中,得乎其下;取乎其下,则无所得矣。}
\qentry{管子·牧民}{倉稟實而知禮節,衣食足而知榮辱。}
\qentry{禮記·曲禮}{賢者狎而敬之,畏而愛之。}
\qentry{诗经·大雅·烝民}{柔則茹之,剛則吐之。}
\qentry{诗经·大雅·烝民}{柔亦不茹,剛亦不吐。}
\qentry{孟子·離婁下}{不祥之實,蔽賢者當之。}
\qentry{明·劉基·賣柑者言}{金玉其外,敗絮其中。}
\qentry{論語·顏淵}{為仁由己,而由人乎哉?}
\qentry{Cicero}{Inter arma enim silent leges.}
\qentry{西塞羅}{將在外,國法有所不從。}
\entry{聞過則喜}{---}{孟子·公孫丑上}{子路,人告知以有過則喜}


\end{document}