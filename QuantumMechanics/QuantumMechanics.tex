\documentclass{ctexart}

% Mathematics Include

\usepackage{amsmath}
\usepackage{amssymb}
\usepackage{amsthm}
\usepackage{amsfonts}
\usepackage{mathrsfs}
\usepackage{enumitem}
\usepackage{braket}
\usepackage{hyperref}
\usepackage[all, pdf]{xy}
\usepackage{wrapfig}
\usepackage{leftidx}

% Physics Include
\usepackage{amsmath}
\usepackage{physics}
\usepackage{siunitx}
\usepackage[makeroom]{cancel}
\usepackage{pstricks}
\usepackage{pstricks-add}
\psset{algebraic=true}

\usepackage[version=4]{mhchem}
\usepackage{array,booktabs}
\usepackage{longtable}
\usepackage{mathtools}
\usepackage[normalem]{ulem}
\usepackage{multicol}

\usepackage{mdframed}
\usepackage{lipsum}% just to generate text for the example

\newmdenv[
  topline=false,
  bottomline=false,
  skipabove=\topsep,
  skipbelow=\topsep
]{siderules}


% Mathematics Head

\newcommand{\pare}[1]{\left(#1\right)}
\newcommand{\blr}[1]{\left[#1\right)}
\newcommand{\lbr}[1]{\left(#1\right]}
\newcommand{\brac}[1]{\left[#1\right]}
\newcommand{\curb}[1]{\left\{#1\right\}}
% \newcommand{\abs}[1]{\left|\, #1 \,\right|}
\newcommand{\rec}[1]{\frac{1}{#1}}
\newcommand{\N}{\mathbb{N}}
\newcommand{\bC}{\mathbb{C}}
\newcommand{\Q}{\mathbb{Q}}
\newcommand{\Z}{\mathbb{Z}}
\newcommand{\R}{\mathbb{R}}
\newcommand{\unk}{\mathcal{X}}
\newcommand{\bu}[3]{#1_{#2}^{\pare{#3}}}
\newcommand{\dref}[1]{定义\ref{def:#1}}
\newcommand{\tref}[1]{定理\ref{thm:#1}}
\newcommand{\lref}[1]{引理\ref{lem:#1}}
\newcommand{\cref}[1]{推论\ref{coll:#1}}
\newcommand{\pref}[1]{命题\ref{prp:#1}}
\newcommand{\rmref}[1]{附注\ref{rm:#1}}
\newcommand{\eref}[1]{例\ref{ex:#1}}
\newcommand{\dcompare}[1]{\textit{平行于\dref{#1}}}
\newcommand{\tcompare}[1]{\textit{平行于\tref{#1}}}
\newcommand{\lcompare}[1]{\textit{平行于\lref{#1}}}
\newcommand{\ecompare}[1]{\textit{平行于\eref{#1}}}
\newcommand{\ccompare}[1]{\textit{平行于\cref{#1}}}
\newcommand{\func}[3]{#1:\, #2 \rightarrow #3}
\newcommand{\overbar}[1]{\mkern 1.5mu\overline{\mkern-1.5mu#1\mkern-1.5mu}\mkern 1.5mu}
\newcommand{\clo}[1]{\overbar{#1}}
\newcommand{\supi}[2]{\overbar{\int_{#1}^{#2}}}
\newcommand{\infi}[2]{\underbar{\int_{#1}^{#2}}}
\newcommand{\setf}{\mathscr}
\newcommand{\bool}{\mathrm{bool}}
\newcommand{\inc}{++}
\newcommand{\defeq}{:=}
\newcommand{\ntuple}{$n$元组}
\newcommand{\card}[1]{\#\pare{#1}}
\newcommand{\setcond}[2]{\curb{#1 \, \left| \, #2 \right.}}
\newcommand{\setcondl}[2]{\curb{\left. #1 \, \right| \, #2}}
\newcommand{\bv}[1]{\mathbf{#1}}
\newcommand{\bfa}{\bv{a}}
\newcommand{\bfb}{\bv{b}}
\newcommand{\bfx}{\bv{x}}
\newcommand{\bfy}{\bv{y}}
\newcommand{\bfe}{\bv{e}}
\newcommand{\bfF}{\bv{F}}
\newcommand{\bff}{\bv{f}}
\newcommand{\bfG}{\bv{G}}
\newcommand{\bfH}{\bv{H}}
\newcommand{\bfg}{\bv{g}}
\newcommand{\bfh}{\bv{h}}
\newcommand{\bfr}{\bv{r}}
\newcommand{\bfk}{\bv{k}}
\newcommand{\bfu}{\bv{u}}
\newcommand{\bfv}{\bv{v}}
\newcommand{\oo}[1]{o\pare{#1}}
\newcommand{\OO}[1]{O\pare{#1}}
% \newcommand{\norm}[1]{\left\| #1 \right\|}
\newcommand{\DD}{\mathbf{D}}
\newcommand{\comp}{\circ}
\newcommand{\const}{\mathrm{const}}
\newcommand{\dist}[2]{d\pare{#1,#2}}
\newcommand{\len}{\ell}
\newcommand{\siga}{$\sigma$-代数}
\newcommand{\cara}{Carath\'{e}odory}
\newcommand{\Gd}{G_\delta}
\newcommand{\Fs}{F_\sigma}
\newcommand{\mmani}{$m$-维流形}
\newcommand{\open}[1]{\mathcal{#1}}
\newcommand{\half}{\frac{1}{2}}
\newcommand{\maxo}[1]{\text{max}\curb{#1}}
\newcommand{\mino}[1]{\text{min}\curb{#1}}
\newcommand{\epsclo}{$\epsilon$-接近}
\newcommand{\close}[1]{$#1$-接近}
\newcommand{\cinf}{$C^\infty$}
\newcommand{\cuno}{$C^1$}
\newcommand{\Int}{\text{Int}\,}
\newcommand{\Ext}{\text{Ext}\,}
\newcommand{\funcf}{\mathcal}
\newcommand{\DDu}{\overbar{\DD}}
\newcommand{\DDl}{\underbar{\DD}}
\newcommand{\Diff}[1]{\mathrm{Diff}_{#1}\,}
\newcommand{\Av}[1]{\mathrm{Av}_{#1}\,}
\newcommand{\Lip}[1]{Lipschitz-$#1$}
\newcommand{\sgn}{\mathrm{sgn}}
\newcommand{\sgnf}{\mathrm{sgn}\,}
\newcommand{\eset}{\varnothing}
\newcommand{\difn}[1]{^{\pare{#1}}}
%\newcommand{\ca}{\mathcal{a}}
%\newcommand{\cb}{\mathcal{b}}
%\newcommand{\cc}{\mathcal{c}}
%\newcommand{\cd}{\mathcal{d}}
%\newcommand{\ce}{\mathcal{e}}
%\newcommand{\cf}{\mathcal{f}}
%\newcommand{\cg}{\mathcal{g}}
%\newcommand{\ch}{\mathcal{h}}
%\newcommand{\ci}{\mathcal{i}}
%\newcommand{\cj}{\mathcal{j}}
%\newcommand{\ck}{\mathcal{k}}
%\newcommand{\cl}{\mathcal{l}}
%\newcommand{\cm}{\mathcal{m}}
%\newcommand{\cn}{\mathcal{n}}
%\newcommand{\co}{\mathcal{o}}
%\newcommand{\cp}{\mathcal{p}}
%\newcommand{\cq}{\mathcal{q}}
%\newcommand{\cr}{\mathcal{r}}
%\newcommand{\cs}{\mathcal{s}}
%\newcommand{\ct}{\mathcal{t}}
%\newcommand{\cu}{\mathcal{u}}
%\newcommand{\cv}{\mathcal{v}}
%\newcommand{\cw}{\mathcal{w}}
%\newcommand{\cx}{\mathcal{x}}
%\newcommand{\cy}{\mathcal{y}}
%\newcommand{\cz}{\mathcal{z}}
\newcommand{\cA}{\mathcal{A}}
\newcommand{\cB}{\mathcal{B}}
\newcommand{\cC}{\mathcal{C}}
\newcommand{\cD}{\mathcal{D}}
%\newcommand{\cE}{\mathcal{E}}
\newcommand{\cF}{\mathcal{F}}
\newcommand{\cG}{\mathcal{G}}
\newcommand{\cH}{\mathcal{H}}
\newcommand{\cI}{\mathcal{I}}
\newcommand{\cJ}{\mathcal{J}}
\newcommand{\cK}{\mathcal{K}}
\newcommand{\cL}{\mathcal{L}}
\newcommand{\cM}{\mathcal{M}}
\newcommand{\cN}{\mathcal{N}}
\newcommand{\cO}{\mathcal{O}}
\newcommand{\cP}{\mathcal{P}}
\newcommand{\cQ}{\mathcal{Q}}
\newcommand{\cR}{\mathcal{R}}
\newcommand{\cS}{\mathcal{S}}
\newcommand{\cT}{\mathcal{T}}
\newcommand{\cU}{\mathcal{U}}
\newcommand{\cV}{\mathcal{V}}
\newcommand{\cW}{\mathcal{W}}
\newcommand{\cX}{\mathcal{X}}
\newcommand{\cY}{\mathcal{Y}}
\newcommand{\cZ}{\mathcal{Z}}
\newcommand{\inter}[1]{\mathring{#1}}
\newcommand{\forest}[3]{对于{#1},存在{#2},使得{#3}}
\newcommand{\tuno}{$T_1$公理}
\newcommand{\isom}{\overset{\sim}{=}}
\newcommand{\diam}{\mathrm{diam}\,}
\newcommand{\ord}[1]{\abs{#1}}
\newcommand{\sbm}[1]{\overbar{#1}}
\newcommand{\inv}[1]{#1^{-1}}
\newcommand{\restr}[2]{#1|_{#2}}
\newcommand{\divs}{|}
\newcommand{\ndivs}{\nmid}
\newcommand{\modeq}[1]{\overbar{#1}}
\newcommand{\ggen}[1]{\langle#1\rangle}
\newcommand{\ggencond}{\braket}

\newcommand{\hd}{H\"{o}lder}

\renewcommand{\proofname}{证明}

\newenvironment{cenum}{\begin{enumerate}\itemsep0em}{\end{enumerate}}

\newtheorem{definition}{定义}[section]
\newtheorem{lemma}{引理}[section]
\newtheorem{theorem}{定理}[section]
\newtheorem{collary}{推论}[section]
\newtheorem{corollary}{推论}[section]
\newtheorem{proposition}{命题}[section]
\newtheorem{axiom}{公理}[section]
\newtheorem{ass}{假设}[section]
\newtheorem{ex}{例}[section]
\newtheorem{remark}{附注}[section]
%\newtheorem*{remark*}{附注}[section]
\newtheorem{reflection}{反射}[section]
\newcommand{\refl}[1]{\vspace{0.5em}\par\noindent\fbox{%
    \parbox{0.97\textwidth}{%
    \begin{reflection}
        #1
    \end{reflection}
    }%
}\vspace{0.5em}\par}
\newcommand{\rref}[1]{反射\ref{refl:#1}}
\newcommand{\tbref}[1]{表\ref{table:#1}}
\allowdisplaybreaks

\newenvironment{aenum}{\begin{enumerate}[label=\textnormal{(\alph*)}]}{\end{enumerate}}

% Physics Head

\DeclareSIUnit\dyne{dynes}

\newcommand{\ddel}[1]{\frac{\partial}{\partial #1}}
\newcommand{\ddelon}[2]{\frac{\partial #1}{\partial #2}}
\newcommand{\dddel}[1]{\frac{\partial^2}{\partial^2 #1}}
\newcommand{\ddt}{\ddel{t}}
\newcommand{\ddT}{\ddel{T}}
\newcommand{\ddV}{\ddel{V}}
\newcommand{\ddr}{\ddel{r}}
\newcommand{\dds}{\ddel{s}}
\newcommand{\ddron}[1]{\ddelon{#1}{r}}
\newcommand{\ddson}[1]{\ddelon{#1}{s}}
\newcommand{\ddton}[1]{\ddelon{#1}{t}}
\newcommand{\ddxon}[1]{\ddelon{#1}{x}}
\newcommand{\ddyon}[1]{\ddelon{#1}{y}}
\newcommand{\ddzon}[1]{\ddelon{#1}{z}}
\newcommand{\ddthon}[1]{\ddelon{#1}{\theta}}
\newcommand{\ddalon}[1]{\ddelon{#1}{\alpha}}
\newcommand{\ddth}{\ddel{\theta}}
\newcommand{\ddph}{\ddel{\phi}}
\newcommand{\dddt}{\dddel{t}}
\newcommand{\dddr}{\dddel{t}}
\newcommand{\dddth}{\dddel{\theta}}
\newcommand{\dddph}{\dddel{\phi}}
\newcommand{\rd}[1]{\mathrm{d} #1}
\newcommand{\dt}{\rd{t}}
\newcommand{\dy}{\rd{y}}
\newcommand{\dx}{\rd{x}}
\newcommand{\edd}[1]{\frac{\mathrm{d}}{\mathrm{d} #1}}
\newcommand{\eddd}[1]{\frac{\mathrm{d}^2}{\mathrm{d}^2 #1}}
\newcommand{\eddon}[2]{\frac{\mathrm{d} #1}{\mathrm{d} #2}}
\newcommand{\edddon}[2]{\frac{\mathrm{d}^2 #1}{\mathrm{d}^2 #2}}
\newcommand{\edt}{\edd{t}}
\newcommand{\edton}[1]{\eddon{#1}{t}}
\newcommand{\edalon}[1]{\eddon{#1}{\alpha}}
\newcommand{\edT}{\edd{T}}
\newcommand{\edr}{\edd{r}}
\newcommand{\edl}{\edd{l}}
\newcommand{\edx}{\edd{x}}
\newcommand{\edth}{\edd{\theta}}
\newcommand{\eddton}[1]{\edddon{#1}{t}}
\newcommand{\eddzon}[1]{\edddon{#1}{z}}
\newcommand{\vect}[1]{\boldsymbol{#1}}
\newcommand{\alp}{\frac{1}{\sqrt{2}}}
\newcommand{\alpi}{\frac{i}{\sqrt{2}}}
\newcommand{\expc}[1]{\langle#1\rangle}
\newcommand{\bkn}[1]{\bra{#1}\ket{#1}}
\newcommand{\bk}[2]{\bra{#1}\ket{#2}}
\newcommand{\bik}[3]{\bra{#1} #2 \ket{#3}}
\newcommand{\vari}[1]{\sigma_{#1}}
\newcommand{\intc}[2]{\left[#1, #2\right]}
\newcommand{\sch}{Schr\"{o}dinger}
\newcommand{\moment}{\boldsymbol{p}}
\newcommand{\coor}{\boldsymbol{x}}
\newcommand{\lapc}{\nabla^2}
% \newcommand{\rec}[1]{\frac{1}{#1}}
\newcommand{\vva}{\boldsymbol{a}}
\newcommand{\vvb}{\boldsymbol{b}}
\newcommand{\vc}{\boldsymbol{c}}
\newcommand{\vd}{\boldsymbol{d}}
\newcommand{\ve}{\boldsymbol{e}}
\newcommand{\vf}{\boldsymbol{f}}
\newcommand{\vg}{\boldsymbol{g}}
\newcommand{\vh}{\boldsymbol{h}}
\newcommand{\vi}{\boldsymbol{i}}
\newcommand{\vj}{\boldsymbol{j}}
\newcommand{\vk}{\boldsymbol{k}}
\newcommand{\vl}{\boldsymbol{l}}
\newcommand{\vm}{\boldsymbol{m}}
\newcommand{\vn}{\boldsymbol{n}}
\newcommand{\vo}{\boldsymbol{o}}
\newcommand{\vp}{\boldsymbol{p}}
\newcommand{\vq}{\boldsymbol{q}}
\newcommand{\vr}{\boldsymbol{r}}
\newcommand{\vs}{\boldsymbol{s}}
\newcommand{\vt}{\boldsymbol{t}}
\newcommand{\vvu}{\boldsymbol{u}}
\newcommand{\vv}{\boldsymbol{v}}
\newcommand{\vw}{\boldsymbol{w}}
\newcommand{\vx}{\boldsymbol{x}}
\newcommand{\vy}{\boldsymbol{y}}
\newcommand{\vz}{\boldsymbol{z}}
\newcommand{\vA}{\boldsymbol{A}}
\newcommand{\vB}{\boldsymbol{B}}
\newcommand{\vC}{\boldsymbol{C}}
\newcommand{\vD}{\boldsymbol{D}}
\newcommand{\vE}{\boldsymbol{E}}
\newcommand{\vF}{\boldsymbol{F}}
\newcommand{\vG}{\boldsymbol{G}}
\newcommand{\vH}{\boldsymbol{H}}
\newcommand{\vI}{\boldsymbol{I}}
\newcommand{\vJ}{\boldsymbol{J}}
\newcommand{\vK}{\boldsymbol{K}}
\newcommand{\vL}{\boldsymbol{L}}
\newcommand{\vM}{\boldsymbol{M}}
\newcommand{\vN}{\boldsymbol{N}}
\newcommand{\vO}{\boldsymbol{O}}
\newcommand{\vP}{\boldsymbol{P}}
\newcommand{\vQ}{\boldsymbol{Q}}
\newcommand{\vR}{\boldsymbol{R}}
\newcommand{\vS}{\boldsymbol{S}}
\newcommand{\vT}{\boldsymbol{T}}
\newcommand{\vU}{\boldsymbol{U}}
\newcommand{\vV}{\boldsymbol{V}}
\newcommand{\vW}{\boldsymbol{W}}
\newcommand{\vX}{\boldsymbol{X}}
\newcommand{\vY}{\boldsymbol{Y}}
\newcommand{\vZ}{\boldsymbol{Z}}
\newcommand{\vzero}{\boldsymbol{0}}
\newcommand{\vomega}{\boldsymbol{\omega}}
%\newcommand{\half}{\frac{1}{2}}
\newcommand{\thalf}{\frac{3}{2}}
\newcommand{\rot}{\nabla\times}
\newcommand{\divg}{\nabla\cdot}
\newcommand{\cE}{\mathcal{E}}
\newcommand{\conclu}[1]{\vspace{1em}\par\noindent\fbox{\parbox{0.97\linewidth}{#1}}\vspace{1em}}
\newcommand{\subentrynote}{$\bullet$}
\newcommand{\keypoint}[1]{\par\begin{siderules}\subentrynote\quad #1 \end{siderules}\par}
\newcommand{\fconclu}{\boxed}
\newcommand{\pair}[2]{#1 \, #2}
\newcommand{\barbelow}[1]{\underline{#1}}
\newcommand{\intn}[2]{\int #1 \,\mathrm{d} #2}
\newcommand{\intu}[3]{\int_0^{#1} #2 \,\mathrm{d} #3}
\newcommand{\intiu}[3]{\int_{-\infty}^{#1} #2 \, \rd{} #3}
\newcommand{\intui}[2]{\int_0^{\infty} #1 \,\mathrm{d} #2}
\newcommand{\intii}[2]{\int_{-\infty}^{\infty} #1 \,\mathrm{d} #2}
\newcommand{\intt}[2]{\int_0^\infty #1 \, \rd{} #2}
\newcommand{\intr}[2]{\int_{-\infty}^{\infty} #1 \, \rd{} #2}
\newcommand{\intbi}[3]{\int_{#1}^{\infty} #2 \, \rd{} #3}
\newcommand{\intab}[4]{\int_{#1}^{#2} #3 \, \rd{} #4}
\newcommand{\supint}[2]{\overbar{\int_{#1}^{#2}}}
\newcommand{\infint}[2]{\barbelow{\int_{#1}^{#2}}}
\newcommand{\intpostf}[1]{\, \rd{} #1}
\newcommand{\bfactor}[1]{e^{-#1/k_BT}}
\newcommand{\pbfactor}[1]{e^{#1/k_BT}}
\newcommand{\dn}[2]{#1^{\pare{#2}}}
\newcommand{\prodg}[1]{\pare{#1}^\times}
\newcommand{\muc}{S_\Omega}
\newcommand{\otw}{\, || \,}

\newcommand{\notion}{\emph}
\newcommand{\iP}{\mathcal{P}}
\newcommand{\eiP}{e^{-\iP}}
\newcommand{\iF}{\mathcal{F}}
\newcommand{\eiF}{e^{-\iF}}
\newcommand{\iG}{\mathcal{G}}

\newcommand{\rc}{r\cos\theta}
\newcommand{\rs}{r\sin\theta}
\newcommand{\sn}{\mathrm{sn}}
\newcommand{\cn}{\mathrm{cn}}
\newcommand{\rdn}{\mathrm{dn}}

\newcommand{\hankel}{H_p^{\pare{1}}}
\newcommand{\hankell}{H_p^{\pare{2}}}
\newcommand{\hhankel}{H_n^{\pare{1}}}
\newcommand{\hhankell}{H_n^{\pare{2}}}
\newcommand{\ber}{\text{ber}\,}
\newcommand{\bei}{\text{bei}\,}
\newcommand{\kker}{\text{ker}\,}
\newcommand{\kei}{\text{kei}\,}
\newcommand{\Ai}{\text{Ai}}
\newcommand{\Bi}{\text{Bi}}

\newcommand{\re}{\text{Re}\,}

\newcommand{\Fp}{F_\phi}
\newcommand{\Ep}{E_\phi}
\newcommand{\Fx}{F_x}
\newcommand{\FF}{\mathbf{F}}
\newcommand{\Ex}{E_x}

%\newcommand{\erf}{\mathrm{erf}}
\newcommand{\erfi}{\mathrm{erfi}}
\newcommand{\erfc}{\mathrm{erfc}}
\newcommand{\ehxs}[1]{e^{-\frac{#1^2}{2}}}
\newcommand{\dcol}[2]{\[ \left.#1 \hspace{1em}\right\vert\hspace{1em} #2 \]}
\newcommand{\titlegamma}{\texorpdfstring{$\Gamma$}{Gamma}}
\newcommand{\titleB}{\texorpdfstring{$B$}{B}}
\newcommand{\switch}[2]{\brac{#1 | #2}}
\newcommand{\SYSexeref}[1]{(舒幼生#1)}
\newcommand{\warning}[1]{\par\textit{注意:#1}}

% Computer Science Head
\usepackage{listings}
\usepackage{color}

\definecolor{dkgreen}{rgb}{0,0.6,0}
\definecolor{gray}{rgb}{0.5,0.5,0.5}
\definecolor{mauve}{rgb}{0.58,0,0.82}

\lstset{frame=tb,
  language=Java,
  aboveskip=3mm,
  belowskip=3mm,
  showstringspaces=false,
  columns=flexible,
  basicstyle={\small\ttfamily},
  numbers=none,
  numberstyle=\tiny\color{gray},
  keywordstyle=\color{blue},
  commentstyle=\color{dkgreen},
  stringstyle=\color{mauve},
  breaklines=true,
  breakatwhitespace=true,
  tabsize=3
}
\lstset{language=Java}
\newcommand{\snp}[1]{\lstinline!#1!}
\newcommand{\term}[2]{\textbf{#1(#2)}}


\begin{document}
\section{基本概念}
\subsection{Stein-Gerlach实验}
银原子经过非均匀磁场,由于自旋的作用,在$z$方向上受力
\[ F_z =  \ddel{z}\pare{\vect{\mu}\cdot\vect{B}} = \mu_z \ddel{z}B_z. \]
若$\mu_z$服从经典物体之分布,则银原子应当形成连续带状。然而实验结果为分立二点,相应角动量为$\pm \hbar /2$。
\par
可以通过这一装置筛选$S_z+$自旋之原子,筛选结果之$S_x+$与$S_x-$均分。若再筛选出其中$S_x+$者,则$S_z+$与$S_z-$均分而非仅有$S_z+$。
\par
这一情形与极化光类似,只需将$S_z\pm$看作$x$方向与$y$方向极化,而$S_x\pm$看作旋转半直角后的$x'$与$y'$方向极化。通过与极化光的类比,可猜测
\[ \ket{S_x;\pm} = \pm\alp\ket{S_z;+} + \alp\ket{S_z;-}, \]
\par
再类比圆偏振光,即
\[ \vect{E} = E_0\pare{\alp \hat{\vect{x}} \cos + \alp \hat{\vect{y}} \sin}. \]
并写作复指数形式,可得
\[ \ket{S_y;\pm} = \alp\ket{S_z;+} \pm  \alpi\ket{S_z;-}, \]
\subsection{动力学}
注意到对任何可观测量$Q$,
\[ \edt\expc{Q} = \edt\bk{\Psi}{Q\Psi} = \bk{\edt\Psi}{Q\Psi} + \bk{\Psi}{\ddt{Q}\Psi} + \bk{\Psi}{Q\edt\Psi}. \]
借助\sch 方程替换$\edt\Psi$,立刻有
\begin{equation}
\label{eq:dQ}
\edt\expc{Q} = \frac{i}{\hbar}\expc{\intc{H}{Q}} + \expc{\ddt{Q}}.
\end{equation}
分别令$Q=\coor$与$Q=\moment$可得Ehrenfest定理。
\subsection{不确定性原理}
记$f=\pare{\hat{A} - \expc{A}}\Psi$,$g=\pare{\hat{B} - \expc{B}}\Psi$,则
\[ \vari{A}^2\vari{B}^2 = \bkn{f}^2\bkn{g}^2 \ge \abs{\bk{f}{g}}^2 \ge \pare{\Im\bk{f}{g}}^2 = \pare{\frac{1}{2i}\pare{\bk{f}{g}-\bk{g}{f}}}^2. \]
再注意
\[ \bk{f}{g} = \expc{\hat{A}\hat{B}} - \expc{A}\expc{B},\quad \bk{g}{f} = \expc{\hat{B}\hat{A}} - \expc{A}\expc{B}. \]
立即可得
\[ \vari{A}^2\vari{B}^2 \ge \pare{\frac{1}{2i}\expc{\intc{A}{B}}}^2. \]
由\eqref{eq:dQ}可得
\[ \vari{E}\vari{Q} \ge \frac{\hbar}{2}\abs{\edt\expc{Q}}. \]
即能量-时间不确定性原理(下式中$\Delta t$为$\expc{Q}$变化一标准差之时间)
\[ \Delta E \Delta t \ge \frac{\hbar}{2}. \]
\section{三维空间的量子力学}
\subsection{分离变量}
以$\Psi=XYZ$分离变量可得三维无限深方势井等价于三个一维之和
\begin{equation}
\label{eq:3dw}
E = \frac{\pi^2\hbar^2\pare{n_x^2+n_y^2+n_z^2}}{2ma^2}.
\end{equation}
\par
对于中心对称势的情形,设$\Psi=RY$并注意
\[ \lapc = \rec{r^2}\ddr\pare{r^2\ddr} + \rec{r^2}\rec{\sin\theta}\ddth\pare{\sin\theta\ddth} + \rec{r^2}\rec{\sin^2\theta}\pare{\dddph}. \]
立刻有
\[ \rec{R}\edr\pare{r^2\edr{R}} - \frac{2mr^2}{\hbar^2}\pare{V\pare{r}-E} = l\pare{l+1}, \]
\[ \rec{Y}\pare{\rec{\sin\theta}\ddth\pare{\sin\theta\ddth Y} + \rec{\sin^2\theta}\dddph Y} = -l\pare{l+1}. \]

\newpage
\section{金属基本概念}
\subsection{Drude模型}
\subsubsection{基本假设}
Drude假定金属价电子可自由活动而原子实不动,通常每个原子贡献1或2个电子,也有高达5个者,相应的自由电子密度在\SI{0.91e22}{\cm^{-3}}到\-\SI{24.7e22}{\cm^{-3}}之间,电子半间距为2至3个Bohr半径。尽管与理想气体的间距相去甚远且电荷间作用较大,Drude仍采用理想气体模型处理电子。
\par
\begin{enumerate}
\item 电子间无相互作用(独立电子假设)以及电子与离子间无相互作用(自由电子假设)。前者在多数情形下成立,后者则需要修正。
\item 模型还电子的碰撞归为与离子间的碰撞而非电子间的碰撞。
\item 模型还假设了电子的平均碰撞频率为$\rec{\tau}$。
\item 且电子的碰撞维持着导体的热平衡。
\end{enumerate}
\subsubsection{直流电}
设电阻率$\rho$,自由电子密度$n$,则
\[ \vE = \rho \vj, \quad \vj = -ne\vv. \]
存在电场时,$\vv$与电阻率$\sigma$可计算为
\begin{equation}
\label{eq:rhotau}
\vv = -\frac{e\vE \tau}{m}, \quad \vj = \pare{\frac{ne^2\tau}{m}}\vE, \quad \sigma = \rec{\rho} =  \pare{\frac{ne^2\tau}{m}}.
\end{equation}
\par
若采用经典的$\half mv^2 = \thalf k_B T$,并借助上式估计平均自由时间,所得平均自由程与原子间距相近。然而,这样估计的速度与实际相去甚远,而与$\tau$相关的物理量如今也大都弃用。
\par
对于电场随时间变化的情形,注意$\dt$内电子碰撞的概率仅为$\dt/\tau$,忽略得
\[ \vp\pare{t+\dt} = \vp\pare{t} - \frac{\dt}{\tau}\vf + \vf\dt + O\pare{\dt}^2. \]
\begin{equation}
\label{eq:dtvp}
\dot{\vp} = -\frac{\vp}{\tau} + \vf.
\end{equation}
未碰撞电子之贡献为二阶项,故得径直忽略之。由上式可见碰撞之结果与摩擦类似。
\par
在Hall效应中,施加磁场$H$,则磁阻
\[ \rho\pare{H} = \frac{E_x}{j_x} \]
与$H$无关。
\par
借助\eqref{eq:dtvp}或直接由Lorentz力带入平衡方程,可得Hall系数
\[ R_H = \frac{E_y}{j_x H} = -\rec{ne}. \]
此结果对碱金属符合较好,对其他金属有相当偏差。日后需要考虑$\omega_c \tau$的影响,即磁场导致电子在自由程内的圆周运动。
\subsubsection{交变电场}
假定交变电场$\vE=\vE_0e^{-i\omega t}$并带入\eqref{eq:dtvp},则
\[ -i\omega\vp=-\frac{\vp}{\tau}-e\vE. \]
立刻有
\[ \vj_0 = -\frac{ne\vp}{m} = \frac{\pare{ne^2/n}\vE}{1/\tau-i\omega}. \]
可将电导率表示为
\[ \sigma = \frac{\sigma_0}{1-i\omega\tau}, \quad \sigma_0 = \frac{ne^2\tau}{m}. \]
上式忽略了磁场的Lorentz力效应,因其数量级可忽略不计。然而假设电场在空间中均匀分布确有不妥,但只要电场的波长$\lambda \gg
 \ell$($\ell$为平均自由程),$\vj = \sigma \vE$仍适用。在此基础上,由Maxwell方程,
 \[ \rot\rot\vE = -\lapc\vE = \frac{i\omega}{c}\rot\vB = \frac{i\omega}{c}\pare{\frac{4\pi\sigma}{c}\vE-\frac{i\omega}{c}\vE}. \]
 亦即
 \[ -\lapc\vE = \frac{\omega^2}{c^2}\epsilon\vE, \quad \epsilon = 1+\frac{4\pi i \sigma}{\omega}. \]
 对高频电场$\omega\tau\gg 1$(紫外光仍然满足波长和频率同时足够高),
 \begin{equation}
 \label{eq:omega}
 \sigma\approx-\frac{\sigma_0}{i\omega\tau},\quad \epsilon = 1-\frac{\omega_p^2}{\omega^2}, \quad \omega_p^2 = \frac{4\pi ne^2}{m}.
 \end{equation}
 若$\omega_p<\omega$则电场指数衰减,电磁波无法通过,反之亦然。
 \par
 借助\eqref{eq:rhotau}表达$\tau$,借助$r_s$表达$n$,可得对于一般金属,
 \[ \omega_p \tau = \SI{1.6e2}{cm^2 m\ohm}\pare{\frac{r_s}{a_0}}^{\frac{3}{2}}\pare{\frac{1}{\rho}}\gg 1. \]
从\eqref{eq:omega}出发还可以类似地推导紫外光穿过的频率阈值,与实际大体相符。
\par
此外,对于$\omega_p=\omega$的情形,可预期内电场与外电场同步等振幅振荡,故电荷密度亦应当有此种震荡。实际上,若此种震荡成立,由
\[ \div \vj = -\dot{\rho} = i\omega\rho,\quad \div \vE = \div \vj/\sigma = 4\pi\rho, \]
可得与$\omega_p=\omega$相合的
\[ 4\pi i \sigma = -\omega. \]
\par
电荷的震荡亦可以通过使电子与背景偏离距离$d$获得,相应$\sigma=nde$,$E=4\pi\sigma$,其产生的电场导致对于内部电荷有如下震荡
\[ m\ddot{d} = -e\cdot 4\pi\sigma = 4\pi ne^2 d. \]
\subsubsection{热导率}
热流满足方程$\vj^q=-\kappa\nabla T$,其中$\kappa$为热导率。Wiedemann–Franz定律表明,$\kappa/\sigma T$近似为Lorenz常数\SI{2.44e-8}{W \ohm/K^2}。假定电子贡献主要热能,则一维下两电子以反方向运动导致
\[ j^q = \half nv\pare{\cE\pare{T\pare{x-v\tau}} - \cE\pare{T\pare{x+v\tau}}} = nv^2\tau\edT \cE \cdot \pare{-\edx T}. \]
注意三维下$\expc{v_x^2} = \frac{1}{3}\expc{v^2}$且$c_V=n\edT \cE$,故
\[ \vj^q = \frac{1}{3}v^2\tau c_V\pare{-\nabla T}, \quad \sigma = \frac{1}{3}v^2\tau c_V = \frac{1}{3}\ell vc_V. \]
对于理想气体,$c_V=\frac{3}{2}nk_B$,$\half mv^2 = \frac{3}{2}k_BT$,故
\[ \frac{\kappa}{\sigma T} = \frac{\frac{1}{3}c_V mv^2}{ne^2 T} = \frac{3}{2}\pare{\frac{k_B}{e}}^2 = \SI{1.11e-8}{W\ohm/K^2}. \]
\par
上述处理之不完善之处在于假定了电子碰撞后的速度完全随机分布,因而产生一常数量级的误差\cite{concepts}。
\subsubsection{温差电与模型缺陷}
考虑高温处的电子运动速度更快,热传导将伴随电子密度的重新分布,导致电势差,其大小近似为
\[ \vE = Q\nabla T. \]
其中$Q$为Seeback系数。仿照前开论述,有
\[ v_Q = \half\pare{v\pare{x-v\tau}-v\pare{x+v\tau}}=-\tau v\edx v = -\tau\edx\frac{v^2}{2}. \]
考虑三维情形且与电场速度平衡,有
\[ \vv_Q = -\frac{\tau}{6}\edT v^2\cdot\nabla T = -\vv_E = \frac{e\vE \tau}{m}. \]
从而
\[ Q = -\frac{\frac{\tau}{6}\edT mv^2}{e\tau} = -\frac{k_B}{2e} = \SI{-0.43e-4}{V/K}. \]
与实际值相差约2个数量级。
\subsection{Sommerfeld模型}
\subsubsection{基本假设}
假定电子服从Fermi-Dirac分布,即
\[ f\pare{\vv} = \frac{\pare{m/\hbar}^3}{4\pi^3}\rec{\exp\pare{\pare{\half mv^2-k_BT_0}/k_BT}+1}, \]
其中$T_0$由归一化条件确定,通常为上千度。不同于Maxwell-Boltzmann分布的指数下降,$f$先平缓维持常数后阶梯骤降至零。
\begin{figure}[!h]
\centering
\begin{pspicture}(0,-0.5)(6,1.5)
\psaxes[labels=none]{->}(0,0)(6,1.5)
\psplot{0}{6}{1/(2.71828^((x-3)*10)+1)}
\psplot{0.165}{6}{7*1/(2.71828^(x*10))}
\end{pspicture}
\caption{Fermi-Dirac分布与Maxwell-Boltzmann分布对比}
\end{figure}
\par
Sommerfeld模型与Drude模型几乎唯一的不同之处仅在于分布函数。
\subsubsection{近零度基态}
假定电子分布在立方导体中,$L=V^{1/3}$,并且不妨假设导体的三对相对面被粘贴\footnote{亦可以不作此种假设而应用无限深方势井\cite{iqm},然而这样做将得到定态。},因此
\[ \psi\pare{\vr + L\hat{\vx}} = \psi\pare{\vr}. \]
参考\eqref{eq:3dw}可得\footnote{\cite{iqm}在此处与\cite{ssph}有所不同,导致\cite{iqm}要限制诸$k>0$,并最终引入$1/8$因子与此处的$2^3$相符。}
\[ \psi \propto e^{i\vk\cdot\vr}, \quad \vk = \frac{2\pi\vn}{L}, \quad E = \frac{\hbar^2\vk^2}{2m}, \quad \vp=\hbar\vk.  \]
\par
设Fermi面半径$k_F$,则(考虑电子自旋导致二重简并)可包含的电子数为($n$为自由电子密度,$n=3/\pare{4\pi r_s^3}$)
\begin{equation}
\label{eq:fmn}
N = 2\pare{\frac{4\pi k_F^3}{3}}/\pare{2\pi/L}^3 = \frac{k_F^3}{3\pi^2}V = nV, \quad \fconclu{n = \frac{k_F^3}{3\pi^2}.}
\end{equation}
立即可得
\[ k_F = \frac{\SI{3.63}{\per\angstrom}}{r_s/a_0}, \quad v_F = \frac{\hbar k_F}{m} = \frac{\SI{4.2e8}{cm/s}}{r_s/a_0}, \quad E_F = \frac{\hbar^2 k_F^2}{2m} = \frac{\SI{50.1}{eV}}{\pare{r_s/a_0}^2}. \]
总能量
\[ E = 2 \intu{k=k_F}{\frac{\hbar^2}{2m}k^2\cdot 4\pi k^2}{^3n} = \frac{4\pi\hbar^2}{5m}k_F^5\cdot \pare{\frac{L}{2\pi}}^3 = \frac{\hbar^2 k_F^2}{10\pi^2 m}V. \]
对应平均每个电子的能量为
\[ \cE = \frac{E}{N} = \frac{E}{V}\frac{V}{N} = \frac{3}{10}\frac{\hbar^2 k_F^2}{m} = \frac{3}{5}E_F. \]
满足$k_BT_F=\cE$的温度$T_F$成为Fermi温度,大小为\SI{1e5}{K}量级。
\par
考虑简并压强
\[ P=-\ddV E \propto -\ddV k_F^2 \propto -\ddV V^{-2/3} = \frac{2}{3}\frac{E}{V}.  \]
体积模量
\[ B = -V\ddV P \propto V^{-5/3} = \frac{5}{3}P = \frac{10}{9}\frac{E}{V} = \frac{2}{3}nE_F = \pare{\frac{6.13}{r_s/a_0}}^5\SI[per-mode=fraction]{e10}{\dyne\per\square\cm}. \]
对碱金属拟合较好,对稀土金属可得到正确数量级。
\subsubsection{Fermi-Dirac分布}
考虑经典的Maxwell-Boltzmann分布,有($F$为Helmholtz自由能)
\[ P_N\pare{E} = \frac{\bfactor{E}}{\sum\bfactor{E_\alpha^N}} = \frac{\bfactor{E}}{\bfactor{F_N}}. \]
其中$E_\alpha^N$表示$N$个粒子的组合态中第$\alpha$能级的能量。存在一个粒子处于能级$i$的概率为
\begin{align*}f_i^N = \sum P_N\pare{E_\alpha^N} &= 1 - \sum P_{N+1}\pare{E_\alpha^{N+1} - \cE_i}\\ 
&= 1-\pbfactor{\pare{\cE_i-\mu}} \sum P_N\pare{E_\alpha^{N+1}}.
\end{align*}
求和对所有$i$能级被占据的组合态进行,$\mu = F_{N+1}-F_N$为化学势。对于充分大的$N$,$N+1$个粒子的组合态与$N$个粒子的相比能量分布相差无几,故$P_N\pare{E_\alpha^{N}}\sim P_N\pare{E_\alpha^{N+1}}$,
\[ f_i^N = 1 - \pbfactor{\pare{\cE_i-\mu}} f_i^N, \quad f_i^N = \rec{\pbfactor{\pare{\cE_i - \mu}} + 1}.\]
$f_i^N$为处在$i$能级的平均粒子数,化学势由归一化条件$\sum_i f_i^N = N$给出。
\subsubsection{热容}
求解恒容比热容,考虑
\[ c_V = \pare{\ddelon{u}{T}}_V, \quad u=\frac{U}{V}, \quad U = 2 \sum E_{\vk}f\pare{E_{\vk}}= 2 \int \frac{L^3}{\pare{2\pi}^3} E_{\vk}f\pare{E_{\vk}}. \]
立即有(并借助相似方法算出$n$)
\[ u = \rec{4\pi^3}\intn{E_{\vk}f\pare{E_{\vk}}}{\vk}, \quad n = \rec{4\pi^3}\intn{f\pare{E_{\vk}}}{\vk}. \]
注意$E=\hbar^2 k^2/2m$,$k^2=2mE/\hbar^2$,$\rd k = m/\hbar^2\pare{2mE/\hbar^2}^{-1/2}\rd E$,
\[ \intn{\rec{4\pi^3}F\pare{E\pare{\vk}}}{\vk} = \intu{\infty}{\frac{m}{\hbar^2\pi^2}\sqrt{\frac{2mE}{\hbar^2}}F\pare{E}}{E} = \intu{\infty}{g\pare{E}F\pare{E}}{E}. \]
借助\eqref{eq:fmn}可得(这一因子也被称作能级密度)
\[ g\pare{E} = \frac{m}{\hbar^2\pi^2}\sqrt{\frac{2mE}{\hbar^2}} = \frac{3}{2}\frac{n}{E_F}\sqrt{\frac{E}{E_F}},\quad \fconclu{ g\pare{E_F} = \frac{mk_F}{\hbar^2\pi^2} = \frac{3}{2}\frac{n}{E_F}. }  \]
分离出$g$的好处在于无关电子假设仅仅在$g$的具体形式上起了作用——分布函数$f$对于有相互作用的粒子也是成立的。
\par
通过分部积分,设$G'=g$,记$x=\pare{E-\mu}/k_BT$,$\rd x = \rd E/k_BT$,
\[ \intu{\infty}{g\pare{E}f\pare{E}}{E} = -\intui{G\pare{E}f'\pare{E}}{E},\quad f'\pare{E} = -\frac{1}{k_BT}\frac{e^x}{\pare{e^x+1}^2}. \]
注意当$\mu\gg k_BT$,$x_{E=0}\sim-\infty$,积分可写为
\[ I = \intii{G\pare{x}\frac{e^x}{\pare{e^x+1}^2}}{x} = \sum_n \rec{n!}\dn{G}{n}\pare{0}\intii{\frac{x^ne^x}{\pare{e^x+1}^2}}{x}. \]
对于奇数的$n$,上述积分为零。对于偶数的$n$,
\begin{align*}
\intii{\frac{x^ne^x}{\pare{e^x+1}^2}}{x} &= \intii{\frac{x^s}{e^x}\pare{1-e^{-x}+e^{-2x}-\cdots}}{x} \\
&= 2n!\sum_k\frac{\pare{-1}^{k+1}}{k^n} = 2n!\pare{1-2^{1-n}}\zeta\pare{n}.
\end{align*}
\begin{align*}
I = \sum_{2|n} 2\dn{G}{n}\pare{0}\pare{1-2^{1-n}}\zeta\pare{n} = \intu{\mu}{g\pare{E}}{E} + \frac{\pi^2\pare{k_BT}^2}{6}g''\pare{0}+\cdots.
\end{align*}

\newpage
\begin{thebibliography}{9}
\bibitem{concepts} 
Stephen J.Blundell and Katherine M.Blundell.
\textit{热物理概念——热力学与统计物理学}. 
清华大学出版社, 2015.
\bibitem{iqm} 
David J.Griffiths.
\textit{量子力学概论}. 
机械工业出版社, 2009.
\bibitem{ssph} 
Ashcroft, Mermin.
\textit{Solid State Physics}. 
世界图书出版社, 2007.
\end{thebibliography}
\end{document}