\documentclass{ctexart}
\usepackage{fourier} 
\usepackage{array}
\usepackage{makecell}
\usepackage{currfile}

\renewcommand\theadalign{bc}
\renewcommand\theadfont{\bfseries}
\renewcommand\theadgape{\Gape[4pt]}
\renewcommand\cellgape{\Gape[4pt]}

\usepackage{color}
\usepackage{ulem}
\usepackage{multicol}
\setlength{\intextsep}{5pt}
\newcommand{\cc}{鹼基互補配對}
\newcommand{\att}{著絲點}
\newcommand{\dna}{脫氧核苷酸}
\newcommand{\reff}{\textsubscript{[來自\currfiledir\currfilename]}}
\newcommand{\parah}[1]{\par\noindent\subentrynote\textbf{#1}\reff\\ \textbf{{\color{red}\hrulefill}}}
\newcommand{\para}[1]{\vspace{1cm}\par\noindent\subentrynote\textbf{#1}\reff\\ \textbf{{\color{red}\hrulefill}}}
\newcommand{\err}[1]{\sout{而不是 #1。}}
\newcommand{\wa}[1]{\textit{#1 は}}
\newcommand{\qref}[3]{\quad---\textit{#1}.p.\-#2.\-#3}
\newcommand{\subentrynote}{$\bullet^{\the\inputlineno}$}
\newcommand{\term}[2]{\par\subentrynote\textit{表述:}#1 謂\textbf{#2}。}
\newcommand{\fterm}[2]{\par\subentrynote\textit{表述:}#1 謂\textbf{#2}}
\newcommand{\cons}[2]{\par\subentrynote\textit{結果:}#1 為\textbf{#2}。}
\newcommand{\meth}[1]{\par\subentrynote\textit{方法:}#1。}
\newcommand{\elem}[2]{\par\subentrynote\textit{從屬:}#1 屬於\textbf{#2}。}
\newcommand{\prop}[2]{\par\subentrynote\textit{性質:}#1 具有\textbf{#2}。}
\newcommand{\nprop}[2]{\par\subentrynote\textit{性質:}#1 不具有\textbf{#2}。}
\newcommand{\defi}[2]{\par\subentrynote\textit{定義:}{\color{blue}{\textit{#1}}}謂\textbf{#2}。}
\newcommand{\fdefi}[2]{\par\subentrynote\textit{定義:}{\color{blue}{\textit{#1}}}謂\textbf{#2}}
\newcommand{\centry}[1]{\par \textit{$*$ #1}}
\newcommand{\sus}[4]{{\par \color{red} \textit{存疑:}#1 \qref{#2}{#3}{#4}}}
\newcommand{\noi}[4]{{\par \color{red} \textit{待補充:}#1 \qref{#2}{#3}{#4}}}
\newcommand{\stress}[1]{{\color{red} #1}}
\newcommand{\qreft}[2]{\qref{\bio\tek}{#1}{#2}}
\newcommand{\vref}[2]{\qref{Vol}{#1}{#2}}
\newcommand{\impl}[1]{\underline{\color{yellow}{[#1]}}}
\newcommand{\termcomp}[4]{
\par\subentrynote\textit{對比:}
\begin{figure}[!htb]
    \centering
    \begin{minipage}{.45\textwidth}
        \term{#1}{#2}
    \end{minipage}%
    \begin{minipage}{.1\textwidth}
    \centering
    $\leftrightarrow$
    \end{minipage}%
    \begin{minipage}{0.45\textwidth}
        \term{#3}{#4}
    \end{minipage}
\end{figure}}
\newcommand{\comp}[2]{
\par\subentrynote\textit{對比:}
\begin{figure}[!htb]
    \centering
    \begin{minipage}{.45\textwidth}
        #1
    \end{minipage}%
    \begin{minipage}{.1\textwidth}
    \centering
    $\leftrightarrow$
    \end{minipage}%
    \begin{minipage}{0.45\textwidth}
        #2
    \end{minipage}
\end{figure}}
\newcommand{\bio}{せ}
\newcommand{\tek}{テ}
\newcommand{\MI}{減數第一次分裂}
\newcommand{\MII}{減數第二次分裂}
\newcommand{\PII}{第二極體}
\newcommand{\PI}{第二極體}
\newcommand{\cf}{cf.}
\newcommand{\TIV}{T\textsubscript{4}}
\newcommand{\Ecoli}{\textit{E}·\textit{coli}}
\newcommand{\rulesep}{\unskip\ \vrule\ }

\begin{document}
\section{生物}
\subsection*{生物·通論}
\par 抽象司法行為禁止——篇目所載的要點須具有具體的存疑或失誤的題目,並與之俱有相當關聯性。
\par 一事不再理——既已審查的題目除非存在要點歸納錯誤,不得再度審查。
\par 表述性的要點,可以為操作之賓語、原因、工具或其他狀語。
\par 定義性的要點,可以為定義本身或足夠精確之描述。
\par 同一要點,視其所欲強調者不同,得給予不同之分類。
\par 要點之順序,視其於所處之章節之位置確定。描述動態過程時,應以要點對應過程之步驟確定其位置;描述靜態時,應以一定位置順序確定之。

\subsection{生物·通論}
\meth{\wa{選擇題之項目排除},选择题对知识性错误项目之排除优先于表述性错误}\vref{1070505}{7}

\subsection{生物·卷0·常識或未歸類}
\term{導管的構成}{植物的死細胞}
\term{有絲分裂之著絲點分離發生於}{有絲分裂後期}\qreft{28}{4}
\elem{玉米、小麥}{單子葉植物}\qreft{6}{lower.left}
\elem{核移植技術}{細胞工程}\qreft{42}{1}
\elem{溶酶體酶}{蛋白質}故其合成處為核糖體。\qreft{42}{2}
\meth{若涉及動物細胞於培養皿內培養,可判定其涉及動物細胞培養技術}\qreft{46}{7.3}
\meth{題目中出現的名詞在必要時應徑直使用}\qreft{92}{2}
\noi{選填顯微注射技術或轉基因技術}\qreft{92}{14.4}\qreft{94}{14.4}
\noi{選填基因工程技術或轉基因技術}\qreft{96}{12.2}
\noi{選填基因工程技術/細胞工程技術或核移植技術}\qreft{51}{7.2}\qreft{60}{6.1}
\sus{激素僅作用於特定細胞}\qreft{51}{4}
\term{細胞核移植之細胞核所移入者}{\stress{去核}卵母細胞}\qreft{28}{4}
\term{以去核卵母細胞為受體細胞之原因}{含有使細胞核表達全能性的物質與營養條件,體積大且易操作}\qreft{96}{11.3}
\term{轉錄之首先步驟}{\stress{RNA聚合酶}識別基因的\stress{啟動子}結構}\qreft{11}{7.1}
\term{啟動子之作用}{為RNA聚合酶提供識別與結合的部位,驅動轉錄}\qreft{72}{13.4}
\term{啟動子與終止子之作用}{啟動和終止\stress{轉錄}的進程}\err{翻譯}\qreft{13}{2}
\noi{格里菲斯實驗之步驟}{\bio\tek}{11}{6.5}
\term{格里菲斯實驗之啟示}{DNA可以從一種生物個體傳遞至另一種生物個體}\qreft{11}{6.5}

\subsection{生物·卷I·細胞與遺傳學}
\subsubsection{遺傳學}
\parah{實驗}
\comp{\fterm{孟德爾「一堆相對性狀的雜交實驗」包含步驟}{
\begin{enumerate}
\item 顯性純合與隱形純合雜交,得到不分離的$F_1$;
\item $F_1$自交得到分離的$F_2$。
\end{enumerate}
}}{\fterm{摩爾根證實基因位於染色體上的實驗包含步驟}{
\begin{enumerate}
\item 顯性雌性與隱性雌性雜交,得到顯性$F_1$;
\item $F_1$\stress{雜交}得到分離的$F_2$。
\end{enumerate}
}}\vref{1070505}{25}
\fterm{孟德爾之假說}{
\begin{enumerate}
\item 生物的性狀由遺傳因子決定;
\item 體細胞的遺傳因子成對存在;
\item 生物體形成配子時,成對的遺傳因子分離進入不同配子;
\item 雌雄配子的結合是隨機的。
\end{enumerate}}
上述假說不涉及染色體。\vref{1070505}{38}
\term{檢驗植物是否顯性純合,最簡便易行的方式}{自交視後代是否性狀分離}\vref{1070505}{21,24}

\para{基因的表達}
\meth{\wa{尋找提高產物之方法},應審查促進產物之產生的基因或物質與抑制產物產生之基因或物質,並添加或去處之}\qreft{74}{13.4}
\nprop{原核生物}{內含子}故其基因結構與人體的基因結構不同。\qreft{73}{5}
\term{欲改變蛋白質中某一氨基酸,應替換之鹼基數量最少}{1個}\err{3個}\qreft{75}{11.2}
\meth{\wa{基因突變},題目中給出終止密碼子時,可考慮突變為終止密碼子之情形}\qreft{76}{15.2}
\term{從基因表達水平分析,遺傳病之患病原因之一可能}{突變後終止密碼子提前出現,翻譯提前終止,形成一場蛋白}\qreft{76}{15.2}
\meth{\wa{表現型數目之判定},應視實際表現出相對性狀之數目決定,\err{單純由基因之組合數量}}\vref{1070505}{35}

\para{染色體與有絲分裂}
\sus{神經細胞之染色體數目為2N?}{vol}{1070505}{29}
\fdefi{同源染色體}{
\begin{enumerate}
\item 成對存在;
\item 形態、大小相同;
\item 一條來自父方,一條來自母方;
\end{enumerate}
的一對染色體。
}上述第二條存在例外,如性染色體。\vref{1070505}{9}
\noi{染色單體之定義}{Vol}{20170505}{9}僅在\att 分離前存在。
\term{同源染色體}{在減數分裂是聯會之性質}不存在例外。\vref{1070505}{11}
\comp{\term{植物細胞有絲分裂末期之特徵}{細胞壁合成,高爾基體活動旺盛}}{\term{動物細胞有絲分裂末期之特徵僅}{細胞膜變形}}\vref{1070505}{5}

\subsection{生物·卷II·基因工程}
\subsubsection{基因工程}
\parah{基因工程通論與限制酶,DNA連接酶}
\term{基因工程之最終目的}{定向改造生物的遺傳性狀}\qreft{4}{lower.1}
\term{基因工程所需的必要酶}{限制酶,\stress{DNA連接酶}}\qreft{71}{1}
\term{限制酶之作用特點}{識別特定的\stress{核苷酸}序列,並在特定位點上切割DNA分子}\qreft{11}{7.7}
\meth{\wa{切割酶的選用},對目的基因應選用可將包含目的基因的一段DNA片段切下的酶,對質粒應選用切下二標記基因之一的酶}\qref{\bio\tek}{69}{8}
\term{在目的基因兩側選用不同的酶切割之原因}{防止目的基因與載體反向連接}\qreft{76}{14.1}
\term{限制酶的切割對象}{有特定\stress{脫氧}核苷酸序列的DNA片段}\qref{\bio\tek}{70}{11.1}
\term{因目的基因事後不存在切割位點致限制酶無法切割}{無法識別切割位點}\qref{\bio\tek}{70}{11.5}
\sus{基因能夠與質粒連接的主要原因是切割後形成相同黏性末端}{\bio\tek}{74}{13.2}
\term{限制酶的相同點之一}{切割後形成的黏性末端相同或互補}選填何種相同點應視上下文決定之。\qref{\bio\tek}{70}{12.3}
\term{DNA連接酶有兩類}{\TIV\ DNA連接酶與\Ecoli\ DNA連接酶}前者連接黏性末端或平末端,後者僅連接黏性末端。\qref{\bio\tek}{70}{12.5}
\centry{\Ecoli 謂Escherichia coli,即大腸桿菌。}






\para{質粒與其他載體}
\meth{\wa{基因工程之判定},涉及以質粒或其他載體導入目的基因者,皆可判定為使用了基因工程技術}\qreft{72}{13.2}
\term{基因工程操作之核心步驟}{基因表達載體之構建}\qreft{6}{lower.right.2}
\term{需表達載體之原因}{目的基因無複製原點與表達所需的啟動子}\qreft{16}{6.3}
\fterm{質粒載體應具備之條件}{\begin{enumerate}
\item 具有一個或多個限制酶切割位點,供外源DNA片段插入
\item 具有標記基因
\item 能自我複製
\end{enumerate}}\qreft{5}{7.1}
\centry{煙草花葉病毒為RNA病毒,而大多數噬菌體為DNA	 病毒。}
\term{基因工程之載體}{質粒,\stress{$\lambda$噬菌體衍生物與動植物病毒}}\qreft{70}{12}
\comp{\meth{受體為動物,選擇動植物病毒}}{\meth{受體為細菌,選擇噬菌體}}\wa{載體之選用}:\qreft{11}{6.2}
\term{標記基因之作用}{供重組DNA的鑑定與選擇}\qreft{3}{lower.right}
\cons{將質粒與目的基因用同一酶切,連接的結果}{質粒自聯,廢棄基因自聯,目的基因與質粒互聯}\qref{\bio\tek}{70}{10}
\term{重組質粒作為表達載體應當具有}{啟動子、終止子、標記基因}\qreft{74}{14.3}
\sus{表達載體無需複製原點}{\bio\tek}{74}{14.3}
\term{噬菌體改造後作為載體,其DNA複製原料來自}{受體細胞}\err{宿主細胞}應當據上下文回答直接對象。\qreft{5}{7.3}
\meth{篩選轉化後細胞所用之抗生素應嚴格視條件之抗性基因決定}\qreft{24}{3}
\term{微生物受體不能在抗生素培養基上生存之主要原因}{重組質粒未導入}\qreft{74}{13.3}
\term{某一生物之基因能插入其他生物的染色體之原因}{其遺傳物質均為DNA,物質組成與空間結構相同}\qreft{70}{14}

\para{基因的獲取}
\term{獲得目的基因之主要途徑}{從已有物種中分離與人工合成}\qreft{74}{14.1}
\term{從植物中獲得mRNA,選用嫩葉而非老業之原因}{嫩葉組織細胞易破碎}\qreft{16}{6.1}
\defi{cDNA}{mRNA逆轉錄之產物}\qreft{94}{14.1}\qreft{72}{1}
\defi{cDNA文庫}{某生物某一發育時期所轉錄的mRNA\stress{全部}經逆轉錄形成的cDNA片段與載體連接而形成的集合}\qreft{73}{11.1}
\term{由mRNA獲得cDNA之原理}{在逆轉錄酶的作用下,以mRNA為模板按照\cc 原則合成cDNA}\qreft{16}{6.2}
\meth{\wa{選用限制酶},應當選擇切割位點盡可能接近目的基因者}\qreft{71}{5}
\sus{人工合成法所合成之序列不唯一}{\bio\tek}{7}{2}
\comp{\meth{從供體DNA開始,可推斷為基因組文庫法}}{\meth{從mRNA開始,可推斷為部分基因文庫法}}\wa{獲得目的基因之方法}:\qreft{10}{1}
\meth{\wa{脫氧核苷酸原料數量的計算},$2^n-\stress{1}$倍的單個DNA分子的原料數量}\qreft{71}{6}
\para{基因的導入}
\meth{\wa{受體生物之選擇},對要求作物之細胞產物的情況,應考慮產物烹飪後失活之可能}\qreft{74}{11.3}
\term{將目的基因導入微生物之方法}{感受態細胞法}\qreft{71}{9}
\term{將目的基因導入微生物前之操作}{用鈣離子處理,使其轉化為感受態細胞}\qreft{76}{14.2}
\noi{交叉引用}{}{}{}
\term{使用細菌作為受體之優點}{繁殖速度快,結構簡單}\qreft{11}{3}
\defi{基因槍法}{用压缩气体动力把粘有DNA 的细微金粉打向细胞,穿过细胞壁、细胞膜、细胞质等构造到达细胞核,完成基因转移}
\term{將目的基因導入單子葉植物之常用方法}{基因槍法}\qreft{6}{lower.left}\qreft{9}{2}
\term{農桿菌轉化法之目的基因插入對象}{Ti質粒之\stress{T-DNA}}\qreft{8}{lower.left}\qreft{74}{14.4}
\centry{Ti質粒之Ti謂Tumour inducing。}
\centry{T-DNA之T謂Transfer。}
\term{目的基因之插入對象}{染色體\stress{DNA}}\qreft{72}{13.3}
\term{整合到染色體上之對象}{目的基因}\err{質粒}\qreft{73}{8}
\term{目的基因插入染色體DNA後結果}{目的基因得以穩定維持與表達}\qreft{8}{mid.right}
\para{基因的檢測}
\defi{杂交探针}{一小段单链DNA片段,用于检测与其互补的核酸序列}在分子雜交技術與DNA分子雜交技術中使用。\qreft{71}{9}
\term{目的基因能否在植物體內穩定遺傳的關鍵}{是否整合至染色體}\qreft{72}{5}
\term{檢查目的基因是否整合至染色體之方法}{DNA分子雜交}\qreft{72}{5}
\term{DNA分子雜交技術中不發生雜交帶者}{非目的基因}\qreft{9}{lower.left}

\para{基因工程的應用}
\defi{轉基因植物}{植物體細胞中被轉入\stress{外源基因}的植物}\err{出現了新基因}\qreft{13}{upper.right.2}
\term{植物轉基因技術之優點}{目的性強,克服遠緣雜交不親和性}\qreft{16}{7.3}
\meth{抗原-抗體雜交技術可確定蛋白質產物之位置,即為提取產物之位置}\qreft{76}{13.3}
\term{轉基因動物作器官移植供體,優點}{避免\stress{免疫}排斥反應}\qreft{74}{14.5}
\term{真核生物作為受體細胞之優勢}{含有內質網與高爾基體,可以對蛋白質進行加工與修飾}\qreft{76}{14.2}
\defi{基因治療}{將健康的\stress{外源}基因導入有基因缺陷的細胞中,以治療疾病}\qreft{74}{15.1}
\term{取出患者細胞,導入正常基因後細胞培養生成組織後重新植入之治療方法}{體外基因治療}\qreft{19}{6.1}
\term{基因治療效果發生之原因}{細胞中能合成\impl{目標產物}}\qreft{74}{15.3}
\noi{交叉引用}{}{}{}
\term{基因工程技術用於基因診斷之原理}{DNA分子雜交}

\para{蛋白質工程}
\fterm{蛋白質工程之流程}{
\begin{enumerate}
\item 預期蛋白質功能;
\item 設計蛋白質的空間結構;
\item 推測蛋白質的氨基酸序列;
\item 合成相應的\dna 序列。
\end{enumerate}}注意上述第二與第三項之順序。\qreft{76}{12.1}
\term{蛋白質工程之原理}{中心法則之逆推}\qreft{76}{14.3}
\term{蛋白質工程之實質}{改造基因}\qreft{76}{14.3}
\fterm{蛋白質工程之手段}{基因修飾與基因合成}\qreft{76}{12.2}
\term{蛋白質工程之蛋白質產物性能提升之\stress{根本}原因}{控制基因合成之位點發生突變}\qreft{19}{3}
\term{對天然基因之改造,選擇操作基因而非操作蛋白質分子之原因}{對蛋白質之改造無法遺傳;對基因之改造操作更為容易,難度降低}\qreft{76}{12.4}
\term{改造蛋白質性質的手段之一}{對蛋白質進行少數胺基酸的替換}\qreft{75}{11.3}
\term{蛋白質工程之目的}{對現有的蛋白質進行改造,或製造一種新的蛋白質,以滿足\stress{人類的生產與生活}的需求}\qreft{76}{12.2}
\subsubsection{植物細胞工程}
\parah{植物組織培養技術·取材與條件}
\meth{\wa{植物組織培養技術之判定}涉及由植物體細胞培養至植株者,皆可判定為使用了植物組織培養技術}\qreft{72}{13.2}
\noi{選填「單倍體育種」「花藥離體培養技術」「植物組織培養技術」之區分。}{\bio\tek}{24}{7.4}
\defi{外植體}{植物組織培養中作為離體培養材料的器官或組織的片段}
\defi{雜種優勢}{雜交種通過繼承其父母的不同的優勢,獲得更好的生物特性}
\defi{細胞全能性}{已分化的細胞具有發育成完整生物體的潛能}\qreft{21}{lower.1}
\fterm{植物細胞表現全能性之條件}{
\begin{enumerate}
\item \stress{脫離母體};
\item 給予適當營養與外界條件。
\end{enumerate}}
\vref{1070505}{2}
\par
卵細胞與受精卵皆有全能性,卵細胞全能性最高。\qreft{24}{1}
\term{植物組織培養過程之條件}{無菌}\qreft{80}{12}
\term{植物組織培養中污染之可能原因}{外植體消毒不徹底}外植體消毒困難\qreft{29}{7.4}
\term{植物組織培養技術之固體培養基內為保障植物生長應當加入}{無機鹽與水}\cf V.p.84\qreft{29}{7.2}
\term{植物組織培養技術之固體培養基內為形成癒傷組織應當加入}{植物激素}\qreft{29}{7.3}
\term{脫分化培養基應加入者}{無機營養物質,生長素與細胞分裂素,有機營養物質}\qreft{79}{9}
\noi{植物組織培養之激素影響與根/芽順序}{\bio\tek}{22}{right.1}\qreft{24}{7.3}
\term{取新生的莖尖細胞培養的原因}{莖尖細胞全能性高與新生細胞無箘無毒}\qreft{24}{7}

\para{植物組織培養技術·脫分化、再分化與結果}
\term{從結果看,分化、脫分化、再分化之實質}{基因的選擇性表達}\qreft{79}{9}
\term{脫分化之實質}{使細胞恢復分裂能力}\qref{\bio\tek}{82}{10}
\term{脫分化過程之條件謂}{避光}\qreft{22}{right.1}
\term{癒傷組織結構特點}{細胞壁更薄,高度液泡化,無定形}\qreft{79}{9}
\par
癒傷組織不含葉綠體,故癒傷組織之分化無需充足光照。\qreft{82}{14}
\par
所填寫培養基之類型名稱,應端視上下文所欲強調之屬性而確定。涉及激素者,應選填「分化培養基」而非「固體培養基」。\qreft{24}{7.3}
\cons{癒傷組織再分化之結果視目的}{胚狀體或幼根和芽}\qreft{22}{upper.r.1}
\par
產物可直接通過癒傷組織獲得。\stress{是否應包含與工廠化生產?}\qref{\bio\tek}{81}{6}
\term{植物組織培養中再分化過程需要激素}{生長素、細胞分裂素}\qreft{22}{right.2}
\noi{各種植物激素之作用}{\bio\tek}{22}{right.2}
\sus{雜種優勢後代不發生性狀分離}{\bio\tek}{81}{5}
\term{轉基因植物之培養所依賴之技術}{轉基因技術與植物組織培養技術}\qreft{24}{4}
\noi{轉基因技術與基因工程技術之定義}{\bio\tek}{24}{4}


\para{人工種子}
\defi{胚狀體}{相當於天然種子的胚}由分生組織構成。
\par
胚狀體與頂芽、腋芽、不定芽並列\err{包含}
\par
人工種子由胚狀體製得,\err{幼苗}\qref{\bio\tek}{81}{7}
\term{人工種子包裹胚狀體的膠質之擬制}{種皮,胚乳、子葉等提供營養的結構}\qref{\bio\tek}{81}{10}
\par
人工種子萌發之植株,可育與否取決於種子之產生方式。體細胞生成之種子可育,配子生成則不可育。\qreft{82}{10}


\para{單倍體育種與花藥離體培養技術}
\elem{花藥離體培養技術}{植物組織培養技術}\qreft{24}{4}
\term{單倍體育種之原理,謂}{生殖細胞之全能性}\qreft{26}{lower.s1.1}
\elem{香蕉}{無花粉的植物}故無法進行花藥離體培養。\qreft{28}{3}
\termcomp{花藥離體培養之條件}{培養基,適宜溫度、pH、光照}{種子萌發之條件}{適宜的溫度、水分}
\qreft{22}{right.2}

\defi{單倍體植株}{配子離體培養所得植株}\qreft{80}{10}
\term{秋水仙素質作用時期}{有絲分裂前期抑制紡錘體形成}\qreft{27}{right.1}
\term{秋水仙素於植物引發之變異}{染色體數目變異}\qreft{26}{lower.s1.1}
\noi{染色體變異類型}{}{}{}
\term{單倍體育種之優點}{明顯縮短育種年限}\qreft{24}{7.4}



\para{脫毒苗的製作}
\term{製作脫毒苗,應當利用者}{無病毒或病毒較少的莖尖或根尖}\qreft{25}{lower.s2.2}

\para{微型繁殖}
\term{微型繁殖技術之意義}{保持品種的遺傳特性}\qreft{82}{13}

\para{突變體利用}
\defi{誘變育種}{對植物的癒傷組織誘變處理,使之突變,再通過誘導分化形成植株}
\term{誘變育種之原理}{基因突變與植物體細胞之分化}\qreft{26}{lower.s2.1}

\para{細胞產物的工廠化生產}
\term{細胞產物的工廠化生產採用的培養基謂}{液體培養基}\qreft{28}{6.2}
\term{在液體培養基中通入空氣的原因}{保障氧氣供應,使細胞和培養液充分接觸}\qreft{28}{6.2}
\term{液體培養後期產物減少的原因}{活細胞數量下降,細胞代謝產物積累,營養物質消耗}\qreft{29}{6.3}




\para{植物體細胞雜交技術}
\defi{植物體細胞雜交}{將不同種的植物體細胞,在一定條件下融合成雜種細胞,並把雜種細胞培育成新的植物體的技術}\qreft{24}{6}
\term{植物體細胞雜交技術包括之手段}{植物體細胞融合,植物組織培養}\qreft{80}{11}
\term{植物細胞去壁所用酶}{纖維素酶與果膠酶}\qreft{21}{upper.\-1}
\sus{酶解法去壁,不應在低滲溶液中進行}{\bio\tek}{79}{5}
\term{植物體細胞融合之原生質體融合產物謂}{融合的原生質體}\qreft{80}{12}
\term{植物體細胞雜交完成標誌}{雜種原生質體再生出細胞壁}\qreft{22}{lower.1}
\meth{細胞融合除考慮三種兩兩融合產物,還應考慮未融合的情形}\qreft{80}{11}
\meth{不同種的植物之間\stress{雜交},應注意AA+BB$\rightarrow$AB}\\ \err{AA+BB$\rightarrow$AABB,不同於融合}\qreft{82}{12}
\term{植物體細胞雜交技術之目的}{獲得新植株}\err{雜種細胞}\qreft{21}{upper.2.6}
\term{植物體細胞雜交技術之意義}{克服遠緣雜交不親和的障礙}\qreft{80}{11}
\term{植物體細胞雜交仍存在之問題}{不能按照人的意願表達兩個親本的性狀}\qreft{80}{12}
\subsubsection{胚胎工程}
\parah{精子}
\term{精子的能量來源}{線粒體與細胞質基質}\qreft{42}{2}
\term{精子形成過程中不同於卵子的最主要特徵}{須變形}\qreft{42}{6.2}
\term{精子形成過程中留下的主要結構}{細胞核與線粒體}\qreft{42}{6.2}
\term{精子的線粒體集中在尾部的原因}{精子尾部游動的能量由線粒體提供}\qreft{90}{10.3}


\para{卵子與受精}
\term{哺乳動物卵子的發生之開始時期}{胚胎性別分化後}\qreft{90}{11.1}
\defi{卵泡}{卵原細胞被卵泡細胞包圍形成者}故卵泡不等於卵泡細胞\qreft{90}{11.2}
\sus{卵子的發生地謂卵巢與輸卵管。}{\bio\tek}{90}{11.1}
\comp{\term{馬的排卵時間}{\MI 前}}{\term{牛、羊或豬的排卵時間}{\MI 後}}因此動物排除卵子的成熟程度不同。\qreft{95}{2}
\term{透明帶形成時間}{\MI 之前}\qreft{44}{upper.left}
\elem{卵原細胞分裂為初級卵母細胞}{有絲分裂},故二者染色體數均為$2N$。\qreft{42}{6.3} 
\term{卵子具備與精子受精的能力之時期}{\MII 中期}\qreft{40}{lower.left}
\begin{figure}[!htb]
    \centering
    \hrulefill\\
    \vspace{0.2cm}
    \begin{minipage}{.45\textwidth}
        \term{精子溶解卵丘細胞之間的物質,穿越放射冠}{頂體反應}
    \end{minipage}%
    \hspace{0.2cm}
    \rulesep
    \hspace{0.2cm}
    \begin{minipage}{0.45\textwidth}
       \term{頂體反應之發生時間}{精子與卵子相遇時}\qreft{94}{12.1}

    \end{minipage}
    \\
    \vspace{0.2cm}
    \hrulefill
    \\
    \vspace{0.2cm}
    \begin{minipage}{.45\textwidth}
        \term{防止多精入卵的第一道屏障}{透明帶反應}\qreft{40}{lower.right.2}
    \end{minipage}%
    \hspace{0.2cm}
    \rulesep
    \hspace{0.2cm}
    \begin{minipage}{0.45\textwidth}
       \term{透明帶反應之發生時間}{精子觸及卵細胞膜的瞬間}
    \end{minipage}
    \\
    \vspace{0.2cm}
    \hrulefill
    \\
    \vspace{0.2cm}
    \begin{minipage}{.45\textwidth}
        \term{防止多精入卵的第二道屏障}{卵細胞膜反應}\qreft{40}{lower.right.2}
    \end{minipage}%
    \hspace{0.2cm}
    \rulesep
    \hspace{0.2cm}
    \begin{minipage}{0.45\textwidth}
       \term{卵細胞膜反應之發生時間}{精子入卵後}
    \end{minipage}
    \vspace{0.2cm}
    \\ \hrulefill
\end{figure}
\prop{精卵識別}{物種特異性}\qreft{89}{5}
\term{精子入卵}{精子外膜與卵細胞膜相互融合}\qreft{40}{lower.\-right.2}
\term{雄原核的形成時間}{精子入卵後}\qreft{89}{2}
\cons{卵子發生}{1個卵子與3個極體}\qreft{39}{lower.right}
\sus{極體數量矛盾}{\bio\tek}{40}{lower.right}
多數哺乳動物\PI 不進行\MII 。
\term{受精標誌}{卵細胞\stress{膜}和透明帶之間觀察到兩個極體}\qreft{40}{lower.right}\qreft{90}{14.3}
\term{雌原核的形成時間}{排出\PII 後}\qreft{89}{2}
\term{受精完成標誌}{雌雄原核的融合}\qreft{40}{lower.right}
受精卵中僅細胞核內的遺傳物質一半來自父方一半來自母方。\qreft{42}{2}



\para{早期胚胎發育}
\defi{卵裂期}{2細胞至桑椹胚期}\qreft{49}{1}
\term{細胞分化至開始時期}{囊胚期}\qreft{39}{upper.right.6}\qreft{90}{12.3}
\term{胚層分化之開始時期}{原腸胚期}\qreft{90}{12.3}
\cons{滋養層發育}{胎盤與胎膜}\qreft{41}{2}

\para{卵母細胞或卵子的採集}
\elem{促性腺激素}{肽類化合物}因此不得口服。\qreft{44}{mid.left.1}
\term{採集卵母細胞之位置}{卵巢}\err{輸卵管}\qreft{95}{4}
\term{大型動物採集卵母細胞之方法}{從卵巢中採集}可自活體動物或屠宰場。\qreft{46}{2}
\term{大型動物採集卵母細胞後處理}{培養成熟}故無需篩選處於\MII 中期的細胞。\qreft{93}{8}
\term{體外受精的意義}{提高卵子的利用率,有助於發揮優良母豬的繁殖潛能}\qreft{91}{6}

\para{精子的採集與體外受精}
\prop{精清}{抑制精子獲能的物質}因此體外受精前對精子離心處理。\qreft{44}{mid.left.2}
\term{精子獲能之常用化學方法}{化學誘導法與培養法}\qreft{92}{12.2}
\comp{\term{對嚙齒動物、豬、兔常用獲能方法}{培養法}}{\term{對\stress{家畜}常用獲能方法}{化學誘導法}}\qreft{96}{10.1}
\term{精子獲能之化學誘導法所使用之試劑}{肝素溶液或鈣離子載體的A23187溶液}\qreft{44}{1}
\sus{精子獲能環境通入二氧化碳之原因謂調節獲能液之pH。}{\bio\tek}{93.3}{}
\term{體外受精技術之原理}{人工模擬體內的環境,包括營養、溫度、pH值等,使卵母細胞成熟,同時使精子或能,最終完成受精}\qreft{92}{13.2}
\noi{選填受精或體外受精之區分}\qreft{93}{11}
\term{體外受精技術之意義}{解決動物生育率低的問題}\qreft{92}{12.4}
\defi{試管嬰兒技術}{體外受精的受精卵經過初步篩選與發育後進行胚胎移植}\qreft{92}{15.1}

\para{人工授精}
\comp{\elem{人工授精}{人工將精液置入子宮內}}{\elem{配種}{自然交配}}\qreft{91}{6}
\term{人工授精的意義}{提高精子的利用率,有利於發揮優良公畜的特性}\qreft{91}{6}
\term{人工授精與體外受精的本質同}{精卵結合}\qreft{91}{6}

\para{早期胚胎培養}
\term{早期胚胎培養之培養液成分}{兩鹽、兩素、兩酸與\stress{動物激素}}其中動物激素具有特殊性故優先填寫\qreft{50}{1}\qreft{96}{10}
\term{早起胚胎培養之條件要求}{\stress{無菌無毒}、營養、溫度和pH與氣體環境}\qreft{93}{11.3}
\noi{各種培養基或培養液成分}{\bio\tek}{50}{1}\qreft{45}{lower.right}強調血清/血漿。\qreft{96}{11}
\term{受精卵培養中,為防止感染可添加者}{抗生素}\qreft{92}{14.3}

\para{胚胎移植}
\term{對供體公牛與母牛之要求}{具有優良的遺傳性狀}\qreft{93}{11.2}
\elem{促性腺激素}{垂體分泌之激素}\qreft{94}{12.4}
\term{胚胎移植之生理基礎}{同種動物供、受體排卵後,生殖器官的生理變化相同;\stress{受體對外來胚胎基本不發生免疫排斥反應};供體胚胎可與受體子宮建立正常的生理和組織聯繫,但遺傳特性不受影響}第二點優先填寫。\qreft{94}{15.5}
\term{同期發情處理之目的}{在胚胎移植前後提供相同的生理環境}\qreft{96}{5.4}
\comp{\term{胚胎移植前,對供體母牛之處理}{同期發情與促性腺激素}}{\term{胚胎移植前,對供體母牛之處理}{同期發情}}\qreft{93}{4}
\term{胚胎移植之胚胎來源}{(體內受精)配種、人工授精,體外受精,細胞核移植之胚胎}\qreft{51}{6.4}
\sus{動物種類不同,進行胚胎移植的早期胚胎所處階段不同,課本p.72.bottom,牛、羊一般培養至桑椹胚或囊胚,小鼠、家兔等可更早}{\bio\tek}{45}{2}\qreft{96}{11.5}
\term{配種(體內受精)後胚胎移植前的步驟}{沖卵}\qreft{48}{mid.right.2}
\term{沖卵之對象}{胚胎}\err{卵子}\qreft{51}{2}
\term{沖卵之生理基礎}{早期胚胎形成後,在一定時間內不會與母體子宮建立組織上的聯繫,而是處於游離狀態}\qreft{48}{mid.right.2}
\meth{\wa{篩選的具體化},胚胎移植前的篩選,可視上下文填寫更確定的篩選形式}如性別鑑定\qreft{94}{14.5}
\term{胚胎移植的意義}{提高優良品種家畜的繁殖能力,加速品種改良}\qreft{48}{mid.left.5}

\para{胚胎分割}
\elem{胚胎分割}{無性生殖}\qreft{49}{upper.right}
\term{\stress{僅囊胚}的分割之要求}{將\stress{內細胞團}均等分割}\qreft{48}{upper.right.6}
\term{早期胚胎不能無限分割之原因}{胚胎分割產生同卵多胎的可能性有限,目前以二分胚成功性最高}\qreft{49}{upper.left}
\term{胚胎分割後移植之操作}{直接將裸半胚植入或先移入預先準備的透明帶}\qreft{51}{3}
\prop{胚胎分割的結果}{形態學差異}因性狀可能受環境影響。\qreft{93}{3}

\para{胚胎幹細胞}
\term{將早期胚胎培養為幹細胞,在培養基中加入者}{抑制因子}\qreft{96}{9.3}
\term{胚胎幹細胞之特點}{體積小,細胞核\stress{大},核仁明顯}\qreft{93}{8}
\term{使幹細胞定向分化之過程}{誘導分化}\qreft{94}{13.4}



\subsection{生物·卷III·實踐}
\subsubsection{DNA與蛋白質技術}
\term{PCR技術之條件要求}{模板、游離的\dna 、引物與DNA聚合酶等}\qreft{74}{13.1}
\term{PCR技術之引物要求}{與模板DNA兩條鏈末端鹼基互補配對}因此若PCR實驗失敗,可考慮重新設計引物。\qreft{10}{2}\qreft{72}{5}
\term{為方便構建重組質粒,引物中可添加適當的}{\stress{限制酶}切割位點}\qreft{72}{11.2}
\term{為防止自聯,引物之間應避免形成}{鹼基互補配對}\qreft{72}{4}
\term{PCR技術之聚合酶應當使用者}{耐高溫的Taq DNA聚合酶}
\centry{Taq謂Thermus aquaticus,aquaticus謂與水相關的,由是知Taq酶由溫泉或深海熱泉細菌取得。}

\end{document}
