\parah{基因工程通論與限制酶,DNA連接酶}
\term{基因工程之最終目的}{定向改造生物的遺傳性狀}\qreft{4}{lower.1}
\term{基因工程所需的必要酶}{限制酶,\stress{DNA連接酶}}\qreft{71}{1}
\term{限制酶之作用特點}{識別特定的\stress{核苷酸}序列,並在特定位點上切割DNA分子}\qreft{11}{7.7}
\meth{\wa{切割酶的選用},對目的基因應選用可將包含目的基因的一段DNA片段切下的酶,對質粒應選用切下二標記基因之一的酶}\qref{\bio\tek}{69}{8}
\term{在目的基因兩側選用不同的酶切割之原因}{防止目的基因與載體反向連接}\qreft{76}{14.1}
\term{限制酶的切割對象}{有特定\stress{脫氧}核苷酸序列的DNA片段}\qref{\bio\tek}{70}{11.1}
\term{因目的基因事後不存在切割位點致限制酶無法切割}{無法識別切割位點}\qref{\bio\tek}{70}{11.5}
\sus{基因能夠與質粒連接的主要原因是切割後形成相同黏性末端}{\bio\tek}{74}{13.2}
\term{限制酶的相同點之一}{切割後形成的黏性末端相同或互補}選填何種相同點應視上下文決定之。\qref{\bio\tek}{70}{12.3}
\term{DNA連接酶有兩類}{\TIV\ DNA連接酶與\Ecoli\ DNA連接酶}前者連接黏性末端或平末端,後者僅連接黏性末端。\qref{\bio\tek}{70}{12.5}
\centry{\Ecoli 謂Escherichia coli,即大腸桿菌。}






\para{質粒與其他載體}
\meth{\wa{基因工程之判定},涉及以質粒或其他載體導入目的基因者,皆可判定為使用了基因工程技術}\qreft{72}{13.2}
\term{基因工程操作之核心步驟}{基因表達載體之構建}\qreft{6}{lower.right.2}
\term{需表達載體之原因}{目的基因無複製原點與表達所需的啟動子}\qreft{16}{6.3}
\fterm{質粒載體應具備之條件}{\begin{enumerate}
\item 具有一個或多個限制酶切割位點,供外源DNA片段插入
\item 具有標記基因
\item 能自我複製
\end{enumerate}}\qreft{5}{7.1}
\centry{煙草花葉病毒為RNA病毒,而大多數噬菌體為DNA	 病毒。}
\term{基因工程之載體}{質粒,\stress{$\lambda$噬菌體衍生物與動植物病毒}}\qreft{70}{12}
\comp{\meth{受體為動物,選擇動植物病毒}}{\meth{受體為細菌,選擇噬菌體}}\wa{載體之選用}:\qreft{11}{6.2}
\term{標記基因之作用}{供重組DNA的鑑定與選擇}\qreft{3}{lower.right}
\cons{將質粒與目的基因用同一酶切,連接的結果}{質粒自聯,廢棄基因自聯,目的基因與質粒互聯}\qref{\bio\tek}{70}{10}
\term{重組質粒作為表達載體應當具有}{啟動子、終止子、標記基因}\qreft{74}{14.3}
\sus{表達載體無需複製原點}{\bio\tek}{74}{14.3}
\term{噬菌體改造後作為載體,其DNA複製原料來自}{受體細胞}\err{宿主細胞}應當據上下文回答直接對象。\qreft{5}{7.3}
\meth{篩選轉化後細胞所用之抗生素應嚴格視條件之抗性基因決定}\qreft{24}{3}
\term{微生物受體不能在抗生素培養基上生存之主要原因}{重組質粒未導入}\qreft{74}{13.3}
\term{某一生物之基因能插入其他生物的染色體之原因}{其遺傳物質均為DNA,物質組成與空間結構相同}\qreft{70}{14}
