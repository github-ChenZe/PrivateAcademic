\para{基因的獲取}
\term{獲得目的基因之主要途徑}{從已有物種中分離與人工合成}\qreft{74}{14.1}
\term{從植物中獲得mRNA,選用嫩葉而非老業之原因}{嫩葉組織細胞易破碎}\qreft{16}{6.1}
\defi{cDNA}{mRNA逆轉錄之產物}\qreft{94}{14.1}\qreft{72}{1}
\defi{cDNA文庫}{某生物某一發育時期所轉錄的mRNA\stress{全部}經逆轉錄形成的cDNA片段與載體連接而形成的集合}\qreft{73}{11.1}
\term{由mRNA獲得cDNA之原理}{在逆轉錄酶的作用下,以mRNA為模板按照\cc 原則合成cDNA}\qreft{16}{6.2}
\meth{\wa{選用限制酶},應當選擇切割位點盡可能接近目的基因者}\qreft{71}{5}
\sus{人工合成法所合成之序列不唯一}{\bio\tek}{7}{2}
\comp{\meth{從供體DNA開始,可推斷為基因組文庫法}}{\meth{從mRNA開始,可推斷為部分基因文庫法}}\wa{獲得目的基因之方法}:\qreft{10}{1}
\meth{\wa{脫氧核苷酸原料數量的計算},$2^n-\stress{1}$倍的單個DNA分子的原料數量}\qreft{71}{6}
\para{基因的導入}
\meth{\wa{受體生物之選擇},對要求作物之細胞產物的情況,應考慮產物烹飪後失活之可能}\qreft{74}{11.3}
\term{將目的基因導入微生物之方法}{感受態細胞法}\qreft{71}{9}
\term{將目的基因導入微生物前之操作}{用鈣離子處理,使其轉化為感受態細胞}\qreft{76}{14.2}
\noi{交叉引用}{}{}{}
\term{使用細菌作為受體之優點}{繁殖速度快,結構簡單}\qreft{11}{3}
\defi{基因槍法}{用压缩气体动力把粘有DNA 的细微金粉打向细胞,穿过细胞壁、细胞膜、细胞质等构造到达细胞核,完成基因转移}
\term{將目的基因導入單子葉植物之常用方法}{基因槍法}\qreft{6}{lower.left}\qreft{9}{2}
\term{農桿菌轉化法之目的基因插入對象}{Ti質粒之\stress{T-DNA}}\qreft{8}{lower.left}\qreft{74}{14.4}
\centry{Ti質粒之Ti謂Tumour inducing。}
\centry{T-DNA之T謂Transfer。}
\term{目的基因之插入對象}{染色體\stress{DNA}}\qreft{72}{13.3}
\term{整合到染色體上之對象}{目的基因}\err{質粒}\qreft{73}{8}
\term{目的基因插入染色體DNA後結果}{目的基因得以穩定維持與表達}\qreft{8}{mid.right}
\para{基因的檢測}
\defi{杂交探针}{一小段单链DNA片段,用于检测与其互补的核酸序列}在分子雜交技術與DNA分子雜交技術中使用。\qreft{71}{9}
\term{目的基因能否在植物體內穩定遺傳的關鍵}{是否整合至染色體}\qreft{72}{5}
\term{檢查目的基因是否整合至染色體之方法}{DNA分子雜交}\qreft{72}{5}
\term{DNA分子雜交技術中不發生雜交帶者}{非目的基因}\qreft{9}{lower.left}
