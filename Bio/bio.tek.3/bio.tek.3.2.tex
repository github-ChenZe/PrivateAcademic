\para{卵母細胞或卵子的採集}
\elem{促性腺激素}{肽類化合物}因此不得口服。\qreft{44}{mid.left.1}
\term{採集卵母細胞之位置}{卵巢}\err{輸卵管}\qreft{95}{4}
\term{大型動物採集卵母細胞之方法}{從卵巢中採集}可自活體動物或屠宰場。\qreft{46}{2}
\term{大型動物採集卵母細胞後處理}{培養成熟}故無需篩選處於\MII 中期的細胞。\qreft{93}{8}
\term{體外受精的意義}{提高卵子的利用率,有助於發揮優良母豬的繁殖潛能}\qreft{91}{6}

\para{精子的採集與體外受精}
\prop{精清}{抑制精子獲能的物質}因此體外受精前對精子離心處理。\qreft{44}{mid.left.2}
\term{精子獲能之常用化學方法}{化學誘導法與培養法}\qreft{92}{12.2}
\comp{\term{對嚙齒動物、豬、兔常用獲能方法}{培養法}}{\term{對\stress{家畜}常用獲能方法}{化學誘導法}}\qreft{96}{10.1}
\term{精子獲能之化學誘導法所使用之試劑}{肝素溶液或鈣離子載體的A23187溶液}\qreft{44}{1}
\sus{精子獲能環境通入二氧化碳之原因謂調節獲能液之pH。}{\bio\tek}{93.3}{}
\term{體外受精技術之原理}{人工模擬體內的環境,包括營養、溫度、pH值等,使卵母細胞成熟,同時使精子或能,最終完成受精}\qreft{92}{13.2}
\noi{選填受精或體外受精之區分}\qreft{93}{11}
\term{體外受精技術之意義}{解決動物生育率低的問題}\qreft{92}{12.4}
\defi{試管嬰兒技術}{體外受精的受精卵經過初步篩選與發育後進行胚胎移植}\qreft{92}{15.1}

\para{人工授精}
\comp{\elem{人工授精}{人工將精液置入子宮內}}{\elem{配種}{自然交配}}\qreft{91}{6}
\term{人工授精的意義}{提高精子的利用率,有利於發揮優良公畜的特性}\qreft{91}{6}
\term{人工授精與體外受精的本質同}{精卵結合}\qreft{91}{6}

\para{早期胚胎培養}
\term{早期胚胎培養之培養液成分}{兩鹽、兩素、兩酸與\stress{動物激素}}其中動物激素具有特殊性故優先填寫\qreft{50}{1}\qreft{96}{10}
\term{早起胚胎培養之條件要求}{\stress{無菌無毒}、營養、溫度和pH與氣體環境}\qreft{93}{11.3}
\noi{各種培養基或培養液成分}{\bio\tek}{50}{1}\qreft{45}{lower.right}強調血清/血漿。\qreft{96}{11}
\term{受精卵培養中,為防止感染可添加者}{抗生素}\qreft{92}{14.3}
