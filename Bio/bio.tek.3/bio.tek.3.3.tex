\para{胚胎移植}
\term{對供體公牛與母牛之要求}{具有優良的遺傳性狀}\qreft{93}{11.2}
\elem{促性腺激素}{垂體分泌之激素}\qreft{94}{12.4}
\term{胚胎移植之生理基礎}{同種動物供、受體排卵後,生殖器官的生理變化相同;\stress{受體對外來胚胎基本不發生免疫排斥反應};供體胚胎可與受體子宮建立正常的生理和組織聯繫,但遺傳特性不受影響}第二點優先填寫。\qreft{94}{15.5}
\term{同期發情處理之目的}{在胚胎移植前後提供相同的生理環境}\qreft{96}{5.4}
\comp{\term{胚胎移植前,對供體母牛之處理}{同期發情與促性腺激素}}{\term{胚胎移植前,對供體母牛之處理}{同期發情}}\qreft{93}{4}
\term{胚胎移植之胚胎來源}{(體內受精)配種、人工授精,體外受精,細胞核移植之胚胎}\qreft{51}{6.4}
\sus{動物種類不同,進行胚胎移植的早期胚胎所處階段不同,課本p.72.bottom,牛、羊一般培養至桑椹胚或囊胚,小鼠、家兔等可更早}{\bio\tek}{45}{2}\qreft{96}{11.5}
\term{配種(體內受精)後胚胎移植前的步驟}{沖卵}\qreft{48}{mid.right.2}
\term{沖卵之對象}{胚胎}\err{卵子}\qreft{51}{2}
\term{沖卵之生理基礎}{早期胚胎形成後,在一定時間內不會與母體子宮建立組織上的聯繫,而是處於游離狀態}\qreft{48}{mid.right.2}
\meth{\wa{篩選的具體化},胚胎移植前的篩選,可視上下文填寫更確定的篩選形式}如性別鑑定\qreft{94}{14.5}
\term{胚胎移植的意義}{提高優良品種家畜的繁殖能力,加速品種改良}\qreft{48}{mid.left.5}

\para{胚胎分割}
\elem{胚胎分割}{無性生殖}\qreft{49}{upper.right}
\term{\stress{僅囊胚}的分割之要求}{將\stress{內細胞團}均等分割}\qreft{48}{upper.right.6}
\term{早期胚胎不能無限分割之原因}{胚胎分割產生同卵多胎的可能性有限,目前以二分胚成功性最高}\qreft{49}{upper.left}
\term{胚胎分割後移植之操作}{直接將裸半胚植入或先移入預先準備的透明帶}\qreft{51}{3}
\prop{胚胎分割的結果}{形態學差異}因性狀可能受環境影響。\qreft{93}{3}

\para{胚胎幹細胞}
\term{將早期胚胎培養為幹細胞,在培養基中加入者}{抑制因子}\qreft{96}{9.3}
\term{胚胎幹細胞之特點}{體積小,細胞核\stress{大},核仁明顯}\qreft{93}{8}
\term{使幹細胞定向分化之過程}{誘導分化}\qreft{94}{13.4}