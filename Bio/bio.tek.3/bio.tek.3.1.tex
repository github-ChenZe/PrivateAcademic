\parah{精子}
\term{精子的能量來源}{線粒體與細胞質基質}\qreft{42}{2}
\term{精子形成過程中不同於卵子的最主要特徵}{須變形}\qreft{42}{6.2}
\term{精子形成過程中留下的主要結構}{細胞核與線粒體}\qreft{42}{6.2}
\term{精子的線粒體集中在尾部的原因}{精子尾部游動的能量由線粒體提供}\qreft{90}{10.3}


\para{卵子與受精}
\term{哺乳動物卵子的發生之開始時期}{胚胎性別分化後}\qreft{90}{11.1}
\defi{卵泡}{卵原細胞被卵泡細胞包圍形成者}故卵泡不等於卵泡細胞\qreft{90}{11.2}
\sus{卵子的發生地謂卵巢與輸卵管。}{\bio\tek}{90}{11.1}
\comp{\term{馬的排卵時間}{\MI 前}}{\term{牛、羊或豬的排卵時間}{\MI 後}}因此動物排除卵子的成熟程度不同。\qreft{95}{2}
\term{透明帶形成時間}{\MI 之前}\qreft{44}{upper.left}
\elem{卵原細胞分裂為初級卵母細胞}{有絲分裂},故二者染色體數均為$2N$。\qreft{42}{6.3} 
\term{卵子具備與精子受精的能力之時期}{\MII 中期}\qreft{40}{lower.left}
\begin{figure}[!htb]
    \centering
    \hrulefill\\
    \vspace{0.2cm}
    \begin{minipage}{.45\textwidth}
        \term{精子溶解卵丘細胞之間的物質,穿越放射冠}{頂體反應}
    \end{minipage}%
    \hspace{0.2cm}
    \rulesep
    \hspace{0.2cm}
    \begin{minipage}{0.45\textwidth}
       \term{頂體反應之發生時間}{精子與卵子相遇時}\qreft{94}{12.1}

    \end{minipage}
    \\
    \vspace{0.2cm}
    \hrulefill
    \\
    \vspace{0.2cm}
    \begin{minipage}{.45\textwidth}
        \term{防止多精入卵的第一道屏障}{透明帶反應}\qreft{40}{lower.right.2}
    \end{minipage}%
    \hspace{0.2cm}
    \rulesep
    \hspace{0.2cm}
    \begin{minipage}{0.45\textwidth}
       \term{透明帶反應之發生時間}{精子觸及卵細胞膜的瞬間}
    \end{minipage}
    \\
    \vspace{0.2cm}
    \hrulefill
    \\
    \vspace{0.2cm}
    \begin{minipage}{.45\textwidth}
        \term{防止多精入卵的第二道屏障}{卵細胞膜反應}\qreft{40}{lower.right.2}
    \end{minipage}%
    \hspace{0.2cm}
    \rulesep
    \hspace{0.2cm}
    \begin{minipage}{0.45\textwidth}
       \term{卵細胞膜反應之發生時間}{精子入卵後}
    \end{minipage}
    \vspace{0.2cm}
    \\ \hrulefill
\end{figure}
\prop{精卵識別}{物種特異性}\qreft{89}{5}
\term{精子入卵}{精子外膜與卵細胞膜相互融合}\qreft{40}{lower.\-right.2}
\term{雄原核的形成時間}{精子入卵後}\qreft{89}{2}
\cons{卵子發生}{1個卵子與3個極體}\qreft{39}{lower.right}
\sus{極體數量矛盾}{\bio\tek}{40}{lower.right}
多數哺乳動物\PI 不進行\MII 。
\term{受精標誌}{卵細胞\stress{膜}和透明帶之間觀察到兩個極體}\qreft{40}{lower.right}\qreft{90}{14.3}
\term{雌原核的形成時間}{排出\PII 後}\qreft{89}{2}
\term{受精完成標誌}{雌雄原核的融合}\qreft{40}{lower.right}
受精卵中僅細胞核內的遺傳物質一半來自父方一半來自母方。\qreft{42}{2}



\para{早期胚胎發育}
\defi{卵裂期}{2細胞至桑椹胚期}\qreft{49}{1}
\term{細胞分化至開始時期}{囊胚期}\qreft{39}{upper.right.6}\qreft{90}{12.3}
\term{胚層分化之開始時期}{原腸胚期}\qreft{90}{12.3}
\cons{滋養層發育}{胎盤與胎膜}\qreft{41}{2}
