\parah{植物組織培養技術·取材與條件}
\meth{\wa{植物組織培養技術之判定}涉及由植物體細胞培養至植株者,皆可判定為使用了植物組織培養技術}\qreft{72}{13.2}
\noi{選填「單倍體育種」「花藥離體培養技術」「植物組織培養技術」之區分。}{\bio\tek}{24}{7.4}
\defi{外植體}{植物組織培養中作為離體培養材料的器官或組織的片段}
\defi{雜種優勢}{雜交種通過繼承其父母的不同的優勢,獲得更好的生物特性}
\defi{細胞全能性}{已分化的細胞具有發育成完整生物體的潛能}\qreft{21}{lower.1}
\fterm{植物細胞表現全能性之條件}{
\begin{enumerate}
\item \stress{脫離母體};
\item 給予適當營養與外界條件。
\end{enumerate}}
\vref{1070505}{2}
\par
卵細胞與受精卵皆有全能性,卵細胞全能性最高。\qreft{24}{1}
\term{植物組織培養過程之條件}{無菌}\qreft{80}{12}
\term{植物組織培養中污染之可能原因}{外植體消毒不徹底}外植體消毒困難\qreft{29}{7.4}
\term{植物組織培養技術之固體培養基內為保障植物生長應當加入}{無機鹽與水}\cf V.p.84\qreft{29}{7.2}
\term{植物組織培養技術之固體培養基內為形成癒傷組織應當加入}{植物激素}\qreft{29}{7.3}
\term{脫分化培養基應加入者}{無機營養物質,生長素與細胞分裂素,有機營養物質}\qreft{79}{9}
\noi{植物組織培養之激素影響與根/芽順序}{\bio\tek}{22}{right.1}\qreft{24}{7.3}
\term{取新生的莖尖細胞培養的原因}{莖尖細胞全能性高與新生細胞無箘無毒}\qreft{24}{7}

\para{植物組織培養技術·脫分化、再分化與結果}
\term{從結果看,分化、脫分化、再分化之實質}{基因的選擇性表達}\qreft{79}{9}
\term{脫分化之實質}{使細胞恢復分裂能力}\qref{\bio\tek}{82}{10}
\term{脫分化過程之條件謂}{避光}\qreft{22}{right.1}
\term{癒傷組織結構特點}{細胞壁更薄,高度液泡化,無定形}\qreft{79}{9}
\par
癒傷組織不含葉綠體,故癒傷組織之分化無需充足光照。\qreft{82}{14}
\par
所填寫培養基之類型名稱,應端視上下文所欲強調之屬性而確定。涉及激素者,應選填「分化培養基」而非「固體培養基」。\qreft{24}{7.3}
\cons{癒傷組織再分化之結果視目的}{胚狀體或幼根和芽}\qreft{22}{upper.r.1}
\par
產物可直接通過癒傷組織獲得。\stress{是否應包含與工廠化生產?}\qref{\bio\tek}{81}{6}
\term{植物組織培養中再分化過程需要激素}{生長素、細胞分裂素}\qreft{22}{right.2}
\noi{各種植物激素之作用}{\bio\tek}{22}{right.2}
\sus{雜種優勢後代不發生性狀分離}{\bio\tek}{81}{5}
\term{轉基因植物之培養所依賴之技術}{轉基因技術與植物組織培養技術}\qreft{24}{4}
\noi{轉基因技術與基因工程技術之定義}{\bio\tek}{24}{4}


\para{人工種子}
\defi{胚狀體}{相當於天然種子的胚}由分生組織構成。
\par
胚狀體與頂芽、腋芽、不定芽並列\err{包含}
\par
人工種子由胚狀體製得,\err{幼苗}\qref{\bio\tek}{81}{7}
\term{人工種子包裹胚狀體的膠質之擬制}{種皮,胚乳、子葉等提供營養的結構}\qref{\bio\tek}{81}{10}
\par
人工種子萌發之植株,可育與否取決於種子之產生方式。體細胞生成之種子可育,配子生成則不可育。\qreft{82}{10}


\para{單倍體育種與花藥離體培養技術}
\elem{花藥離體培養技術}{植物組織培養技術}\qreft{24}{4}
\term{單倍體育種之原理,謂}{生殖細胞之全能性}\qreft{26}{lower.s1.1}
\elem{香蕉}{無花粉的植物}故無法進行花藥離體培養。\qreft{28}{3}
\termcomp{花藥離體培養之條件}{培養基,適宜溫度、pH、光照}{種子萌發之條件}{適宜的溫度、水分}
\qreft{22}{right.2}

\defi{單倍體植株}{配子離體培養所得植株}\qreft{80}{10}
\term{秋水仙素質作用時期}{有絲分裂前期抑制紡錘體形成}\qreft{27}{right.1}
\term{秋水仙素於植物引發之變異}{染色體數目變異}\qreft{26}{lower.s1.1}
\noi{染色體變異類型}{}{}{}
\term{單倍體育種之優點}{明顯縮短育種年限}\qreft{24}{7.4}



\para{脫毒苗的製作}
\term{製作脫毒苗,應當利用者}{無病毒或病毒較少的莖尖或根尖}\qreft{25}{lower.s2.2}

\para{微型繁殖}
\term{微型繁殖技術之意義}{保持品種的遺傳特性}\qreft{82}{13}

\para{突變體利用}
\defi{誘變育種}{對植物的癒傷組織誘變處理,使之突變,再通過誘導分化形成植株}
\term{誘變育種之原理}{基因突變與植物體細胞之分化}\qreft{26}{lower.s2.1}

\para{細胞產物的工廠化生產}
\term{細胞產物的工廠化生產採用的培養基謂}{液體培養基}\qreft{28}{6.2}
\term{在液體培養基中通入空氣的原因}{保障氧氣供應,使細胞和培養液充分接觸}\qreft{28}{6.2}
\term{液體培養後期產物減少的原因}{活細胞數量下降,細胞代謝產物積累,營養物質消耗}\qreft{29}{6.3}




\para{植物體細胞雜交技術}
\defi{植物體細胞雜交}{將不同種的植物體細胞,在一定條件下融合成雜種細胞,並把雜種細胞培育成新的植物體的技術}\qreft{24}{6}
\term{植物體細胞雜交技術包括之手段}{植物體細胞融合,植物組織培養}\qreft{80}{11}
\term{植物細胞去壁所用酶}{纖維素酶與果膠酶}\qreft{21}{upper.\-1}
\sus{酶解法去壁,不應在低滲溶液中進行}{\bio\tek}{79}{5}
\term{植物體細胞融合之原生質體融合產物謂}{融合的原生質體}\qreft{80}{12}
\term{植物體細胞雜交完成標誌}{雜種原生質體再生出細胞壁}\qreft{22}{lower.1}
\meth{細胞融合除考慮三種兩兩融合產物,還應考慮未融合的情形}\qreft{80}{11}
\meth{不同種的植物之間\stress{雜交},應注意AA+BB$\rightarrow$AB}\\ \err{AA+BB$\rightarrow$AABB,不同於融合}\qreft{82}{12}
\term{植物體細胞雜交技術之目的}{獲得新植株}\err{雜種細胞}\qreft{21}{upper.2.6}
\term{植物體細胞雜交技術之意義}{克服遠緣雜交不親和的障礙}\qreft{80}{11}
\term{植物體細胞雜交仍存在之問題}{不能按照人的意願表達兩個親本的性狀}\qreft{80}{12}