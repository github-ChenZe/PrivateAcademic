% !TEX encoding = UTF-8 Unicode
\documentclass[UTF-8]{ctexart}

% Mathematics Include

\usepackage{amsmath}
\usepackage{amssymb}
\usepackage{amsthm}
\usepackage{amsfonts}
\usepackage{mathrsfs}
\usepackage{enumitem}
\usepackage{braket}
\usepackage{hyperref}
\usepackage[all, pdf]{xy}
\usepackage{wrapfig}
\usepackage{leftidx}

% Physics Include
\usepackage{amsmath}
\usepackage{physics}
\usepackage{siunitx}
\usepackage[makeroom]{cancel}
\usepackage{pstricks}
\usepackage{pstricks-add}
\psset{algebraic=true}

\usepackage[version=4]{mhchem}
\usepackage{array,booktabs}
\usepackage{longtable}
\usepackage{mathtools}
\usepackage[normalem]{ulem}
\usepackage{multicol}

\usepackage{mdframed}
\usepackage{lipsum}% just to generate text for the example

\newmdenv[
  topline=false,
  bottomline=false,
  skipabove=\topsep,
  skipbelow=\topsep
]{siderules}


% Mathematics Head

\newcommand{\pare}[1]{\left(#1\right)}
\newcommand{\blr}[1]{\left[#1\right)}
\newcommand{\lbr}[1]{\left(#1\right]}
\newcommand{\brac}[1]{\left[#1\right]}
\newcommand{\curb}[1]{\left\{#1\right\}}
% \newcommand{\abs}[1]{\left|\, #1 \,\right|}
\newcommand{\rec}[1]{\frac{1}{#1}}
\newcommand{\N}{\mathbb{N}}
\newcommand{\bC}{\mathbb{C}}
\newcommand{\Q}{\mathbb{Q}}
\newcommand{\Z}{\mathbb{Z}}
\newcommand{\R}{\mathbb{R}}
\newcommand{\unk}{\mathcal{X}}
\newcommand{\bu}[3]{#1_{#2}^{\pare{#3}}}
\newcommand{\dref}[1]{定义\ref{def:#1}}
\newcommand{\tref}[1]{定理\ref{thm:#1}}
\newcommand{\lref}[1]{引理\ref{lem:#1}}
\newcommand{\cref}[1]{推论\ref{coll:#1}}
\newcommand{\pref}[1]{命题\ref{prp:#1}}
\newcommand{\rmref}[1]{附注\ref{rm:#1}}
\newcommand{\eref}[1]{例\ref{ex:#1}}
\newcommand{\dcompare}[1]{\textit{平行于\dref{#1}}}
\newcommand{\tcompare}[1]{\textit{平行于\tref{#1}}}
\newcommand{\lcompare}[1]{\textit{平行于\lref{#1}}}
\newcommand{\ecompare}[1]{\textit{平行于\eref{#1}}}
\newcommand{\ccompare}[1]{\textit{平行于\cref{#1}}}
\newcommand{\func}[3]{#1:\, #2 \rightarrow #3}
\newcommand{\overbar}[1]{\mkern 1.5mu\overline{\mkern-1.5mu#1\mkern-1.5mu}\mkern 1.5mu}
\newcommand{\clo}[1]{\overbar{#1}}
\newcommand{\supi}[2]{\overbar{\int_{#1}^{#2}}}
\newcommand{\infi}[2]{\underbar{\int_{#1}^{#2}}}
\newcommand{\setf}{\mathscr}
\newcommand{\bool}{\mathrm{bool}}
\newcommand{\inc}{++}
\newcommand{\defeq}{:=}
\newcommand{\ntuple}{$n$元组}
\newcommand{\card}[1]{\#\pare{#1}}
\newcommand{\setcond}[2]{\curb{#1 \, \left| \, #2 \right.}}
\newcommand{\setcondl}[2]{\curb{\left. #1 \, \right| \, #2}}
\newcommand{\bv}[1]{\mathbf{#1}}
\newcommand{\bfa}{\bv{a}}
\newcommand{\bfb}{\bv{b}}
\newcommand{\bfx}{\bv{x}}
\newcommand{\bfy}{\bv{y}}
\newcommand{\bfe}{\bv{e}}
\newcommand{\bfF}{\bv{F}}
\newcommand{\bff}{\bv{f}}
\newcommand{\bfG}{\bv{G}}
\newcommand{\bfH}{\bv{H}}
\newcommand{\bfg}{\bv{g}}
\newcommand{\bfh}{\bv{h}}
\newcommand{\bfr}{\bv{r}}
\newcommand{\bfk}{\bv{k}}
\newcommand{\bfu}{\bv{u}}
\newcommand{\bfv}{\bv{v}}
\newcommand{\oo}[1]{o\pare{#1}}
\newcommand{\OO}[1]{O\pare{#1}}
% \newcommand{\norm}[1]{\left\| #1 \right\|}
\newcommand{\DD}{\mathbf{D}}
\newcommand{\comp}{\circ}
\newcommand{\const}{\mathrm{const}}
\newcommand{\dist}[2]{d\pare{#1,#2}}
\newcommand{\len}{\ell}
\newcommand{\siga}{$\sigma$-代数}
\newcommand{\cara}{Carath\'{e}odory}
\newcommand{\Gd}{G_\delta}
\newcommand{\Fs}{F_\sigma}
\newcommand{\mmani}{$m$-维流形}
\newcommand{\open}[1]{\mathcal{#1}}
\newcommand{\half}{\frac{1}{2}}
\newcommand{\maxo}[1]{\text{max}\curb{#1}}
\newcommand{\mino}[1]{\text{min}\curb{#1}}
\newcommand{\epsclo}{$\epsilon$-接近}
\newcommand{\close}[1]{$#1$-接近}
\newcommand{\cinf}{$C^\infty$}
\newcommand{\cuno}{$C^1$}
\newcommand{\Int}{\text{Int}\,}
\newcommand{\Ext}{\text{Ext}\,}
\newcommand{\funcf}{\mathcal}
\newcommand{\DDu}{\overbar{\DD}}
\newcommand{\DDl}{\underbar{\DD}}
\newcommand{\Diff}[1]{\mathrm{Diff}_{#1}\,}
\newcommand{\Av}[1]{\mathrm{Av}_{#1}\,}
\newcommand{\Lip}[1]{Lipschitz-$#1$}
\newcommand{\sgn}{\mathrm{sgn}}
\newcommand{\sgnf}{\mathrm{sgn}\,}
\newcommand{\eset}{\varnothing}
\newcommand{\difn}[1]{^{\pare{#1}}}
%\newcommand{\ca}{\mathcal{a}}
%\newcommand{\cb}{\mathcal{b}}
%\newcommand{\cc}{\mathcal{c}}
%\newcommand{\cd}{\mathcal{d}}
%\newcommand{\ce}{\mathcal{e}}
%\newcommand{\cf}{\mathcal{f}}
%\newcommand{\cg}{\mathcal{g}}
%\newcommand{\ch}{\mathcal{h}}
%\newcommand{\ci}{\mathcal{i}}
%\newcommand{\cj}{\mathcal{j}}
%\newcommand{\ck}{\mathcal{k}}
%\newcommand{\cl}{\mathcal{l}}
%\newcommand{\cm}{\mathcal{m}}
%\newcommand{\cn}{\mathcal{n}}
%\newcommand{\co}{\mathcal{o}}
%\newcommand{\cp}{\mathcal{p}}
%\newcommand{\cq}{\mathcal{q}}
%\newcommand{\cr}{\mathcal{r}}
%\newcommand{\cs}{\mathcal{s}}
%\newcommand{\ct}{\mathcal{t}}
%\newcommand{\cu}{\mathcal{u}}
%\newcommand{\cv}{\mathcal{v}}
%\newcommand{\cw}{\mathcal{w}}
%\newcommand{\cx}{\mathcal{x}}
%\newcommand{\cy}{\mathcal{y}}
%\newcommand{\cz}{\mathcal{z}}
\newcommand{\cA}{\mathcal{A}}
\newcommand{\cB}{\mathcal{B}}
\newcommand{\cC}{\mathcal{C}}
\newcommand{\cD}{\mathcal{D}}
%\newcommand{\cE}{\mathcal{E}}
\newcommand{\cF}{\mathcal{F}}
\newcommand{\cG}{\mathcal{G}}
\newcommand{\cH}{\mathcal{H}}
\newcommand{\cI}{\mathcal{I}}
\newcommand{\cJ}{\mathcal{J}}
\newcommand{\cK}{\mathcal{K}}
\newcommand{\cL}{\mathcal{L}}
\newcommand{\cM}{\mathcal{M}}
\newcommand{\cN}{\mathcal{N}}
\newcommand{\cO}{\mathcal{O}}
\newcommand{\cP}{\mathcal{P}}
\newcommand{\cQ}{\mathcal{Q}}
\newcommand{\cR}{\mathcal{R}}
\newcommand{\cS}{\mathcal{S}}
\newcommand{\cT}{\mathcal{T}}
\newcommand{\cU}{\mathcal{U}}
\newcommand{\cV}{\mathcal{V}}
\newcommand{\cW}{\mathcal{W}}
\newcommand{\cX}{\mathcal{X}}
\newcommand{\cY}{\mathcal{Y}}
\newcommand{\cZ}{\mathcal{Z}}
\newcommand{\inter}[1]{\mathring{#1}}
\newcommand{\forest}[3]{对于{#1},存在{#2},使得{#3}}
\newcommand{\tuno}{$T_1$公理}
\newcommand{\isom}{\overset{\sim}{=}}
\newcommand{\diam}{\mathrm{diam}\,}
\newcommand{\ord}[1]{\abs{#1}}
\newcommand{\sbm}[1]{\overbar{#1}}
\newcommand{\inv}[1]{#1^{-1}}
\newcommand{\restr}[2]{#1|_{#2}}
\newcommand{\divs}{|}
\newcommand{\ndivs}{\nmid}
\newcommand{\modeq}[1]{\overbar{#1}}
\newcommand{\ggen}[1]{\langle#1\rangle}
\newcommand{\ggencond}{\braket}

\newcommand{\hd}{H\"{o}lder}

\renewcommand{\proofname}{证明}

\newenvironment{cenum}{\begin{enumerate}\itemsep0em}{\end{enumerate}}

\newtheorem{definition}{定义}[section]
\newtheorem{lemma}{引理}[section]
\newtheorem{theorem}{定理}[section]
\newtheorem{collary}{推论}[section]
\newtheorem{corollary}{推论}[section]
\newtheorem{proposition}{命题}[section]
\newtheorem{axiom}{公理}[section]
\newtheorem{ass}{假设}[section]
\newtheorem{ex}{例}[section]
\newtheorem{remark}{附注}[section]
%\newtheorem*{remark*}{附注}[section]
\newtheorem{reflection}{反射}[section]
\newcommand{\refl}[1]{\vspace{0.5em}\par\noindent\fbox{%
    \parbox{0.97\textwidth}{%
    \begin{reflection}
        #1
    \end{reflection}
    }%
}\vspace{0.5em}\par}
\newcommand{\rref}[1]{反射\ref{refl:#1}}
\newcommand{\tbref}[1]{表\ref{table:#1}}
\allowdisplaybreaks

\newenvironment{aenum}{\begin{enumerate}[label=\textnormal{(\alph*)}]}{\end{enumerate}}

% Physics Head

\DeclareSIUnit\dyne{dynes}

\newcommand{\ddel}[1]{\frac{\partial}{\partial #1}}
\newcommand{\ddelon}[2]{\frac{\partial #1}{\partial #2}}
\newcommand{\dddel}[1]{\frac{\partial^2}{\partial^2 #1}}
\newcommand{\ddt}{\ddel{t}}
\newcommand{\ddT}{\ddel{T}}
\newcommand{\ddV}{\ddel{V}}
\newcommand{\ddr}{\ddel{r}}
\newcommand{\dds}{\ddel{s}}
\newcommand{\ddron}[1]{\ddelon{#1}{r}}
\newcommand{\ddson}[1]{\ddelon{#1}{s}}
\newcommand{\ddton}[1]{\ddelon{#1}{t}}
\newcommand{\ddxon}[1]{\ddelon{#1}{x}}
\newcommand{\ddyon}[1]{\ddelon{#1}{y}}
\newcommand{\ddzon}[1]{\ddelon{#1}{z}}
\newcommand{\ddthon}[1]{\ddelon{#1}{\theta}}
\newcommand{\ddalon}[1]{\ddelon{#1}{\alpha}}
\newcommand{\ddth}{\ddel{\theta}}
\newcommand{\ddph}{\ddel{\phi}}
\newcommand{\dddt}{\dddel{t}}
\newcommand{\dddr}{\dddel{t}}
\newcommand{\dddth}{\dddel{\theta}}
\newcommand{\dddph}{\dddel{\phi}}
\newcommand{\rd}[1]{\mathrm{d} #1}
\newcommand{\dt}{\rd{t}}
\newcommand{\dy}{\rd{y}}
\newcommand{\dx}{\rd{x}}
\newcommand{\edd}[1]{\frac{\mathrm{d}}{\mathrm{d} #1}}
\newcommand{\eddd}[1]{\frac{\mathrm{d}^2}{\mathrm{d}^2 #1}}
\newcommand{\eddon}[2]{\frac{\mathrm{d} #1}{\mathrm{d} #2}}
\newcommand{\edddon}[2]{\frac{\mathrm{d}^2 #1}{\mathrm{d}^2 #2}}
\newcommand{\edt}{\edd{t}}
\newcommand{\edton}[1]{\eddon{#1}{t}}
\newcommand{\edalon}[1]{\eddon{#1}{\alpha}}
\newcommand{\edT}{\edd{T}}
\newcommand{\edr}{\edd{r}}
\newcommand{\edl}{\edd{l}}
\newcommand{\edx}{\edd{x}}
\newcommand{\edth}{\edd{\theta}}
\newcommand{\eddton}[1]{\edddon{#1}{t}}
\newcommand{\eddzon}[1]{\edddon{#1}{z}}
\newcommand{\vect}[1]{\boldsymbol{#1}}
\newcommand{\alp}{\frac{1}{\sqrt{2}}}
\newcommand{\alpi}{\frac{i}{\sqrt{2}}}
\newcommand{\expc}[1]{\langle#1\rangle}
\newcommand{\bkn}[1]{\bra{#1}\ket{#1}}
\newcommand{\bk}[2]{\bra{#1}\ket{#2}}
\newcommand{\bik}[3]{\bra{#1} #2 \ket{#3}}
\newcommand{\vari}[1]{\sigma_{#1}}
\newcommand{\intc}[2]{\left[#1, #2\right]}
\newcommand{\sch}{Schr\"{o}dinger}
\newcommand{\moment}{\boldsymbol{p}}
\newcommand{\coor}{\boldsymbol{x}}
\newcommand{\lapc}{\nabla^2}
% \newcommand{\rec}[1]{\frac{1}{#1}}
\newcommand{\vva}{\boldsymbol{a}}
\newcommand{\vvb}{\boldsymbol{b}}
\newcommand{\vc}{\boldsymbol{c}}
\newcommand{\vd}{\boldsymbol{d}}
\newcommand{\ve}{\boldsymbol{e}}
\newcommand{\vf}{\boldsymbol{f}}
\newcommand{\vg}{\boldsymbol{g}}
\newcommand{\vh}{\boldsymbol{h}}
\newcommand{\vi}{\boldsymbol{i}}
\newcommand{\vj}{\boldsymbol{j}}
\newcommand{\vk}{\boldsymbol{k}}
\newcommand{\vl}{\boldsymbol{l}}
\newcommand{\vm}{\boldsymbol{m}}
\newcommand{\vn}{\boldsymbol{n}}
\newcommand{\vo}{\boldsymbol{o}}
\newcommand{\vp}{\boldsymbol{p}}
\newcommand{\vq}{\boldsymbol{q}}
\newcommand{\vr}{\boldsymbol{r}}
\newcommand{\vs}{\boldsymbol{s}}
\newcommand{\vt}{\boldsymbol{t}}
\newcommand{\vvu}{\boldsymbol{u}}
\newcommand{\vv}{\boldsymbol{v}}
\newcommand{\vw}{\boldsymbol{w}}
\newcommand{\vx}{\boldsymbol{x}}
\newcommand{\vy}{\boldsymbol{y}}
\newcommand{\vz}{\boldsymbol{z}}
\newcommand{\vA}{\boldsymbol{A}}
\newcommand{\vB}{\boldsymbol{B}}
\newcommand{\vC}{\boldsymbol{C}}
\newcommand{\vD}{\boldsymbol{D}}
\newcommand{\vE}{\boldsymbol{E}}
\newcommand{\vF}{\boldsymbol{F}}
\newcommand{\vG}{\boldsymbol{G}}
\newcommand{\vH}{\boldsymbol{H}}
\newcommand{\vI}{\boldsymbol{I}}
\newcommand{\vJ}{\boldsymbol{J}}
\newcommand{\vK}{\boldsymbol{K}}
\newcommand{\vL}{\boldsymbol{L}}
\newcommand{\vM}{\boldsymbol{M}}
\newcommand{\vN}{\boldsymbol{N}}
\newcommand{\vO}{\boldsymbol{O}}
\newcommand{\vP}{\boldsymbol{P}}
\newcommand{\vQ}{\boldsymbol{Q}}
\newcommand{\vR}{\boldsymbol{R}}
\newcommand{\vS}{\boldsymbol{S}}
\newcommand{\vT}{\boldsymbol{T}}
\newcommand{\vU}{\boldsymbol{U}}
\newcommand{\vV}{\boldsymbol{V}}
\newcommand{\vW}{\boldsymbol{W}}
\newcommand{\vX}{\boldsymbol{X}}
\newcommand{\vY}{\boldsymbol{Y}}
\newcommand{\vZ}{\boldsymbol{Z}}
\newcommand{\vzero}{\boldsymbol{0}}
\newcommand{\vomega}{\boldsymbol{\omega}}
%\newcommand{\half}{\frac{1}{2}}
\newcommand{\thalf}{\frac{3}{2}}
\newcommand{\rot}{\nabla\times}
\newcommand{\divg}{\nabla\cdot}
\newcommand{\cE}{\mathcal{E}}
\newcommand{\conclu}[1]{\vspace{1em}\par\noindent\fbox{\parbox{0.97\linewidth}{#1}}\vspace{1em}}
\newcommand{\subentrynote}{$\bullet$}
\newcommand{\keypoint}[1]{\par\begin{siderules}\subentrynote\quad #1 \end{siderules}\par}
\newcommand{\fconclu}{\boxed}
\newcommand{\pair}[2]{#1 \, #2}
\newcommand{\barbelow}[1]{\underline{#1}}
\newcommand{\intn}[2]{\int #1 \,\mathrm{d} #2}
\newcommand{\intu}[3]{\int_0^{#1} #2 \,\mathrm{d} #3}
\newcommand{\intiu}[3]{\int_{-\infty}^{#1} #2 \, \rd{} #3}
\newcommand{\intui}[2]{\int_0^{\infty} #1 \,\mathrm{d} #2}
\newcommand{\intii}[2]{\int_{-\infty}^{\infty} #1 \,\mathrm{d} #2}
\newcommand{\intt}[2]{\int_0^\infty #1 \, \rd{} #2}
\newcommand{\intr}[2]{\int_{-\infty}^{\infty} #1 \, \rd{} #2}
\newcommand{\intbi}[3]{\int_{#1}^{\infty} #2 \, \rd{} #3}
\newcommand{\intab}[4]{\int_{#1}^{#2} #3 \, \rd{} #4}
\newcommand{\supint}[2]{\overbar{\int_{#1}^{#2}}}
\newcommand{\infint}[2]{\barbelow{\int_{#1}^{#2}}}
\newcommand{\intpostf}[1]{\, \rd{} #1}
\newcommand{\bfactor}[1]{e^{-#1/k_BT}}
\newcommand{\pbfactor}[1]{e^{#1/k_BT}}
\newcommand{\dn}[2]{#1^{\pare{#2}}}
\newcommand{\prodg}[1]{\pare{#1}^\times}
\newcommand{\muc}{S_\Omega}
\newcommand{\otw}{\, || \,}

\newcommand{\notion}{\emph}
\newcommand{\iP}{\mathcal{P}}
\newcommand{\eiP}{e^{-\iP}}
\newcommand{\iF}{\mathcal{F}}
\newcommand{\eiF}{e^{-\iF}}
\newcommand{\iG}{\mathcal{G}}

\newcommand{\rc}{r\cos\theta}
\newcommand{\rs}{r\sin\theta}
\newcommand{\sn}{\mathrm{sn}}
\newcommand{\cn}{\mathrm{cn}}
\newcommand{\rdn}{\mathrm{dn}}

\newcommand{\hankel}{H_p^{\pare{1}}}
\newcommand{\hankell}{H_p^{\pare{2}}}
\newcommand{\hhankel}{H_n^{\pare{1}}}
\newcommand{\hhankell}{H_n^{\pare{2}}}
\newcommand{\ber}{\text{ber}\,}
\newcommand{\bei}{\text{bei}\,}
\newcommand{\kker}{\text{ker}\,}
\newcommand{\kei}{\text{kei}\,}
\newcommand{\Ai}{\text{Ai}}
\newcommand{\Bi}{\text{Bi}}

\newcommand{\re}{\text{Re}\,}

\newcommand{\Fp}{F_\phi}
\newcommand{\Ep}{E_\phi}
\newcommand{\Fx}{F_x}
\newcommand{\FF}{\mathbf{F}}
\newcommand{\Ex}{E_x}

%\newcommand{\erf}{\mathrm{erf}}
\newcommand{\erfi}{\mathrm{erfi}}
\newcommand{\erfc}{\mathrm{erfc}}
\newcommand{\ehxs}[1]{e^{-\frac{#1^2}{2}}}
\newcommand{\dcol}[2]{\[ \left.#1 \hspace{1em}\right\vert\hspace{1em} #2 \]}
\newcommand{\titlegamma}{\texorpdfstring{$\Gamma$}{Gamma}}
\newcommand{\titleB}{\texorpdfstring{$B$}{B}}
\newcommand{\switch}[2]{\brac{#1 | #2}}
\newcommand{\SYSexeref}[1]{(舒幼生#1)}
\newcommand{\warning}[1]{\par\textit{注意:#1}}

% Computer Science Head
\usepackage{listings}
\usepackage{color}

\definecolor{dkgreen}{rgb}{0,0.6,0}
\definecolor{gray}{rgb}{0.5,0.5,0.5}
\definecolor{mauve}{rgb}{0.58,0,0.82}

\lstset{frame=tb,
  language=Java,
  aboveskip=3mm,
  belowskip=3mm,
  showstringspaces=false,
  columns=flexible,
  basicstyle={\small\ttfamily},
  numbers=none,
  numberstyle=\tiny\color{gray},
  keywordstyle=\color{blue},
  commentstyle=\color{dkgreen},
  stringstyle=\color{mauve},
  breaklines=true,
  breakatwhitespace=true,
  tabsize=3
}
\lstset{language=Java}
\newcommand{\snp}[1]{\lstinline!#1!}
\newcommand{\term}[2]{\textbf{#1(#2)}}

\title{数学物理方法 笔记}
\author{C.Z.}
\begin{document}
  \maketitle
  \tableofcontents
  \section*{落单的奇技淫巧}
  暂时...没有诶. 算上这个吧.
  \refl{
  \label{refl:abi}
  对于
  \[ \frac{a+bi}{a^2+b^2} \]
  形式的复数,可以将之写为
  \[ \frac{1}{r}e^{i\varphi}. \]
  其中$re^{i\varphi} = a+bi$.}
  \refl{
  \label{refl:xd2x}
  若方程中存在$x$与$x''$,可径在两侧乘以$x'$后积分.
  }
  此技巧在\emph{Classical Mechanics}的奇技淫巧导出机械能守恒中用到,在本书中的\ref{par:y2y}与\ref{sssec:gamma2pd}中有用到.
  \refl{
  \label{refl:et2}
  \[e^{-t^2} = \frac{1}{t} t e^{-t^2} = -\frac{1}{t} \cdot \half \edt{e^{-t^2}}.\]
  }
  上述式可用于求渐进展开. 与之相对应的有
  \[ f\pare{t} = \edt{t}f\pare{t} \]
  这样的换元积分.
  \refl{
  \label{refl:fact2}
  双阶乘
  \[
    \begin{cases}
      \pare{2n-1}!! = \pare{2n-1}!/\pare{2^{n-1} \pare{n-1}!},\\
      \pare{2n}!! = 2^n \pare{n}!.
    \end{cases}
  \]
  上式反过来用到是挺经常的.
  }
  \refl{
    \label{refl:dxd}
    \[ v\DD x \DD u - u \DD x \DD v = \DD x\pare{v\DD u - u\DD v}. \]
    参考Bessel函数的部分. 对微分方程由奇效.
  }
  \section{常微分方程}
  \subsection{引论}
  \label{ssec:ode_int}
  含有偏导数的微分方程称为\notion{偏微分方程},否则称为\notion{常微分方程}. 例子有LRC电路中的
  \[ L \eddton{I} + R \edt{I} + \frac{I}{C} = \edt{V}.\]
  \par
  微分方程的\notion{阶}为方程中导数的最大阶数.
  \par
  \notion{线性}微分方程者如
  \[ a_0 y + a_1 y' + a_2 y'' + a_3 y''' + \dots = b. \]
  \par
  $n$阶线性微分方程的解存在$n$个待定常数. 但对于非线性方程,情形可能并非如此,参见\ref{par:nonl}.
  \subsection{分离变量法}
  若可让方程一端仅存在未知函数与其导数之积,另一端仅存在自变量,则可以径直积分而求解,如
  \[ \edt{N} = -\lambda N \]
  可分离为
  \[ \frac{\rd{} N}{N} = -\lambda. \]
  \par
  又如
  \[ xy' = y+1 \]
  可分离为
  \[ \frac{y'}{y+1} = \frac{1}{x}. \]
  其解为过一点的直线系,而其等势面$u$可以由
  \[ u' = -\frac{1}{y'} = -\frac{x}{y+1} = -\frac{x}{u+1} \]
  求解. 可将上式分解为
  \[ u' \pare{u+1} = -x, \]
  故等势面为同心圆系.
  \par
  待定常数数目不等于阶数者如
  \[ y' = \sqrt{1-y^2}, \]
  可将之分离为
  \[ \frac{y'}{\sqrt{1-y^2}} = 1 \]
  得
  \[ y = \sin \pare{x+\alpha}. \]
  然而实际解仅仅是正弦函数之一部分,剩余部分取$1$或$-1$. 这导致经过$\pare{0,1}$的解有无数多个\label{par:nonl}.
  \refl{对于一阶方程,试将其化为$ f\pare{y}y' = g\pare{x} $的形式,则有
  \[ \iF\pare{y} = \iG\pare{x} + C. \]
  其中
  \[ \iF'=f,\quad \iG'=g. \]}
  \subsection{一阶线性微分方程}
  \notion{一阶线性微分方程}者如
  \[ y' + P\pare{x}y = Q\pare{x}. \]
  若$Q=0$则可径依分离变量法求解,即
  \[ \frac{y'}{y} = -P\pare{x}. \]
  若令
  \[ \iP'=P, \]
  有
  \[ y = \eiP. \]
  同时
  \begin{equation}
    \label{eq:yp}
    \pare{ye^\iP}' = e^\iP \pare{y'+\iP y}.
  \end{equation}
  得
  \[ \pare{ye^\iP}' = Qe^\iP. \]
  \refl{对于$y'+Py=Q$,可径解如下\[ y = \eiP\pare{ \int Q e^\iP + c }. \]}
  \begin{ex}
    \[ y' + \frac{6x}{1+x^2}\,y = \frac{2x}{1+x^2}. \]
    因而
    \[ P = \frac{6x}{1+x^2}, \quad Q = \frac{2x}{1+x^2}, \]
    \[ \iP = 3\ln\pare{1+x^2}, \quad e^\iP = \pare{1+x^2}^3. \]
    可求解$y$.
  \end{ex}
  \subsection{奇技淫巧}
  \begin{ex}[Bernoulli方程]
    \[ y' + Py = Qy^n \]
    可通过令
    \[ z = y^{1-n}, \]
    两边除以$y$有
    \[ \pare{\ln y}' + P = \frac{Q}{z}. \]
    注意
    \[ \ln z = \pare{1-n}\ln y. \]
    故
    \[ \frac{1}{1-n}\frac{z'}{z} + P = \frac{Q}{z}. \]
    即
    \[ \frac{1}{1-n}z' + Pz = Q. \]
  \end{ex}
  \refl{
    \label{refl:bern}
    对于
    \[ y'+P\pare{x}y = Q\pare{x}y^n, \]
    应作换元$z=y^{1-n}$,后有
    \[ z' + \pare{n-1}P\pare{x}z = \pare{n-1}Q\pare{x}. \]
  }
  \paragraph{全微分与积分因子}
  \[ y' + Py = Q \]
  可等价于
  \[ y' = \frac{[P]\pare{x,y}}{[Q]\pare{x,y}}. \]
  注意这里的$P$和$Q$与上式中不同. 或者更广义地,
  \[ [P]\dx + [Q]\dy = 0. \]
  若
  \[ \ddelon{[P]}{y} = \ddelon{[Q]}{x}, \]
  则有
  \[ \rd{} F = [P]\dx + [Q]\dy = 0. \]
  知$F\pare{x,y}=\const$为$y$与$x$之关系.\footnote{尽管这时常可以化简问题,但似乎不太值得花大功夫搜索$\unk$.}
  \refl{
    对于$P\pare{x,y}\rd{} x + Q\pare{x,y}\rd{} y = 0$,可以尝试两边同时乘以$\unk$,使得
    \[ \ddelon{[P]}{y} = \ddelon{[Q]}{x}, \]
    则可以求出$F\pare{x,y}=\const$之$F$.
  }
  \par
  对于
  \[ \pare{Py - Q}\dx + \dy  = 0, \]
  虽不满足全微分条件,然而欲拼凑
  \[ \unk\pare{Py - Q}\dx + \unk\dy  = 0, \]
  \[ [P] \dx + [Q] \dy = 0, \]
  而要求
  \[ \ddelon{\unk\pare{Py - Q}}{y} = \unk P = \ddelon{\unk}{x}, \]
  者,
  \[ \unk = e^\iP \]
  为是. 此处$\unk$称为\notion{积分因子}.
  \notion{势函数}$F\pare{x,y}的全微分已知,故可以直接积分$
  \[ F\pare{x,y} = \int \unk\pare{Py - Q}{y} \, \rd{} x+ \int \unk \, \rd{} y. \]
  先计算$F$关于$x$于$y=0$的变化,再计算关于$y$的变化,故
  \[ F = -\int e^\iP Q \, \rd{} x+ e^\iP y = \const. \]
  与\ref{eq:yp}可见一致.
  \begin{ex}
  \label{ex:bern}
    对于方程
    \[ 4y'y + 2y^2 + x = 0, \]
    除了将其化为Bernoulli方程外,还可以化为
    \[ 4y \dy + \pare{2y^2 + x} \dx = 0. \]
    积分因子可寻找如下
    \[ \eddon{}{x}\pare{4\unk y} = \eddon{}{y}\pare{2y^2\unk+x\unk}, \]
    得
    \[ \unk = e^x. \]
    故全微分形式为
    \[ 4ye^x\,\dy+\pare{2y^2e^x+xe^x}\dx=0, \]
    先由$x=0$对$y$积分,再对$x$积分,有
    \[ 2y^2e^x + xe^x -e^x + c = 0. \]
  \end{ex}
  \paragraph{齐次方程}如果
  \[ P\pare{x,y}\dx + Q\pare{x,y}\dy = 0 \]
  中$P$与$Q$为次数相同的\notion{齐次多项式},则可以有
  \newcommand{\yx}{\frac{y}{x}}
  \[ y' = f\pare{\yx}. \]
  令$y=xv$可求解之. 此对于结果$y$为$x$之隐函数的情形尤其有用.
  \refl{
    对于齐次方程
    \[ y' = f\pare{\yx} \]
    有
    \[ \iF\pare{\yx} = \log x + C. \]
    其中
    \[ \iF'\pare{v} = \frac{1}{f\pare{v}-v}. \]
  }
  \par
  亦可以在两边乘以
  \[ \unk = \frac{1}{xP+yQ} \]
  而令其成为齐次方程. 盖所求微分变为
  \[ \frac{P\dx+Q\dy}{xP+yQ} = \frac{f\pare{u}\dx+\dy}{x\pare{f\pare{u}+u}}. \]
  欲证其为齐次,需
  \[ \frac{1}{x} \ddelon{}{y}\frac{f}{f+u} = \ddelon{}{x}\frac{1}{x}\frac{1}{f+u}, \]
  若令
  \[ G = \frac{1}{f+u}, \]
  有
  \[ \frac{1}{x} \ddelon{}{y}fG = \ddelon{}{x}\frac{G}{x}. \]
  故需要
  \[ \ddelon{u}{y} \eddon{fG}{u} = x \ddelon{u}{x} \eddon{}{u}\frac{G}{x}, \]
  \[ f'G + fG' = -uG' - G .\]
  等价于
  \[ \frac{G'}{G}\pare{f+u} = f'+1. \]
  \paragraph{换元}换元对积分有用,对微分方程也有用,前述Bernoulli方程即是一例.
  \subsection{二阶常系数齐次方程}
  形如
  \[ a_2 y'' + a_1 y' + a_o y = 0 \]
  的方程,可引入微分算子$\DD = \edt{}$求解. 将上述方程写为
  \[ \pare{\DD-\alpha}\pare{\DD-\beta}y = 0. \]
  其中$\alpha$与$\beta$为$a_2 x^2 + a_1 x + a_0 = 0$的两根.
  \par
  $\DD$有本征值为$\alpha$的本征函数$e^{\alpha t}$,故
  \[ \pare{\DD-\alpha}y=0 \]
  有解
  \[ y=ce^{\alpha t}. \]
  还注意到如下的\emph{位移公式},
  \refl{
    \[\pare{\DD-\alpha} = e^{\DD\alpha t} \DD e^{-\alpha t}.\]
  }
  此外,$\DD$与非常数并不对易,
  \[ \DD x = x\DD + 1. \]
  可得
  \begin{equation}
    \label{eq:dpmx}
    \pare{\DD-x}\pare{\DD+x} = \DD^2-x^2+1.
  \end{equation}
  \par
  由上可得上述二阶方程有两特解(另一解将$\alpha$换成$\beta$),通解为两特解的线性组合
  \refl{
  \[ y=c_1 e^{\alpha t} + c_2 e^{\beta t}. \]
  }
  \par
  对于两根皆为$\alpha$的情况,有
  \[ \pare{\DD-\alpha}y = ce^{\alpha t}. \]
  代入位移公式后积分,有
  \refl{
  \[ y = \pare{c_1x + c_2}e^{\alpha t}. \]
  }
  若$\alpha$与$\beta$为一对共轭复数$r \pm i\omega$,则实部对应指数,虚部对应角频率
  \refl{
  \[ y=e^{rt}\pare{c_1 \sin \omega t + c_2 \cos \omega t} = ce^{rt}\sin\pare{\omega t + \varphi}. \]}
  \paragraph{阻尼-谐振子}阻尼-谐振子系统满足方程
  \label{par:hd}
  \[ my''=-ky-ly'. \]
  若令
  \[ \omega^2 = \frac{k}{m}, \quad b = \frac{l}{2m}, \]
  则特征方程的判别式为
  \[ \Delta = b^2 - \omega^2. \]
  若弹簧具有优势,则$\Delta < 0$;反之若阻尼占优势则$\Delta > 0$. 可分类为
  \begin{enumerate}
    \item $\Delta < 0$,弹簧优势,振幅指数衰减的振动;
    \item $\Delta = 0$,势均力敌,振幅呈伪指数衰减;
    \item $\Delta > 0$,阻尼优势,振幅指数衰减.
  \end{enumerate}
  对于LRC电路可有类似结论.
  \subsection{非齐次二阶常系数方程}
  对于
  \[ a_2 y'' + a_1 y' + a_0 = f\pare{t}, \]
  通解为
  \[ y = y_c + y_p. \]
  其中$y_c$为齐次方程的通解,$y_p$为上述方程的任一特解.
  \par
  对于指数\footnote{若为三角函数亦得径化虚指数而适用. 此处假定$\alpha \neq \beta \neq \omega$.}的$f\pare{t}=e^{\omega t}$,由位移公式
  \[ \pare{\DD-\alpha}\pare{\DD-\beta}y = e^{\alpha t} \DD e^{-\alpha t} u = e^{\omega t}, \]
  得
  \[ u = \frac{1}{\omega - \alpha}e^{\omega t} + c_1 e^{\alpha t}. \]
  第二项可视为通解之一部分. 再度(同理)积分有\footnote{若$a_2=1$,则分母径为$\omega^2 + a_1 \omega + a_2$.}
  \refl{
  \label{refl:yp}
  \[ y_p = \frac{1}{\pare{\omega - \alpha}\pare{\omega - \beta}}e^{\omega t}. \]}
  对于一般的伪指数$f\pare{t}=P_n\pare{t} e^{\omega t}$及其叠加,可依\emph{待定系数法}求解.
  \refl{
  \begin{equation*}
    y_p =
    \begin{dcases}
        e^{\omega t}Q_n\pare{t} &\omega \neq \alpha \neq \beta,\\
        te^{\omega t}Q_n\pare{t} &\omega = \alpha \neq \beta,\\
        t^2e^{\omega t}Q_n\pare{t} &\omega = \alpha = \beta.
    \end{dcases}
  \end{equation*}}
  注意到当右侧为纯多项式时,特解亦为纯多项式.
  \paragraph{受迫振动}参考\ref{par:hd},对于受迫振动,有
  \[ y'' + 2by' + \omega^2 y = F \sin \omega' t. \]
  可视为
  \[ y'' + 2by' + \omega^2 y = F e^{i\omega' t} \]
  之虚部. 由\rref{yp}可得
  \[ y = \frac{1}{-\omega'^2 + 2bi\omega' + \omega^2}e^{i\omega' t}. \]
  系数又可以写为(参考\rref{abi})
  \[ C = \frac{\pare{\omega^2-\omega'^2} - 2bi\omega'}{\pare{\omega^2-\omega'^2}^2 + \pare{2b\omega'}^2} = \frac{e^{-i\varphi}}{R}. \]
  故特解为
  \[ y_p = \frac{F}{R}\sin\pare{\omega't - \varphi}. \]
  此为\emph{稳态解},因通解在$t$增长时急剧减小.
  \par
  上式中
  \[ R = \sqrt{\pare{\omega^2-\omega'^2}^2 + \pare{2b\omega'}^2}. \]
  对于给定的本征频率$\omega$,当$\omega'^2=\omega^2-2b^2$时“阻碍”$R$最小. 对于其他的$\omega'$,阻碍都会(视$b$的大小)相对增大. 调频收音机通过这一原理将给定的频率分离出来.\footnote{参考Walter Lewin的电磁学课程.}
  \paragraph{借助Fourier级数求解}当右侧$f\pare{t}$为周期函数,可径将其Fourier展开并逐项求解后叠加. 逐项求解时可借助\rref{yp}.
  \subsection{特殊二阶方程}
  \paragraph{$y$消失的情形}在本节中不假设方程为线性.
  \refl{
  $y$消失时可设$y'=p$,$y''=p'$,后转为一阶方程.
  }
  \paragraph{$x$消失的情形}可以如上将方程转为$p$对$y$的方程,故得到一阶方程. 其中\[ p=y',\quad p'=\edt{p}=\eddon{p}{y}y'=p\eddon{p}{y}.\]求解后$p$对$y$的关系可知,再行求解一阶方程.
  \refl{
  $x$消失时可设$y'=p$,$y''=p'$,后两度求解一阶方程.
  }
  \begin{ex}
  例如对
  \[ 4y''+2y'^2+y=0, \]
  先有\footnote{$p$上的撇号表示对$y$的求导.}
  \[ 4pp' + 2p^2 + y=0. \]
  可化为Bernoulli方程
  \[ p' + \half p = -\frac{1}{4}yp^{-1}, \]
  并借助\rref{bern}或\eref{bern}求解,代换$z=p^2$,有
  \[ p^2 = \half\pare{y+1}-ce^y. \]
  故原方程化为超越方程
  \[ y'^2 = \half\pare{y+1}-ce^y. \]
  \end{ex}
  \paragraph{$y''+f\pare{y}=0$的形式}\label{par:y2y}此情形包含在上例中,但有更简单的解法.
  \refl{
    对于
    \[ y''+ f\pare{y} = 0, \]
    两侧同时乘以$y'$后积分有
    \[ \half y'^2 + \iF\pare{y} = \const. \]
    其中
    \[ \iF'=f. \]
  }
  如果将$y$看做位置,$F=-f$看做力,上述方程即动能定理
  \[ \half mv^2 = \int F \dx. \]
  \paragraph{Cauchy-Euler方程}
  对于方程\[ a_2 t^2 y'' + a_1 t y' + a_0 y = f\pare{t}, \]
  若为换元
  \[ t = e^z, \]
  则
  \[ y' = \edt{y} = \edt{z}\eddon{y}{z} = \frac{1}{t}\eddon{y}{z}, \]
  \[ y'' = \edt{}\edt{y} = \edt{z}\eddon{}{z}{\frac{1}{t}\eddon{y}{z}} = -\frac{1}{t^2}\eddon{y}{z}+\frac{1}{t}\eddzon{y}. \]
  代入有
  \[ a_2 y'' + \pare{a_1-a_2}y'+a_0y=f\pare{e^z}. \]
  \refl{
  方程\[ a_2 t^2 y'' + a_1 t y' + a_0 y = f\pare{t}, \]
  关于$z=\ln t$的求导有
  \[ a_2 y'' + \pare{a_1-a_2}y'+a_0y=f\pare{e^z}. \]
  可由此将$y$作为$z$的函数解出,再得到$y$对$t$的形式.
  }
  \paragraph{降次}对于
  \[ y''+f\pare{x}y'+g\pare{x}y=0, \]如果已知一解$u$,则设$y=uv$,有
  \begin{align*}
    y'' &= u''v + 2u'v' + uv'',\\
    f\pare{x}y' &= fu'v + fuv',\\
    g\pare{x}y &= guv, \\
    y'' + fy' + gy &= uv'' + \pare{2u'+fu}v' + \pare{u''+fu'+gu}v\\
    &= uv''+\pare{2u'+fu}v'\\
    &= 0.
  \end{align*}
  可径直分离变量求解.
  \refl{
    \label{refl:uv}
    已知\[ y''+f\pare{x}y'+g\pare{x}y=0 \]一解$u$,另一解$y=uv$满足
    \[ v' = \frac{c}{u^2} e^{-\iF}. \]
    其中\[ \iF'=f. \]
  }
  \begin{ex}
     对于\[ y'' - 2y' + y = 0, \]
     一解为$y=e^x$,又$\iF = -2x$,遂得
     \[ v'=c, \quad v' = cx, \]
     知另解为
     \[ y = xe^x. \]
  \end{ex}
  \subsection{Laplace变换}
  \subsection{Laplace变换用于微分方程}
  \subsection{卷积}
  \subsection{Dirac Delta函数}
  \subsection{Green函数}
  \section{特殊函数}
  \subsection{引论}
  特殊函数之间的联系或者恒等式可能对发现一些关系以及化简有帮助.
  \subsection{阶乘函数}
  对
  \[ \int_0^\infty e^{-\alpha x} \, \rd{} x = \frac{1}{\alpha} \]
  求导$n$次,有
  \[ \int_0^\infty x^n e^{-\alpha x} \, \rd{} x = \frac{n!}{\alpha^{n+1}}. \]
  特别地,
  \refl{
  \begin{equation}
    \label{eq:fact}
    \int_0^\infty x^n e^{-x} \, \rd{} x = n!.
  \end{equation}}
  \par
  对指数求导是一个计算含有指数的积分的常用技巧,如动力学理论中的积分
  \[ \int_0^\infty t^{2n} e^{-\alpha x^2} \, \rd{} x \]
  亦可以由(参考\ref{eq:ex2})下式导出.
  \refl{
  \label{refl:eax2}
  \[ \int_0^\infty e^{-\alpha x^2} \, \rd{} x = \half\sqrt{\frac{\pi}{\alpha}} \]}
  \subsection{\titlegamma 函数}
  \notion{$\Gamma$函数}定义为
  \[ \Gamma\pare{p} = \int_0^\infty x^{p-1}e^{-x}\, \rd{} x, \quad p>0. \]
  比照\ref{eq:fact}可知
  \begin{equation}
    \label{eq:gammarec}
    \Gamma\pare{n} = \pare{n-1}!.
  \end{equation}
  且对前式分部积分亦可以得到
  \[ \Gamma\pare{p+1} = p\Gamma\pare{p}. \]
  \subsection{\titlegamma 函数于负数}
  比照\ref{eq:gammarec}可知
  \[ \Gamma\pare{p} = \frac{1}{p}\Gamma\pare{p+1}. \]
  因此,$\Gamma$函数于负整数有奇点,但对其他负数可以定义.
  \subsection{\titlegamma 函数的一些公式}
  首级奇技淫巧谓
    \[ \pare{\int_0^\infty e^{-x^2} \, \rd{} x}^2 = \int_0^\frac{\pi}{2}\int_0^\infty e^{-r^2}\, \rd{} r \, \rd{} \theta = \frac{\pi}{4}, \]
  故
  \begin{equation}
    \label{eq:ex2}
    \int_0^\infty e^{-x^2} \, \rd{} x = \frac{\sqrt{\pi}}{2}.
  \end{equation}
  令$u=x^2$,
  \[ \half \int_0^\infty \sqrt{u}e^{-u} \, \rd{} u = \frac{\sqrt{\pi}}{2}. \]
  即
  \[ \Gamma\pare{\half} = \sqrt{\pi}. \]
  \par
  上述技巧亦可以倒过来使用,亦即
  \begin{equation}
    \label{eq:ux2}
    \intt{u^{p-1}e^{-u}}{u} = 2\intt{x^{2p-1}e^{-x^2}}{x},
  \end{equation}
  这在下文会有意想不到的应用.
  \par
  欲求出
  \[ \Gamma\pare{1-p}, \]
  可以考虑
  \begin{align*}
    \Gamma\pare{p}\Gamma\pare{1-p} &= \intt{\intt{u^{p-1}v^{-p}e^{-u-v}}{u}}{v}\\
    &=\intt{\intt{\frac{1}{u}\pare{\frac{u}{v}}^p e^{-u\pare{1+\frac{v}{u}}}}{u}}{v}\\
    &=\intt{\intt{t^{-p}e^{-u\pare{1+t}}}{u}}{t}\\
    &=\intt{\frac{t^{-p}}{1+t}}{t}.
  \end{align*}
  故有
  \refl{
  \label{refl:gammarefl}
  \[ \Gamma\pare{p}\Gamma\pare{1-p} = \frac{\pi}{\sin \pi p}. \]}
  亦可以由此得到
  \[ \Gamma\pare{\half} = \sqrt{\pi}. \]
  \par
  上述求$\Gamma\pare{p}\Gamma\pare{1-p}$的方法亦可以用于求
  \begin{align*}
    \Gamma\pare{p}\Gamma\pare{s-p} &= \intt{\intt{u^{p-1}v^{s-p-1}e^{-u-v}}{u}}{v} \\
    &= \intt{\intt{\frac{v^s}{uv}\pare{\frac{u}{v}}^{p}e^{u\pare{1+\frac{v}{u}}}}{u}}{v} \\
    &= \intt{\intt{\pare{tu}^{s-1}t^{-p}e^{-u\pare{1+t}}}{u}}{t} \\
    &= \intt{\frac{t^{q-1}}{\pare{1+t}^s}\intt{\pare{u\pare{1+t}}^{s-1}e^{-u\pare{1+t}}}{u\pare{1+t}}}{t} \\
    & = \Gamma\pare{p+q} \intt{\frac{t^{q-1}}{\pare{1+t}^{p+q}}}{t}.
  \end{align*}
  式中$p+q=s$,故有
  \refl{
  \[ \frac{\Gamma\pare{p}\Gamma\pare{q}}{\Gamma\pare{p+q}} = \intt{\frac{t^{q-1}}{\pare{1+t}^{p+q}}}{t}, \]}
  下文将看见,右端可以被记作$B\pare{p,q}$. 在此之前,亦可以引用\ref{eq:ux2}以相似于\ref{eq:ex2}的奇技淫巧(转化为极坐标)求之.
  \begin{align*}
    \Gamma\pare{p}\Gamma\pare{q} &= 4\intt{\intt{x^{2p-1}y^{2q-1}e^{-\pare{x^2+y^2}}}{x}}{y} \\
    &= 4 \intu{\frac{\pi}{2}}{\intt{\pare{\rc}^{2p-1}\pare{\rs}^{2q-1}e^{-r^2}}{r}}{\theta} \\
    &= 4 \intt{r^{2p+2q-2}e^{-r^2}}{r}\intu{\frac{\pi}{2}}{\pare{\cos\theta}^{2p-1}\pare{\sin\theta}^{2q-1}}{\theta} \\
    &= 2\Gamma\pare{p+q}\intu{\frac{\pi}{2}}{\pare{\cos\theta}^{2p-1}\pare{\sin\theta}^{2q-1}}{\theta}.
  \end{align*}
  故有
  \refl{
  \[ \frac{\Gamma\pare{p}\Gamma\pare{q}}{\Gamma\pare{p+q}}  = 2\intu{\frac{\pi}{2}}{\pare{\cos\theta}^{2p-1}\pare{\sin\theta}^{2q-1}}{\theta}. \]}
  \subsection{\titleB 函数}
  \notion{\titleB 函数}定义为
  \[ B\pare{p,q} = \intu{1}{x^{p-1}\pare{1-x}^{q-1}}{x}. \]
  有一堆关于它的恒等式,如
  \[ B\pare{p,q} = B\pare{q,p}. \]
  若令$x=\sin\theta$,还有
  \begin{align}
  \label{eq:bsc}
    B\pare{p,q} &= \intu{\frac{\pi}{2}}{\pare{\sin^2\theta}^{p-1}\pare{\cos^2\theta}^{q-1}2\sin\theta\cos\theta}{\theta} \\
    &= 2 \intu{\frac{\pi}{2}}{\pare{\sin\theta}^{2p-1}\pare{\cos\theta}^{2q-1}}{\theta},
  \end{align}
  立知
  \[ B\pare{p,q} = \frac{\Gamma\pare{p}\Gamma\pare{q}}{\Gamma\pare{p+q}}, \]
  \refl{
  \[ \frac{\Gamma\pare{p}\Gamma\pare{q}}{\Gamma\pare{p+q}} = \intu{1}{x^{p-1}\pare{1-x}^{q-1}}{x}. \]
  }
  故又有
  \[ B\pare{p,q} = \intt{\frac{t^{q-1}}{\pare{1+t}^{p+q}}}{t}. \]
  \subsection{与\titlegamma 函数的联系:加倍公式}
  在\ref{eq:bsc}中令$p=q$有
  \begin{align*}
    B\pare{p,p} &= 2 \intu{\frac{\pi}{2}}{\pare{\sin\theta\cos\theta}^{2p-1}}{\theta} \\
    &= \frac{2}{2^{2p-1}}\intu{\frac{\pi}{2}}{\pare{\sin2\theta}^{2p-1}}{\theta} \\
    &= \frac{1}{2^{2p-1}}\intu{\pi}{\pare{\sin\theta}^{2p-1}}{\theta} \\
    &= \frac{2}{2^{2p-1}}\intu{\frac{\pi}{2}}{\pare{\sin\theta}^{2p-1}}{\theta} \\
    &=\frac{1}{2^{2p-1}}B\pare{p,\frac{1}{2}}.
  \end{align*}
  立刻得到
  \[ \frac{\Gamma\pare{p}^2}{\Gamma\pare{2p}} = \frac{1}{2^{2p-1}}\frac{\Gamma\pare{p}\sqrt{\pi}}{\Gamma\pare{p+\half}}, \]
  即\notion{Legendre加倍公式}
  \refl{
  \label{refl:gamma2p}
  \[ \Gamma\pare{2p} = \frac{2^{2p-1}}{\sqrt{\pi}}\Gamma\pare{p}\Gamma\pare{p+\half}. \]}
  \begin{ex}
  \label{ex:sin2n}
  积分
  \begin{align*}
   \intu{\frac{\pi}{2}}{\sin^{2n}\theta}{\theta} &= \half B\pare{n+\half,\half} = \half\frac{\Gamma\pare{n+\half}\Gamma\pare{\half}}{\Gamma\pare{n+1}}\\
   &= \frac{\pi}{2^{2n}}\frac{\Gamma\pare{2n}}{\Gamma\pare{n}\Gamma\pare{n+1}} = \frac{\pi}{2^{2n}}C_{2n-1}^{n}.
  \end{align*}
  \end{ex}
  \subsection{应用:与单摆的联系}
  \label{sssec:gamma2pd}
  单摆(自水平起摆)的方程为
  \[ \rd{}ot{\theta} = -\half\sin\theta, \]
  可以两侧同乘$\dot{\theta}$后积分即得到
  \[ \dot{\theta}^2 = \cos{\theta}, \]
  是故
  \[ \rd{} t = \frac{\rd{} \theta}{\sqrt{\cos\theta}}, \]
  \[ \frac{T}{4} = \intu{\frac{\pi}{4}}{\frac{1}{\sqrt{\cos\theta}}}{\theta}, \]
  有
  \[ T = 4B\pare{\half, \frac{1}{4}}. \]
  \subsection{误差函数}
  误差函数在统计学中有所应用,其定义为\footnote{具体定义可能存在分歧.}
  \[ \erf\pare{x} = \frac{2}{\sqrt{\pi}}\intu{x}{e^{-t^2}}{t}. \]
  可以视为正态分布下方的面积.应当注意,标准正态分布为
  \[ f\pare{x} = \frac{1}{\sqrt{2\pi}} e^{-\frac{x^2}{2}}. \]
  其累积函数为
  \[ \Phi\pare{x} = \frac{1}{\sqrt{2\pi}} \intiu{x}{\ehxs{t}}{t} = \half + \half \erf\pare{\frac{x}{\sqrt{2}}}. \]
  上式亦可倒转而为
  \[ \erf\pare{x} = 2\Phi\pare{x\sqrt{2}}-1. \]
  \par 误差函数的若干性质如奇函数
  \[ \erf\pare{x} = -\erf\pare{-x}, \]
  以及无穷远处的极限如
  \[ \erf\pare{\infty} = 1. \]
  因此将误差函数视为正态分布下方面积的2倍似乎更为妥当.
  \par
  由上式立知
  \[ \erfc\pare{x} = \frac{2}{\sqrt{\pi}} \intbi{x}{e^{-t^2}}{t} = 1-\erf\pare{x}. \]
  \par
  还可定义虚误差函数
  \[ \erfi\pare{x} = \frac{2}{\sqrt{\pi}}\intu{x}{e^{t^2}}{t} = \erf\pare{ix}. \]
  \subsection{级数展开}
  注意\[ e^{-t^2} = 1-t^2+\frac{t^4}{2!}-\cdots, \]
  积分后有
  \refl{
    \[ \erf\pare{x} = \frac{2}{\sqrt{\pi}}\pare{x-\frac{x^3}{3}+\frac{x^5}{5\cdot2!}-\cdots}. \]
  }
  上式对于小的$x$较为有用.对于大的$x$,应当考虑
  \[ \erfc\pare{x} = \frac{2}{\sqrt{\pi}}\intbi{x}{e^{-t^2}}{t} \]
  的分部积分,只需要在每个步骤中作如下代换(参考\rref{et2})
  \[ e^{-t^2} = \frac{1}{t} t e^{-t^2} = -\frac{1}{t} \cdot \half \edt{e^{-t^2}}, \]
  对$\frac{1}{t}$微分便可以将分部积分的结果后项多乘一$\frac{1}{t^2}$,故得展开
  \refl{
    \[ \erfc\pare{x} = 1 - \erf\pare{x} \sim \frac{e^{-x^2}}{x\sqrt{\pi}}\pare{1-\frac{1}{2x^2} + \frac{1\cdot 3}{\pare{2x^2}^2} - \frac{1\cdot 3\cdot 5}{\pare{2x^2}^3} +\cdots }. \]
  }
  使用$\sim$是表示上述级数的渐进性. 上述级数对任意$x$发散,只有截断若干项才能得到有意义的结果.
  \paragraph{误差估计}若截断二项,则由分部积分的步骤知
  \[ \erfc\pare{x} =  \frac{e^{-x^2}}{x\sqrt{\pi}}\pare{1-\frac{1}{2x^2}} + O\pare{\intbi{x}{t^{-4}e^{-t^2}}{t}}. \]
  余项可借助\rref{et2}估计为
  \[ \intbi{x}{t^{-4}e^{-t^2}}{t} = \intbi{x}{t^{-5}\pare{te^{-t^2}}}{t} < \frac{1}{x^5}\intbi{x}{\pare{te^{-t^2}}}{t} = O\pare{\frac{e^{-x^2}}{x^5}}. \]
  对于大的$x$远小于截断项,故可称上开级数为\emph{渐进级数}.
  \paragraph{对数积分}定义对数积分如下
  \newcommand{\Li}{\mathrm{Li}}
  \[ \Li\pare{x} = \intu{x}{\frac{1}{\log t}}{t}. \]
  可以通过分部积分将上述积分渐进展开为
  \begin{align*}
    \Li\pare{x} &= \intu{x}{\frac{1}{\log t}}{t} \\
    &= \intu{x}{\pare{\edt{}t}\frac{1}{\log t}}{t} \\
    &= \frac{x}{\log x} + \intu{x}{\pare{\edt{}t}\frac{1}{\log^2 t}}{t}\\
    &= \frac{x}{\log x} \pare{\sum \frac{n!}{\log^n x}}.
  \end{align*}
  可以截断为
  \[ \frac{t}{\log t} + \frac{t}{\log^2 t} \sim \frac{t}{\log t - 1}. \]
  分母上的$1$称为\emph{Legendre常数}\footnote{这恐怕是一个笑话.也许它应该是1.08366.}.
  \subsection{Laplace方法:Stirling公式}
  试图计算积分
  \dcol{\intab{a}{b}{e^{-s\Phi\pare{x}}\psi\pare{x}}{x}}
  {\intt{e^{s\log x - sx}\frac{1}{x}}{x}}
  时,可以尝试寻找$\Phi\pare{x}$的最小值点$x_0$,并考察其附近的情形,有
  \dcol{e^{-s\Phi\pare{x_0}}\intab{a}{b}{e^{-s\frac{\Phi''\pare{x_0}}{2}\pare{x-x_0}^2\varphi\pare{x}}\psi\pare{x}}{x}}
  {e^{-s}\intt{e^{-s\half\pare{x-1}^2\varphi\pare{x}}\frac{1}{x}}{x}.}
  其中
  \dcol{\varphi\pare{x} = 1+O\pare{x-x_0}}
  {\varphi\pare{x} = 1-\frac{2}{3}\pare{x-1}+\cdots.}
  可以令
  \dcol{y=\pare{x-x_0}\sqrt{\varphi\pare{x}}}
  {y=\pare{x-1}\sqrt{1-\frac{2}{3}\pare{x-1}+\cdots},}
  则$\dy$与$\dx$几乎可径直互换,惟应注意
  \dcol{\frac{\dy}{\dx} = 1+O\pare{y}}
  {\frac{\dy}{\dx} = 1 - \frac{2}{3}y+\cdots.}
  此时可以试图做换元积分,而$x$仅在$x_0$附近得保证与$y$有一一对应,上述$\dy$与$\dx$的互换也仅在有限范围内成立. 譬如右侧的
  \[ \varphi\pare{x} = \frac{2\pare{x-\log x -1}}{\pare{x-1}^2} \]
  在$x=0$附近便已发散.故仅仅估计$x_0$附近区域的积分,此时对应的$y$应当从负数穿过零变为正数,有
  \begin{align*}
  &\intab{x_0-}{x_0 +}{e^{-s\Phi\pare{x}}\psi\pare{x}}{x} = \\ & \psi\pare{x_0}\intab{-}{+}{e^{-s\frac{\Phi''\pare{x_0}}{2}y^2}}{y} + O\pare{\intab{-}{+}{e^{-s\frac{\Phi''\pare{x_0}}{2}y^2}\left\vert y\right\vert}{y}}
  \end{align*}
  \hrulefill
  \begin{align*}
  &\intab{1-}{1+}{e^{-s\half\pare{x-1}^2\varphi\pare{x}}\frac{1}{x}}{x} = \\& \psi\pare{x_0}\intab{-}{+}{e^{-\half sy^2}}{y} + O\pare{\intab{-}{+}{e^{-\half sy^2}\left\vert y\right\vert}{y}}
  \end{align*}
  第一项积分可求出为
  \dcol{s^{-\half}\pare{\frac{2\pi}{\Phi''\pare{x_0}}}^{\half}+O\pare{e^{-\delta s}}}
  {\pare{\frac{2\pi}{s}}^{\half}+O\pare{e^{-\delta s}}.}
  第二项(余项)为$O\pare{1/s}$. 上开区间变更忽略掉的项为
  \[ \int e^{-s\Phi\pare{x}}\psi\pare{x} \]
  关于$s$指数递减,故忽略之.
  \par
  经由上述步骤可得
  \[ \intt{e^{s\log x - sx}\frac{1}{x}}{x} \sim e^{-s}\frac{\sqrt{2\pi}}{s^{\half}}. \]
  故有
  \begin{align*}
  \Gamma\pare{s} =& \intt{e^{s\log x - x}\frac{1}{x}}{x}\\ =& \intt{e^{s\log sx - sx}\frac{1}{sx}}{sx} \\=& e^{s\log s}\intt{e^{s\log x - sx}\frac{1}{x}}{x}\\
  =& \sqrt{2\pi} e^{s\log s-s}s^{-\half}.
  \end{align*}
  稍加功夫,便可得到如下的\emph{Stirling公式}(带有渐进项).
  \refl{
    \label{refl:st}
    \[ p! = \Gamma\pare{p+1} = p^pe^{-p}\sqrt{2\pi p}\pare{1+\frac{1}{12p}+\frac{1}{288p^2}+\cdots}. \]
  }
  上述渐进依然发散. 相对误差随$p$的增加而减小,绝对误差则增大.
  \paragraph{Wallis乘积}Wallis乘积的解法之一为
  \begin{align*}
    \Pi\pare{\frac{2n}{2n-1}\cdot\frac{2n}{2n+1}} &= \frac{1}{2k+1}\cdot\frac{2^{4k}\pare{k!}^4}{\pare{\pare{2k}!}^2} \\
    &\sim \frac{\pi}{2}.
  \end{align*}
  第一行可以由如下技巧求得
  \newcommand{\x}{\times}
  \begin{align*}
  &\frac{2\x2\x4\x4\x6\x6}{\pare{1\x3}\x\pare{3\x5}\x\pare{5\x7}} =\\ &\frac{2\x2\x4\x4\x6\x6}{\pare{1\x2\x3\x4}\x\pare{3\x4\x5\x6}\x\pare{5\x6\x7\x8}}\\ &\cdot\pare{2\x4\x4\x6\x6\x8}.
  \end{align*}
  \paragraph{高阶近似}前述Laplace方法稍加改动可获得更精确的Stirling近似. 欲得到$1/288p^2$项,需在$\varphi$与$\psi$的Taylor展开中保留至$t^4$项,即在
  \[ \intt{e^{-s\pare{t-\log t -1}}\frac{1}{t}}{t} \]
  中,注意到
  \begin{align*}
  & t-\log t-1\\
  =& \half \pare{t-1}^2\pare{1-\frac{2 \pare{t-1}}{3}+\frac{1}{2} \pare{t-1}^2-\frac{2}{5} \pare{t-1}^3+\frac{1}{3} \pare{t-1}^4}. 
  \end{align*}
  以及
  \begin{align*}
    &\frac{1}{t} = 1-\pare{t-1}+\pare{t-1}^2-\pare{t-1}^3+\pare{t-1}^4.
  \end{align*}
  令$u=t-1$,将积分近似为
  \[ \intab{0-}{0+}{e^{-\half su^2\pare{1-\frac{2 u}{3}+\frac{1}{2} u^2-\frac{2}{5} u^3+\frac{1}{3} u^4}}\pare{1-u+u^2-u^3+u^4}}{u}. \]
  复令
  \[ y=u\sqrt{1-\frac{2 u}{3}+\frac{1}{2} u^2-\frac{2}{5} u^3+\frac{1}{3} u^4} \]
  并截断到4阶项,有
  \[ y= u\pare{1-\frac{u}{3}+\frac{7}{36}u^2-\frac{73}{540}u^3+\frac{1331}{12960}u^4}. \]
  于是到了最精彩也是最困难的一步,对$y$求逆(保留4阶项),有
  \[ u=y\pare{1+\frac{y}{3}+\frac{1}{36}y^2-\frac{1}{270}y^3+\frac{1}{4320}y^4}. \]
  欲换元积分,则需要$u$对$y$的导数,
  \[ \eddon{u}{y} = 1+\frac{2}{3}y+\frac{1}{12}y^2-\frac{2}{135}y^3+\frac{1}{864}y^4. \]
  于是有(截断后)
  \[ \pare{1-\frac{2 u}{3}+\frac{1}{2} u^2-\frac{2}{5} u^3+\frac{1}{3} u^4} \eddon{u}{y} = 1-\frac{y}{3}+\frac{1}{12}y^2-\frac{2}{135}y^3+\frac{1}{864}y^4. \]
  因此原来的积分为
  \[ \intab{0-}{0+}{e^{-\half s y^2}\pare{1-\frac{y}{3}+\frac{1}{12}y^2-\frac{2}{135}y^3+\frac{1}{864}y^4}}{y}. \]
  借助\rref{eax2},并注意奇数次幂积分为$O\pare{e^{-\delta s}}$,有
  \begin{align*}
  \intab{0-}{0+}{e^{-\half s y^2}}{y} &= \sqrt{\frac{2\pi}{s}}, \\
  \intab{0-}{0+}{e^{-\half s y^2}y^2}{y} &= \sqrt{2\pi}s^{-\frac{3}{2}}, \\
  \intab{0-}{0+}{e^{-\half s y^2}y^2}{y} &= 3\sqrt{2\pi}s^{-\frac{5}{2}}.
  \end{align*}
  故上述积分近似为
  \[ \sqrt{\frac{2\pi}{s}}\pare{1+\frac{1}{12s}+\frac{1}{288s^2}}, \]
  可以得到\rref{st}的近似.
  \subsection{椭圆积分与椭圆函数}
  \paragraph{椭圆的周长}欲计算椭圆
  \begin{equation*}
    \begin{cases}
      x = \cos \theta\\
      y = b \sin \theta
    \end{cases}
  \end{equation*}
  的周长,注意到$e^2=1-b^2$,
  \[ \rd{} s^2 = \rd{} x^2 + \rd{} y^2 = \pare{\cos^2\theta+b^2\sin^2\theta}\rd{}\theta^2. \]
  故
  \[ L=\intu{2\pi}{\sqrt{1-e^2\sin^2\theta}}{\theta} = 4\intu{\frac{\pi}{2}}{\sqrt{1-e^2\sin^2\theta}}{\theta}. \]
  式中的积分为完全椭圆积分. 若不欲计算全椭圆周长而仅需部分周长,则积分至对应的$\theta$即可.注意$\theta$并非对应点与轴之间的夹角,而是其anomaly. \par
  定义\emph{椭圆积分}的\emph{Legendre形式}如下:
  \refl{ 
  \begin{align*}
  &\Fp\pare{\phi, k} = \intu{\phi}{\frac{1}{\sqrt{1-k^2\sin^2\theta}}}{\theta},\quad & 0\le k \le 1,\\
  &\Ep\pare{\phi, k} = \intu{\phi}{\sqrt{1-k^2\sin^2\theta}}{\theta},\quad & 0\le k\le 1.
  \end{align*}
  }
  如果作换元$t=\sin\theta$,并令$x=\sin\phi$,可以将积分转化为\emph{Jacobi形式}:
  \refl{
    \begin{align*}
  &\Fx\pare{x, k} = \intu{x}{\frac{1}{\sqrt{1-t^2}\sqrt{1-k^2t^2}}}{t},\quad & 0\le k \le 1,\\
  &\Ex\pare{x, k} = \intu{x}{\frac{\sqrt{1-k^2t^2}}{\sqrt{1-t^2}}}{t},\quad & 0\le k\le 1.
  \end{align*}
  }
  对于积分上限恰好为$\pi/4$(或$1$)的情况,可以定义\emph{完全椭圆积分}:
  \refl{
  \begin{align*}
    K\pare{k} &= \Fp\pare{\frac{\pi}{2},k}=\Fx\pare{1,k},\\
    E\pare{k} &= \Ep\pare{\frac{\pi}{2},k}=\Ex\pare{1,k}.
   \end{align*}
  }
  由周期性和对称性还可以得到
  \begin{align*}
    \Ep\pare{\phi_1,k} + \Ep\pare{\phi_2,k} &= \intab{-\phi_1}{\phi_2}{\sqrt{1-k^2\sin^2\theta}}{\theta}.\\
    \Ep\pare{n\pi\pm\phi,k} &= 2nK\pm\Ep\pare{\phi,k}.\\
  \end{align*}
  对$\Fp$有完全类似的公式.
  \paragraph{与单摆的联系}参考\ref{sssec:gamma2pd},对于起始摆角$\alpha$,有
  \[ \dot{\theta}^2=\frac{2g}{l}\pare{\cos\theta-\cos\alpha}. \]
  直接积分,有
  \[ \intu{\alpha}{\frac{1}{\sqrt{\cos\theta-\cos\alpha}}}{\theta} = \sqrt{\frac{2g}{l}}\frac{T_\alpha}{4}, \]
  应用倍角公式并换元,
  \[ \cos\theta-\cos\alpha = 2\sin^2\frac{\alpha}{2}-2\sin^2\frac{\theta}{2} =  2k^2-2\sin^2\psi. \]
  积分可以写为
  \begin{align*}
    \sqrt{2}\intu{\frac{\alpha}{2}}{\frac{1}{\sqrt{k^2-\sin^2\psi}}}{\psi} &= \sqrt{2}\intu{k}{\frac{1}{\sqrt{k^2-x^2}\sqrt{1-x^2}}}{x}\\
    &= \sqrt{2}\intu{1}{\frac{1}{\sqrt{1-t^2}\sqrt{1-k^2t^2}}}{t}\\
    &= \sqrt{2}K\pare{\sin\frac{\alpha}{2}}.
  \end{align*}
  立即有
  \[ T_\alpha = 4\sqrt{\frac{l}{g}}K\pare{\sin\frac{\alpha}{2}}. \]
  \paragraph{级数展开}注意到
  \[ \frac{1}{\sqrt{1-x^2}} = \sum \frac{\pare{2n-1}!!}{2^n n!}x^{2n}, \]
  再借助\eref{sin2n}以及\rref{fact2},有
  \begin{align*}
  K\pare{k} &= \pi \sum \frac{\pare{2n-1}!!}{2^n n!}\frac{\pare{2n-1}!}{2^{2n}n!\pare{2n-1}!}k^{2n}\\
  &= \frac{\pi}{2}\sum\pare{\frac{\pare{2n-1}!!}{\pare{2n}!!}}^2k^{2n}.
  \end{align*}
  立知
  \begin{align*}
  T_\alpha &= 2\pi\sqrt{\frac{l}{g}}\pare{1+\pare{\half}^2\sin^2\frac{\alpha}{2}+\pare{\frac{1\cdot 3}{2\cdot 4}}^2\sin^4\frac{\alpha}{2}+\cdots} \\
  &= 2\pi\sqrt{\frac{l}{g}}\pare{1+\frac{\alpha^2}{16}+\cdots} .
  \end{align*}
  \paragraph{椭圆函数}比照
  \[ u = \intu{x}{\frac{1}{\sqrt{1-t^2}}}{t} = \sin^{-1}x, \]
  定义
  \[ u = \intu{x}{\frac{1}{\sqrt{1-t^2}\sqrt{1-k^2t^2}}}{t} = \sn^{-1}\,x. \]
  因此有
  \[ \sn\, u = \sin \phi. \]
  相似地定义
  \[ \cn\, u = \cos \phi = \sqrt{1-\sn\, u^2}, \]
  \[ \rdn\, u = \eddon{\phi}{u} = \frac{1}{\rd{} u/\rd{} \phi} = \sqrt{1-k^2\sn^2\,u}. \]
  注意由于$\sn$是某个角的$\sin$,上述诸函数均为周期函数,周期随$k$有变.
  \section{微分方程的级数解}
  \subsection{引论}
  本章将考察各种线性的二阶方程,其系数可能为$x$的函数,因此解析解的获得可能有困难. 这是激动人心的一章,尤其是在读了《量子力学概论》之后.
  \begin{ex}
    考虑方程
    \[ y' = 2xy. \]
    设
    \[ y = \sum a_n x^n, \]
    上述方程变为
    \[ \pare{n+2}a_{n+2} = 2a_n, \]
    且
    \[ a_\text{odd} = 0. \]
    得
    \[ a_{2n} = \frac{a_0}{n!}, \]
    \[ y = ce^{x^2}. \]
  \end{ex}
  \refl{
  \label{refl:subst}
  可以遵守如\tbref{subst}所示的替换法则.牢记当$n<0$时定义$a_n=0$.
  }
  \begin{table}[!ht]
  \centering
    \begin{tabular}{|c|c|c|c|}
    \hline
    因子 & $1$ & $x$ & $x^2$ \\
    \hline
    $y$ & $a_n$ & $a_{n-1}$ & $a_{n-2}$ \\
    \hline
    $y'$ & $\pare{n+1}a_{n+1}$ & $na_n$ & $\pare{n-1}a_{n-1}$ \\
    \hline
    $y''$ & $\pare{n+1}\pare{n+2}a_{n+2}$ & $n\pare{n+1}a_{n+1}$ & $\pare{n-1}na_n$ \\
    \hline
    \end{tabular}
    \caption{替换法则}
    \label{table:subst}
  \end{table}
  \subsection{Legendre方程}
  如下的\emph{Legendre方程}
  \[ \pare{1-x^2}y'' - 2xy' + l\pare{l+1}y = 0, \]
  经过\rref{subst}的替换,得到
  \[ \pare{n+1}\pare{n+2}a_{n+2} + \pare{l^2+l-n^2-n}a_n = 0. \]
  注意到
  \[ l^2-n^2+l-n = \pare{l-n}\pare{l+n+1}, \]
  得
  \[ a_{n+2} = -\frac{\pare{l-n}\pare{l+n+1}}{\pare{n+1}\pare{n+2}}a_n. \]
  立刻有
  \begin{align*}
    y = & a_0\pare{1-\frac{l\pare{l+1}}{2!}x^2 + \frac{l\pare{l+1}\pare{l-2}\pare{l+3}}{4!}x^4-\cdots}\\
    & + a_1\pare{x-\frac{\pare{l-1}\pare{l+2}}{3!}x^3 + \frac{\pare{l-1}\pare{l+2}\pare{l-3}\pare{l+4}}{5!}x^5-\cdots}.
  \end{align*}
  若级数无限延续,则当$x^2=1$时其发散.
  \paragraph{Legendre多项式}当$l$为非负整数时,上述展开中奇数幂或偶数幂部分之一将被截断. 若$l$为非负偶数,则上式偶数幂部分被截断,奇数幂无限延伸,故可设$a_1=0$而清除之. 故此时对应的$y$为偶数幂多项式,当$l$为奇数时则对应奇数幂多项式. 如此产生的多项式序列,若选择首项系数使得$x=1$时$y=1$,则称之为\emph{Legendre多项式}. 前几个Legendre多项式为
  \[ P_0 = 1, \qquad P_1 = x, \qquad P_2 = \half\pare{3x^2-1}.\]
  注意当$l$为偶数或奇数时,$P_l$为偶函数或奇函数,故
  \[ P_l\pare{-1} = \pare{-1}^l. \]
  还有,此处可以应用\rref{uv}获得\emph{第二类Legendre}函数,其中
  \[ f\pare{x} = -\frac{2x}{1-x^2},\quad \iF\pare{x} = \log\pare{1-x^2},\quad \eiF = \frac{1}{1-x^2}. \]
  \paragraph{特征值问题}
  Legendre多项式可视为算子
  \[ \FF\pare{\DD} = \pare{x^2-1}\DD^2-2x\DD = \DD\pare{x^2-1}\DD \]
  的本征值为$l\pare{l+1}$的本征函数.
  \refl{
    \[ \DD f\DD P_l = l\pare{l+1}P_l. \]
  }
  \subsection{Leibniz求导法则}
  \refl{
    \[ \DD^n\pare{fg} = \sum C_n^r \DD^r f \DD^{n-r} g. \]
    特别地,
    \[ [\DD^n, x] = n\DD^{n-1}. \]
  }
  \begin{ex}
  \label{ex:DfD}
    \begin{alignat*}{2}
    \DD^{l+1}f &= f \DD^{l+1} + \pare{l+1}&&f'\DD^{l} + l\pare{l+1}\frac{f''}{2}\DD^{l-1} + \cdots
    \end{alignat*}
    若$f$为首一二次多项式,$f''=2$,则
    \[ [\DD^{l+1}, f] = \pare{l+1}f'\DD^l + l\pare{l+1}\DD^{l-1}. \]
  \end{ex}
  \subsection{Rodrigues公式}
  令$f = x^2-1$,由\eref{DfD},
  \begin{align*}
    \DD f\DD \DD ^lf^l &= \DD \pare{\DD ^{l+1}f - \pare{l+1}f'\DD ^l - l\pare{l+1}\DD ^{l-1}}f^l \\
    &= \DD ^{l+2}ff^l - 2\pare{l+1}\DD x\DD ^lf^l - l\pare{l+1}\DD ^lf^l.
  \end{align*}
  第一项
  \begin{align*}
    \DD ^{l+2}ff^l &= \DD ^{l+1}\pare{l+1}f'f^l \\
    &= 2\pare{l+1}\DD ^{l+1}xf^l \\
    &= 2\pare{l+1}x\DD ^lf^l + 2\pare{l+1}\pare{l+1}\DD ^lf^l.
  \end{align*}
  第二项
  \begin{align*}
    -2\pare{l+1}\DD x\DD ^lf^l &= -2\pare{l+1}\pare{x\DD +1}\DD ^lf^l,
  \end{align*}
  知
  \[ \DD f\DD \DD ^lf^l = l\pare{l+1}\DD ^lf^l. \]
  再注意到
  \[ \DD ^lf^l = \DD ^l\pare{x+1}^l\pare{x-1}^l, \]
  借助Leibniz法则,$x=1$时仅有
  \[ \pare{x+1}^l\DD ^l\pare{x-1}^l = 2^l l!. \]
  一项非零(因$x-1$的因子已经被悉数微分).
  故
  \refl{
    \label{refl:pl}
    \[ P_l = \frac{1}{2^l l!}\DD ^l\pare{x^2-1}^l. \]
  }
  \paragraph{Rodrigues公式之父}设$p$为二次多项式,$q$为一次多项式,$w$(可能不是有理函数)满足
  \begin{equation}
    \label{eq:w}
    \pare{wp}' = wq.
  \end{equation}
  $w$的一阶导数可以写为
  \[ \frac{w'}{w} = \frac{q-p'}{p}, \]
  二阶导数可以写为
  \[ \frac{w''}{w} = \frac{\pare{q-p'}\pare{q-2p'}+q'-p''}{p}. \]
  \par
  现在定义
  \[ \FF = \frac{1}{w}\DD w, \]
  以及
  \[ y_n = \FF^np^n = \frac{1}{w}\DD^nwp^n. \]
  由一阶导数的公式,有
  \[ p\DD\pare{wp^n} = wp^n\pare{\pare{n-1}p'+q}. \]
  对上式微分$n+1$次并注意$p$为二次而$q$为一次,得到
  \begin{alignat*}{2}
    p\FF^{n+2}p^n + \pare{n+1}p' & \FF^{n+1}p^n + && \frac{n\pare{n+1}}{2}  p''\FF^np^n \\
    = \pare{\pare{n-1}p'+q} & \FF^{n+1}p^n + &&\pare{\pare{n+1}\pare{\pare{n- 1}p''+q'}} \FF^np^n.
  \end{alignat*}
  合并$\FF p$的同类项,有
  \[ p\FF^{n+2}p^n + \pare{2p'-q}\FF^{n+1}p^n - \pare{\frac{n^2-n-2}{2}p''+\pare{n+1}q'}\FF^np^n = 0. \]
  或者
  \[ p\FF^2y_n + \pare{2p'-q}\FF y_n - \pare{\frac{n^2-n-2}{2}p''+\pare{n+1}q'}y_n = 0. \]
  $\FF$可以消去如下
  \begin{equation*}
    \FF = \DD + \frac{w'}{w} = \DD + \frac{q-p'}{p}.
  \end{equation*}
  $\FF^2$可以消去为
  \begin{align*}
    \FF^2 &= \frac{1}{w}\DD^2w = \DD^2+2\frac{w'}{w}\DD + \frac{w''}{w} \\
           &= \DD^2 + \frac{2\pare{q-p'}}{p}\DD + \frac{1}{p}\pare{\pare{q-p'}\pare{q-2p'}+q'-p''}.
  \end{align*}
  代回得
  \[ py''_n + qy'_n - \pare{\frac{n^2-n}{2}p''+nq'}y_n = 0. \]
  因此,有
  \refl{
    若$p$为二次多项式,$q$为一次多项式,则
    \[ y_n = \FF^n p^n \]
    是$p\DD^2+q\DD$本征值为
    \[ \frac{n^2-n}{2}p''+nq' \]
    的函数. 其中
    \[ \log w = \int \frac{q}{p} - \log p. \]
  }
  可以看出,前述的Rodrigues公式是这一反射的特例.
  \paragraph{正交性}$P_l$与所有次数小于$l$的多项式正交,因为
  \[ \intab{-1}{1}{P_l}{x} = 0, \]
  而
  \[ 2^ll!\intab{-1}{1}{x^mP_l}{x} = \left. x^mD^{l-1}\pare{x^2-1}^l\right\vert_{x=-1}^{1} - \int x^{m-1}\cdots.  \]
  第一项由对称性知为零,第二项由归纳假设为零.
  \subsection{生成函数与递归关系}
  令
  \[ \Phi\pare{x,h} = \pare{1-2xh+h^2}^{-\half}, \]
  有(参考《电动力学导论》)
  \refl{
    \[\pare{1-2xh+h^2}^{-\half} = \sum P_l\pare{x}h^l.\]
  }
  Taylor展开$1/\sqrt{1-x}$可以验证前若干项目. 令人信服的证明则需要先验证当$x=1$,有
  \[ \Phi\pare{x,h} = \frac{1}{1-h} = 1+h+h^2+\cdots, \]
  故系数多项式在$x=1$时均为$1$. 再注意到
  \[ \pare{1-x^2}\Phi_{xx}-2x\Phi_x+h\pare{h\Phi}_{hh} = 0, \]
  其中下标表示求导. 因而(参考\rref{subst})
  \[ \pare{1-x^2}P'' -2xP' + l\pare{l+1}P = 0. \]
  从生成函数还可以得到相当多的递推关系,从而得以在已知次数小的Legendre多项式的情况下推出次数大的多项式. 例如,借助
  \[ \pare{1-2xh+h^2}\Phi_h = \pare{x-h}\Phi, \]
  即
  \[ h\Phi_h = xh\pare{h\Phi}_h + xh^2\Phi_h-h^2\pare{h\Phi}_h, \]
  有
  \refl{
    \label{refl:lpl1}
    \[ lP_l = \pare{2l-1}xP_{l-1} - \pare{l-1}P_{l-2}. \]
  }
  \begin{ex}
    \label{ex:xpp}
    \begin{align*}
      \frac{\int xP_{n-1}P_n}{\int P_{2n-1}^2} &= \frac{2n-1}{2}\pare{\int\frac{nP_n+\pare{n-1}P_{n-2}}{2n-1}P_n} \\
      &= \frac{n}{2n+1}.
    \end{align*}
    同理有(更简单的方法为注意奇函数乘偶函数为奇函数)
    \[ \int xP_n^2 = 0. \]
  \end{ex}
  又借助
  \[ \pare{x-h}\Phi_x = h\Phi_h, \]
  有(书中证明正交关系时有用)
  \refl{
    \label{refl:lpl2}
    \[ lP_l = xP'_l - P'_{l-1}. \]
  }
  对\rref{lpl1}微分,有
  \[ lP'_l = \pare{2l-1}P_{l-1} + \pare{2l-1}xP'_{l-1}-\pare{l-1}P'_{l-2}. \]
  再注意\rref{lpl2}可以写为
  \[ \pare{l-1}^2P_{l-1} = \pare{l-1}xP'_{l-1}-\pare{l-1}P'_{l-2}, \]
  从而
  \[ \pare{l^2-\pare{l-1}^2}P_{l-1} + \pare{2l-1}xP'_{l-1}-\pare{l-1}P'_{l-2} = l^2P_{l-1}+lxP'_{l-1}. \]
  \refl{
    \label{refl:lpl3}
    \[ lP_{l-1} = P'_l - xP'_{l-1}. \]
  }
  在\rref{lpl2}两侧乘$x$并与\rref{lpl3}相减,得到
  \[ lP_{l-1} - lxP_l = \pare{1-x^2}P'_l. \]
  亦即
  \refl{
    \label{refl:lpl4}
    \[ \pare{1-x^2}P'_l = lP_{l-1}-lxP_l. \]
  }
  在\rref{lpl3}两侧乘$x$并与\rref{lpl2}相减,得到
  \[ lxP_{l-1} - lP_l= \pare{1-x^2}P'_{l-1}. \]
  亦即
  \refl{
    \label{refl:lpl5}
    \[ \pare{1-x^2}P'_{l-1} = lxP_{l-1}-lP_l. \]
  }
  在\rref{lpl4}两侧微分,有
  \[ -DfDP_l = lP'_{l-1} - lxP'_l - lP_l. \]
  由\rref{lpl2},
  \[ lxP'_l = l^2P_l+lP'_{l-1}, \]
  故
  \[ DfDP_l = l\pare{l+1}P_l. \]
  这再次证明上述反射皆定义了Legendre多项式.
  \par
  最后,将\rref{lpl3}写作
  \[ P'_{l+1} - xP'_l = \pare{l+1}P_l. \]
  由\rref{lpl2},
  \[ xP'_l = lPl + P'_{l-1}, \]
  得到(快速获得正交关系的途径之一)
  \refl{
    \label{refl:lpl6}
    \[ \pare{2l+1}P_l = P'_{l+1}-P'_{l-1}. \]
  }
  \paragraph{势展开}场源$r$与检验点$R$之间的距离可以写为
  \[ d = R\sqrt{1-2\frac{r}{R}\cos\theta+\pare{\frac{r}{R}}^2}. \]
  对于反比势,有
  \[ V = \frac{K}{R}\pare{1-\frac{2r}{R}\cos\theta+\pare{\frac{r}{R}}^2}^{-\half}. \]
  注意$h=r/R$,有
  \[ V = \frac{K}{R} \sum \frac{r^l}{R^l}P_l\pare{\cos\theta}. \]
  \par
  若场源的位置远离检验点,例如要考虑其他行星对水星的摄动,则可以互换$r$与$R$,上述方程仍然不变.
  \subsection{完备正交函数集}
  定义
  \[ \bk{A}{B} = \intab{}{}{A^*B}{x}, \]
  若
  \[ \bk{A}{B} = 0 \]
  则称$A$与$B$正交.
  \par
  Legendre多项式$\curb{P_0,\cdots,P_{n}}$是$\curb{1,\cdots,x^{n}}$Gram-Schmidt正交化的结果. 故单位区间内的所有Riemann可积函数均可以表示为Legendre多项式的和(参考如《傅立叶分析导论》,或《实变函数论》p.192),因此Legendre多项式是一完备集合.
  \par
  注意$\curb{P_0,\cdots,P_{n-1}}$与$\curb{1,\cdots,x^{n-1}}$张成完全相同的多项式空间,并假设其可由Gram-Schmidt过程获得. 则对$xP_{n-1}$应用此过程,有(忽略比例因子)
  \[ \pare{P_n} = xP_{n-1} + \frac{\int xP_{n-1}^2}{\int P_{n-1}^2}P_{n-1} + \frac{\int x P_{n-2}P_{n-1}}{\int P_{n-2}^2}P_{n-2}. \]
  由\eref{xpp},这正是\rref{lpl1}(忽略比例因子)
  \[ \pare{P_n} = xP_{n-1} + \frac{n-1}{2n-1}P_{n-2}. \]
  \par
  之所以可以如此应用Gram-Schmidt过程,在于$\alpha x^n$与$x^n$挖去\[ \curb{1,\cdots,x^{n-1}} \]分量的结果成比例. 而$\alpha x^n + O\pare{x^{n-1}}$挖去的结果也相同,故最终仅仅相差一比例因子.
  \subsection{正交关系}
  \label{sssec:oth}
  $\FF = \DD f\DD$是Hermitian算子,
  \begin{align*}
    \intab{-1}{1}{P\DD f\DD Q}{x} &= \intab{-1}{1}{P}{f\DD Q} \\
    &= Pf\DD Q\left\vert_{x=-1}^{1} - \intab{-1}{1}{\pare{f\DD Q}\pare{\DD P}}{x} \right. \\
    &= -\intab{-1}{1}{f\DD P}{Q} \\
    &= -Qf\DD P\left\vert_{x=-1}^{1} + \intab{-1}{1}{Q\DD fDP}{x}\right. \\
    &= \intab{-1}{1}{Q\DD f\DD P}.
  \end{align*}
  边界项的消除完全承蒙$f\pare{\pm1}=0$. 由此可得
  \[ \bik{P_m}{\FF}{P_n} = \lambda_m\bk{P_m}{P_n} = \lambda_n\bk{P_m}{P_n}. \]
  因此,若$m\ne n$,有
  \[ \intab{-1}{1}{P_mP_n}{x} = 0. \]
  \par
  注意任意的$f\pare{x}$本身均为Hermitian算子,而Hermitian算子的加减仍然为Hermitian算子,因此
  \[ \DD f \DD - \frac{m^2}{1-x^2} \]
  也是Hermitian算子,故其本征函数为正交系(关联Legendre函数).
  \subsection{自交关系:归一化}
  借助\rref{lpl3},
  \begin{align*}
    \pare{2l+1}\intab{-1}{1}{P_l^2}{x} &= \intab{-1}{1}{P_l\pare{P'_{l+1}-P'_{l-1}}}{x} \\
    &= \intab{-1}{1}{P_l}{P_{l+1}} \\
    &= \left.P_lP_{l+1}\right\vert_{x=-1}^{1} - \intab{-1}{1}{P_{l+1}P'_l}{x} \\
    &= 2.
  \end{align*}
  其中若干积分项的消失承蒙$P_l$与次数小于其的所有多项式正交的事实.
  \refl{
    \label{refl:PP}
    \[ \intab{-1}{1}{P_lP_m}{x} = \frac{2}{2l+1}\delta_{lm}. \]
  }
  \subsection{Legendre级数}
  \begin{ex}
    将
    \[ 
       f\pare{x} = \begin{cases}
       0, \quad -1<x<0,\\
       1, \quad 0<x<1
       \end{cases}
    \]
    展开为Legendre级数
    \[ f=\sum c_l P_l, \]
    有
    \[ \bk{f}{P_l} = c_l\bk{P_l}{P_l}. \]
  \end{ex}
  \refl{
    对于
    \[ f=\sum c_l P_l, \]
    有
    \[ c_l = \frac{2l+1}{2}\bk{f}{P_l}. \]
  }
  上述逼近可以证明为最佳逼近(参考《傅立叶分析导论》). 直接注意到$P_l$为正交系后应用Pythagorean定理即可.
  \subsection{关联Legendre函数}
  对于方程
  \[ \pare{1-x^2}y''-2xy'+\pare{l\pare{l+1} - \frac{m^2}{1-x^2}}y = 0,\qquad m\le l, \]
  令
  \[ f=1-x^2, \]
  可以尝试代换
  \[ y=\pare{1-x^2}^{m/2}u = f^{m/2}u. \]
  从而
  \[ y' = f^{m/2}u' - mxf^{m/2-1}u, \]
  \begin{align*}
    y'' =& f^{m/2}u'' - mxf^{m/2-1}u' \\
         & -mf^{m/2-1}u + m\pare{m-2}x^2f^{m/2-2}u\\
         & -mxf^{m/2-1}u'.
  \end{align*}
  将各项系数列出如下
  \begin{table}[!ht]
    \centering
    \begin{tabular}{|c|c|c|c|}
    \hline
     & $f^{m/2+1}$ & $f^{m/2}$ & $f^{m/2-1}$ \\
    \hline
     $y''$项 & $u''$ & $-mu-mxu'-mxu'$ & $m\pare{m-2}x^2u$ \\
    \hline
     $y'$项 & $$ & $-2xu'$ & $2mx^2u$ \\
    \hline
     $y$项 & $$ & $l\pare{l+1}u$ & $-m^2u$ \\
    \hline
    \end{tabular}
  \end{table}
  合并$u$项给出
  \[ f^{m/2} \pare{\pare{l\pare{l+1}-m\pare{m+1}}u}, \]
  合并$u'$项给出
  \[ -2f^{m/2}\pare{m+1}xu, \]
  合并$u''$项给出
  \[ f^{m/2+1}u''. \]
  故方程化为
  \[ \pare{1-x^2}u''-2\pare{m+1}xu'+\pare{l\pare{l+1} - m\pare{m+1}}u = 0. \]
  若对上式微分,有
  \[ \pare{1-x^2}\pare{u'}'' - 2\pare{m+1+1}x\pare{u'}' - \pare{l\pare{l+1} - \pare{m+1}\pare{m+2}}u = 0. \]
  这正是对前式替换$m$为$m+1$的结果. 由是可以得到,对$m+1$的解刚好为对$m$的解求导的结果. 而对于$m$为0的情况,$u$恰为Legendre多项式. 是故
  \refl{
    \label{refl:pml}
    方程
    \[ \pare{\DD f \DD + \pare{l\pare{l+1}-\frac{m^2}{1-x^2}}}y = 0 \]
    的解为
    \[  P_l^m = \pare{1-x^2}^{m/2}\DD^mP_l. \]
  }
  应当注意到,Legendre多项式就是$m=0$时的关联Legendre函数.
  \begin{ex}
    对于方程
    \[ \frac{1}{\sin\theta} \eddon{}{\theta}\pare{\sin\theta\eddon{y}{\theta}} + \pare{l\pare{l+1}-\frac{m^2}{\sin^2\theta}}y = 0, \]
    令$x=\cos\theta$,有
    \[ \sin\theta\eddon{}{\theta} = -f\DD, \]
    而
    \[ \frac{1}{\sin\theta}\eddon{}{\theta} = -\DD. \]
    因此方程等价于
    \[ \DD f \DD y + \pare{l\pare{l+1}-\frac{m^2}{1-x^2}}y = 0. \]
    故解为
    \[ y = P_l^m\pare{\cos\theta}. \]
  \end{ex}
  由于
  \[ \DD f \DD + -\frac{m^2}{1-x^2} \]
  是Hermitian算子(参考\ref{sssec:oth}),故对同一$m$,其本征函数为正交函数系.
  \paragraph{关联Legendre函数的正交与自交}
  由Leibniz法则,
  \begin{align*}
  L &= \rec{\pare{l-m}!}\DD^{l-m}\pare{x-1}^l\pare{x+1}^l =\\ &\sum \rec{r!\pare{l-m-r}!} \curb{\DD^r \pare{x-1}^l} \curb{\DD^{l-m-r}\pare{x+1}^l},\\
  R &=\pare{x-1}^m\pare{x+1}^m\rec{\pare{l+m}!}\DD^{l+m}\pare{x-1}^l\pare{x+1}^l =\\ &\sum \rec{r!\pare{l+m-r}!} \curb{\pare{x-1}^m\DD^r \pare{x-1}^l} \curb{\pare{x+1}^m\DD^{l+m-r}\pare{x+1}^l},
  \end{align*}
  对齐两侧的项,以右侧$r \mapsto r+m$,变为
  \[ \sum \rec{\pare{r+m}!\pare{l-r}!} \curb{\pare{x-1}^m\DD^{r+m} \pare{x-1}^l} \curb{\pare{x+1}^m\DD^{l-r}\pare{x+1}^l}. \]
  因此,对齐后
  \begin{align*}
    L &\sim \frac{l\times \cdots \times \pare{l-r+1}\times l \times \cdots \times \pare{m+r+1}}{r!\pare{l-m-r}!},\\
    R &\sim \frac{l\times \cdots \times \pare{l-m-r+1}\times l \times \cdots \times \pare{r+1}}{\pare{r+m}!\pare{l-r}!}.
  \end{align*}
  消去可知两侧相等,故(注意$f=1-x^2$而不是$x^2-1$)
  \begin{equation}
  \label{eq:lmlm} \DD^{l-m}f^l = \pare{-1}^m\frac{\pare{l-m}!}{\pare{l+m}!}f^m\DD^{l+m}f^l.
  \end{equation}
  \refl{
    \[ \rec{\pare{l+m}!}\pare{x^2-1}^{m/2}\DD^{l+m}\pare{x^2-1}^l \]
    在变换$m \mapsto -m$下不变.
  }
  在\rref{pml}中代入\rref{pl},有
  \[ P_l^m = \rec{2^ll!}f^{m/2}\DD^{l+m}f^l. \]
  因此,由这一公式求出的
  \[ P_l^{-m} = \rec{2^ll!}f^{-m/2}\DD^{l-m}f^l = \pare{-1}^m \frac{\pare{l-m}!}{\pare{l+m}!} P_l^m. \]
  有些作者在\rref{pml}中采用了$\abs{m}$替换$m$,此时
  \[ P_l^{-m} = P_l^{m}. \]
  \par
  若将$\DD^{l+m}$利用\ref{eq:lmlm}代换,有
  \[ P_l^m = \pare{-1}^m\rec{2^ll!}f^{-m/2}\frac{\pare{l+m}!}{\pare{l-m}!}\DD^{l-m}f^l. \]
  又注意由\rref{pml},
  \[ P_l^m = f^{m/2}\DD^{l+m}f^l, \]
  因此
  \[ \pare{P_l^m}^2 = \frac{\pare{l+m}!}{\pare{l-m}!} \rec{2^ll!} \DD^{l+m}f^l \DD^{l-m}f^l,  \]
  分部积分后由\rref{pl}与\rref{PP},自交结果为
  \refl{
  \[ \bk{P_l^m}{P_l^m} = \frac{\pare{l+m}!}{\pare{l-m}!}\frac{2}{2l+1}.  \]
  }
  \subsection{广义幂级数}
  对于解的幂级数中含有$x$的分数幂的情况,例如
  \[ y = x^s \sum a_n x^n \]
  仍可以借助类似于\rref{subst}的方法求解,唯应注意$n\mapsto n+s$.
  \refl{
  \label{refl:substs}
  可以遵守如\tbref{substs}所示的替换法则,其中$n'=n+s$.牢记当$n<0$时定义$a_n=0$.
  }
  \begin{table}[!ht]
  \centering
    \begin{tabular}{|c|c|c|c|}
    \hline
    因子 & $1$ & $x$ & $x^2$ \\
    \hline
    $y$ & $a_n$ & $a_{n-1}$ & $a_{n-2}$ \\
    \hline
    $y'$ & $\pare{n'+1}a_{n+1}$ & $n'a_n$ & $\pare{n'-1}a_{n-1}$ \\
    \hline
    $y''$ & $\pare{n'+1}\pare{n'+2}a_{n+2}$ & $n'\pare{n'+1}a_{n+1}$ & $\pare{n'-1}n'a_n$ \\
    \hline
    \end{tabular}
    \caption{替换法则}
    \label{table:substs}
  \end{table}
  \begin{ex}
    对于方程
    \[ x^2y''+4xy' + \pare{x^2+2}y=0, \]
    有
    \[ \pare{n+s-1}\pare{n+s}a_n + 4\pare{n+s}a_n + a_{n-2} + 2a_n = 0. \]
    当$n=0$,有
    \[ \pare{s-1}sa_0 + 4sa_0 + 2a_0 = 0. \]
    仅有二解为$s=-2$与$s=-1$.
    \par
    对于$s=-1$的情形,$a_{2n+1}=0$且
    \[ a_n = -\frac{a_{n-2}}{n\pare{n+1}}, \]
    故
    \[ y = \frac{a_0\sin x}{x}. \]
  \end{ex}
  \subsection{Bessel方程}
  考虑方程
  \[ x^2y''+xy'+\pare{x^2-p^2}y = 0, \]
  可以考虑代换$x\pare{xy'}' = x^2y'' + xy'$,从而化之为
  \[ x\pare{xy'}' + \pare{x^2-p^2}y = 0. \]
  若不考虑上述代换,仍可以借助\rref{substs},有
  \[ \pare{n+s-1}\pare{n+s}a_n + \pare{n+s}a_n + a_{n-2} - p^2a_n = 0. \]
  或者
  \[ \pare{\pare{n+s}^2-p^2}a_n + a_{n-2} = 0, \]
  \[ a_n = -\frac{a_{n-2}}{\pare{n+s}^2-p^2}. \]
  此处考虑$a_0\ne 0$的解,从而$a_1 = 0$且$s=p$(还有另一解$s=-p$留待下节).因此
  \[ a_{2n} = -\frac{a_{2n-2}}{4n\pare{n+p}} = \pare{-1}^n\frac{\Gamma\pare{1}}{\Gamma\pare{n+1}}\frac{\Gamma\pare{1+p}}{\Gamma\pare{n+1+p}}\frac{a_0}{2^{2n}}. \]
  且
  \[ y = \sum a_{2n}x^{2n+p}, \]
  故若令
  \[ a_0 = \rec{2^p\Gamma\pare{p+1}}, \]
  有
  \refl{
    \label{refl:jps}
    \[ J_p\pare{x} = \sum \frac{\pare{-1}^n}{\Gamma\pare{n+1}\Gamma\pare{n+1+p}}\pare{\frac{x}{2}}^{2n+p} \]
    满足
    \[ \pare{x\DD}^2y + x^2y-p^2y = 0. \]
  }
  \subsection{Bessel方程的第二解}
  可直接定义(或者,原封不动)
  \[ J_{-p}\pare{x} = \sum \frac{\pare{-1}^n}{\Gamma\pare{n+1}\Gamma\pare{n+1-p}}\pare{\frac{x}{2}}^{2n-p}, \]
  代换$n\mapsto n+p$有
  \[ J_{-p}\pare{x} = \sum_{n=-p} \frac{\pare{-1}^{n+p}}{\Gamma\pare{n+p+1}\Gamma\pare{n+1}}\pare{\frac{x}{2}}^{2n+p}, \]
  未对齐的项目由于$n<p$而系数消失.故
  \refl{
  对于整数$p$,
    \[ J_{-p} = \pare{-1}^pJ_p. \]
  }
  对于非整数的$p$,$J_p$与$J_{-p}$为二独立解,故可组合而给出Bessel方程的通解. 对于一般的$p$,第二解由Neumann函数给出
  \refl{
    \label{refl:newman}
    \[ N_p = Y_p = \frac{\cos\pare{\pi p} J_p - J_{-p}}{\sin \pare{\pi p}} \]
    满足Bessel方程.
  }
  当$p$非整数时,$N_p$与$J_p$线性无关. 而对于整数的$p$,在$x\ne 0$极限存在,而$x=0$发散,故独立于$J_p$.
  \par
  特别地,对于半奇整数,直接有
  \[ N_{\pare{2n+1}/2}=\pare{-1}^{n+1}J_{-\pare{2n+1}/2}. \]
  这并不矛盾,半奇整数的两个$J$本就线性独立.
  \par
  \begin{ex}
    \label{ex:p-half}
    对于$p=-\half$,借助\rref{gamma2p}有
    \begin{align*}
    J_{-\half} &= \pare{\frac{x}{2}}^{-\half} \sum \frac{\pare{-1}^n}{\Gamma\pare{n+1}\Gamma\pare{n+\half}}\pare{\frac{x}{2}}^{2n} \\
    &= \pare{\frac{x}{2}}^{-\half} \sum \frac{\pare{-1}^n2^{2n}}{\sqrt{\pi}\Gamma\pare{2n+1}}\pare{\frac{x}{2}}^{2n} \\
    &= \sqrt{\frac{2}{\pi x}}\cos x.
    \end{align*}
    参考上文令$n=0$,直接有
    \[ Y_\half = -\sqrt{\frac{2}{\pi x}}\cos x. \]
  \end{ex}
  \subsection{Bessel函数的零点}
  所有(除了$p=0$)的Bessel函数都可以近似看做振幅以$x^{-1/2}$衰减的三角函数,其零点间距接近$\pi$. 而Neumann函数也有类似的性质(参考后文的渐进公式,以及《物理学家用的数学方法》,Arfken).
  \subsection{递归关系}
  本节的反射对Neumann函数均适用.
  \par 
  对
  \[ x^pJ_p = \sum \frac{\pare{-1}^n}{\Gamma\pare{n+1}\Gamma\pare{n+1+p}}\pare{\frac{x}{2}}^{2n+2p} \]
  求导,
  \begin{align*}
    \DD x^p J_p &= \sum \frac{\pare{-1}^n \pare{n+p}}{\Gamma\pare{n+1}\Gamma\pare{n+1+p}}\pare{\frac{x}{2}}^{2n+2p-1} \\
    &= x^p \sum \frac{\pare{-1}^n}{\Gamma\pare{n+1}\Gamma\pare{n+1+p-1}}\pare{\frac{x}{2}}^{2n+\pare{p-1}} \\
    &= x^p J_{p-1}.
  \end{align*}
  \refl{
  \label{refl:jpp0}
    \[ \pare{x^pJ_p}' = x^pJ_{p-1}. \]
  }
  而对
  \[ x^{-p}J_p = \sum \frac{\pare{-1}^n}{\Gamma\pare{n+1}\Gamma\pare{n+1+p}}\pare{\frac{x}{2}}^{2n} \]
  求导,
  \begin{align*}
    \DD x^{-p} J_p &= \sum \frac{\pare{-1}^n n}{\Gamma\pare{n+1}\Gamma\pare{n+1+p}}\pare{\frac{x}{2}}^{2n} \\
    &= \sum \frac{\pare{-1}^n}{\Gamma\pare{n}\Gamma\pare{n+1+p}}\pare{\frac{x}{2}}^{2n-1} \\
    &= x^{-p} \sum \frac{\pare{-1}^{n+1}}{\Gamma\pare{n+1}\Gamma\pare{n+1+p+1}}\pare{\frac{x}{2}}^{2n+p+1} \\
    &= -x^{-p}J_{p+1}.
  \end{align*}
  \refl{
  \label{refl:jpp1}
    \[ \pare{x^{-p}J_p}' = -x^{-p}J_{p+1}. \]
  }
  \begin{ex}
    \label{ex:phalf}
    对\eref{p-half}应用上述反射,有
    \[ \pare{x^{\half}J_{-\half}}' = -\sqrt{\frac{2}{\pi}}\sin x = -x^{\half}J_{\half}. \]
    即
    \[ J_\half = \sqrt{\frac{2}{\pi x}}\sin x. \]
  \end{ex}
  \paragraph{半整数阶的初等函数通项}借助前例以及上述反射,可以证明
  \[ \sqrt{\frac{\pi}{2x}}J_{n+\half}\pare{x}=x^n\pare{-\rec{x}\DD}^n\frac{\sin x}{x}. \]
  上式要求
  \[ \sqrt{\frac{\pi}{2x}}J_{n+1+\half} \pare{x} = x^{n+1}\pare{-\rec{x}\DD}^{n+1}\frac{\sin x}{x}, \]
  而递推关系要求
  \begin{align*}
    J_{n+1+\half} &= -x^{n+\half}\DD\pare{x^{-n-\half}J_{n+\half}} \\
    &= -x^{n+\half}\DD\pare{x^{-n-\half}\sqrt{\frac{2x}{\pi}}x^n\pare{-\rec{x}\DD}^n\frac{\sin x}{x}}.
  \end{align*}
  可见二者一致,故通项公式满足递推条件. 又可见其首项相符,故
  \refl{
    \label{refl:elej}
    \[ J_{n+\half}\pare{x} = \sqrt{\frac{2x}{\pi}}x^n\pare{-\rec{x}\DD}^n\pare{\frac{\sin x}{x}}. \]
  }
  而参考\eref{p-half}后段,直接将$\sin$替换为$-\cos$,可得
  \refl{
    \label{refl:eley}
    \[ Y_{n+\half} = -\sqrt{\frac{2x}{\pi}}x^n\pare{-\rec{x}\DD}^n\pare{\frac{\cos x}{x}}. \]
  }
  将\rref{jpp0}与\rref{jpp1}的微分展开,有
  \begin{align}
    px^{p-1}J_p+x^pJ'_p &= x^pJ_{p-1},\\
  \label{eq:jpd}
    -px^{-p-1}J_p + x^{-p}J'_p &= -x^{-p}J_{p+1}.
  \end{align}
  立即得到
  \refl{
  \label{refl:jpp2}
    \[ J'_p = -\frac{p}{x}J_p + J_{p-1} = \frac{p}{x}J_p - J_{p+1}. \]
  }
  将第二条微分乘以$x^{2p}$,得
  \[ -px^{p-1}J_p + x^{p}J'_p = -x^{p}J_{p+1}. \]
  前式减后式,有
  \refl{
  \label{refl:jpp3}
    \[ \frac{2p}{x}J_p = J_{p-1}+J_{p+1}. \]
  }
  前式加后式,有
  \refl{
  \label{refl:jpp4}
    \[ 2J'_p = J_{p-1} - J_{p+1}. \]
  }
  下面证明$J_0$的积分形式,
  \refl{
    \[ J_0\pare{x} = \frac{1}{\pi}\intu{\pi}{\cos\pare{x\sin\theta}}{\theta}. \]
  }
  将$\cos$级数展开并借助\eref{sin2n},即
  \begin{align*}
    \frac{1}{\pi}\intu{\pi}{\cos\pare{x\sin\theta}}{\theta} &= \frac{1}{\pi}\intu{\pi}{\sum \frac{\pare{-1}^n}{\pare{2n}!}\pare{\sin^{2n}\theta} x^{2n}}{\theta} \\
    &= \frac{2}{\pi}\sum \pare{-1}^n\frac{\pi}{2^{2n}}\frac{\Gamma\pare{2n}}{\Gamma\pare{n}\Gamma\pare{n+1}\Gamma\pare{2n+1}} x^{2n}\\
    &= \sum \frac{\pare{-1}^n}{\Gamma\pare{n+1}\Gamma\pare{n+1}}\pare{\frac{x}{2}}^{2n} \\
    &= J_0\pare{x}.
  \end{align*}
  欲求出$J_0$的Laplace变换,可经由
  \begin{align*}
    &\re \frac{1}{\pi}\intt{\intu{\pi}{e^{ix\sin\theta}e^{-tx}}{\theta}}{x}\\
    =&\re \frac{1}{\pi}\intu{\pi}{\rec{t-i\sin\theta}}{\theta}\\
    =& \frac{1}{\pi}\intu{\pi}{\frac{t}{t^2+\sin^2\theta}}{\theta}\\
    =& \frac{1}{2\pi}\intu{2\pi}{\frac{a}{a^2+\pare{1-\cos\theta}/2}}{\theta}\\
    =& \sqrt{\frac{1}{1+t^2}}.
  \end{align*}
  虽然上式仅适用于$t>0$,亦不妨将其延拓至$t=0$处,从而
  \[ \intt{J_0\pare{x}}{x} = 1. \]
  直接在\rref{jpp1}中令$p=0$并积分,有
  \[ \intt{J_1\pare{x}}{x} = J_0\pare{0} - J_0\pare{\infty} = 1. \]
  对\rref{jpp4}积分,有
  \[ \intt{J_0\pare{x}}{x} = \intt{J_{2n}\pare{x}}{x} \]
  以及
  \[ \intt{J_1\pare{x}}{x} = \intt{J_{2n+1}\pare{x}}{x}. \]
  因此,
  \refl{
    \[ \intt{J_n\pare{x}}{x} = 1. \]
  }
  \subsection{Bessel函数用于更广泛的微分方程}
	\refl{
	  \label{refl:genj}
	  方程
	  \[ y''+\frac{1-2a}{x}y'+\pare{\pare{bcx^{c-1}}^2+\frac{a^2-p^2c^2}{x^2}}y=0 \]
	  有解
	  \[ y = x^a Z_p \pare{bx^c}. \]
	  其中$Z$为$J$和$N$的任意线性组合.
	}
	令
	\[ y=x^2J\pare{z},\quad z=bx^c. \]
	\begin{align*}
	  y'' &= a\pare{a-1}x^{a-2}J+2ax^{a-1}J'+x^aJ'',\\
	  \frac{1-2a}{x}y' &= \pare{1-2a}ax^{a-2}J+\pare{1-2a}x^{a-1}J',\\
	  \pare{bcx^{c-1}}^2y &=\pare{bc}^2x^{a+2c-2}J,\\
	  \frac{a^2-p^2c^2}{x^2}y &=\pare{a^2-p^2c^2}x^{a-2}J.\\
	\end{align*}
	从而
	\[ c^2\pare{b^2x^{2c}-p^2}J+xJ'+x^2J'' = 0. \]
	前若干式的求导均对$x$进行,下列各式的撇号都表示对$z$的求导.
	\begin{align*}
	  x\eddon{}{x}J &= x \eddon{z}{x} \eddon{}{z}J = xbcx^{c-1}J' = czJ'.\\
	  x^2\edx{J} &= x^2 \eddon{z}{x}\eddon{}{z}\eddon{z}{x}J' \\
	  &= x^2bcx^{c-1}\eddon{}{z}\pare{bcx^{c-1}J'} \\
	  &= \pare{\frac{z}{b}}^{\rec{c}}cz\eddon{}{z}\pare{bc\pare{\frac{z}{b}}^{1-\rec{c}}J'}\\
	  &=\pare{\frac{z}{b}}^{\rec{c}}cz\pare{\pare{c-1}\pare{\frac{z}{b}}^{-\rec{c}}+bc\pare{\frac{z}{b}}^{1-\rec{c}}J''}\\
	  &= c\pare{c-1}zJ'+c^2z^2J''.
	\end{align*}
	最终
	\[ c^2\pare{z^2-p^2}J+\pare{c+c\pare{c-1}}zJ'+c^2z^2J''=0. \]
	故得证.
	\par
	在上述反射中令$a=0$,$b=K$,$c=1$,得到
	\refl{
	  \label{refl:jk}
	  方程
	  \[ x\DD x\DD y + \pare{K^2x^2-p^2}y = 0 \]
	  的解为$J_p\pare{Kx}$与$N_p\pare{Kx}$.
	}
	\subsection{其他Bessel函数}
	\paragraph{Hankel函数}定义如下,可参照$e^i = \cos + i\sin$.
  \refl{
    \begin{align*}
      \hankel\pare{x} &= J_p\pare{x} + iN_p\pare{x}, \\
      \hankell \pare{x} &= J_p\pare{x} - iN_p\pare{x}.
    \end{align*}
  }
  \paragraph{修正Bessel函数}定义如下,可参照$\sinh = -i\sin i$以及$\cosh = \cos i$.
  \refl{
    \begin{align*}
      I_p\pare{x} &= i^{-p}J_p\pare{ix},\\
      K_p\pare{x} &= \frac{\pi}{2}i^{p+1}\hankel\pare{ix}.
    \end{align*}
  }
  $i$因子的存在使得上述二函数将实数映为实数.
  \paragraph{球Bessel函数}对于方程
  \[ x^2y''+2xy'+\pare{x^2-n\pare{n+1}}y = 0, \]
  借助\rref{genj},令$a=-1/2$,$b=1$,$c=1$,$p=n+1/2$.可得
  \[ y = \frac{1}{\sqrt{x}}J_{n+\half}\pare{x}. \]
  借助\rref{elej}与\rref{eley},有
  \refl{
    \begin{align*}
      j_n\pare{x} &= \sqrt{\frac{\pi}{2x}}J_{n+\half}\pare{x} = x^n\pare{-\rec{x}\DD}^n\pare{\frac{\sin x}{x}},\\
      y_n\pare{x} &= \sqrt{\frac{\pi}{2x}}Y_{n+\half}\pare{x} = -x^n\pare{-\rec{x}\DD}^n\pare{\frac{\cos x}{x}},\\
      \hhankel &= j_n + iy_n,\\
      \hhankell &= j_n - iy_n.
    \end{align*}
  }
  $j_n$与$y_n$满足的递推关系可修正\rref{jpp0}及以下而得.
  \paragraph{Kelvin函数}对于下列方程
  \[ y''+\frac{1}{x}y'-iy=0, \]
  借助\rref{genj},可得
  \[ y = Z_0\pare{i^{3/2}x}. \]
  故定义四个函数
  \begin{align*}
    J_0\pare{i^{3/2}x} &= \ber x + i\, \bei x,\\
    K_0\pare{i^{1/2}x} &= \kker x + i\, \kei x.
  \end{align*}
  \paragraph{Airy函数}对于方程
  \[ y''-xy=0, \]
  借助\rref{genj},可得
  \[ y = \sqrt{x}Z_{1/3}\pare{\frac{2}{3}ix^{3/2}}. \]
  故定义
  \begin{align*}
    \Ai\pare{x} &= \rec{\pi}\sqrt{\frac{x}{3}}K_{1/3}\pare{\frac{2}{3}x^{3/2}},\\
    \Bi\pare{x} &= \sqrt{\frac{x}{3}}\pare{I_{-1/3}\pare{\frac{2}{3}x^{3/2}}+I_{1/3}\pare{\frac{2}{3}x^{3/2}}}.
  \end{align*}
  \subsection{应用:伸长的单摆}
  在单摆的Lagrange方程
  \[ \edt{}\pare{ml^2\dot{\theta}} + mgl\sin\theta = 0 \]
  中令$l = l_0 + vt $,立即有
  \[ \edt{} = v\eddon{}{l}. \]
  故
  \[ mv^2l^2 \eddon{\theta}{l} + 2mv^2l\eddon{\theta}{l} + mgl\theta = 0, \]
  即
  \[ l \theta'' + 2\theta' + \frac{g}{v^2}\theta = 0. \]
  在\rref{genj}中令$a=-\half$,$b=2g^{1/2}/v$,$c=\half$,$p=1$,有
  \[ \theta=l^{-\half}Z_1\pare{bl^{\half}}. \]
  复令$u=bl^{1/2}$,有
  \[ \theta = Au^{-1}J_1\pare{u} + Bu^{-1}N_1\pare{u}. \]
  借助\rref{jpp1},有
  \[ \eddon{\theta}{u} = -\pare{Au^{-1}J_2\pare{u} + Bu^{-1}N_2\pare{u}}. \]
  故可以根据初始条件确定$A$与$B$的值. 
  \paragraph{边值问题}对于$\theta=\theta_0$,$\dot{\theta}=0$的初始条件,先证如下等式.
  \[ \DD\pare{x\pare{J_pJ'_{-p}-J_{-p}J'_p}} = 0. \]
  径借助\rref{dxd},在Bessel方程两侧分别乘$J_p$与$J_{-p}$后相减得到. 由是立得
  \[ J_pJ'_{-p}-J_{-p}J'_p = \frac{c}{x}. \]
  借助\rref{jps}提取$-1$次项,
  \[ -\frac{1}{\Gamma\pare{1+p}}\frac{p}{\Gamma\pare{1-p}} - \frac{1}{\Gamma\pare{1-p}}\frac{p}{\Gamma\pare{1+p}} = c. \]
  由\rref{gammarefl},得
  \[ c = -\frac{2 \sin p\pi}{\pi}. \]
  又由\rref{newman},
  \[ J_pN'_p-J'_pN_p = \frac{2}{\pi x}. \]
  借助\eqref{eq:jpd}(在两侧乘$x^p$),可得
  \[ J_nN_{n+1} - J_{n+1}N_n = -\frac{2}{\pi x}. \]
  于是可径令
  \[ A=-\frac{\pi u_0^2}{2}\theta_0N_2\pare{u_0},\quad B = \frac{\pi u_0^2}{2}\theta_0J_2\pare{u_0}. \]
  易知上述$A$,$B$满足条件.
  \subsection{正交关系与自交关系}
  考虑二函数$u=J_p\pare{\alpha x}$与$v=J_p\pare{\beta x}$(就像$\sin n\pi x$与$\sin m\pi x$),由\rref{jk}分别满足
  \begin{align*}
    x\pare{xu'}' + \pare{\alpha^2 x^2-p^2}u = 0, \\
    x\pare{xv'}' + \pare{\alpha^2 x^2-p^2}v = 0.
  \end{align*}
  上式乘$v$,下式乘$u$后相减得
  \[ v\pare{xu'}' - u\pare{xv'}' + \pare{\alpha^2-\beta^2}xuv = 0. \]
  借助\rref{dxd},得
  \[ \DD x \pare{v\DD u - u \DD v} + \pare{\alpha^2-\beta^2}xuv = 0. \]
  因而
  \[ \bik{u}{x}{v} = -\frac{J_p\pare{\beta}\alpha J'_p\pare{\alpha} - J_p\pare{\alpha}\beta J'_p\pare{\beta}}{\alpha^2-\beta^2}.  \]
  其中内积的积分区间为$\curb{0,1}$.
  \par
  设$\alpha$与$\beta$为$J_p$的二零点,若$\alpha \ne \beta$,内积为$0$. 若$\alpha = \beta$,先设$\beta$接近实际零点,故
  \[ \bik{u}{x}{v} = \frac{J_p\pare{\beta}\alpha J'_p\pare{\alpha}}{\pare{\beta-\alpha}\pare{\alpha+\beta}} = \frac{\pare{J_p\pare{\beta}-J_p\pare{\alpha}}\alpha J'_p\pare{\alpha}}{\pare{\beta-\alpha}\pare{\alpha+\beta}} = \half J'^2_p\pare{\alpha}. \]
  因而
  \refl{
    \[ \bik{J_p\pare{\alpha x}}{x}{J_p\pare{\beta x}} = \half J'^2_p\pare{\alpha}\delta_{\alpha\beta}. \]
  }
  借助\rref{jpp3}及以下,有
  \refl{
    \[ \bik{J_p\pare{\alpha x}}{x}{J_p\pare{\alpha x}} = \half J'^2_p\pare{\alpha} = \half J^2_{p+1}\pare{\alpha} = \half J^2_{p-1}\pare{\alpha}. \]
  }
  \subsection{鞍点法:渐进公式}
  对于大的$x$,有各类Bessel函数有如下渐进形式:
  \refl{
    \[ J_p\pare{x}\sim \sqrt{\frac{2}{\pi x}}\cos\pare{x-\frac{2p+1}{4}\pi}. \]
  }
  \refl{
    \[ N_p\pare{x}\sim \sqrt{\frac{2}{\pi x}}\sin\pare{x-\frac{2p+1}{4}\pi}. \]
  }
  \subsection{Fuchs定理}
  对于方程
  \[ y''+f\pare{x}y'+g\pare{x}y=0, \]
  若$f$与$g$可以展开为Taylor级数,则二解或为二Frobenius级数,或其一$S_1$为级数,另解为$S_1\ln x+S_2$(其中$S_2$亦为Frobenius级数). 第二种情况仅当特征方程的根相差为整数时出现.
  \subsection{Hermite函数,Laguerre函数,阶梯算子}
  \paragraph{Hermite函数}对于方程
  \[ y''_n-x^2y_n = -\pare{2n+1}y_n,\quad n=0,1,2,\cdots. \]
  可由多种方法应对之.
  \paragraph{阶梯算子}由\eqref{eq:dpmx},$\pare{\DD-x}\pare{\DD+x}=\DD^2-x^2+1$,$\pare{\DD+x}\pare{\DD-x}=\DD^2-x^2-1$. 定义$\oplus=\DD+x$,$\ominus=\DD-x$,方程可写为
  \begin{align*}
    \ominus\oplus y_n &= -2ny_n,\\
    \oplus\ominus y_n &= -2\pare{n+1}y_n.
  \end{align*}
  进而
  \begin{align*}
    \ominus\oplus\ominus y_n &= -2\pare{n+1}\ominus y_n,\\
    \oplus\ominus\oplus y_n &= -2n\oplus y_n.
  \end{align*}
  故$\ominus y_n$给出$n+1$的解,而$\oplus y_n$给出$n-1$的解. 因此,若对于复数的$n$要求$y_n$被消灭,还需要
  \[ \pare{\DD+x}y_0 = 0. \]
  立得
  \[ y_0 = e^{-x^2/2}. \]
  由是还可得
  \[ e^{x^2/2}\DD e^{-x^2/2} = \DD -x, \]
  从而
  \[ e^{x^2/2}\DD^n e^{-x^2/2} = \pare{\DD -x}^n. \]
  故
  \refl{
    \[ y_n = \pare{\DD-x}^n e^{-x^2/2} = e^{x^2/2}\DD^n e^{-x^2} \]
    是$\DD^2-x^2$本征值为$-2\pare{n+1}$的本征函数.
  }
  Hermite函数的正交关系则有赖于算子$\DD^2-x^2$为Hermite算子(谐振子能量),故不同的Hermite函数彼此正交(内积的积分区间为实数轴). 自交结果可计算为
  \begin{align*}
    \intr{y^2_{n+1}}{x} &= \intr{\pare{\pare{\DD-x}y}^2}{x}\\
    &= \intr{\pare{\DD y_n}^2}{x} - \intr{2xy_ny'_n}{x}+\intr{x^2y^2_n}{x}\\
    &= -\intr{y\DD^2y}{x} - \intr{2xy_ny'_n}{x}+\intr{x^2y^2_n}{x}.
  \end{align*}
  并注意
  \[ \DD^2y_n = x^2y_n - \pare{2n+1}y_n. \]
  还有
  \[ \intr{xy_ny'_n}{x} = -\intr{y^2_n}{x} - \intr{xy_ny'_n}. \]
  故
  \[ \intr{2xy_ny'_n}{x} = -\intr{y^2_n}{x}. \]
  由是
  \[ \intr{y^2_{n+1}}{x} = 2\pare{n+1}\intr{y^2_{n}}{x}. \]
  \refl{
    \[ \bk{y_n}{y_m} = \sqrt{\pi}2^nn!\delta_{nm}. \]
  }
  由Hermite函数引出的Hermite多项式定义如下:
  \refl{
    \[ H_n\pare{x} = \pare{-1}^n e^{x^2} \DD^n e^{-x^2}. \]
  }
  注意$y_n = e^{-x^2/2}H_n$,代入Hermite方程
  \[ e^{x^2/2}\DD^2 e^{-x^2/2}H_n - x^2 H_n = -\pare{2n+1}H_n. \]
  故
  \[ \pare{\DD-x}^2H_n-x^2H_n=-\pare{2n+1}H_n, \]
  即
  \refl{
    $H_n$是
    \[ y''-2xy'+2ny = 0 \]
    的解.
  }
  设函数
  \[ \Phi\pare{x,h} = e^{2xh-h^2} = \sum H_n\pare{x}\frac{h^n}{n!}. \]
  其满足
  \[ \Phi_{xx} - 2x\Phi_x+2h\Phi_h = 0. \]
  故
  \[ H_n''-2xH_n'+2nH_n = 0. \]
  欲得到此处的$H_n$正是上述的Hermite多项式,还需要另外的论证. 可以考虑证明
  \[ e^{-\pare{h-x}^2} = \sum e^{-x^2}H_n\pare{x}\frac{h^n}{n!}. \]
  也就是在$h=0$处,
  \[ e^{-\pare{h-x}^2} \]
  的$n$阶导数要等于(参考前文$H_n$的定义)
  \[ \pare{-1}^n\DD^n e^{-x^2}. \]
  这是显然的,可以将对$h$的求导看做对$-x$的求导,从而得到前面的$\pare{-1}^n$因子.
  \refl{
    \[ e^{2xh-h^2} = \sum H_n\pare{x}\frac{h^n}{n!}. \]
  }
  由
  \[ \Phi_x - 2h\Phi = 0, \]
  \refl{
    \[ H'_n = 2nH_{n-1}. \]
  }
  由
  \[ \Phi_h = 2x\Phi - 2h\Phi, \]
  \refl{
    \[ H_{n+1} = 2xH_n - 2nH_{n-1}. \]
  }
  \paragraph{幂级数}
\end{document}