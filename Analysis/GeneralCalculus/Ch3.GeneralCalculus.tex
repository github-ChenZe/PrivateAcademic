%Ch1.GeneralSetTheory.tex
\ifx\allfiles\undefined
\documentclass{ctexrep}
\usepackage{amsmath}
\usepackage{amssymb}
\usepackage{amsthm}
\usepackage{amsfonts}
\usepackage{mathrsfs}
\usepackage{enumitem}
\usepackage{braket}
\usepackage{hyperref}


\newcommand{\pare}[1]{\left(#1\right)}
\newcommand{\blr}[1]{\left[#1\right)}
\newcommand{\lbr}[1]{\left(#1\right]}
\newcommand{\brac}[1]{\left[#1\right]}
\newcommand{\curb}[1]{\left\{#1\right\}}
\newcommand{\abs}[1]{\left|\, #1 \,\right|}
\newcommand{\rec}[1]{\frac{1}{#1}}
\newcommand{\N}{\mathbb{N}}
\newcommand{\Q}{\mathbb{Q}}
\newcommand{\Z}{\mathbb{Z}}
\newcommand{\R}{\mathbb{R}}
\newcommand{\unk}{\mathcal{X}}
\newcommand{\bu}[3]{#1_{#2}^{\pare{#3}}}
\newcommand{\dref}[1]{定义\ref{def:#1}}
\newcommand{\tref}[1]{定理\ref{thm:#1}}
\newcommand{\lref}[1]{引理\ref{lem:#1}}
\newcommand{\cref}[1]{推论\ref{coll:#1}}
\newcommand{\pref}[1]{命题\ref{prp:#1}}
\newcommand{\func}[3]{#1:\, #2 \rightarrow #3}
\newcommand{\overbar}[1]{\mkern 1.5mu\overline{\mkern-1.5mu#1\mkern-1.5mu}\mkern 1.5mu}
\newcommand{\clo}[1]{\overbar{#1}}
\newcommand{\supi}[2]{\overbar{\int_{#1}^{#2}}}
\newcommand{\infi}[2]{\underbar{\int_{#1}^{#2}}}
\newcommand{\setf}{\mathscr}
\newcommand{\bool}{\mathrm{bool}}
\newcommand{\inc}{++}
\newcommand{\defeq}{:=}
\newcommand{\ntuple}{$n$元组}
\newcommand{\card}[1]{\#\pare{#1}}
\newcommand{\setcond}[2]{\curb{#1 \, \left| \, #2 \right.}}
\newcommand{\setcondl}[2]{\curb{\left. #1 \, \right| \, #2}}
\newcommand{\bv}[1]{\mathbf{#1}}
\newcommand{\bfa}{\bv{a}}
\newcommand{\bfb}{\bv{b}}
\newcommand{\bfx}{\bv{x}}
\newcommand{\bfy}{\bv{y}}
\newcommand{\bfe}{\bv{e}}
\newcommand{\bfF}{\bv{F}}
\newcommand{\bff}{\bv{f}}
\newcommand{\bfG}{\bv{G}}
\newcommand{\bfH}{\bv{H}}
\newcommand{\bfg}{\bv{g}}
\newcommand{\bfh}{\bv{h}}
\newcommand{\bfr}{\bv{r}}
\newcommand{\bfk}{\bv{k}}
\newcommand{\bfu}{\bv{u}}
\newcommand{\bfv}{\bv{v}}
\newcommand{\oo}[1]{o\pare{#1}}
\newcommand{\OO}[1]{O\pare{#1}}
\newcommand{\norm}[1]{\left\| #1 \right\|}
\newcommand{\DD}{\mathbf{D}}
\newcommand{\comp}{\circ}
\newcommand{\const}{\mathrm{const}}
\newcommand{\dist}[2]{d\pare{#1,#2}}
\newcommand{\len}{\ell}
\newcommand{\siga}{$\sigma$-代数}
\newcommand{\cara}{Carath\'{e}odory}
\newcommand{\Gd}{G_\delta}
\newcommand{\Fs}{F_\sigma}
\newcommand{\mmani}{$m$-维流形}
\newcommand{\open}[1]{\mathcal{#1}}
\newcommand{\half}{\frac{1}{2}}
\newcommand{\maxo}[1]{\text{max}\curb{#1}}
\newcommand{\mino}[1]{\text{min}\curb{#1}}
\newcommand{\epsclo}{$\epsilon$-接近}
\newcommand{\close}[1]{$#1$-接近}
\newcommand{\cinf}{$C^\infty$}
\newcommand{\cuno}{$C^1$}
\newcommand{\Int}{\text{Int}\,}
\newcommand{\Ext}{\text{Ext}\,}
\newcommand{\funcf}{\mathcal}
\newcommand{\DDu}{\overbar{\DD}}
\newcommand{\DDl}{\underbar{\DD}}
\newcommand{\Diff}[1]{\mathrm{Diff}_{#1}\,}
\newcommand{\Av}[1]{\mathrm{Av}_{#1}\,}
\newcommand{\Lip}[1]{Lipschitz-$#1$}
\newcommand{\sgn}[1]{\mathrm{sgn}}
\newcommand{\eset}{\varnothing}
\newcommand{\cT}{\mathcal{T}}
\newcommand{\cS}{\mathcal{S}}
\newcommand{\cG}{\mathcal{G}}
\newcommand{\cF}{\mathcal{F}}
\newcommand{\cC}{\mathcal{C}}
\newcommand{\cB}{\mathcal{B}}

\newcommand{\hd}{H\"{o}lder}

\renewcommand{\proofname}{证明}

\newenvironment{cenum}{\begin{enumerate}\itemsep0em}{\end{enumerate}}

\newtheorem{definition}{定义}[section]
\newtheorem{lemma}{引理}[section]
\newtheorem{theorem}{定理}[section]
\newtheorem{collary}{推论}[section]
\newtheorem{proposition}{命题}[section]
\newtheorem{axiom}{公理}[section]
\newtheorem{ex}{例}[section]

\newenvironment{aenum}{\begin{enumerate}[label=\textnormal{(\alph*)}]}{\end{enumerate}}
\begin{document}
\fi

%Content

\chapter{普通微积分}
  \section{微分}
  \subsection{实函数的导数}
  \begin{definition}
    极限
    \[ f'\pare{x} = \frac{f\pare{t\rightarrow x} - f\pare{x}}{\pare{t\rightarrow x}-x} \]
    若存在则谓$f\pare{x}$的导数。
  \end{definition}
  \begin{theorem}
    函数在可微点连续。
  \end{theorem}
  \begin{theorem}
    若$f$和$g$都在$x$处可微,则此处
    \begin{cenum}
      \item $\pare{f+g}' = f'+g'$;
      \item $\pare{fg}'=f'g+fg'$;
      \item $\pare{f/g}'=\pare{f'g-fg'}/\pare{g^2}.$
    \end{cenum}
  \end{theorem}
  \begin{theorem}
    若$f$连续且在$x$处可微,$g$在$f\pare{x}$处可微,则
    \[ \pare{g\comp f}'\pare{x} = g'\pare{f\pare{x}}f'\pare{x}. \]
  \end{theorem}
  \begin{ex}
    $f\pare{x}=x\sin 1/x \otw 0$,则$f$在$x=0$处不可微。
  \end{ex}
  \begin{ex}
    $f\pare{x}=x^2\sin 1/x \otw 0$在任意点可微且$f'\pare{0}=0$,则$f$处处可微但$f'$不连续。
  \end{ex}
  \subsection{中值定理}
  \begin{definition}
    若存在$\delta>0$满足$d\pare{p,q}<\delta\Rightarrow f\pare{q}\le f\pare{p}$则谓之局部极大值。类似定义极小值。
  \end{definition}
  \begin{theorem}
    $f\pare{x}$若取得局部极大(小)值而$f'\pare{x}$存在,则$f'\pare{x}=0$。
  \end{theorem}
  \begin{theorem}
    $\brac{a,b}$上连续的$f$和$g$在$\pare{a,b}$上可微则存在$x$满足
    \[ \brac{f\pare{b}-f\pare{a}} g'\pare{x} = \brac{g\pare{b}-g\pare{a}}f'\pare{x}. \]
  \end{theorem}
  \begin{proof}
    两侧做差并发现原函数在两段点水平。
  \end{proof}
  取$g\pare{x}=x$,有
  \begin{theorem}
    $\brac{a,b}$上连续的$f$可微则存在$x$满足
    \[ {f\pare{b}-f\pare{a}} = \pare{b-a}f'\pare{x}. \]
  \end{theorem}
  \begin{theorem}
    在$\pare{a,b}$内可微的f若恒有$f'\pare{x}\ge 0$则$f$单调递增,类似有常数和递减。
  \end{theorem}
  \subsection{导数的连续性}
  \begin{theorem}
    设$f$在$\brac{a,b}$可微,又设$f'\pare{a}<\lambda<f'\pare{b}$,则存在$f'\pare{x}=\lambda$。
  \end{theorem}
  \begin{proof}
    构造$g\pare{x}=f\pare{x}-\lambda x$并注意其取得最小值即可。
  \end{proof}
  \begin{corollary}
    若$f$在$\brac{a,b}$上可微,则$f'$不能有第一类间断。
  \end{corollary}
  \subsection{L'Hospital法则}
  \begin{theorem}
    若$f$和$g$在$\pare{a,b}$($a$和$b$可取无穷)内可微且$g'\pare{x}\neq 0$,若$f\pare{x\rightarrow a}\rightarrow 0$以及$g\pare{x\rightarrow a}\rightarrow 0$或$g\pare{x\rightarrow a}\rightarrow \pm\infty$则
    \[ \frac{f'\pare{x\rightarrow a}}{g'\pare{x\rightarrow a}}\rightarrow A \Rightarrow \frac{f\pare{x\rightarrow a}}{g\pare{x\rightarrow a}} \rightarrow A. \]
    上述$a$替换为$b$仍然成立。
  \end{theorem}
  \begin{proof}
    对任意$q>A$,当$x,y,t$足够接近$a$时,有$f'\pare{t}/g'\pare{t}<q^-$,故
    \[ \frac{f\pare{x}-f\pare{y}}{g\pare{x}-g\pare{y}} = \frac{f'\pare{t}}{g'\pare{t}} < q^-. \]
    对双零的情形单方面让$y\rightarrow0$即可,当$g\pare{x\rightarrow a}\rightarrow a$时令$x,y$再次足够接近$a$则两边通分有
    \[ \frac{f\pare{x}}{g\pare{x}} < q^- - q^- \frac{g\pare{y}}{g\pare{x}}+\frac{f\pare{y}}{f\pare{x}}. \]
  \end{proof}
  \subsection{高阶导数}
  \begin{definition}
    若$f$在一个区间上有导数$f'$,$f'$可微则谓二阶导数$f''$,类似定义高阶者。
  \end{definition}
  \subsection{Taylor定理}
  \begin{theorem}
    若$f$为闭区间上实函数,$f\difn{n-1}$连续且$f\difn{n}$存在,令
    \[ P\pare{t} = \sum_{k=0}^{n-1} \frac{f\difn{k}\pare{\alpha}}{k!}\pare{t-\alpha}^k. \]
    则对于给定的$\beta$,存在$\brac{\alpha, \beta}$间一点$x$满足
    \[ f\pare{\beta} = P\pare{\beta} + \frac{f\difn{n}\pare{x}}{n!}\pare{\beta-\alpha}^n. \]
  \end{theorem}
  \begin{proof}
    定义$g\pare{t} = f\pare{t} - P\pare{t} - M\pare{t-\alpha}^n$,注意$g\difn{0}=\cdots=g\difn{n-1}\pare{\alpha}=0$。选取$M$满足$g\pare{\beta}=0$,故存在$g'\pare{\beta^-_1}=0$,存在$g''\pare{\beta^-_2}=0$等。
  \end{proof}
  \subsection{向量值函数的微分}
  前开加减乘除法则将$f$设定为向量值函数后仍然有效,然而L'Hospital法则不再有效。
  \begin{ex}
    定义$f\pare{x}=x$与$g\pare{x}=x+x^2 e^{i/x^2}$,则显然$f\pare{x}/g\pare{x}\rightarrow1$。然而,$g'=1+\pare{2x-2i/x}e^{i/x^2}$,$\abs{f'\pare{x\rightarrow0}/g'\pare{x\rightarrow0}}=0$。
  \end{ex}
  类似于中值定理,有
  \begin{theorem}
    \label{thm:vfab}
    设连续的$\bff$在$\pare{a,b}$内可微并将$\brac{a,b}$映入$\R^k$,则必有$x\in\pare{a,b}$满足$\abs{\bff\pare{b}-\bff\pare{a}}\le\pare{b-a}\abs{\bff'\pare{x}}$。
  \end{theorem}
  \begin{proof}
    不妨设$a=0$,$b=1$,$\bff\pare{a}=\vzero$。令$\varphi\pare{t}=\bff\pare{t}\cdot\pare{1}$后中值定理
    \[ \bff\pare{1}^2 = \bff'\pare{x}\cdot \bff\pare{1} \le \abs{\bff'\pare{x}}\abs{\bff\pare{1}}. \qedhere \]
  \end{proof}
  \section{Riemann-Stieltjes积分}
  \subsection{积分的定义和存在性}
  \begin{definition}
    $\brac{a,b}$的分法谓其间一列非减点集。在每个区间$\brac{x_{i-1},x_i}$内定义$M_i=\sup f$,$m_i=\inf f$,又定义
    \begin{align*}
      U\pare{P,f} &= \sum M_i \Delta x_i, \\
      L\pare{P,f} &= \sum m_i \Delta x_i.
    \end{align*}
    定义上积分与下积分为
    \begin{align*}
      \supint{a}{b} f\intpostf{x} &= \inf U\pare{P,f},\\
      \infint{a}{b} f\intpostf{x} &= \sup L\pare{P,f}.
    \end{align*}
    如果上下积分相等,则谓之Riemann可积。
  \end{definition}
  易见有界函数的上下积分均存在。
  \begin{definition}
    设$\alpha$是$\brac{a,b}$上的单调递增函数,在前开定义中将$\Delta x_i$替换为$\Delta \alpha_i$则可得Riemann-Stieltjes积分。
  \end{definition}
  \begin{theorem}
    对于$P$的加细$P^*$有
    \begin{equation*}
      L\pare{P,f,\alpha}\le L\pare{P^*,f,\alpha}\le U\pare{P^*,f,\alpha}\le U\pare{P,f,\alpha}.
    \end{equation*}
  \end{theorem}
  \begin{theorem}
    $\infint{}{} \le \supint{}{}.$
  \end{theorem}
  \begin{theorem}
    $\brac{a,b}$上的$f$可积当且仅当对于任意$\epsilon$都存在分法满足
    \begin{equation}
      \label{eq:ULe}
      U\pare{P,f,\alpha} - L\pare{P,f,\alpha} < \epsilon.
    \end{equation}
  \end{theorem}
  \begin{proof}
    注意若条件满足,则意味着$\abs{\supint{}{}-\infint{}{}}<\epsilon$。
  \end{proof}
  \begin{theorem}
    如果\eqref{eq:ULe}成立,则加细后仍然成立。
  \end{theorem}
  \begin{theorem}
    \label{thm:stbetweenx}
    如果\eqref{eq:ULe}成立,在$\brac{x_{i-1},x_i}$之间选取$s_i$和$t_i$,则
    \[ \sum \abs{f\pare{s_i}-f\pare{t_i}}\Delta\alpha_i < \epsilon. \]
  \end{theorem}
  \begin{theorem}
    \label{thm:intervalrepr}
    在前开题设下,若$f$可积,则
    \[ \abs{\sum f\pare{t_i}\Delta\alpha_i - \int_{a}^{b} f \intpostf{\alpha}}<\epsilon. \]
  \end{theorem}
  \begin{theorem}
    闭区间上连续函数对任意$\alpha$可积。
  \end{theorem}
  \begin{theorem}
    闭区间上单调函数对单调连续的$\alpha$可积。
  \end{theorem}
  \begin{proof}
    选取分割$P_n$满足$\Delta\alpha_i=\Delta\alpha/n$,从而$U-L=\Delta\alpha_i\Delta f$。 
  \end{proof}
  \begin{theorem}
    若$f$在$\brac{a,b}$上有界且只有有限间断点,$\alpha$在$f$的每个间断点上连续,则$f$可积。
  \end{theorem}
  \begin{proof}
    选取小区间包含$f$的诸间断点且其内$\alpha$连续,这些区间内因$\alpha$连续而上下差任意小,剩余区间内由于$f$连续而上下差任意小。
  \end{proof}
  \begin{theorem}
    $\brac{a,b}$上的$f$可积而$\phi$在其像上连续,则$\phi\comp f$可积。
  \end{theorem}
  \begin{proof}
    若$\Delta s<\delta \Rightarrow \Delta\phi<\epsilon$,选取$P$使$U\pare{f}-L\pare{f}<\delta^2$,将区间分两类,$\Delta f < \delta$及相反,后者的区间满足$\sum\Delta\alpha_i<\delta$。总的$\sum\Delta\phi\comp f$为小量。
  \end{proof}
  \subsection{积分及其性质}
  \begin{theorem}
    对于积分,下列性质成立:
    \begin{cenum}
      \item 闭区间可积函数构成线性空间,积分是线性算子;
      \item $f_1\le f_2\Rightarrow \int f_1 \le \int f_2$;
      \item 若$\brac{a,b}$上$f$可积,则$\brac{a,c}$与$\brac{c,b}$上可积且$\int_a^c + \int_c^b=\int_a^b$;
      \item 若$f$可积且$\abs{f}\le M$,则$\abs{\int f} \le M \brac{\alpha\pare{b}-\alpha\pare{a}}$;
      \item 积分与可积性对$\alpha$为线性。
    \end{cenum}
  \end{theorem}
  \begin{theorem}
    若$f$和$g$皆可积则$fg$可积。
  \end{theorem}
  \begin{theorem}
    若$f$可积则$\abs{f}$可积且$\abs{\int f}\le \int \abs{f}$。
  \end{theorem}
  \begin{definition}
    单位阶跃函数定义为$I\pare{x}=\chi_{\pare{0,+\infty}}$。
  \end{definition}
  \begin{definition}
    若$f$在$\brac{a,b}$上有界且$\alpha\pare{x}=I\pare{x-s}=\chi_{\pare{s,+\infty}}$,则
    \[ \int_{a}^{b} f \intpostf{\alpha} = f\pare{s}. \]
  \end{definition}
  \begin{theorem}
    设$\curb{c_n>0}$且$\sum c_n$收敛,$\curb{s_n}$两两不同,
    \[ \alpha\pare{x} = \sum c_n I\pare{x-s_n}. \]
    若$f$在$\brac{a,b}$上连续,则
    \[ \int_a^b f\intpostf{\alpha} = \sum c_n f\pare{s_n}. \]
  \end{theorem}
  \begin{proof}
    选取足够大的$N$满足$\sum c_n-\sum^N c_n < \epsilon$后将$\alpha$拆分即可。
  \end{proof}
  \begin{theorem}
    若闭区间上$\alpha$单调递增且$\alpha'$可积,$f$有界,则$f$可积当且仅当$f\alpha'$可积,且
    \[ \int f\intpostf{\alpha} = \int f\alpha'. \]
  \end{theorem}
  \begin{proof}
    设$U\pare{P,\alpha'}-L\pare{P,\alpha'}<\epsilon$而$f$上界为$M$,根据中值定理与\tref{stbetweenx},
    \[ \sum f_i\Delta\alpha_i = \sum f_i\alpha'\pare{s_i}\Delta x, \quad \sum f_i \abs{\alpha'\pare{s_i} - \alpha'\pare{t_i}} \Delta x \le M\epsilon. \]
    \[ \abs{\sum f\pare{s_i}\Delta\alpha_i - \sum f\pare{s_i}\alpha'\pare{s_i}\Delta x_i} \le M\epsilon. \]
    因此,$\abs{\switch{L}{U}\pare{f,\alpha}-\switch{L}{U}\pare{f,\alpha'}}\le M\epsilon$,任意加细后仍然成立。
  \end{proof}
  \begin{theorem}
    设有严格递增的满射$\func{\varphi}{\brac{A,B}}{\brac{a,b}}$,$f$在$\brac{a,b}$上对$\alpha$可积,定义$\beta=\alpha\comp\varphi$而$g=f\comp \varphi$,则$g$对$\beta$可积且
    \[ \int_A^B g \intpostf{\beta} = \int_a^b f\intpostf{\alpha}. \]
  \end{theorem}
  \begin{proof}
    注意$\varphi$只是一个坐标变换,把分法映射到分法即可。
  \end{proof}
  取$\alpha\pare{x}=x$,就有
  \[ \int_a^b f\pare{x}\intpostf{x} = \int_A^B  f\pare{\varphi\pare{y}}\varphi'\pare{y}\intpostf{y}. \]
  \subsection{积分与微分}
  \begin{theorem}
    设$f$可积且
    \[ F\pare{x} = \int_a^x f\pare{x}\intpostf{x}, \]
    则$F$在$\brac{a,b}$上一致连续且若$f$在$x_0$处连续则$F$可微且$F'\pare{x_0} = f\pare{x_0}$。
  \end{theorem}
  \begin{proof}
    做差,将$F$的差和积分相互转换。
  \end{proof}
  \begin{theorem}
    \label{thm:elecalculus}
    若$\brac{a,b}$上$f$可积且$F'=f$,则
    \[ \int_a^b f\intpostf{x} = F\pare{b}-F\pare{a}. \]
  \end{theorem}
  \begin{proof}
    在某分割内选取有代表性的$t_i$后对下式调用\tref{intervalrepr}。
    \[ \sum f\pare{t_i}\Delta x_i = F\pare{b}-F\pare{a}.\qedhere \]
  \end{proof}
  \subsection{向量值函数的积分}
  \begin{definition}
    定义区间上的向量值函数的积分以诸分量的积分为分量。
  \end{definition}
  \begin{theorem}[\tcompare{elecalculus}]
    若可积的$\vf$和$\vF$将$\brac{a,b}$映入$R^k$,且$\vF'=\vf$则
    \[ \int_a^b \vf = \vF\pare{b} - \vF\pare{a}. \]
  \end{theorem}
  \begin{theorem}
    \label{thm:ineqtriint}
    若$\vf$如上且$\vf$对$\alpha$可积,则$\abs{\vf}$可积且
    \[ \abs{\int_a^b\vf} \le \int_a^b \abs{\vf}. \]
  \end{theorem}
  \begin{proof}
    可积性由连续函数复合可知,由Schwarz不等式,有
    \[ \abs{\int\vf}^2 = \sum\int \pare{\int f_i}\cdot  f_i \le \int \abs{\int \vf} \abs{\vf}. \qedhere \]
  \end{proof}
  \subsection{可求长曲线}
  \begin{definition}
    将闭区间映入$\R^k$内的映射谓曲线,双射谓弧,区间端点处取值相等谓闭曲线。
  \end{definition}
  \begin{definition}
    对闭区间的分法$P$定义
    \[ \Lambda\pare{P,\gamma} = \sum\abs{\gamma\pare{x_i}-\gamma\pare{x_{i-1}}}. \]
    定义$\Lambda\pare{\gamma}=\sup \Lambda\pare{P,\gamma}$。其值有限则谓曲线可求长。
  \end{definition}
  \begin{theorem}
    $\gamma'$连续则曲线可求长且
    \[ \Lambda\pare{\gamma} = \int_a^b \abs{\gamma'\pare{t}}\intpostf{t}. \]
  \end{theorem}
  \begin{proof}
    对$\gamma'$调用积分的三角不等式\tref{ineqtriint}的
    \[ \Lambda\pare{\gamma}\le \int \abs{\gamma'}. \]
    反向的不等式注意小区间上的下列不等式加和即可
    \begin{align*}
      \int \abs{\gamma'} &\le \abs{\gamma'\pare{x_i}}\Delta x_i + \epsilon\Delta x_i\\
      &\le \abs{\int \gamma'} + \abs{\int\brac{\gamma'\pare{x_i}-\gamma'}} + \epsilon \Delta x_i.\qedhere
    \end{align*}
  \end{proof}
  
%ContentEnds
 
\ifx\allfiles\undefined %如果位置放错,可能出现意外中断
\end{document}
\fi