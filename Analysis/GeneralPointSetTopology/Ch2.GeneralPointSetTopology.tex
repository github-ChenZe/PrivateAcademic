%Ch2.GeneralPointSetTopology.tex
\ifx\allfiles\undefined
\documentclass{ctexrep}
\usepackage{amsmath}
\usepackage{amssymb}
\usepackage{amsthm}
\usepackage{amsfonts}
\usepackage{mathrsfs}
\usepackage{enumitem}
\usepackage{braket}
\usepackage{hyperref}


\newcommand{\pare}[1]{\left(#1\right)}
\newcommand{\blr}[1]{\left[#1\right)}
\newcommand{\lbr}[1]{\left(#1\right]}
\newcommand{\brac}[1]{\left[#1\right]}
\newcommand{\curb}[1]{\left\{#1\right\}}
\newcommand{\abs}[1]{\left|\, #1 \,\right|}
\newcommand{\rec}[1]{\frac{1}{#1}}
\newcommand{\N}{\mathbb{N}}
\newcommand{\Q}{\mathbb{Q}}
\newcommand{\Z}{\mathbb{Z}}
\newcommand{\R}{\mathbb{R}}
\newcommand{\unk}{\mathcal{X}}
\newcommand{\bu}[3]{#1_{#2}^{\pare{#3}}}
\newcommand{\dref}[1]{定义\ref{def:#1}}
\newcommand{\tref}[1]{定理\ref{thm:#1}}
\newcommand{\lref}[1]{引理\ref{lem:#1}}
\newcommand{\cref}[1]{推论\ref{coll:#1}}
\newcommand{\pref}[1]{命题\ref{prp:#1}}
\newcommand{\func}[3]{#1:\, #2 \rightarrow #3}
\newcommand{\overbar}[1]{\mkern 1.5mu\overline{\mkern-1.5mu#1\mkern-1.5mu}\mkern 1.5mu}
\newcommand{\clo}[1]{\overbar{#1}}
\newcommand{\supi}[2]{\overbar{\int_{#1}^{#2}}}
\newcommand{\infi}[2]{\underbar{\int_{#1}^{#2}}}
\newcommand{\setf}{\mathscr}
\newcommand{\bool}{\mathrm{bool}}
\newcommand{\inc}{++}
\newcommand{\defeq}{:=}
\newcommand{\ntuple}{$n$元组}
\newcommand{\card}[1]{\#\pare{#1}}
\newcommand{\setcond}[2]{\curb{#1 \, \left| \, #2 \right.}}
\newcommand{\setcondl}[2]{\curb{\left. #1 \, \right| \, #2}}
\newcommand{\bv}[1]{\mathbf{#1}}
\newcommand{\bfa}{\bv{a}}
\newcommand{\bfb}{\bv{b}}
\newcommand{\bfx}{\bv{x}}
\newcommand{\bfy}{\bv{y}}
\newcommand{\bfe}{\bv{e}}
\newcommand{\bfF}{\bv{F}}
\newcommand{\bff}{\bv{f}}
\newcommand{\bfG}{\bv{G}}
\newcommand{\bfH}{\bv{H}}
\newcommand{\bfg}{\bv{g}}
\newcommand{\bfh}{\bv{h}}
\newcommand{\bfr}{\bv{r}}
\newcommand{\bfk}{\bv{k}}
\newcommand{\bfu}{\bv{u}}
\newcommand{\bfv}{\bv{v}}
\newcommand{\oo}[1]{o\pare{#1}}
\newcommand{\OO}[1]{O\pare{#1}}
\newcommand{\norm}[1]{\left\| #1 \right\|}
\newcommand{\DD}{\mathbf{D}}
\newcommand{\comp}{\circ}
\newcommand{\const}{\mathrm{const}}
\newcommand{\dist}[2]{d\pare{#1,#2}}
\newcommand{\len}{\ell}
\newcommand{\siga}{$\sigma$-代数}
\newcommand{\cara}{Carath\'{e}odory}
\newcommand{\Gd}{G_\delta}
\newcommand{\Fs}{F_\sigma}
\newcommand{\mmani}{$m$-维流形}
\newcommand{\open}[1]{\mathcal{#1}}
\newcommand{\half}{\frac{1}{2}}
\newcommand{\maxo}[1]{\text{max}\curb{#1}}
\newcommand{\mino}[1]{\text{min}\curb{#1}}
\newcommand{\epsclo}{$\epsilon$-接近}
\newcommand{\close}[1]{$#1$-接近}
\newcommand{\cinf}{$C^\infty$}
\newcommand{\cuno}{$C^1$}
\newcommand{\Int}{\text{Int}\,}
\newcommand{\Ext}{\text{Ext}\,}
\newcommand{\funcf}{\mathcal}
\newcommand{\DDu}{\overbar{\DD}}
\newcommand{\DDl}{\underbar{\DD}}
\newcommand{\Diff}[1]{\mathrm{Diff}_{#1}\,}
\newcommand{\Av}[1]{\mathrm{Av}_{#1}\,}
\newcommand{\Lip}[1]{Lipschitz-$#1$}
\newcommand{\sgn}[1]{\mathrm{sgn}}
\newcommand{\eset}{\varnothing}
\newcommand{\cT}{\mathcal{T}}
\newcommand{\cS}{\mathcal{S}}
\newcommand{\cG}{\mathcal{G}}
\newcommand{\cF}{\mathcal{F}}
\newcommand{\cC}{\mathcal{C}}
\newcommand{\cB}{\mathcal{B}}

\newcommand{\hd}{H\"{o}lder}

\renewcommand{\proofname}{证明}

\newenvironment{cenum}{\begin{enumerate}\itemsep0em}{\end{enumerate}}

\newtheorem{definition}{定义}[section]
\newtheorem{lemma}{引理}[section]
\newtheorem{theorem}{定理}[section]
\newtheorem{collary}{推论}[section]
\newtheorem{proposition}{命题}[section]
\newtheorem{axiom}{公理}[section]
\newtheorem{ex}{例}[section]

\newenvironment{aenum}{\begin{enumerate}[label=\textnormal{(\alph*)}]}{\end{enumerate}}
\begin{document}
\fi

%Content

\chapter{普通点集拓扑}
  \section{拓扑空间与连续函数}
  \subsection{拓扑空间}
  \begin{definition}
  集合$X$上的一个拓扑$\cT$谓$X$的一满足如下条件的子集族:
  \begin{cenum}
    \item $\curb{\eset, X} \in \cT$;
    \item $\cT$中元素的任意并仍在$\cT$中;
    \item $\cT$中元素的有限交仍在$\cT$中。
  \end{cenum}
  \end{definition} 
  \begin{definition}
  $X$的所有子集构成的拓扑谓离散拓扑。
  \end{definition}
  \begin{definition}
  由$X$和$\eset$构成的拓扑谓密着拓扑。
  \end{definition}
  \begin{definition}
  由$X$本身与所有满足$X-U$为有限集的$U$构成的拓扑谓有限补拓扑。
  \end{definition}
  \begin{definition}
  $\cT'\supset\cT$则$\cT'$细于$\cT$,反之则谓粗于。
  \end{definition}
  如果把开集比做石子,把石子打碎就得到更细的拓扑。
  \subsection{拓扑的基}
  \begin{definition}
  \label{def:tb}
  基$\cB$谓满足如下条件的子集族:
  \begin{cenum}
    \item 对任意$x\in X$,存在$B\in\cB$满足$x\in B$;
    \item 对任意$x\in B_1\cap B_2$,存在$B$满足$x\in B$且$B\subset B_1\cap B_2$。
  \end{cenum}
  \end{definition}
  注意此定义不针对具体的拓扑。
  \begin{ex}
  平面上的圆域和矩形域构成的集族都构成基。
  \end{ex}
  \begin{definition}
  满足\dref{tb}的$\cB$生成的拓扑为所有满足对$x\in U$,存在$x\in B\subset U$的$U$的集族。
  \end{definition}
  可以直接验证上述定义构成一个拓扑。对所有$x$取对应的$x\in B_x$后将诸$B_x$并起,可得等价的表述
  \begin{theorem}
  若$\cB$为$\cT$的基,则$\cT$为$\cB$中元素并的族。
  \end{theorem}
  \begin{theorem}
  设$\cC$为开集族,若对于任意开集$U$中任意$x$,存在$C\in\cC$满足$x\in C\subset U$,则$\cC$为$\cT$的基。
  \end{theorem}
  \begin{proof}
  容易验证$\cC$为基。再分别证$\cC\subset \curb{U}$与$\curb{\cup C}\supset \curb{U}$。
  \end{proof}
  \begin{theorem}
  设$\cB$于$\cB'$分别生成$\cT$与$\cT'$,则$\cT'$细于$\cT$当且仅当对任意$x\in B$存在$x\in B'\subset B$。
  \end{theorem}
  \begin{proof}
  强行带入定义,即任意$U$均在$\cT'$内即可。
  \end{proof}
  \begin{definition}
  $\R$上的$\pare{a,b}$生成的拓扑谓标准拓扑。
  \end{definition}
  \begin{definition}
  $\R$上$\blr{a,b}$生成的拓扑谓下限拓扑,记作$\R_\ell$。
  \end{definition}
  \begin{definition}
  $\R$上$\pare{a,b}$与$\pare{a,b} - \curb{\rec{n}}$生成的拓扑谓K-拓扑,记作$\R_K$。
  \end{definition}
  \begin{lemma}
  $\R_\ell$与$\R_K$严格细于标准拓扑,但它们之间不可比较。
  \end{lemma}
  \begin{proof}
  $\R_K$\emph{严格}细于的证明只需考虑$x=0$与$B=\pare{-1,1}-\curb{1/n}$,同一个集合可证$\R_\ell$不细于$R_K$。
  \end{proof}
  \begin{definition}
  子基$\cS$谓满足$\cup S=X$的集族。
  \end{definition}
  \begin{definition}
  子基生成的拓扑谓$\cS$中有限交的所有并。
  \end{definition}
  可以直接验证$\curb{\cap S}$为一个基,故其确实生成一拓扑。
  \subsection{序拓扑}
  \begin{definition}
  具有全序关系的$X$上的序拓扑谓所有$\pare{a,b}$,$\lbr{a,\max X}$,$\blr{\min X, b}$生成的拓扑。
  \end{definition}
  \begin{ex}
  $\Z_+$上的序拓扑是离散拓扑。然而$X=\curb{1,2}\times\Z_+$的字典序拓扑下单点集$1\times1$并非开集。
  \end{ex}
  \begin{definition}
  全序集$X$中$a$决定的射线谓开射线$\pare{a,+\infty}$,$\pare{-\infty,a}$,$\blr{a,+\infty}$,$\lbr{-\infty,a}$。
  \end{definition}
  所有开射线构成$X$的序拓扑的子基。
  \subsection{积拓扑}
  \begin{definition}
  $X\times Y$上的积拓扑谓所有$U\times V$的集族$\cB$生成的拓扑,其中$U$与$V$为$X$与$Y$中的开集。
  \end{definition}
  \begin{theorem}
  若$\cB$与$\cC$分别为$X$与$Y$的基,则$\cB\times\cC$为$X\times Y$的基。
  \end{theorem}
  \begin{definition}
  投射$\pi_1\pare{x,y}=x$,$\pi_2\pare{x,y}=y$。
  \end{definition}
  \begin{theorem}
  如下的$\cS$构成$X\times Y$的一子基,其中$U$和$V$分别为$X$与$Y$中的开集。
  \[ \cS = \curb{\pi_1^{-1}\pare{U}}\cup\curb{\pi_2^{-1}\pare{V}}. \]
  \end{theorem}
  \subsection{子空间拓扑}
  \begin{definition}
  对$X$的子集$Y$定义子空间拓扑,其中$U$为$X$中的开集。
  \[ \cT_Y = \curb{Y\cap U}. \]
  \end{definition}
  \begin{theorem}
  若$\cB$为$X$的一个基,则
  \[ \cB_Y = \setcond{B\cap Y}{B \in \cB} \]
  谓$Y$的子空间拓扑的一个基。
  \end{theorem}
  \begin{lemma}
  若$Y$为$X$中开集而$U$为$Y$中开集,则$U$为$X$中开集。
  \end{lemma}
  \begin{theorem}
  若$A\subset X$,$B\subset Y$,则$A\times B$的积拓扑与其自$X\times Y$继承的子空间拓扑相符。
  \end{theorem}
  然而,对于序拓扑无类似结论。
  \begin{ex}
  考虑$X=\R$而$Y=\brac{0,1}$,$Y$上的序拓扑与子空间拓扑相符。
  \end{ex}
  \begin{ex}
  考虑$X=\R$而$Y=\blr{0,1}\cup\curb{2}$,子空间拓扑中$\curb{2}$为开集,二者不符。
  \end{ex}
  \begin{ex}
  考虑$X=\R^2$而$Y=\brac{0,1}\times\brac{0,1}$,则$\half\times\lbr{\half,1}$为子空间拓扑的开集但不是序拓扑的开集。
  \end{ex}
  \begin{definition}
  子集$Y$称为凸的,如果对$Y$中$a<b$皆有$\pare{a,b}\subset Y$。
  \end{definition}
  \begin{theorem}
  设$X$为全序集,$Y$为凸子集,则子空间拓扑与序拓扑一致。
  \end{theorem}
  \begin{proof}
  借助开射线构造子基后证明其相互包含即可。
  \end{proof}
  \subsection{闭集与极限点}
  \subsubsection{Hausdorff空间}
  \begin{definition}
    若$X$中任意两不同点存在无交邻域,则称$X$为一Hausdorff空间(Hausdorff space)。
  \end{definition}
  \section{连通性与紧致性}
  \subsection{紧致空间}
  \begin{definition}
    $X$的子集族$\setf{C}$称为具有有限交性质(finite intersection property),如果$\setf{C}$的任意有限子族交非空。
  \end{definition}
  \begin{theorem}
    \label{thm:finiteinters}
    $X$是紧致的当且仅当$X$中具有有限交性质的每一个闭集族$\setf{C}$,其交非空。
  \end{theorem}
  \begin{proof}
    这些集合的补是一堆开集,这些开集中的任意有限个都不能覆盖$X$,但$X$是紧致的,所以它们合起来也不能覆盖$X$。
  \end{proof}
  \subsection{实直线上的紧致子空间}
  \begin{theorem}
    \label{thm:uncountableR}
    非空紧致Hausforff空间$X$,若无孤立点则不可数。
  \end{theorem}
  \begin{proof}
    对于$X$的任意元素$x$,由Hausdorff性质皆可以选取一非空开集$V$,满足$x \notin \clo{V}$。\par
    假设有$\func{f}{\Z_+}{X}$,则可以选取$V_1$其闭包不包含$x$,且可选取$V_2 \subset V_1$其闭包不包含$x_2$,以此类推。考虑
    \[ \clo{V}_1 \supset \clo{V}_2 \supset \cdots, \]
    由$x$的紧致性与\tref{finiteinters},知其交非空故有元素$x$在诸$x_n$之外。
  \end{proof}
  \subsection{极限点紧致性}
  \begin{definition}
    度量空间内的映射$f$,若
    \[ \dist{f\pare{x}}{f\pare{y}} < \dist{x}{y}, \]
    则称$f$为收紧映射(shrinking map)。
  \end{definition}
    \begin{definition}
    度量空间内的映射$f$,若
    \[ \dist{f\pare{x}}{f\pare{y}} \le \alpha\dist{x}{y}, \]
    其中$\alpha < 1$,则称$f$为压缩映射(contraction map)。
  \end{definition}
  \begin{theorem}
    \label{thm:fixp0}
    若$X$为完备度量空间,则压缩映射存在不动点。
  \end{theorem}


%ContentEnds
 
\ifx\allfiles\undefined %如果位置放错,可能出现意外中断
\end{document}
\fi