%Ch1.GeneralSetTheory.tex
\ifx\allfiles\undefined
\documentclass{ctexrep}
% Mathematics Include

\usepackage{amsmath}
\usepackage{amssymb}
\usepackage{amsthm}
\usepackage{amsfonts}
\usepackage{mathrsfs}
\usepackage{enumitem}
\usepackage{braket}
\usepackage{hyperref}
\usepackage[all, pdf]{xy}

% Physics Include
\usepackage{amsmath}
\usepackage{physics}
\usepackage{siunitx}
\usepackage[makeroom]{cancel}
\usepackage{pstricks}
\usepackage{pstricks-add}
\psset{algebraic=true}

\usepackage[version=4]{mhchem}
\usepackage{array,booktabs}
\usepackage{longtable}
\usepackage{mathtools}
\usepackage[normalem]{ulem}
\usepackage{multicol}

% Mathematics Head

\newcommand{\pare}[1]{\left(#1\right)}
\newcommand{\blr}[1]{\left[#1\right)}
\newcommand{\lbr}[1]{\left(#1\right]}
\newcommand{\brac}[1]{\left[#1\right]}
\newcommand{\curb}[1]{\left\{#1\right\}}
% \newcommand{\abs}[1]{\left|\, #1 \,\right|}
\newcommand{\rec}[1]{\frac{1}{#1}}
\newcommand{\N}{\mathbb{N}}
\newcommand{\bC}{\mathbb{C}}
\newcommand{\Q}{\mathbb{Q}}
\newcommand{\Z}{\mathbb{Z}}
\newcommand{\R}{\mathbb{R}}
\newcommand{\unk}{\mathcal{X}}
\newcommand{\bu}[3]{#1_{#2}^{\pare{#3}}}
\newcommand{\dref}[1]{定义\ref{def:#1}}
\newcommand{\tref}[1]{定理\ref{thm:#1}}
\newcommand{\lref}[1]{引理\ref{lem:#1}}
\newcommand{\cref}[1]{推论\ref{coll:#1}}
\newcommand{\pref}[1]{命题\ref{prp:#1}}
\newcommand{\eref}[1]{例\ref{ex:#1}}
\newcommand{\func}[3]{#1:\, #2 \rightarrow #3}
\newcommand{\overbar}[1]{\mkern 1.5mu\overline{\mkern-1.5mu#1\mkern-1.5mu}\mkern 1.5mu}
\newcommand{\clo}[1]{\overbar{#1}}
\newcommand{\supi}[2]{\overbar{\int_{#1}^{#2}}}
\newcommand{\infi}[2]{\underbar{\int_{#1}^{#2}}}
\newcommand{\setf}{\mathscr}
\newcommand{\bool}{\mathrm{bool}}
\newcommand{\inc}{++}
\newcommand{\defeq}{:=}
\newcommand{\ntuple}{$n$元组}
\newcommand{\card}[1]{\#\pare{#1}}
\newcommand{\setcond}[2]{\curb{#1 \, \left| \, #2 \right.}}
\newcommand{\setcondl}[2]{\curb{\left. #1 \, \right| \, #2}}
\newcommand{\bv}[1]{\mathbf{#1}}
\newcommand{\bfa}{\bv{a}}
\newcommand{\bfb}{\bv{b}}
\newcommand{\bfx}{\bv{x}}
\newcommand{\bfy}{\bv{y}}
\newcommand{\bfe}{\bv{e}}
\newcommand{\bfF}{\bv{F}}
\newcommand{\bff}{\bv{f}}
\newcommand{\bfG}{\bv{G}}
\newcommand{\bfH}{\bv{H}}
\newcommand{\bfg}{\bv{g}}
\newcommand{\bfh}{\bv{h}}
\newcommand{\bfr}{\bv{r}}
\newcommand{\bfk}{\bv{k}}
\newcommand{\bfu}{\bv{u}}
\newcommand{\bfv}{\bv{v}}
\newcommand{\oo}[1]{o\pare{#1}}
\newcommand{\OO}[1]{O\pare{#1}}
% \newcommand{\norm}[1]{\left\| #1 \right\|}
\newcommand{\DD}{\mathbf{D}}
\newcommand{\comp}{\circ}
\newcommand{\const}{\mathrm{const}}
\newcommand{\dist}[2]{d\pare{#1,#2}}
\newcommand{\len}{\ell}
\newcommand{\siga}{$\sigma$-代数}
\newcommand{\cara}{Carath\'{e}odory}
\newcommand{\Gd}{G_\delta}
\newcommand{\Fs}{F_\sigma}
\newcommand{\mmani}{$m$-维流形}
\newcommand{\open}[1]{\mathcal{#1}}
\newcommand{\half}{\frac{1}{2}}
\newcommand{\maxo}[1]{\text{max}\curb{#1}}
\newcommand{\mino}[1]{\text{min}\curb{#1}}
\newcommand{\epsclo}{$\epsilon$-接近}
\newcommand{\close}[1]{$#1$-接近}
\newcommand{\cinf}{$C^\infty$}
\newcommand{\cuno}{$C^1$}
\newcommand{\Int}{\text{Int}\,}
\newcommand{\Ext}{\text{Ext}\,}
\newcommand{\funcf}{\mathcal}
\newcommand{\DDu}{\overbar{\DD}}
\newcommand{\DDl}{\underbar{\DD}}
\newcommand{\Diff}[1]{\mathrm{Diff}_{#1}\,}
\newcommand{\Av}[1]{\mathrm{Av}_{#1}\,}
\newcommand{\Lip}[1]{Lipschitz-$#1$}
\newcommand{\sgn}{\mathrm{sgn}}
\newcommand{\eset}{\varnothing}
\newcommand{\cT}{\mathcal{T}}
\newcommand{\cS}{\mathcal{S}}
\newcommand{\cG}{\mathcal{G}}
\newcommand{\cF}{\mathcal{F}}
\newcommand{\cC}{\mathcal{C}}
\newcommand{\cB}{\mathcal{B}}
\newcommand{\inter}[1]{\mathring{#1}}
\newcommand{\forest}[3]{对于{#1},存在{#2},使得{#3}}
\newcommand{\tuno}{$T_1$公理}
\newcommand{\isom}{\overset{\sim}{=}}
\newcommand{\diam}{\mathrm{diam}\,}
\newcommand{\ord}[1]{\abs{#1}}
\newcommand{\sbm}[1]{\overbar{#1}}
\newcommand{\inv}[1]{#1^{-1}}
\newcommand{\restr}[2]{#1|_{#2}}
\newcommand{\divs}{|}
\newcommand{\ndivs}{\nmid}
\newcommand{\modeq}[1]{\overbar{#1}}
\newcommand{\ggen}[1]{\langle#1\rangle}
\newcommand{\ggencond}{\braket}

\newcommand{\hd}{H\"{o}lder}

\renewcommand{\proofname}{证明}

\newenvironment{cenum}{\begin{enumerate}\itemsep0em}{\end{enumerate}}

\newtheorem{definition}{定义}[section]
\newtheorem{lemma}{引理}[section]
\newtheorem{theorem}{定理}[section]
\newtheorem{collary}{推论}[section]
\newtheorem{corollary}{推论}[section]
\newtheorem{proposition}{命题}[section]
\newtheorem{axiom}{公理}[section]
\newtheorem{ex}{例}[section]
\newtheorem{reflection}{反射}[section]
\newcommand{\refl}[1]{\vspace{0.5em}\par\noindent\fbox{%
    \parbox{0.9\textwidth}{%
    \begin{reflection}
        #1
    \end{reflection}
    }%
}\vspace{0.5em}\par}
\newcommand{\rref}[1]{反射\ref{refl:#1}}
\newcommand{\tbref}[1]{表\ref{table:#1}}
\allowdisplaybreaks

\newenvironment{aenum}{\begin{enumerate}[label=\textnormal{(\alph*)}]}{\end{enumerate}}

% Physics Head

\DeclareSIUnit\dyne{dynes}

\newcommand{\ddel}[1]{\frac{\partial}{\partial #1}}
\newcommand{\ddelon}[2]{\frac{\partial #1}{\partial #2}}
\newcommand{\dddel}[1]{\frac{\partial^2}{\partial^2 #1}}
\newcommand{\ddt}{\ddel{t}}
\newcommand{\ddT}{\ddel{T}}
\newcommand{\ddV}{\ddel{V}}
\newcommand{\ddr}{\ddel{r}}
\newcommand{\ddth}{\ddel{\theta}}
\newcommand{\ddph}{\ddel{\phi}}
\newcommand{\dddt}{\dddel{t}}
\newcommand{\dddr}{\dddel{t}}
\newcommand{\dddth}{\dddel{\theta}}
\newcommand{\dddph}{\dddel{\phi}}
\newcommand{\rd}[1]{\mathrm{d} #1}
\newcommand{\dt}{\rd{t}}
\newcommand{\dy}{\rd{y}}
\newcommand{\dx}{\rd{x}}
\newcommand{\edd}[1]{\frac{\mathrm{d}}{\mathrm{d} #1}}
\newcommand{\eddd}[1]{\frac{\mathrm{d}^2}{\mathrm{d}^2 #1}}
\newcommand{\eddon}[2]{\frac{\mathrm{d} #1}{\mathrm{d} #2}}
\newcommand{\edddon}[2]{\frac{\mathrm{d}^2 #1}{\mathrm{d}^2 #2}}
\newcommand{\edt}{\edd{t}}
\newcommand{\edton}[1]{\eddon{#1}{t}}
\newcommand{\edT}{\edd{T}}
\newcommand{\edr}{\edd{r}}
\newcommand{\edl}{\edd{l}}
\newcommand{\edx}{\edd{x}}
\newcommand{\edth}{\edd{\theta}}
\newcommand{\eddton}[1]{\edddon{#1}{t}}
\newcommand{\eddzon}[1]{\edddon{#1}{z}}
\newcommand{\vect}[1]{\boldsymbol{#1}}
\newcommand{\alp}{\frac{1}{\sqrt{2}}}
\newcommand{\alpi}{\frac{i}{\sqrt{2}}}
\newcommand{\expc}[1]{\langle#1\rangle}
\newcommand{\bkn}[1]{\bra{#1}\ket{#1}}
\newcommand{\bk}[2]{\bra{#1}\ket{#2}}
\newcommand{\bik}[3]{\bra{#1} #2 \ket{#3}}
\newcommand{\vari}[1]{\sigma_{#1}}
\newcommand{\intc}[2]{\left[#1, #2\right]}
\newcommand{\sch}{Schr\"{o}dinger}
\newcommand{\moment}{\boldsymbol{p}}
\newcommand{\coor}{\boldsymbol{x}}
\newcommand{\lapc}{\nabla^2}
% \newcommand{\rec}[1]{\frac{1}{#1}}
\newcommand{\vva}{\boldsymbol{a}}
\newcommand{\vvb}{\boldsymbol{b}}
\newcommand{\vc}{\boldsymbol{c}}
\newcommand{\vd}{\boldsymbol{d}}
\newcommand{\ve}{\boldsymbol{e}}
\newcommand{\vf}{\boldsymbol{f}}
\newcommand{\vg}{\boldsymbol{g}}
\newcommand{\vh}{\boldsymbol{h}}
\newcommand{\vi}{\boldsymbol{i}}
\newcommand{\vj}{\boldsymbol{j}}
\newcommand{\vk}{\boldsymbol{k}}
\newcommand{\vl}{\boldsymbol{l}}
\newcommand{\vm}{\boldsymbol{m}}
\newcommand{\vn}{\boldsymbol{n}}
\newcommand{\vo}{\boldsymbol{o}}
\newcommand{\vp}{\boldsymbol{p}}
\newcommand{\vq}{\boldsymbol{q}}
\newcommand{\vr}{\boldsymbol{r}}
\newcommand{\vs}{\boldsymbol{s}}
\newcommand{\vt}{\boldsymbol{t}}
\newcommand{\vvu}{\boldsymbol{u}}
\newcommand{\vv}{\boldsymbol{v}}
\newcommand{\vw}{\boldsymbol{w}}
\newcommand{\vx}{\boldsymbol{x}}
\newcommand{\vy}{\boldsymbol{y}}
\newcommand{\vz}{\boldsymbol{z}}
\newcommand{\vA}{\boldsymbol{A}}
\newcommand{\vB}{\boldsymbol{B}}
\newcommand{\vC}{\boldsymbol{C}}
\newcommand{\vD}{\boldsymbol{D}}
\newcommand{\vE}{\boldsymbol{E}}
\newcommand{\vF}{\boldsymbol{F}}
\newcommand{\vG}{\boldsymbol{G}}
\newcommand{\vH}{\boldsymbol{H}}
\newcommand{\vI}{\boldsymbol{I}}
\newcommand{\vJ}{\boldsymbol{J}}
\newcommand{\vK}{\boldsymbol{K}}
\newcommand{\vL}{\boldsymbol{L}}
\newcommand{\vM}{\boldsymbol{M}}
\newcommand{\vN}{\boldsymbol{N}}
\newcommand{\vO}{\boldsymbol{O}}
\newcommand{\vP}{\boldsymbol{P}}
\newcommand{\vQ}{\boldsymbol{Q}}
\newcommand{\vR}{\boldsymbol{R}}
\newcommand{\vS}{\boldsymbol{S}}
\newcommand{\vT}{\boldsymbol{T}}
\newcommand{\vU}{\boldsymbol{U}}
\newcommand{\vV}{\boldsymbol{V}}
\newcommand{\vW}{\boldsymbol{W}}
\newcommand{\vX}{\boldsymbol{X}}
\newcommand{\vY}{\boldsymbol{Y}}
\newcommand{\vZ}{\boldsymbol{Z}}
\newcommand{\vzero}{\boldsymbol{0}}
\newcommand{\vomega}{\boldsymbol{\omega}}
%\newcommand{\half}{\frac{1}{2}}
\newcommand{\thalf}{\frac{3}{2}}
\newcommand{\rot}{\nabla\times}
\newcommand{\divg}{\nabla\cdot}
\newcommand{\cE}{\mathcal{E}}
\newcommand{\conclu}[1]{\vspace{1em}\par\noindent\fbox{\parbox{0.9\textwidth}{#1}}\vspace{1em}}
\newcommand{\subentrynote}{$\bullet$}
\newcommand{\keypoint}[1]{\par\subentrynote\quad #1 \par}
\newcommand{\fconclu}{\boxed}
\newcommand{\pair}[2]{#1 \, #2}
\newcommand{\intn}[2]{\int #1 \,\mathrm{d} #2}
\newcommand{\intu}[3]{\int_0^{#1} #2 \,\mathrm{d} #3}
\newcommand{\intiu}[3]{\int_{-\infty}^{#1} #2 \, \rd{} #3}
\newcommand{\intui}[2]{\int_0^{\infty} #1 \,\mathrm{d} #2}
\newcommand{\intii}[2]{\int_{-\infty}^{\infty} #1 \,\mathrm{d} #2}
\newcommand{\intt}[2]{\int_0^\infty #1 \, \rd{} #2}
\newcommand{\intr}[2]{\int_{-\infty}^{\infty} #1 \, \rd{} #2}
\newcommand{\intbi}[3]{\int_{#1}^{\infty} #2 \, \rd{} #3}
\newcommand{\intab}[4]{\int_{#1}^{#2} #3 \, \rd{} #4}
\newcommand{\bfactor}[1]{e^{-#1/k_BT}}
\newcommand{\pbfactor}[1]{e^{#1/k_BT}}
\newcommand{\dn}[2]{#1^{\pare{#2}}}
\newcommand{\prodg}[1]{\pare{#1}^\times}

\newcommand{\notion}{\emph}
\newcommand{\iP}{\mathcal{P}}
\newcommand{\eiP}{e^{-\iP}}
\newcommand{\iF}{\mathcal{F}}
\newcommand{\eiF}{e^{-\iF}}
\newcommand{\iG}{\mathcal{G}}

\newcommand{\rc}{r\cos\theta}
\newcommand{\rs}{r\sin\theta}
\newcommand{\sn}{\mathrm{sn}}
\newcommand{\cn}{\mathrm{cn}}
\newcommand{\rdn}{\mathrm{dn}}

\newcommand{\hankel}{H_p^{\pare{1}}}
\newcommand{\hankell}{H_p^{\pare{2}}}
\newcommand{\hhankel}{H_n^{\pare{1}}}
\newcommand{\hhankell}{H_n^{\pare{2}}}
\newcommand{\ber}{\text{ber}\,}
\newcommand{\bei}{\text{bei}\,}
\newcommand{\kker}{\text{ker}\,}
\newcommand{\kei}{\text{kei}\,}
\newcommand{\Ai}{\text{Ai}}
\newcommand{\Bi}{\text{Bi}}

\newcommand{\re}{\text{Re}\,}

\newcommand{\Fp}{F_\phi}
\newcommand{\Ep}{E_\phi}
\newcommand{\Fx}{F_x}
\newcommand{\FF}{\mathbf{F}}
\newcommand{\Ex}{E_x}

%\newcommand{\erf}{\mathrm{erf}}
\newcommand{\erfi}{\mathrm{erfi}}
\newcommand{\erfc}{\mathrm{erfc}}
\newcommand{\ehxs}[1]{e^{-\frac{#1^2}{2}}}
\newcommand{\dcol}[2]{\[ \left.#1 \hspace{1em}\right\vert\hspace{1em} #2 \]}
\newcommand{\titlegamma}{\texorpdfstring{$\Gamma$}{Gamma}}
\newcommand{\titleB}{\texorpdfstring{$B$}{B}}

% Computer Science Head
\usepackage{listings}
\usepackage{color}

\definecolor{dkgreen}{rgb}{0,0.6,0}
\definecolor{gray}{rgb}{0.5,0.5,0.5}
\definecolor{mauve}{rgb}{0.58,0,0.82}

\lstset{frame=tb,
  language=Java,
  aboveskip=3mm,
  belowskip=3mm,
  showstringspaces=false,
  columns=flexible,
  basicstyle={\small\ttfamily},
  numbers=none,
  numberstyle=\tiny\color{gray},
  keywordstyle=\color{blue},
  commentstyle=\color{dkgreen},
  stringstyle=\color{mauve},
  breaklines=true,
  breakatwhitespace=true,
  tabsize=3
}
\lstset{language=Java}
\newcommand{\snp}[1]{\lstinline!#1!}
\newcommand{\term}[2]{\textbf{#1(#2)}}
\begin{document}
\fi

%Content

\chapter{普通集合论}
  \section{集与势}
  \subsection{自然数公理及其运算}
  \begin{axiom}
    \[ 0 \in \N. \]
  \end{axiom}
  \begin{axiom}
    若$n\in \N$,则
    \[ n\inc\in \N. \]
  \end{axiom}
  并且作如下定义:
  \begin{definition}
    \[ 1 \defeq 0\inc, \quad 2 \defeq 1\inc, \quad 3 \defeq 2\inc\cdots. \]
  \end{definition}
  然而这并不完备,因为完全可以定义$3\inc=0$从而令之回环。为此,制定第三条公理。
  \begin{axiom}
    $0$不是任何自然数的后继。
  \end{axiom}
  尽管如此,仍然可以定义$4\inc=4$或$4\inc=4$。诸多解决方法之一为如下第四条。
  \begin{axiom}
    后继相同的自然数相等。
  \end{axiom}
  为了保证除了$0,1,2,3,4\cdots$外没有其他杂质,复制定第五条。
  \begin{axiom}[数学归纳法]
    若$P\pare{0}$为真,且$P\pare{n}$为真推出$P\pare{n\inc}$为真,则对于任意$n\in \N$,$P\pare{n}$为真。
  \end{axiom}
  $\N$的存在性视为一公理。
  \subsubsection{加法}
  \begin{definition}
    定义操作$+$为$0+m=0$,$\pare{n\inc}+m=\pare{n+m}\inc$。
  \end{definition}
  \begin{lemma}
    对于任意自然数$n$,$n+0=n$。
  \end{lemma}
  \begin{proof}
    由$0+0=0$开始,归纳可得。
  \end{proof}
  \begin{lemma}
    $n+\pare{m\inc} = \pare{n+m}\inc$。
  \end{lemma}
  \begin{proof}
    先由$0+\pare{m\inc} = \pare{0+m}\inc$开始,对$n$归纳。
  \end{proof}
  可直接推知$n\inc = n+1$。
  \begin{proposition}
    \[ n+m=m+n. \]
  \end{proposition}
  \begin{proof}
    先由$0+m=m+0$,后借助前一命题对$n$归纳.
  \end{proof}
  \begin{proposition}
    \[ \pare{a+b}+c = a+\pare{b+c}. \]
  \end{proposition}
  \begin{proof}
    先由$\pare{a+b}+0=a+\pare{b+0}$,后对$c$归纳可得。
  \end{proof}
  \begin{proposition}
    若$a+b=a+c$,则$b=c$。
  \end{proposition}
  \begin{proof}
    从$0+b=0+c$开始,对$a$归纳可得。
  \end{proof}
  \begin{definition}
    非零自然数为正。
  \end{definition}
  \begin{proposition}
    若$a$为正而$b$为自然数,则$a+b$为正。
  \end{proposition}
  \begin{proof}
    从$b=0$开始归纳可得。
  \end{proof}
  \begin{collary}
    若自然数$a$和$b$满足$a+b=0$则$a=b=0$。
  \end{collary}
  \begin{proposition}
    正数存在前置元素。
  \end{proposition}
  \begin{proof}
    归纳可证,注意在$P$中排除$0$。
  \end{proof}
  \begin{definition}[自然数的序]
    若存在$a$,使$n=m+a$,则称$n\ge m$。若$m\ne n$,则称$n>m$。
  \end{definition}
  \begin{proposition}
    \begin{aenum}
      \item $a\ge a.$
      \item 若$a\ge b$且$b \ge c$,则$a \ge c$。
      \item 若$a \ge b$且$b \ge a$,则$a=b$。
      \item $a\ge b$当且仅当$a+c \ge b+c$。
      \item $a < b$当且仅当$a\inc \le b$。
      \item $a < b$当且仅当存在正数$d$,$b=a+d$。
    \end{aenum}
  \end{proposition}
  \begin{proposition}
    下列三个命题中有且仅有一成立:
    \[ a<b, \quad a=b,\quad a>b. \]
  \end{proposition}
  \begin{proof}
    互斥是显然的。下证必有一成立,故成立全序。先设$a=0$,后对$a$归纳可得(归纳时可对$a$分类)。
  \end{proof}
  \begin{proposition}[强归纳原理]
    若对于任意$m\ge m_0$有如下蕴含关系:任意$m_0\le m' <m$有$P\pare{m'}$推出$P\pare{m}$,则$P$对一切$m \ge m_0$成立。
  \end{proposition}
  \subsubsection{乘法}
  \begin{definition}
    定义操作$\times$为$0\times m = 0$,$\pare{n\inc}\times m = n \times m + n$。
  \end{definition}
  \begin{lemma}
    \[ n \times m = m \times n. \]
  \end{lemma}
  \begin{lemma}
    \label{lem:mn0}
    若$n \times m = 0$,则$n$与$m$中至少一者为$0$。且正数相乘为正。
  \end{lemma}
  \begin{proposition}
    \[ a\pare{b+c} = ab+ac. \]
  \end{proposition}
  \begin{proof}
    先证$a\pare{b+0} = ab + a\times 0$,后对$c$归纳。
  \end{proof}
  \begin{proposition}
    \[ \pare{a\times b} \times c= a \times \pare{b \times c}. \]
  \end{proposition}
  \begin{proposition}
    若$a<b$且$c$为正数,则$ac<bc$。
  \end{proposition}
  \begin{proof}
    在$b=a+d$两侧用$c$乘。
  \end{proof}
  \begin{proposition}
    若$ac=bc$且$c$不为$0$,则$a=b$。
  \end{proposition}
  \begin{proof}
    $a<b$,$a=b$与$a>b$有且仅有一成立,两侧乘$c$知必定为$a=b$。
  \end{proof}
  \begin{proposition}
    对自然数$n$与正数$q$,存在$0\le r <q$满足$n=mq+r$。
  \end{proposition}
  \begin{proof}
    对于任意$q$,命题对$n=0$成立,后对$n$归纳。
  \end{proof}
  \begin{definition}
    定义幂运算如下:$m^0=1$,$m^{n\inc}=m^n\times m$。
  \end{definition}
  \subsection{集公理及其运算}
  \begin{definition}
    若$A$的所有元素都是$B$的元素,则称$A$为$B$的子集,$B$为$A$的超集。记作
    \[ A \subset B \quad \text{或} \quad B \supset A. \]
  \end{definition}
  \begin{definition}
    若$A \subset B$且$B\subset A$,则称$A$与$B$相等,记作
    \[ A=B. \]
  \end{definition}
  \begin{definition}
    若$S$包含$A$与$B$中的所有元素又不包含其他元素,称$S$为$A$与$B$的并集,记作
    \[ S = A \cup B. \]
  \end{definition}
  \begin{definition}
    若$S$包含$A$与$B$的所有共同元素又不包含其他元素,称$S$为$A$与$B$的交集,记作
    \[ S = A \cap B. \]
  \end{definition}
  \begin{definition}
    若$S$包含属于$A$且不属于$B$的一切元素又不包含其它元素,称$S$为$A$与$B$的差集,记作
    \[ S = A-B \quad \text{或} \quad S = \complement_A B. \]
  \end{definition}
  \begin{theorem}
    \[ A \cap \pare{\bigcup S_i} = \bigcup \pare{A \cap S_i}, \]
    \[ A \cup \pare{\bigcap S_i} = \bigcap \pare{A \cup S_i}, \]
    \[ A - \pare{B \cup C} = \pare{A-B} \cap \pare{A-C}, \]
    \[ A - \pare{B \cap C} = \pare{A-B} \cup \pare{A-C}. \]
    后二条称为DeMorgan律,即并的补等于补的交,交的补等于补的并。
  \end{theorem}
  \subsubsection{Cartesian乘积}
  \begin{definition}
    $X_1,\cdots,X_n$的Cartesian乘积定义为所有\ntuple 
    \[ \pare{x_1,\cdots,x_n} \]
    的集合。
  \end{definition}
  \begin{lemma}[有限选择]
    若有限诸个$X_i$非空,则$\Pi X_i$非空。
  \end{lemma}
  \begin{proof}
    对$n$归纳即可。
  \end{proof}
  \subsection{映射}
  \begin{definition}
    集合$A$中的一个等价关系,是满足左列三条件的一关系$C$:
    \begin{enumerate}
      \item (自反性)对任意$x$,有$xCx$;
      \item (对称性)若$xCy$,则$yCx$;
      \item (传递性)若$xCy$,$yCz$,则$xCz$。
    \end{enumerate}
  \end{definition}
  不是等价关系的关系,如近似同向,即二向量之间夹角为锐角。
  \begin{definition}
    与$x$等价的所有元素为$x$的等价类。
  \end{definition}
  \begin{definition}
    若存在法则$\varphi$,使得$A$的任意元素存在$B$的唯一元素与之对应,$B$的任意元素存在$A$的唯一元素与之对应,则称$\varphi$建立了$A$与$B$的一一对应。
  \end{definition}
  \begin{definition}
    若$A$与$B$见能建立一一对应,则称$A$与$B$对等,或$A$与$B$具有相同基数,记作
    \[ A \sim B. \]
    $A$的基数记作
    \[ \card{A}. \]
  \end{definition}
  \par
  值得注意的是,两个不等大小的圆、两条不同长度的线段、以及自然数全体与偶数之间,都可以轻易建立起一一对应。
  \begin{theorem}
    对等是等价关系。
  \end{theorem}
  \begin{theorem}
    无交对等集的并仍然对等,即若
    \[ A_n \cap A_{n'} = \varnothing,\quad B_n \cap B_{n'} = \varnothing \quad \pare{n \ne n'}, \]
    且
    \[ A_n \sim B_n, \]
    则
    \[ \bigcup A_n \sim \bigcup B_n. \]
  \end{theorem}
  \subsubsection{有限集}
  \begin{definition}
    若集合$A$与$\pare{1,\cdots,n}$对等,称集合具有基数$n$且为有限集。
  \end{definition}
  \begin{proposition}
    有限集的基数唯一。
  \end{proposition}
  \begin{proof}
    对基数作归纳即可。注意到可证集合挖去一元素后基数为其前置数。
  \end{proof}
  \subsection{可数集}
  \begin{definition}
    与$\N$对等的集合称为可数集,或者称为可数的。
  \end{definition}
  \begin{theorem}
    $A$为可数的当且仅当可以编号$A$的元素使得
    \[ A = \curb{a_1, a_2, a_3, \cdots, a_n, \cdots}. \]
  \end{theorem}
  \begin{theorem}
    无穷集$A$必含有可数子集.
  \end{theorem}
  \begin{proof}
    每次取一元素并编号,取之不尽。
  \end{proof}
  \begin{theorem}
    \label{thm:coutablesub}
    可数集的任何无穷子集皆可数,
  \end{theorem}
  \begin{proof}
    将$A$的元素一一编号列出,逐个检查,遇到子集$B$的元素即自增计数器。
  \end{proof}
  \begin{theorem}
    \[ n+\aleph_0\sim\aleph_0. \]
  \end{theorem}
  \begin{proof}
    先列出有限集的元素,后列出可数集的元素,重新编号即可。
  \end{proof}
  \begin{theorem}
    \[ n\cdot\aleph_0\sim\aleph_0. \]
  \end{theorem}
  \begin{proof}
    逐个列出各集的第一个元素,后列出各集第二个元素,以此类推,重新编号即可。
  \end{proof}
  \begin{theorem}
    \[ \aleph_0 \cdot n \sim \aleph_0. \]
  \end{theorem}
  \begin{proof}
    先写出第一个集合的所有元素,后写出第二个集合的所有元素,以此类推,重新编号即可。
  \end{proof}
  \begin{theorem}
    \label{thm:countableprod}
    \[ \aleph_0\cdot\aleph_0\sim\aleph_0. \]
  \end{theorem}
  \begin{proof}
    \begin{align*}
      A_1 &= \curb{\bu{a}{1}{1},\bu{a}{2}{1},\bu{a}{3}{1},\cdots},\\
      A_2 &= \curb{\bu{a}{1}{2},\bu{a}{2}{2},\bu{a}{3}{2},\cdots},\\
      A_3 &= \curb{\bu{a}{1}{3},\bu{a}{2}{3},\bu{a}{3}{3},\cdots},\\
      &\cdots
    \end{align*}
    沿着反对角线列出各个元素,即
    \[ S = \curb{\bu{a}{1}{1},\bu{a}{2}{1},\bu{a}{1}{2},\bu{a}{3}{1},\bu{a}{2}{2},\bu{a}{1}{3},\bu{a}{4}{1},\cdots}, \]
    可知$S$可数。
  \end{proof}
  \begin{theorem}
    $\Q$为可数集。
  \end{theorem}
  \begin{proof}
    将$\Q$写为$\N^2$去掉可约者,去掉前集合可数,去掉后由\tref{coutablesub}知仍然可数。
  \end{proof}
  \begin{collary}
    任意区间内的$\Q$可数。
  \end{collary}
  \begin{proof}
    仍然借助\tref{coutablesub}。
  \end{proof}
  \begin{theorem}
    无穷集$M$与可数集或有限集$A$的并,其势仍然不变。
  \end{theorem}
  \begin{proof}
    $M$含有一可数集$D$,此可数集与$D \cup A$可一一对应,故$M$的势不变。
  \end{proof}
  \begin{theorem}
    不可数集$M$除去一有限或可数子集$A$,仍有
    \[ M-A \sim M. \]
  \end{theorem}
  \begin{proof}
    $M$去除后仍为无限集,因此据上一定理,有$\pare{M-A} \cup A \sim M-A$。
  \end{proof}
  \begin{collary}
    无穷集包含与自身对等的一子集。
  \end{collary}
  \par
  此时发现一有限集不可能具有的性质,故有如下归功于R. Dedekind的定义。
  \begin{definition}
  	  包含与自身对等的真子集的集合称为无穷集。
  \end{definition}
  \begin{theorem}
    \[ \aleph_0^n \sim \aleph_0. \]
  \end{theorem}
  \begin{proof}
    数学归纳法结合\tref{countableprod}可证。
  \end{proof}
  \begin{collary}
    代数数全体为可数集。
  \end{collary}
  \subsection{连续统的势}
  此处仅暂时借用实数的定义。其具体定义留待下文。
  \begin{theorem}
    线段$U=\brac{0,1}$不可数。
  \end{theorem}
  \begin{proof}
    可以参考\tref{uncountableR}。
    \par
    三等分区间,存在一不包含$x_1$的闭区间。将其再度三等分,存在一不包含$x_2$的闭区间,以此类推。故存在闭区间套,其交在$x_n$之外。
  \end{proof}
  \begin{definition}
    若$A$与$\brac{0,1}$对等,则称$A$具有连续统的势$\aleph$。
  \end{definition}
  \begin{theorem}
    $\brac{a,b}$,$\left(a,b\right]$,$\left[a,b\right)$与$\pare{a,b}$的势均为$\aleph$。
  \end{theorem}
  \begin{proof}
    只需注意无穷集挖去有限集后与原来的无穷集对等。
  \end{proof}
  \begin{theorem}
    \[ n\aleph\sim\aleph. \]
  \end{theorem}
  \begin{theorem}
    \[ \aleph_0\aleph\sim\aleph. \]
  \end{theorem}
  \begin{proof}
    取$\left[0,0.9\right),\left[0.9,0.99\right),\left[0.99,0.999\right),\cdots$分别映射即可。
  \end{proof}
  \begin{collary}
    $\R$的势为$\aleph$。
  \end{collary}
  \begin{collary}
    $\R-\Q$的势为$\aleph$。
  \end{collary}
  \begin{collary}
    超越数的势为$\aleph$。
  \end{collary}
  \begin{theorem}
    正整数列全体的势为$\aleph$。
  \end{theorem}
  \begin{proof}
    直接注意将正整数列写成相应连分数可以与无理数一一对应。亦可以将数列中的正整数看作二进制小数的零位索引差而获得此对应。
  \end{proof}
  \begin{theorem}
    \[ \aleph^n \sim \aleph. \]
  \end{theorem}
  \begin{proof}
    第一个$\aleph$与正整数列全体$\curb{a_n}$对应,第二个与$\curb{b_n}$对应,则
    \[ \pare{a_1,b_1,a_2,b_2,\cdots} \]
    与$\aleph^2$一一对应。对于一般的$\aleph^n$,上述论证仍然适用。
  \end{proof}
  \begin{collary}
    $\R^2\sim\aleph.$
  \end{collary}
  \begin{collary}
    $\R^2\sim\aleph.$
  \end{collary}
  \begin{collary}
    \[ \aleph \cdot \aleph \sim \aleph. \]
  \end{collary}
  \begin{theorem}
    \[ \aleph^{\aleph_0} \sim \aleph. \]
  \end{theorem}
  \begin{proof}
    第一个$\aleph$与正整数列全体$\curb{a_n}$对应,第二个与$\curb{b_n}$对应,以此类推。最终将可数个正整数列的直积按照与\tref{countableprod}相同的办法映射为整数列全体。
  \end{proof}
  \begin{theorem}
    \[ \bool^{\aleph_0}\sim\aleph. \]
  \end{theorem}
  \begin{proof}
    注意bool序列与二进制小数序列的对应即可。
  \end{proof}
  \begin{collary}
    \label{coll:2n}
    若$A\sim 2$,则
    \[ A^{\aleph_0}\sim\aleph. \]
  \end{collary}
  \subsection{势的比较}
  \begin{definition}
    若两个集合对等,则称其具有相同的势。予每个对等的等价类一记号,称此记号为等价类中任一集合的势。
  \end{definition}
  \begin{definition}
    若$A$与$B$不对等,且$B$有子集与$A$对等,称$A$的势小于$B$的势。
  \end{definition}
  \begin{theorem}
    $\brac{0,1}$上的所有实函数的集合的势大于$\aleph_1$。
  \end{theorem}
  \begin{proof}
    设$t$对应$F\pare{t,x}$,则$G\pare{x}=F\pare{x,x}+1$不在任意一个$t$的值域内。
  \end{proof}
  \begin{definition}
    称$\brac{0,1}$上的所有实函数的集合的势为$\aleph_2$。
  \end{definition}
  \begin{theorem}
    集合与其子集族不对等。
  \end{theorem}
  \begin{proof}
    假设$x$映射为$\unk$,则将所有$x\notin\unk$的$x$并起来得到$\mathcal{Y}$,并设元素$y$映射到这个集合。若$y\in\mathcal{Y}$,则$y$不满足条件而应被除名。若$y\notin\mathcal{Y}$,则$y$满足条件而应该处在集合内。
  \end{proof}
  \begin{definition}
    若$M$的势为$\mu$,则$M$的子集族的势为$2^\mu$。
  \end{definition}
  \begin{theorem}
    \[ \aleph_1=2^{\aleph_0}. \]
  \end{theorem}
  \begin{proof}
    这正是\cref{2n}。
  \end{proof}
  \begin{theorem}
    设$A_0\supset A_1 \supset A_2$,若$A_2\sim A_0$,则$A_1 \sim A_0$。
  \end{theorem}
  \begin{proof}
    假设$\varphi$为$A_0$到$A_2$的一一对应,则可设$\varphi\pare{A_1}=A_3$,且$A_3 \subset A_2$。由于是同一映射,故\[A_0-A_1\sim A_2-A_3.\]
    \par
    此时$A_2\subset A_1$且$A_1\sim A_3$,故可设一一对应为$\psi$,如法炮制$\psi\pare{A_2}=A_4$,又有\[A_1-A_2 \sim A_3-A_4.\]
    可以此类推,并设$A_\infty = D$,即
    \[ \bigcap A_i = D. \]
    又
    \begin{align*}
      A &= \pare{A_0-A_1}+\pare{A_1-A_2}+\pare{A_2-A_3}+\pare{A_3-A_4}+\cdots+D,\\
      A_1 &= \pare{A_1-A_2}+\pare{A_2-A_3}+\pare{A_3-A_4}+\pare{A_4-A_5}+\cdots+D.
    \end{align*}
    左上与右下对等,右上与左下相同,故$A$与$A_1$对等。
  \end{proof}
  \begin{theorem}[E. Schr\"{o}der--F. Bernstein]设$A$与$B$彼此与对方一子集对等,则彼此对等。
  \end{theorem}
  \begin{proof}
    设
    \[ \varphi\pare{A} = B^*,\qquad \psi\pare{B}=A^*. \]
    则
    \[ \psi\varphi\pare{A} = \psi\pare{B^*} = A^{**}, \]
    即$A$与$A^{**}$对等。由上一定理,$A$与$A^*$对等,故与$B$对等。
  \end{proof}
  \begin{collary}
    势之间大于、等于、小于择一成立。
  \end{collary}
  \begin{proof}
    若同时大于且小于,则由势的大于小于的定义可得上一定理的题设,故两势相等。
  \end{proof}
  \begin{collary}
    势的小于具有传递性。
  \end{collary}
  \begin{proof}
    把仲叔通过$\varphi$和$\psi$映射到伯,得到伯下二嵌套子集。若伯叔对等,则仲亦然矣。
  \end{proof}
  \begin{theorem}
    $\brac{0,1}$上连续函数集的势为$\aleph_1$。
  \end{theorem}
  \begin{proof}
    注意连续函数仅仅取决于$\Q$处的值。
  \end{proof}
  \section{数系}
  \subsection{序关系}
  \begin{definition}
    关系$C$称为全序关系,若满足
    \begin{enumerate}
      \item 对任意$x\ne y$的$x$和$y$,$xCy$与$yCx$二者有一成立;
      \item 不存在$xCx$;
      \item 若$xCy$,$yCz$,则$xCz$。
    \end{enumerate}
  \end{definition}
  \begin{definition}
    对于$a<b$,称
    \[ \setcond{x}{a<x<b} \]
    为开区间。若其为空集,则称$b$为$a$的紧接后元。
  \end{definition}
  前述自然数已被赋予一全序关系,$n\inc$为$n$的紧接后元。
  \begin{definition}
    对于Castesian乘积$A\times B$,可定义字典序关系:当$a_1<b_1$且$a_2<b_2$,有
    \[ a_1\times b_1 < a_2 \times b_2. \]
  \end{definition}
  可证其为一全序关系。
  \begin{definition}
    若$A$具有全序关系$<$,若对于任意$x\in A$有$x\le b$,则$b$为最大元。类似定义最小元。
  \end{definition}
  \begin{definition}
    $A$的子集$A_0$是有上界的,如果存在$b$使对任意$a_0\in A_0$有$a_0\le b$。若所有上界的集合存在最小元,则称之上确界。类似定义下界与下确界。
  \end{definition}
  \begin{definition}
    若$A$的任意有上界的非空子集$A_0$均有上确界,则称$A$具有上确界性质。类似定义下确界性质。
  \end{definition}
  \begin{theorem}
    集合$A$具有上确界性质当且仅当其具有下确界性质。
  \end{theorem}
  \begin{proof}
	  假设集合$A$具有上确界性质且$A_0$为有下界的一非空子集,则其所有下界的集合$B_0$存在一上确界。设$B_0$的所有上界集合为$C_0$,则上确界性质表明存在$c_0$为\emph{所有下界的所有上界的最小元},即$C_0$存在最小元$c_0$。
	  \par
	  如果存在一个元素$a_0\le c_0$,则必有$c_0=a_0$,否则$a_0$可替换$c_0$的位置。故对于任意$a_0$,有$c_0\le a_0$(全序的二择一成立)。因此$c_0\in B_0$。
	  \par
	  而$A_0$的下确界定义为\emph{所有下界的最大元},即$B_0$的最大元。可以证明$c_0$为其最大元。否则$c_0$不能为上界。
  \end{proof}
  \subsection{整数}
  \begin{definition}
    整数是形如$a-b$的表达式,且视$a-b=c-d$当且仅当$a+d=b+c$。
  \end{definition}
  此处利用等价关系将$\Z$视为$\N^2$的商构造,需要验证等价关系的自反性,对称性和传递性。前而者显然,传递性要求在
  \begin{alignat*}{2}
    a-b&=c-d, \quad a+d &= b+c,\\
    c-d&=e-f, \quad c+f &= d+e
  \end{alignat*}
  的假设下证明
  \[ a+f = b+e. \]
  将两式相加并消去即可。
  \begin{definition}
    整数的和定义为
    \[ \pare{a-b}+\pare{c-d} = \pare{a+c}-\pare{b+d}. \]
  \end{definition}
  \begin{definition}
    整数的积定义为
    \[ \pare{a-b}\times\pare{c-d} = \pare{ac+bd}-\pare{ad+bc}. \]
  \end{definition}
  \begin{theorem}
    上述二定义在等价类内相容,即结果与代表元的选取无关。
  \end{theorem}
  \begin{proof}
    只证乘法的部分。设$a-b=a'-b'$,即
    \[ a+b' = a'+b. \]
    便需要证
    \[ ac+bd+a'd+b'c=a'c+b'd+ad+bc. \]
    合并有
    \[ c\pare{a+b'}+d\pare{a'+b} = c\pare{a'+b}+d\pare{a+b'}. \]
    由假设知成立。
  \end{proof}
  可以将$n-0$与自然数$n$对应,从而保持上述加法与乘法的结构,此之谓同构。
  \begin{definition}
    定义$\pare{a-b}$的负为$\pare{b-a}$,记作$-\pare{a-b}$。
  \end{definition}
  同样容易证明其于等价类内相容。
  \begin{definition}
    定义负数为正自然数所对应整数的负。
  \end{definition}
  \begin{theorem}[三歧性(trichofomy)]
    任意整数成立如下三个命题之一:
    \begin{aenum}
      \item $x$为零;
      \item $x$对应正的自然数;
      \item $x$对一个正自然数的负。
    \end{aenum}
  \end{theorem}
  \begin{proof}
    对$x=a-b$中$a$和$b$的大小关系分类即可。
    \par
    为了证明三者中仅有一成立,分若干类讨论。当$x$为零时显然不能为正自然数。若其为负自然数则对一正的$n$,有$0-0=0-n$,从而$n=0$,矛盾。若同时为正负则相似可证矛盾。
  \end{proof}
  事实上也可以通过假定三歧性来定义整数,这是大陆教科书的做法,导致了运算验证上的极大混乱。
  \begin{proposition}
    \label{prp:xyyx}
    对于整数,有
    \begin{aenum}
      \item $x+y = y+x$;
      \item $\pare{x+y}+z = x+\pare{y+z}$;
      \item $x+0 = 0+x = x$;
      \item $x+\pare{-x}=\pare{-x}+x=0$;
      \item $xy=yx$;
      \item $\pare{xy}z=x\pare{yz}$;
      \item $x1=1x=x$;
      \item $x\pare{y+z} = xy + xz$;
      \item $\pare{y+z}x = yx+zy$。
    \end{aenum}
  \end{proposition}
  \begin{proof}
    设$x=a-b$后完全展开消去即可。
  \end{proof}
  这一定理表明整数构成以交换环。
  \begin{definition}
    定义整数的减法为$a-b=a+\pare{-b}$。
  \end{definition}
  对于等价类相容的检验,可以略去,因为减法直接借助了经过验证的加法与取负的定义。也容易验证对于自然数的$a$和$b$,等价类$a-b$与差$a-b=\pare{a-0}+\pare{0-b}$相同。此外,还易证一个数减去它自身将得到零。
  \begin{proposition}
    \[ \pare{-1}\times a = -a. \]
  \end{proposition}
  \begin{proposition}
    若整数$ab=0$,则$a=0$或$b=0$。
  \end{proposition}
  \begin{proof}
    借助上一命题,在两侧乘$-1$将$a$与$b$均强制转化为正数后调用\lref{mn0}。
  \end{proof}
  \begin{collary}
    若$ac=bc$且$c\ne 0$,则$a=b$。
  \end{collary}
  \begin{proof}
    即$\pare{a-b}c=0$。
  \end{proof}
  \begin{lemma}[整数的序]
    \begin{aenum}
      \item $a>b$当且仅当$a-b$为正;
      \item 若$a>b$则$a+c>b+c$;
      \item 若$a>b$且$c$为正,则$ac>bc$;
      \item 若$a>b$则$-a<-b$;
      \item 若$a>b$且$b>c$,则$a>c$;
      \item $a>b$,$b>a$,$a=b$有且仅有一成立。
    \end{aenum}
  \end{lemma}
  \begin{proof}
    借助\pref{xyyx}易得。最后一条可借助$a-b$的正负三歧性。
  \end{proof}
  \subsection{有理数}
  \begin{definition}
    有理数是形如$a/b$的表达式,其中$a$和$b$为整数且$b\ne 0$。两个有理数相等当且仅当$ad=bc$。
  \end{definition}
  \begin{definition}
    有理数的和定义为
    \[ \pare{a/b}+\pare{c/d} = \pare{ad+bc}/\pare{bd}. \]
  \end{definition}
  \begin{definition}
    有理数的积定义为
    \[ \pare{a/b}\times\pare{c/d} = \pare{ac}/\pare{bd}. \]
  \end{definition}
  \begin{definition}
    有理数的负定义为
    \[ -\pare{a/b}=\pare{-a}/b. \]
  \end{definition}
  \begin{theorem}
    上述三定义在等价类内相容。
  \end{theorem}
  \begin{proof}
    只证加法的部分,设$ab'=a'b$,需证
    \[ \pare{ad+bc}\pare{b'd} = \pare{a'd+b'c}\pare{bd}. \]
    展开有
    \[ adb'd + bcb'd = a'dbd + b'cbd. \]
    由假设知成立。
  \end{proof}
  注意到$a/1$与整数$a$可同构。
  \begin{definition}
    有理数的倒数定义为
    \[ \pare{a/b}^{-1} = \pare{b/a}. \]
  \end{definition}
  容易验证,倒数也是等价类相容的。
  \begin{theorem}
    对于有理数,有
    \begin{aenum}
      \item $x+y = y+x$;
      \item $\pare{x+y}+z = x+\pare{y+z}$;
      \item $x+0 = 0+x = x$;
      \item $x+\pare{-x}=\pare{-x}+x=0$;
      \item $xy=yx$;
      \item $\pare{xy}z=x\pare{yz}$;
      \item $x1=1x=x$;
      \item $x\pare{y+z} = xy + xz$;
      \item $\pare{y+z}x = yx+zy$。
    \end{aenum}
  \end{theorem}
  \begin{proof}
    同样设$x=a/b$完全展开后消去。
  \end{proof}
  \begin{definition}
    定义有理数的商
    \[ x/y=x\times y^{-1}. \]
  \end{definition}
  \begin{definition}
    一个有理数$x$称为正数,如果对于某两个正整数$a$和$b$有$x=a/b$。称其为负数,如果它是一个正数的负。
  \end{definition}
  \begin{theorem}[三歧性(trichofomy)]
    任意有理数成立如下三个命题之一:
    \begin{aenum}
      \item $x$为零;
      \item $x$是正的有理数;
      \item $x$是负的有理数。
    \end{aenum}
  \end{theorem}
  \begin{proof}
    对$x=a/b$中$a$和$b$的正负分类即可。
    \par
    为了证明三者中仅有一成立,分若干类讨论。当$x$为零时可证其非正且非负。若$x$同时为正负,则展开后借助整数的三歧性可得矛盾。
  \end{proof}
  \begin{definition}
    $x>y$当且仅当$x-y$是正的有理数,$x<y$当且仅当$x-y$是负的。
  \end{definition}
  \begin{theorem}
    设$x$,$y$,$z$均为有理数,则
    \begin{aenum}
      \item $x=y$,$x>y$与$x<y$有且仅有一成立;
      \item $x<y$当且仅当$y>x$;
      \item 若$x<y$,$y<z$,则$x<z$;
      \item 若$x<z$,则$x+z<y+z$;
      \item 若$x<y$且$z$为正,则$xz<yz$。
    \end{aenum}
  \end{theorem}
  \subsection{实数}
  \subsection{实数的Dedekind构造}
  \subsection{实数作为有理数的Cauchy序列}
  \section{逻辑}
  \subsection{归纳定义原理}
  \subsection{无限集与选择公理}
  \subsection{良序集}
  \subsection{极大原理}
  \subsection{良序原理与选择公理}

%ContentEnds
 
\ifx\allfiles\undefined %如果位置放错,可能出现意外中断
\end{document}
\fi