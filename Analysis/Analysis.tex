%!TEX TS-program = xelatex  
%!TEX encoding = UTF-8 Unicode  
  
\documentclass{ctexrep}
% Mathematics Include

\usepackage{amsmath}
\usepackage{amssymb}
\usepackage{amsthm}
\usepackage{amsfonts}
\usepackage{mathrsfs}
\usepackage{enumitem}
\usepackage{braket}
\usepackage{hyperref}
\usepackage[all, pdf]{xy}

% Physics Include
\usepackage{amsmath}
\usepackage{physics}
\usepackage{siunitx}
\usepackage[makeroom]{cancel}
\usepackage{pstricks}
\usepackage{pstricks-add}
\psset{algebraic=true}

\usepackage[version=4]{mhchem}
\usepackage{array,booktabs}
\usepackage{longtable}
\usepackage{mathtools}
\usepackage[normalem]{ulem}
\usepackage{multicol}

% Mathematics Head

\newcommand{\pare}[1]{\left(#1\right)}
\newcommand{\blr}[1]{\left[#1\right)}
\newcommand{\lbr}[1]{\left(#1\right]}
\newcommand{\brac}[1]{\left[#1\right]}
\newcommand{\curb}[1]{\left\{#1\right\}}
% \newcommand{\abs}[1]{\left|\, #1 \,\right|}
\newcommand{\rec}[1]{\frac{1}{#1}}
\newcommand{\N}{\mathbb{N}}
\newcommand{\bC}{\mathbb{C}}
\newcommand{\Q}{\mathbb{Q}}
\newcommand{\Z}{\mathbb{Z}}
\newcommand{\R}{\mathbb{R}}
\newcommand{\unk}{\mathcal{X}}
\newcommand{\bu}[3]{#1_{#2}^{\pare{#3}}}
\newcommand{\dref}[1]{定义\ref{def:#1}}
\newcommand{\tref}[1]{定理\ref{thm:#1}}
\newcommand{\lref}[1]{引理\ref{lem:#1}}
\newcommand{\cref}[1]{推论\ref{coll:#1}}
\newcommand{\pref}[1]{命题\ref{prp:#1}}
\newcommand{\eref}[1]{例\ref{ex:#1}}
\newcommand{\func}[3]{#1:\, #2 \rightarrow #3}
\newcommand{\overbar}[1]{\mkern 1.5mu\overline{\mkern-1.5mu#1\mkern-1.5mu}\mkern 1.5mu}
\newcommand{\clo}[1]{\overbar{#1}}
\newcommand{\supi}[2]{\overbar{\int_{#1}^{#2}}}
\newcommand{\infi}[2]{\underbar{\int_{#1}^{#2}}}
\newcommand{\setf}{\mathscr}
\newcommand{\bool}{\mathrm{bool}}
\newcommand{\inc}{++}
\newcommand{\defeq}{:=}
\newcommand{\ntuple}{$n$元组}
\newcommand{\card}[1]{\#\pare{#1}}
\newcommand{\setcond}[2]{\curb{#1 \, \left| \, #2 \right.}}
\newcommand{\setcondl}[2]{\curb{\left. #1 \, \right| \, #2}}
\newcommand{\bv}[1]{\mathbf{#1}}
\newcommand{\bfa}{\bv{a}}
\newcommand{\bfb}{\bv{b}}
\newcommand{\bfx}{\bv{x}}
\newcommand{\bfy}{\bv{y}}
\newcommand{\bfe}{\bv{e}}
\newcommand{\bfF}{\bv{F}}
\newcommand{\bff}{\bv{f}}
\newcommand{\bfG}{\bv{G}}
\newcommand{\bfH}{\bv{H}}
\newcommand{\bfg}{\bv{g}}
\newcommand{\bfh}{\bv{h}}
\newcommand{\bfr}{\bv{r}}
\newcommand{\bfk}{\bv{k}}
\newcommand{\bfu}{\bv{u}}
\newcommand{\bfv}{\bv{v}}
\newcommand{\oo}[1]{o\pare{#1}}
\newcommand{\OO}[1]{O\pare{#1}}
% \newcommand{\norm}[1]{\left\| #1 \right\|}
\newcommand{\DD}{\mathbf{D}}
\newcommand{\comp}{\circ}
\newcommand{\const}{\mathrm{const}}
\newcommand{\dist}[2]{d\pare{#1,#2}}
\newcommand{\len}{\ell}
\newcommand{\siga}{$\sigma$-代数}
\newcommand{\cara}{Carath\'{e}odory}
\newcommand{\Gd}{G_\delta}
\newcommand{\Fs}{F_\sigma}
\newcommand{\mmani}{$m$-维流形}
\newcommand{\open}[1]{\mathcal{#1}}
\newcommand{\half}{\frac{1}{2}}
\newcommand{\maxo}[1]{\text{max}\curb{#1}}
\newcommand{\mino}[1]{\text{min}\curb{#1}}
\newcommand{\epsclo}{$\epsilon$-接近}
\newcommand{\close}[1]{$#1$-接近}
\newcommand{\cinf}{$C^\infty$}
\newcommand{\cuno}{$C^1$}
\newcommand{\Int}{\text{Int}\,}
\newcommand{\Ext}{\text{Ext}\,}
\newcommand{\funcf}{\mathcal}
\newcommand{\DDu}{\overbar{\DD}}
\newcommand{\DDl}{\underbar{\DD}}
\newcommand{\Diff}[1]{\mathrm{Diff}_{#1}\,}
\newcommand{\Av}[1]{\mathrm{Av}_{#1}\,}
\newcommand{\Lip}[1]{Lipschitz-$#1$}
\newcommand{\sgn}{\mathrm{sgn}}
\newcommand{\eset}{\varnothing}
\newcommand{\cT}{\mathcal{T}}
\newcommand{\cS}{\mathcal{S}}
\newcommand{\cG}{\mathcal{G}}
\newcommand{\cF}{\mathcal{F}}
\newcommand{\cC}{\mathcal{C}}
\newcommand{\cB}{\mathcal{B}}
\newcommand{\inter}[1]{\mathring{#1}}
\newcommand{\forest}[3]{对于{#1},存在{#2},使得{#3}}
\newcommand{\tuno}{$T_1$公理}
\newcommand{\isom}{\overset{\sim}{=}}
\newcommand{\diam}{\mathrm{diam}\,}
\newcommand{\ord}[1]{\abs{#1}}
\newcommand{\sbm}[1]{\overbar{#1}}
\newcommand{\inv}[1]{#1^{-1}}
\newcommand{\restr}[2]{#1|_{#2}}
\newcommand{\divs}{|}
\newcommand{\ndivs}{\nmid}
\newcommand{\modeq}[1]{\overbar{#1}}
\newcommand{\ggen}[1]{\langle#1\rangle}
\newcommand{\ggencond}{\braket}

\newcommand{\hd}{H\"{o}lder}

\renewcommand{\proofname}{证明}

\newenvironment{cenum}{\begin{enumerate}\itemsep0em}{\end{enumerate}}

\newtheorem{definition}{定义}[section]
\newtheorem{lemma}{引理}[section]
\newtheorem{theorem}{定理}[section]
\newtheorem{collary}{推论}[section]
\newtheorem{corollary}{推论}[section]
\newtheorem{proposition}{命题}[section]
\newtheorem{axiom}{公理}[section]
\newtheorem{ex}{例}[section]
\newtheorem{reflection}{反射}[section]
\newcommand{\refl}[1]{\vspace{0.5em}\par\noindent\fbox{%
    \parbox{0.9\textwidth}{%
    \begin{reflection}
        #1
    \end{reflection}
    }%
}\vspace{0.5em}\par}
\newcommand{\rref}[1]{反射\ref{refl:#1}}
\newcommand{\tbref}[1]{表\ref{table:#1}}
\allowdisplaybreaks

\newenvironment{aenum}{\begin{enumerate}[label=\textnormal{(\alph*)}]}{\end{enumerate}}

% Physics Head

\DeclareSIUnit\dyne{dynes}

\newcommand{\ddel}[1]{\frac{\partial}{\partial #1}}
\newcommand{\ddelon}[2]{\frac{\partial #1}{\partial #2}}
\newcommand{\dddel}[1]{\frac{\partial^2}{\partial^2 #1}}
\newcommand{\ddt}{\ddel{t}}
\newcommand{\ddT}{\ddel{T}}
\newcommand{\ddV}{\ddel{V}}
\newcommand{\ddr}{\ddel{r}}
\newcommand{\ddth}{\ddel{\theta}}
\newcommand{\ddph}{\ddel{\phi}}
\newcommand{\dddt}{\dddel{t}}
\newcommand{\dddr}{\dddel{t}}
\newcommand{\dddth}{\dddel{\theta}}
\newcommand{\dddph}{\dddel{\phi}}
\newcommand{\rd}[1]{\mathrm{d} #1}
\newcommand{\dt}{\rd{t}}
\newcommand{\dy}{\rd{y}}
\newcommand{\dx}{\rd{x}}
\newcommand{\edd}[1]{\frac{\mathrm{d}}{\mathrm{d} #1}}
\newcommand{\eddd}[1]{\frac{\mathrm{d}^2}{\mathrm{d}^2 #1}}
\newcommand{\eddon}[2]{\frac{\mathrm{d} #1}{\mathrm{d} #2}}
\newcommand{\edddon}[2]{\frac{\mathrm{d}^2 #1}{\mathrm{d}^2 #2}}
\newcommand{\edt}{\edd{t}}
\newcommand{\edton}[1]{\eddon{#1}{t}}
\newcommand{\edT}{\edd{T}}
\newcommand{\edr}{\edd{r}}
\newcommand{\edl}{\edd{l}}
\newcommand{\edx}{\edd{x}}
\newcommand{\edth}{\edd{\theta}}
\newcommand{\eddton}[1]{\edddon{#1}{t}}
\newcommand{\eddzon}[1]{\edddon{#1}{z}}
\newcommand{\vect}[1]{\boldsymbol{#1}}
\newcommand{\alp}{\frac{1}{\sqrt{2}}}
\newcommand{\alpi}{\frac{i}{\sqrt{2}}}
\newcommand{\expc}[1]{\langle#1\rangle}
\newcommand{\bkn}[1]{\bra{#1}\ket{#1}}
\newcommand{\bk}[2]{\bra{#1}\ket{#2}}
\newcommand{\bik}[3]{\bra{#1} #2 \ket{#3}}
\newcommand{\vari}[1]{\sigma_{#1}}
\newcommand{\intc}[2]{\left[#1, #2\right]}
\newcommand{\sch}{Schr\"{o}dinger}
\newcommand{\moment}{\boldsymbol{p}}
\newcommand{\coor}{\boldsymbol{x}}
\newcommand{\lapc}{\nabla^2}
% \newcommand{\rec}[1]{\frac{1}{#1}}
\newcommand{\vva}{\boldsymbol{a}}
\newcommand{\vvb}{\boldsymbol{b}}
\newcommand{\vc}{\boldsymbol{c}}
\newcommand{\vd}{\boldsymbol{d}}
\newcommand{\ve}{\boldsymbol{e}}
\newcommand{\vf}{\boldsymbol{f}}
\newcommand{\vg}{\boldsymbol{g}}
\newcommand{\vh}{\boldsymbol{h}}
\newcommand{\vi}{\boldsymbol{i}}
\newcommand{\vj}{\boldsymbol{j}}
\newcommand{\vk}{\boldsymbol{k}}
\newcommand{\vl}{\boldsymbol{l}}
\newcommand{\vm}{\boldsymbol{m}}
\newcommand{\vn}{\boldsymbol{n}}
\newcommand{\vo}{\boldsymbol{o}}
\newcommand{\vp}{\boldsymbol{p}}
\newcommand{\vq}{\boldsymbol{q}}
\newcommand{\vr}{\boldsymbol{r}}
\newcommand{\vs}{\boldsymbol{s}}
\newcommand{\vt}{\boldsymbol{t}}
\newcommand{\vvu}{\boldsymbol{u}}
\newcommand{\vv}{\boldsymbol{v}}
\newcommand{\vw}{\boldsymbol{w}}
\newcommand{\vx}{\boldsymbol{x}}
\newcommand{\vy}{\boldsymbol{y}}
\newcommand{\vz}{\boldsymbol{z}}
\newcommand{\vA}{\boldsymbol{A}}
\newcommand{\vB}{\boldsymbol{B}}
\newcommand{\vC}{\boldsymbol{C}}
\newcommand{\vD}{\boldsymbol{D}}
\newcommand{\vE}{\boldsymbol{E}}
\newcommand{\vF}{\boldsymbol{F}}
\newcommand{\vG}{\boldsymbol{G}}
\newcommand{\vH}{\boldsymbol{H}}
\newcommand{\vI}{\boldsymbol{I}}
\newcommand{\vJ}{\boldsymbol{J}}
\newcommand{\vK}{\boldsymbol{K}}
\newcommand{\vL}{\boldsymbol{L}}
\newcommand{\vM}{\boldsymbol{M}}
\newcommand{\vN}{\boldsymbol{N}}
\newcommand{\vO}{\boldsymbol{O}}
\newcommand{\vP}{\boldsymbol{P}}
\newcommand{\vQ}{\boldsymbol{Q}}
\newcommand{\vR}{\boldsymbol{R}}
\newcommand{\vS}{\boldsymbol{S}}
\newcommand{\vT}{\boldsymbol{T}}
\newcommand{\vU}{\boldsymbol{U}}
\newcommand{\vV}{\boldsymbol{V}}
\newcommand{\vW}{\boldsymbol{W}}
\newcommand{\vX}{\boldsymbol{X}}
\newcommand{\vY}{\boldsymbol{Y}}
\newcommand{\vZ}{\boldsymbol{Z}}
\newcommand{\vzero}{\boldsymbol{0}}
\newcommand{\vomega}{\boldsymbol{\omega}}
%\newcommand{\half}{\frac{1}{2}}
\newcommand{\thalf}{\frac{3}{2}}
\newcommand{\rot}{\nabla\times}
\newcommand{\divg}{\nabla\cdot}
\newcommand{\cE}{\mathcal{E}}
\newcommand{\conclu}[1]{\vspace{1em}\par\noindent\fbox{\parbox{0.9\textwidth}{#1}}\vspace{1em}}
\newcommand{\subentrynote}{$\bullet$}
\newcommand{\keypoint}[1]{\par\subentrynote\quad #1 \par}
\newcommand{\fconclu}{\boxed}
\newcommand{\pair}[2]{#1 \, #2}
\newcommand{\intn}[2]{\int #1 \,\mathrm{d} #2}
\newcommand{\intu}[3]{\int_0^{#1} #2 \,\mathrm{d} #3}
\newcommand{\intiu}[3]{\int_{-\infty}^{#1} #2 \, \rd{} #3}
\newcommand{\intui}[2]{\int_0^{\infty} #1 \,\mathrm{d} #2}
\newcommand{\intii}[2]{\int_{-\infty}^{\infty} #1 \,\mathrm{d} #2}
\newcommand{\intt}[2]{\int_0^\infty #1 \, \rd{} #2}
\newcommand{\intr}[2]{\int_{-\infty}^{\infty} #1 \, \rd{} #2}
\newcommand{\intbi}[3]{\int_{#1}^{\infty} #2 \, \rd{} #3}
\newcommand{\intab}[4]{\int_{#1}^{#2} #3 \, \rd{} #4}
\newcommand{\bfactor}[1]{e^{-#1/k_BT}}
\newcommand{\pbfactor}[1]{e^{#1/k_BT}}
\newcommand{\dn}[2]{#1^{\pare{#2}}}
\newcommand{\prodg}[1]{\pare{#1}^\times}

\newcommand{\notion}{\emph}
\newcommand{\iP}{\mathcal{P}}
\newcommand{\eiP}{e^{-\iP}}
\newcommand{\iF}{\mathcal{F}}
\newcommand{\eiF}{e^{-\iF}}
\newcommand{\iG}{\mathcal{G}}

\newcommand{\rc}{r\cos\theta}
\newcommand{\rs}{r\sin\theta}
\newcommand{\sn}{\mathrm{sn}}
\newcommand{\cn}{\mathrm{cn}}
\newcommand{\rdn}{\mathrm{dn}}

\newcommand{\hankel}{H_p^{\pare{1}}}
\newcommand{\hankell}{H_p^{\pare{2}}}
\newcommand{\hhankel}{H_n^{\pare{1}}}
\newcommand{\hhankell}{H_n^{\pare{2}}}
\newcommand{\ber}{\text{ber}\,}
\newcommand{\bei}{\text{bei}\,}
\newcommand{\kker}{\text{ker}\,}
\newcommand{\kei}{\text{kei}\,}
\newcommand{\Ai}{\text{Ai}}
\newcommand{\Bi}{\text{Bi}}

\newcommand{\re}{\text{Re}\,}

\newcommand{\Fp}{F_\phi}
\newcommand{\Ep}{E_\phi}
\newcommand{\Fx}{F_x}
\newcommand{\FF}{\mathbf{F}}
\newcommand{\Ex}{E_x}

%\newcommand{\erf}{\mathrm{erf}}
\newcommand{\erfi}{\mathrm{erfi}}
\newcommand{\erfc}{\mathrm{erfc}}
\newcommand{\ehxs}[1]{e^{-\frac{#1^2}{2}}}
\newcommand{\dcol}[2]{\[ \left.#1 \hspace{1em}\right\vert\hspace{1em} #2 \]}
\newcommand{\titlegamma}{\texorpdfstring{$\Gamma$}{Gamma}}
\newcommand{\titleB}{\texorpdfstring{$B$}{B}}

% Computer Science Head
\usepackage{listings}
\usepackage{color}

\definecolor{dkgreen}{rgb}{0,0.6,0}
\definecolor{gray}{rgb}{0.5,0.5,0.5}
\definecolor{mauve}{rgb}{0.58,0,0.82}

\lstset{frame=tb,
  language=Java,
  aboveskip=3mm,
  belowskip=3mm,
  showstringspaces=false,
  columns=flexible,
  basicstyle={\small\ttfamily},
  numbers=none,
  numberstyle=\tiny\color{gray},
  keywordstyle=\color{blue},
  commentstyle=\color{dkgreen},
  stringstyle=\color{mauve},
  breaklines=true,
  breakatwhitespace=true,
  tabsize=3
}
\lstset{language=Java}
\newcommand{\snp}[1]{\lstinline!#1!}
\newcommand{\term}[2]{\textbf{#1(#2)}}
\def\allfiles{}

\title{数学分析,实分析与复分析笔记}
\author{C.Z.}
\allowdisplaybreaks
\begin{document}
  \maketitle
  \tableofcontents
  \section*{补足}
  ZF集合论公理以及映射的定义仍待加入。《陶哲轩实分析》有对ZF公理的完善陈述,可供援用。
  \par
  自然数的集构造也有待加入。
  \par

  
  %Ch1.GeneralSetTheory.tex
\ifx\allfiles\undefined
\documentclass{ctexrep}
% Mathematics Include

\usepackage{amsmath}
\usepackage{amssymb}
\usepackage{amsthm}
\usepackage{amsfonts}
\usepackage{mathrsfs}
\usepackage{enumitem}
\usepackage{braket}
\usepackage{hyperref}
\usepackage[all, pdf]{xy}

% Physics Include
\usepackage{amsmath}
\usepackage{physics}
\usepackage{siunitx}
\usepackage[makeroom]{cancel}
\usepackage{pstricks}
\usepackage{pstricks-add}
\psset{algebraic=true}

\usepackage[version=4]{mhchem}
\usepackage{array,booktabs}
\usepackage{longtable}
\usepackage{mathtools}
\usepackage[normalem]{ulem}
\usepackage{multicol}

% Mathematics Head

\newcommand{\pare}[1]{\left(#1\right)}
\newcommand{\blr}[1]{\left[#1\right)}
\newcommand{\lbr}[1]{\left(#1\right]}
\newcommand{\brac}[1]{\left[#1\right]}
\newcommand{\curb}[1]{\left\{#1\right\}}
% \newcommand{\abs}[1]{\left|\, #1 \,\right|}
\newcommand{\rec}[1]{\frac{1}{#1}}
\newcommand{\N}{\mathbb{N}}
\newcommand{\bC}{\mathbb{C}}
\newcommand{\Q}{\mathbb{Q}}
\newcommand{\Z}{\mathbb{Z}}
\newcommand{\R}{\mathbb{R}}
\newcommand{\unk}{\mathcal{X}}
\newcommand{\bu}[3]{#1_{#2}^{\pare{#3}}}
\newcommand{\dref}[1]{定义\ref{def:#1}}
\newcommand{\tref}[1]{定理\ref{thm:#1}}
\newcommand{\lref}[1]{引理\ref{lem:#1}}
\newcommand{\cref}[1]{推论\ref{coll:#1}}
\newcommand{\pref}[1]{命题\ref{prp:#1}}
\newcommand{\eref}[1]{例\ref{ex:#1}}
\newcommand{\func}[3]{#1:\, #2 \rightarrow #3}
\newcommand{\overbar}[1]{\mkern 1.5mu\overline{\mkern-1.5mu#1\mkern-1.5mu}\mkern 1.5mu}
\newcommand{\clo}[1]{\overbar{#1}}
\newcommand{\supi}[2]{\overbar{\int_{#1}^{#2}}}
\newcommand{\infi}[2]{\underbar{\int_{#1}^{#2}}}
\newcommand{\setf}{\mathscr}
\newcommand{\bool}{\mathrm{bool}}
\newcommand{\inc}{++}
\newcommand{\defeq}{:=}
\newcommand{\ntuple}{$n$元组}
\newcommand{\card}[1]{\#\pare{#1}}
\newcommand{\setcond}[2]{\curb{#1 \, \left| \, #2 \right.}}
\newcommand{\setcondl}[2]{\curb{\left. #1 \, \right| \, #2}}
\newcommand{\bv}[1]{\mathbf{#1}}
\newcommand{\bfa}{\bv{a}}
\newcommand{\bfb}{\bv{b}}
\newcommand{\bfx}{\bv{x}}
\newcommand{\bfy}{\bv{y}}
\newcommand{\bfe}{\bv{e}}
\newcommand{\bfF}{\bv{F}}
\newcommand{\bff}{\bv{f}}
\newcommand{\bfG}{\bv{G}}
\newcommand{\bfH}{\bv{H}}
\newcommand{\bfg}{\bv{g}}
\newcommand{\bfh}{\bv{h}}
\newcommand{\bfr}{\bv{r}}
\newcommand{\bfk}{\bv{k}}
\newcommand{\bfu}{\bv{u}}
\newcommand{\bfv}{\bv{v}}
\newcommand{\oo}[1]{o\pare{#1}}
\newcommand{\OO}[1]{O\pare{#1}}
% \newcommand{\norm}[1]{\left\| #1 \right\|}
\newcommand{\DD}{\mathbf{D}}
\newcommand{\comp}{\circ}
\newcommand{\const}{\mathrm{const}}
\newcommand{\dist}[2]{d\pare{#1,#2}}
\newcommand{\len}{\ell}
\newcommand{\siga}{$\sigma$-代数}
\newcommand{\cara}{Carath\'{e}odory}
\newcommand{\Gd}{G_\delta}
\newcommand{\Fs}{F_\sigma}
\newcommand{\mmani}{$m$-维流形}
\newcommand{\open}[1]{\mathcal{#1}}
\newcommand{\half}{\frac{1}{2}}
\newcommand{\maxo}[1]{\text{max}\curb{#1}}
\newcommand{\mino}[1]{\text{min}\curb{#1}}
\newcommand{\epsclo}{$\epsilon$-接近}
\newcommand{\close}[1]{$#1$-接近}
\newcommand{\cinf}{$C^\infty$}
\newcommand{\cuno}{$C^1$}
\newcommand{\Int}{\text{Int}\,}
\newcommand{\Ext}{\text{Ext}\,}
\newcommand{\funcf}{\mathcal}
\newcommand{\DDu}{\overbar{\DD}}
\newcommand{\DDl}{\underbar{\DD}}
\newcommand{\Diff}[1]{\mathrm{Diff}_{#1}\,}
\newcommand{\Av}[1]{\mathrm{Av}_{#1}\,}
\newcommand{\Lip}[1]{Lipschitz-$#1$}
\newcommand{\sgn}{\mathrm{sgn}}
\newcommand{\eset}{\varnothing}
\newcommand{\cT}{\mathcal{T}}
\newcommand{\cS}{\mathcal{S}}
\newcommand{\cG}{\mathcal{G}}
\newcommand{\cF}{\mathcal{F}}
\newcommand{\cC}{\mathcal{C}}
\newcommand{\cB}{\mathcal{B}}
\newcommand{\inter}[1]{\mathring{#1}}
\newcommand{\forest}[3]{对于{#1},存在{#2},使得{#3}}
\newcommand{\tuno}{$T_1$公理}
\newcommand{\isom}{\overset{\sim}{=}}
\newcommand{\diam}{\mathrm{diam}\,}
\newcommand{\ord}[1]{\abs{#1}}
\newcommand{\sbm}[1]{\overbar{#1}}
\newcommand{\inv}[1]{#1^{-1}}
\newcommand{\restr}[2]{#1|_{#2}}
\newcommand{\divs}{|}
\newcommand{\ndivs}{\nmid}
\newcommand{\modeq}[1]{\overbar{#1}}
\newcommand{\ggen}[1]{\langle#1\rangle}
\newcommand{\ggencond}{\braket}

\newcommand{\hd}{H\"{o}lder}

\renewcommand{\proofname}{证明}

\newenvironment{cenum}{\begin{enumerate}\itemsep0em}{\end{enumerate}}

\newtheorem{definition}{定义}[section]
\newtheorem{lemma}{引理}[section]
\newtheorem{theorem}{定理}[section]
\newtheorem{collary}{推论}[section]
\newtheorem{corollary}{推论}[section]
\newtheorem{proposition}{命题}[section]
\newtheorem{axiom}{公理}[section]
\newtheorem{ex}{例}[section]
\newtheorem{reflection}{反射}[section]
\newcommand{\refl}[1]{\vspace{0.5em}\par\noindent\fbox{%
    \parbox{0.9\textwidth}{%
    \begin{reflection}
        #1
    \end{reflection}
    }%
}\vspace{0.5em}\par}
\newcommand{\rref}[1]{反射\ref{refl:#1}}
\newcommand{\tbref}[1]{表\ref{table:#1}}
\allowdisplaybreaks

\newenvironment{aenum}{\begin{enumerate}[label=\textnormal{(\alph*)}]}{\end{enumerate}}

% Physics Head

\DeclareSIUnit\dyne{dynes}

\newcommand{\ddel}[1]{\frac{\partial}{\partial #1}}
\newcommand{\ddelon}[2]{\frac{\partial #1}{\partial #2}}
\newcommand{\dddel}[1]{\frac{\partial^2}{\partial^2 #1}}
\newcommand{\ddt}{\ddel{t}}
\newcommand{\ddT}{\ddel{T}}
\newcommand{\ddV}{\ddel{V}}
\newcommand{\ddr}{\ddel{r}}
\newcommand{\ddth}{\ddel{\theta}}
\newcommand{\ddph}{\ddel{\phi}}
\newcommand{\dddt}{\dddel{t}}
\newcommand{\dddr}{\dddel{t}}
\newcommand{\dddth}{\dddel{\theta}}
\newcommand{\dddph}{\dddel{\phi}}
\newcommand{\rd}[1]{\mathrm{d} #1}
\newcommand{\dt}{\rd{t}}
\newcommand{\dy}{\rd{y}}
\newcommand{\dx}{\rd{x}}
\newcommand{\edd}[1]{\frac{\mathrm{d}}{\mathrm{d} #1}}
\newcommand{\eddd}[1]{\frac{\mathrm{d}^2}{\mathrm{d}^2 #1}}
\newcommand{\eddon}[2]{\frac{\mathrm{d} #1}{\mathrm{d} #2}}
\newcommand{\edddon}[2]{\frac{\mathrm{d}^2 #1}{\mathrm{d}^2 #2}}
\newcommand{\edt}{\edd{t}}
\newcommand{\edton}[1]{\eddon{#1}{t}}
\newcommand{\edT}{\edd{T}}
\newcommand{\edr}{\edd{r}}
\newcommand{\edl}{\edd{l}}
\newcommand{\edx}{\edd{x}}
\newcommand{\edth}{\edd{\theta}}
\newcommand{\eddton}[1]{\edddon{#1}{t}}
\newcommand{\eddzon}[1]{\edddon{#1}{z}}
\newcommand{\vect}[1]{\boldsymbol{#1}}
\newcommand{\alp}{\frac{1}{\sqrt{2}}}
\newcommand{\alpi}{\frac{i}{\sqrt{2}}}
\newcommand{\expc}[1]{\langle#1\rangle}
\newcommand{\bkn}[1]{\bra{#1}\ket{#1}}
\newcommand{\bk}[2]{\bra{#1}\ket{#2}}
\newcommand{\bik}[3]{\bra{#1} #2 \ket{#3}}
\newcommand{\vari}[1]{\sigma_{#1}}
\newcommand{\intc}[2]{\left[#1, #2\right]}
\newcommand{\sch}{Schr\"{o}dinger}
\newcommand{\moment}{\boldsymbol{p}}
\newcommand{\coor}{\boldsymbol{x}}
\newcommand{\lapc}{\nabla^2}
% \newcommand{\rec}[1]{\frac{1}{#1}}
\newcommand{\vva}{\boldsymbol{a}}
\newcommand{\vvb}{\boldsymbol{b}}
\newcommand{\vc}{\boldsymbol{c}}
\newcommand{\vd}{\boldsymbol{d}}
\newcommand{\ve}{\boldsymbol{e}}
\newcommand{\vf}{\boldsymbol{f}}
\newcommand{\vg}{\boldsymbol{g}}
\newcommand{\vh}{\boldsymbol{h}}
\newcommand{\vi}{\boldsymbol{i}}
\newcommand{\vj}{\boldsymbol{j}}
\newcommand{\vk}{\boldsymbol{k}}
\newcommand{\vl}{\boldsymbol{l}}
\newcommand{\vm}{\boldsymbol{m}}
\newcommand{\vn}{\boldsymbol{n}}
\newcommand{\vo}{\boldsymbol{o}}
\newcommand{\vp}{\boldsymbol{p}}
\newcommand{\vq}{\boldsymbol{q}}
\newcommand{\vr}{\boldsymbol{r}}
\newcommand{\vs}{\boldsymbol{s}}
\newcommand{\vt}{\boldsymbol{t}}
\newcommand{\vvu}{\boldsymbol{u}}
\newcommand{\vv}{\boldsymbol{v}}
\newcommand{\vw}{\boldsymbol{w}}
\newcommand{\vx}{\boldsymbol{x}}
\newcommand{\vy}{\boldsymbol{y}}
\newcommand{\vz}{\boldsymbol{z}}
\newcommand{\vA}{\boldsymbol{A}}
\newcommand{\vB}{\boldsymbol{B}}
\newcommand{\vC}{\boldsymbol{C}}
\newcommand{\vD}{\boldsymbol{D}}
\newcommand{\vE}{\boldsymbol{E}}
\newcommand{\vF}{\boldsymbol{F}}
\newcommand{\vG}{\boldsymbol{G}}
\newcommand{\vH}{\boldsymbol{H}}
\newcommand{\vI}{\boldsymbol{I}}
\newcommand{\vJ}{\boldsymbol{J}}
\newcommand{\vK}{\boldsymbol{K}}
\newcommand{\vL}{\boldsymbol{L}}
\newcommand{\vM}{\boldsymbol{M}}
\newcommand{\vN}{\boldsymbol{N}}
\newcommand{\vO}{\boldsymbol{O}}
\newcommand{\vP}{\boldsymbol{P}}
\newcommand{\vQ}{\boldsymbol{Q}}
\newcommand{\vR}{\boldsymbol{R}}
\newcommand{\vS}{\boldsymbol{S}}
\newcommand{\vT}{\boldsymbol{T}}
\newcommand{\vU}{\boldsymbol{U}}
\newcommand{\vV}{\boldsymbol{V}}
\newcommand{\vW}{\boldsymbol{W}}
\newcommand{\vX}{\boldsymbol{X}}
\newcommand{\vY}{\boldsymbol{Y}}
\newcommand{\vZ}{\boldsymbol{Z}}
\newcommand{\vzero}{\boldsymbol{0}}
\newcommand{\vomega}{\boldsymbol{\omega}}
%\newcommand{\half}{\frac{1}{2}}
\newcommand{\thalf}{\frac{3}{2}}
\newcommand{\rot}{\nabla\times}
\newcommand{\divg}{\nabla\cdot}
\newcommand{\cE}{\mathcal{E}}
\newcommand{\conclu}[1]{\vspace{1em}\par\noindent\fbox{\parbox{0.9\textwidth}{#1}}\vspace{1em}}
\newcommand{\subentrynote}{$\bullet$}
\newcommand{\keypoint}[1]{\par\subentrynote\quad #1 \par}
\newcommand{\fconclu}{\boxed}
\newcommand{\pair}[2]{#1 \, #2}
\newcommand{\intn}[2]{\int #1 \,\mathrm{d} #2}
\newcommand{\intu}[3]{\int_0^{#1} #2 \,\mathrm{d} #3}
\newcommand{\intiu}[3]{\int_{-\infty}^{#1} #2 \, \rd{} #3}
\newcommand{\intui}[2]{\int_0^{\infty} #1 \,\mathrm{d} #2}
\newcommand{\intii}[2]{\int_{-\infty}^{\infty} #1 \,\mathrm{d} #2}
\newcommand{\intt}[2]{\int_0^\infty #1 \, \rd{} #2}
\newcommand{\intr}[2]{\int_{-\infty}^{\infty} #1 \, \rd{} #2}
\newcommand{\intbi}[3]{\int_{#1}^{\infty} #2 \, \rd{} #3}
\newcommand{\intab}[4]{\int_{#1}^{#2} #3 \, \rd{} #4}
\newcommand{\bfactor}[1]{e^{-#1/k_BT}}
\newcommand{\pbfactor}[1]{e^{#1/k_BT}}
\newcommand{\dn}[2]{#1^{\pare{#2}}}
\newcommand{\prodg}[1]{\pare{#1}^\times}

\newcommand{\notion}{\emph}
\newcommand{\iP}{\mathcal{P}}
\newcommand{\eiP}{e^{-\iP}}
\newcommand{\iF}{\mathcal{F}}
\newcommand{\eiF}{e^{-\iF}}
\newcommand{\iG}{\mathcal{G}}

\newcommand{\rc}{r\cos\theta}
\newcommand{\rs}{r\sin\theta}
\newcommand{\sn}{\mathrm{sn}}
\newcommand{\cn}{\mathrm{cn}}
\newcommand{\rdn}{\mathrm{dn}}

\newcommand{\hankel}{H_p^{\pare{1}}}
\newcommand{\hankell}{H_p^{\pare{2}}}
\newcommand{\hhankel}{H_n^{\pare{1}}}
\newcommand{\hhankell}{H_n^{\pare{2}}}
\newcommand{\ber}{\text{ber}\,}
\newcommand{\bei}{\text{bei}\,}
\newcommand{\kker}{\text{ker}\,}
\newcommand{\kei}{\text{kei}\,}
\newcommand{\Ai}{\text{Ai}}
\newcommand{\Bi}{\text{Bi}}

\newcommand{\re}{\text{Re}\,}

\newcommand{\Fp}{F_\phi}
\newcommand{\Ep}{E_\phi}
\newcommand{\Fx}{F_x}
\newcommand{\FF}{\mathbf{F}}
\newcommand{\Ex}{E_x}

%\newcommand{\erf}{\mathrm{erf}}
\newcommand{\erfi}{\mathrm{erfi}}
\newcommand{\erfc}{\mathrm{erfc}}
\newcommand{\ehxs}[1]{e^{-\frac{#1^2}{2}}}
\newcommand{\dcol}[2]{\[ \left.#1 \hspace{1em}\right\vert\hspace{1em} #2 \]}
\newcommand{\titlegamma}{\texorpdfstring{$\Gamma$}{Gamma}}
\newcommand{\titleB}{\texorpdfstring{$B$}{B}}

% Computer Science Head
\usepackage{listings}
\usepackage{color}

\definecolor{dkgreen}{rgb}{0,0.6,0}
\definecolor{gray}{rgb}{0.5,0.5,0.5}
\definecolor{mauve}{rgb}{0.58,0,0.82}

\lstset{frame=tb,
  language=Java,
  aboveskip=3mm,
  belowskip=3mm,
  showstringspaces=false,
  columns=flexible,
  basicstyle={\small\ttfamily},
  numbers=none,
  numberstyle=\tiny\color{gray},
  keywordstyle=\color{blue},
  commentstyle=\color{dkgreen},
  stringstyle=\color{mauve},
  breaklines=true,
  breakatwhitespace=true,
  tabsize=3
}
\lstset{language=Java}
\newcommand{\snp}[1]{\lstinline!#1!}
\newcommand{\term}[2]{\textbf{#1(#2)}}
\begin{document}
\fi

%Content

\chapter{普通集合论}
  \section{集与势}
  \subsection{自然数公理及其运算}
  \begin{axiom}
    \[ 0 \in \N. \]
  \end{axiom}
  \begin{axiom}
    若$n\in \N$,则
    \[ n\inc\in \N. \]
  \end{axiom}
  并且作如下定义:
  \begin{definition}
    \[ 1 \defeq 0\inc, \quad 2 \defeq 1\inc, \quad 3 \defeq 2\inc\cdots. \]
  \end{definition}
  然而这并不完备,因为完全可以定义$3\inc=0$从而令之回环。为此,制定第三条公理。
  \begin{axiom}
    $0$不是任何自然数的后继。
  \end{axiom}
  尽管如此,仍然可以定义$4\inc=4$或$4\inc=4$。诸多解决方法之一为如下第四条。
  \begin{axiom}
    后继相同的自然数相等。
  \end{axiom}
  为了保证除了$0,1,2,3,4\cdots$外没有其他杂质,复制定第五条。
  \begin{axiom}[数学归纳法]
    若$P\pare{0}$为真,且$P\pare{n}$为真推出$P\pare{n\inc}$为真,则对于任意$n\in \N$,$P\pare{n}$为真。
  \end{axiom}
  $\N$的存在性视为一公理。
  \subsubsection{加法}
  \begin{definition}
    定义操作$+$为$0+m=0$,$\pare{n\inc}+m=\pare{n+m}\inc$。
  \end{definition}
  \begin{lemma}
    对于任意自然数$n$,$n+0=n$。
  \end{lemma}
  \begin{proof}
    由$0+0=0$开始,归纳可得。
  \end{proof}
  \begin{lemma}
    $n+\pare{m\inc} = \pare{n+m}\inc$。
  \end{lemma}
  \begin{proof}
    先由$0+\pare{m\inc} = \pare{0+m}\inc$开始,对$n$归纳。
  \end{proof}
  可直接推知$n\inc = n+1$。
  \begin{proposition}
    \[ n+m=m+n. \]
  \end{proposition}
  \begin{proof}
    先由$0+m=m+0$,后借助前一命题对$n$归纳.
  \end{proof}
  \begin{proposition}
    \[ \pare{a+b}+c = a+\pare{b+c}. \]
  \end{proposition}
  \begin{proof}
    先由$\pare{a+b}+0=a+\pare{b+0}$,后对$c$归纳可得。
  \end{proof}
  \begin{proposition}
    若$a+b=a+c$,则$b=c$。
  \end{proposition}
  \begin{proof}
    从$0+b=0+c$开始,对$a$归纳可得。
  \end{proof}
  \begin{definition}
    非零自然数为正。
  \end{definition}
  \begin{proposition}
    若$a$为正而$b$为自然数,则$a+b$为正。
  \end{proposition}
  \begin{proof}
    从$b=0$开始归纳可得。
  \end{proof}
  \begin{collary}
    若自然数$a$和$b$满足$a+b=0$则$a=b=0$。
  \end{collary}
  \begin{proposition}
    正数存在前置元素。
  \end{proposition}
  \begin{proof}
    归纳可证,注意在$P$中排除$0$。
  \end{proof}
  \begin{definition}[自然数的序]
    若存在$a$,使$n=m+a$,则称$n\ge m$。若$m\ne n$,则称$n>m$。
  \end{definition}
  \begin{proposition}
    \begin{aenum}
      \item $a\ge a.$
      \item 若$a\ge b$且$b \ge c$,则$a \ge c$。
      \item 若$a \ge b$且$b \ge a$,则$a=b$。
      \item $a\ge b$当且仅当$a+c \ge b+c$。
      \item $a < b$当且仅当$a\inc \le b$。
      \item $a < b$当且仅当存在正数$d$,$b=a+d$。
    \end{aenum}
  \end{proposition}
  \begin{proposition}
    下列三个命题中有且仅有一成立:
    \[ a<b, \quad a=b,\quad a>b. \]
  \end{proposition}
  \begin{proof}
    互斥是显然的。下证必有一成立,故成立全序。先设$a=0$,后对$a$归纳可得(归纳时可对$a$分类)。
  \end{proof}
  \begin{proposition}[强归纳原理]
    若对于任意$m\ge m_0$有如下蕴含关系:任意$m_0\le m' <m$有$P\pare{m'}$推出$P\pare{m}$,则$P$对一切$m \ge m_0$成立。
  \end{proposition}
  \subsubsection{乘法}
  \begin{definition}
    定义操作$\times$为$0\times m = 0$,$\pare{n\inc}\times m = n \times m + n$。
  \end{definition}
  \begin{lemma}
    \[ n \times m = m \times n. \]
  \end{lemma}
  \begin{lemma}
    \label{lem:mn0}
    若$n \times m = 0$,则$n$与$m$中至少一者为$0$。且正数相乘为正。
  \end{lemma}
  \begin{proposition}
    \[ a\pare{b+c} = ab+ac. \]
  \end{proposition}
  \begin{proof}
    先证$a\pare{b+0} = ab + a\times 0$,后对$c$归纳。
  \end{proof}
  \begin{proposition}
    \[ \pare{a\times b} \times c= a \times \pare{b \times c}. \]
  \end{proposition}
  \begin{proposition}
    若$a<b$且$c$为正数,则$ac<bc$。
  \end{proposition}
  \begin{proof}
    在$b=a+d$两侧用$c$乘。
  \end{proof}
  \begin{proposition}
    若$ac=bc$且$c$不为$0$,则$a=b$。
  \end{proposition}
  \begin{proof}
    $a<b$,$a=b$与$a>b$有且仅有一成立,两侧乘$c$知必定为$a=b$。
  \end{proof}
  \begin{proposition}
    对自然数$n$与正数$q$,存在$0\le r <q$满足$n=mq+r$。
  \end{proposition}
  \begin{proof}
    对于任意$q$,命题对$n=0$成立,后对$n$归纳。
  \end{proof}
  \begin{definition}
    定义幂运算如下:$m^0=1$,$m^{n\inc}=m^n\times m$。
  \end{definition}
  \subsection{集公理及其运算}
  \begin{definition}
    若$A$的所有元素都是$B$的元素,则称$A$为$B$的子集,$B$为$A$的超集。记作
    \[ A \subset B \quad \text{或} \quad B \supset A. \]
  \end{definition}
  \begin{definition}
    若$A \subset B$且$B\subset A$,则称$A$与$B$相等,记作
    \[ A=B. \]
  \end{definition}
  \begin{definition}
    若$S$包含$A$与$B$中的所有元素又不包含其他元素,称$S$为$A$与$B$的并集,记作
    \[ S = A \cup B. \]
  \end{definition}
  \begin{definition}
    若$S$包含$A$与$B$的所有共同元素又不包含其他元素,称$S$为$A$与$B$的交集,记作
    \[ S = A \cap B. \]
  \end{definition}
  \begin{definition}
    若$S$包含属于$A$且不属于$B$的一切元素又不包含其它元素,称$S$为$A$与$B$的差集,记作
    \[ S = A-B \quad \text{或} \quad S = \complement_A B. \]
  \end{definition}
  \begin{theorem}
    \[ A \cap \pare{\bigcup S_i} = \bigcup \pare{A \cap S_i}, \]
    \[ A \cup \pare{\bigcap S_i} = \bigcap \pare{A \cup S_i}, \]
    \[ A - \pare{B \cup C} = \pare{A-B} \cap \pare{A-C}, \]
    \[ A - \pare{B \cap C} = \pare{A-B} \cup \pare{A-C}. \]
    后二条称为DeMorgan律,即并的补等于补的交,交的补等于补的并。
  \end{theorem}
  \subsubsection{Cartesian乘积}
  \begin{definition}
    $X_1,\cdots,X_n$的Cartesian乘积定义为所有\ntuple 
    \[ \pare{x_1,\cdots,x_n} \]
    的集合。
  \end{definition}
  \begin{lemma}[有限选择]
    若有限诸个$X_i$非空,则$\Pi X_i$非空。
  \end{lemma}
  \begin{proof}
    对$n$归纳即可。
  \end{proof}
  \subsection{映射}
  \begin{definition}
    集合$A$中的一个等价关系,是满足左列三条件的一关系$C$:
    \begin{enumerate}
      \item (自反性)对任意$x$,有$xCx$;
      \item (对称性)若$xCy$,则$yCx$;
      \item (传递性)若$xCy$,$yCz$,则$xCz$。
    \end{enumerate}
  \end{definition}
  不是等价关系的关系,如近似同向,即二向量之间夹角为锐角。
  \begin{definition}
    与$x$等价的所有元素为$x$的等价类。
  \end{definition}
  \begin{definition}
    若存在法则$\varphi$,使得$A$的任意元素存在$B$的唯一元素与之对应,$B$的任意元素存在$A$的唯一元素与之对应,则称$\varphi$建立了$A$与$B$的一一对应。
  \end{definition}
  \begin{definition}
    若$A$与$B$见能建立一一对应,则称$A$与$B$对等,或$A$与$B$具有相同基数,记作
    \[ A \sim B. \]
    $A$的基数记作
    \[ \card{A}. \]
  \end{definition}
  \par
  值得注意的是,两个不等大小的圆、两条不同长度的线段、以及自然数全体与偶数之间,都可以轻易建立起一一对应。
  \begin{theorem}
    对等是等价关系。
  \end{theorem}
  \begin{theorem}
    无交对等集的并仍然对等,即若
    \[ A_n \cap A_{n'} = \varnothing,\quad B_n \cap B_{n'} = \varnothing \quad \pare{n \ne n'}, \]
    且
    \[ A_n \sim B_n, \]
    则
    \[ \bigcup A_n \sim \bigcup B_n. \]
  \end{theorem}
  \subsubsection{有限集}
  \begin{definition}
    若集合$A$与$\pare{1,\cdots,n}$对等,称集合具有基数$n$且为有限集。
  \end{definition}
  \begin{proposition}
    有限集的基数唯一。
  \end{proposition}
  \begin{proof}
    对基数作归纳即可。注意到可证集合挖去一元素后基数为其前置数。
  \end{proof}
  \subsection{可数集}
  \begin{definition}
    与$\N$对等的集合称为可数集,或者称为可数的。
  \end{definition}
  \begin{theorem}
    $A$为可数的当且仅当可以编号$A$的元素使得
    \[ A = \curb{a_1, a_2, a_3, \cdots, a_n, \cdots}. \]
  \end{theorem}
  \begin{theorem}
    无穷集$A$必含有可数子集.
  \end{theorem}
  \begin{proof}
    每次取一元素并编号,取之不尽。
  \end{proof}
  \begin{theorem}
    \label{thm:coutablesub}
    可数集的任何无穷子集皆可数,
  \end{theorem}
  \begin{proof}
    将$A$的元素一一编号列出,逐个检查,遇到子集$B$的元素即自增计数器。
  \end{proof}
  \begin{theorem}
    \[ n+\aleph_0\sim\aleph_0. \]
  \end{theorem}
  \begin{proof}
    先列出有限集的元素,后列出可数集的元素,重新编号即可。
  \end{proof}
  \begin{theorem}
    \[ n\cdot\aleph_0\sim\aleph_0. \]
  \end{theorem}
  \begin{proof}
    逐个列出各集的第一个元素,后列出各集第二个元素,以此类推,重新编号即可。
  \end{proof}
  \begin{theorem}
    \[ \aleph_0 \cdot n \sim \aleph_0. \]
  \end{theorem}
  \begin{proof}
    先写出第一个集合的所有元素,后写出第二个集合的所有元素,以此类推,重新编号即可。
  \end{proof}
  \begin{theorem}
    \label{thm:countableprod}
    \[ \aleph_0\cdot\aleph_0\sim\aleph_0. \]
  \end{theorem}
  \begin{proof}
    \begin{align*}
      A_1 &= \curb{\bu{a}{1}{1},\bu{a}{2}{1},\bu{a}{3}{1},\cdots},\\
      A_2 &= \curb{\bu{a}{1}{2},\bu{a}{2}{2},\bu{a}{3}{2},\cdots},\\
      A_3 &= \curb{\bu{a}{1}{3},\bu{a}{2}{3},\bu{a}{3}{3},\cdots},\\
      &\cdots
    \end{align*}
    沿着反对角线列出各个元素,即
    \[ S = \curb{\bu{a}{1}{1},\bu{a}{2}{1},\bu{a}{1}{2},\bu{a}{3}{1},\bu{a}{2}{2},\bu{a}{1}{3},\bu{a}{4}{1},\cdots}, \]
    可知$S$可数。
  \end{proof}
  \begin{theorem}
    $\Q$为可数集。
  \end{theorem}
  \begin{proof}
    将$\Q$写为$\N^2$去掉可约者,去掉前集合可数,去掉后由\tref{coutablesub}知仍然可数。
  \end{proof}
  \begin{collary}
    任意区间内的$\Q$可数。
  \end{collary}
  \begin{proof}
    仍然借助\tref{coutablesub}。
  \end{proof}
  \begin{theorem}
    无穷集$M$与可数集或有限集$A$的并,其势仍然不变。
  \end{theorem}
  \begin{proof}
    $M$含有一可数集$D$,此可数集与$D \cup A$可一一对应,故$M$的势不变。
  \end{proof}
  \begin{theorem}
    不可数集$M$除去一有限或可数子集$A$,仍有
    \[ M-A \sim M. \]
  \end{theorem}
  \begin{proof}
    $M$去除后仍为无限集,因此据上一定理,有$\pare{M-A} \cup A \sim M-A$。
  \end{proof}
  \begin{collary}
    无穷集包含与自身对等的一子集。
  \end{collary}
  \par
  此时发现一有限集不可能具有的性质,故有如下归功于R. Dedekind的定义。
  \begin{definition}
  	  包含与自身对等的真子集的集合称为无穷集。
  \end{definition}
  \begin{theorem}
    \[ \aleph_0^n \sim \aleph_0. \]
  \end{theorem}
  \begin{proof}
    数学归纳法结合\tref{countableprod}可证。
  \end{proof}
  \begin{collary}
    代数数全体为可数集。
  \end{collary}
  \subsection{连续统的势}
  此处仅暂时借用实数的定义。其具体定义留待下文。
  \begin{theorem}
    线段$U=\brac{0,1}$不可数。
  \end{theorem}
  \begin{proof}
    可以参考\tref{uncountableR}。
    \par
    三等分区间,存在一不包含$x_1$的闭区间。将其再度三等分,存在一不包含$x_2$的闭区间,以此类推。故存在闭区间套,其交在$x_n$之外。
  \end{proof}
  \begin{definition}
    若$A$与$\brac{0,1}$对等,则称$A$具有连续统的势$\aleph$。
  \end{definition}
  \begin{theorem}
    $\brac{a,b}$,$\left(a,b\right]$,$\left[a,b\right)$与$\pare{a,b}$的势均为$\aleph$。
  \end{theorem}
  \begin{proof}
    只需注意无穷集挖去有限集后与原来的无穷集对等。
  \end{proof}
  \begin{theorem}
    \[ n\aleph\sim\aleph. \]
  \end{theorem}
  \begin{theorem}
    \[ \aleph_0\aleph\sim\aleph. \]
  \end{theorem}
  \begin{proof}
    取$\left[0,0.9\right),\left[0.9,0.99\right),\left[0.99,0.999\right),\cdots$分别映射即可。
  \end{proof}
  \begin{collary}
    $\R$的势为$\aleph$。
  \end{collary}
  \begin{collary}
    $\R-\Q$的势为$\aleph$。
  \end{collary}
  \begin{collary}
    超越数的势为$\aleph$。
  \end{collary}
  \begin{theorem}
    正整数列全体的势为$\aleph$。
  \end{theorem}
  \begin{proof}
    直接注意将正整数列写成相应连分数可以与无理数一一对应。亦可以将数列中的正整数看作二进制小数的零位索引差而获得此对应。
  \end{proof}
  \begin{theorem}
    \[ \aleph^n \sim \aleph. \]
  \end{theorem}
  \begin{proof}
    第一个$\aleph$与正整数列全体$\curb{a_n}$对应,第二个与$\curb{b_n}$对应,则
    \[ \pare{a_1,b_1,a_2,b_2,\cdots} \]
    与$\aleph^2$一一对应。对于一般的$\aleph^n$,上述论证仍然适用。
  \end{proof}
  \begin{collary}
    $\R^2\sim\aleph.$
  \end{collary}
  \begin{collary}
    $\R^2\sim\aleph.$
  \end{collary}
  \begin{collary}
    \[ \aleph \cdot \aleph \sim \aleph. \]
  \end{collary}
  \begin{theorem}
    \[ \aleph^{\aleph_0} \sim \aleph. \]
  \end{theorem}
  \begin{proof}
    第一个$\aleph$与正整数列全体$\curb{a_n}$对应,第二个与$\curb{b_n}$对应,以此类推。最终将可数个正整数列的直积按照与\tref{countableprod}相同的办法映射为整数列全体。
  \end{proof}
  \begin{theorem}
    \[ \bool^{\aleph_0}\sim\aleph. \]
  \end{theorem}
  \begin{proof}
    注意bool序列与二进制小数序列的对应即可。
  \end{proof}
  \begin{collary}
    \label{coll:2n}
    若$A\sim 2$,则
    \[ A^{\aleph_0}\sim\aleph. \]
  \end{collary}
  \subsection{势的比较}
  \begin{definition}
    若两个集合对等,则称其具有相同的势。予每个对等的等价类一记号,称此记号为等价类中任一集合的势。
  \end{definition}
  \begin{definition}
    若$A$与$B$不对等,且$B$有子集与$A$对等,称$A$的势小于$B$的势。
  \end{definition}
  \begin{theorem}
    $\brac{0,1}$上的所有实函数的集合的势大于$\aleph_1$。
  \end{theorem}
  \begin{proof}
    设$t$对应$F\pare{t,x}$,则$G\pare{x}=F\pare{x,x}+1$不在任意一个$t$的值域内。
  \end{proof}
  \begin{definition}
    称$\brac{0,1}$上的所有实函数的集合的势为$\aleph_2$。
  \end{definition}
  \begin{theorem}
    集合与其子集族不对等。
  \end{theorem}
  \begin{proof}
    假设$x$映射为$\unk$,则将所有$x\notin\unk$的$x$并起来得到$\mathcal{Y}$,并设元素$y$映射到这个集合。若$y\in\mathcal{Y}$,则$y$不满足条件而应被除名。若$y\notin\mathcal{Y}$,则$y$满足条件而应该处在集合内。
  \end{proof}
  \begin{definition}
    若$M$的势为$\mu$,则$M$的子集族的势为$2^\mu$。
  \end{definition}
  \begin{theorem}
    \[ \aleph_1=2^{\aleph_0}. \]
  \end{theorem}
  \begin{proof}
    这正是\cref{2n}。
  \end{proof}
  \begin{theorem}
    设$A_0\supset A_1 \supset A_2$,若$A_2\sim A_0$,则$A_1 \sim A_0$。
  \end{theorem}
  \begin{proof}
    假设$\varphi$为$A_0$到$A_2$的一一对应,则可设$\varphi\pare{A_1}=A_3$,且$A_3 \subset A_2$。由于是同一映射,故\[A_0-A_1\sim A_2-A_3.\]
    \par
    此时$A_2\subset A_1$且$A_1\sim A_3$,故可设一一对应为$\psi$,如法炮制$\psi\pare{A_2}=A_4$,又有\[A_1-A_2 \sim A_3-A_4.\]
    可以此类推,并设$A_\infty = D$,即
    \[ \bigcap A_i = D. \]
    又
    \begin{align*}
      A &= \pare{A_0-A_1}+\pare{A_1-A_2}+\pare{A_2-A_3}+\pare{A_3-A_4}+\cdots+D,\\
      A_1 &= \pare{A_1-A_2}+\pare{A_2-A_3}+\pare{A_3-A_4}+\pare{A_4-A_5}+\cdots+D.
    \end{align*}
    左上与右下对等,右上与左下相同,故$A$与$A_1$对等。
  \end{proof}
  \begin{theorem}[E. Schr\"{o}der--F. Bernstein]设$A$与$B$彼此与对方一子集对等,则彼此对等。
  \end{theorem}
  \begin{proof}
    设
    \[ \varphi\pare{A} = B^*,\qquad \psi\pare{B}=A^*. \]
    则
    \[ \psi\varphi\pare{A} = \psi\pare{B^*} = A^{**}, \]
    即$A$与$A^{**}$对等。由上一定理,$A$与$A^*$对等,故与$B$对等。
  \end{proof}
  \begin{collary}
    势之间大于、等于、小于择一成立。
  \end{collary}
  \begin{proof}
    若同时大于且小于,则由势的大于小于的定义可得上一定理的题设,故两势相等。
  \end{proof}
  \begin{collary}
    势的小于具有传递性。
  \end{collary}
  \begin{proof}
    把仲叔通过$\varphi$和$\psi$映射到伯,得到伯下二嵌套子集。若伯叔对等,则仲亦然矣。
  \end{proof}
  \begin{theorem}
    $\brac{0,1}$上连续函数集的势为$\aleph_1$。
  \end{theorem}
  \begin{proof}
    注意连续函数仅仅取决于$\Q$处的值。
  \end{proof}
  \section{数系}
  \subsection{序关系}
  \begin{definition}
    关系$C$称为全序关系,若满足
    \begin{enumerate}
      \item 对任意$x\ne y$的$x$和$y$,$xCy$与$yCx$二者有一成立;
      \item 不存在$xCx$;
      \item 若$xCy$,$yCz$,则$xCz$。
    \end{enumerate}
  \end{definition}
  \begin{definition}
    对于$a<b$,称
    \[ \setcond{x}{a<x<b} \]
    为开区间。若其为空集,则称$b$为$a$的紧接后元。
  \end{definition}
  前述自然数已被赋予一全序关系,$n\inc$为$n$的紧接后元。
  \begin{definition}
    对于Castesian乘积$A\times B$,可定义字典序关系:当$a_1<b_1$且$a_2<b_2$,有
    \[ a_1\times b_1 < a_2 \times b_2. \]
  \end{definition}
  可证其为一全序关系。
  \begin{definition}
    若$A$具有全序关系$<$,若对于任意$x\in A$有$x\le b$,则$b$为最大元。类似定义最小元。
  \end{definition}
  \begin{definition}
    $A$的子集$A_0$是有上界的,如果存在$b$使对任意$a_0\in A_0$有$a_0\le b$。若所有上界的集合存在最小元,则称之上确界。类似定义下界与下确界。
  \end{definition}
  \begin{definition}
    若$A$的任意有上界的非空子集$A_0$均有上确界,则称$A$具有上确界性质。类似定义下确界性质。
  \end{definition}
  \begin{theorem}
    集合$A$具有上确界性质当且仅当其具有下确界性质。
  \end{theorem}
  \begin{proof}
	  假设集合$A$具有上确界性质且$A_0$为有下界的一非空子集,则其所有下界的集合$B_0$存在一上确界。设$B_0$的所有上界集合为$C_0$,则上确界性质表明存在$c_0$为\emph{所有下界的所有上界的最小元},即$C_0$存在最小元$c_0$。
	  \par
	  如果存在一个元素$a_0\le c_0$,则必有$c_0=a_0$,否则$a_0$可替换$c_0$的位置。故对于任意$a_0$,有$c_0\le a_0$(全序的二择一成立)。因此$c_0\in B_0$。
	  \par
	  而$A_0$的下确界定义为\emph{所有下界的最大元},即$B_0$的最大元。可以证明$c_0$为其最大元。否则$c_0$不能为上界。
  \end{proof}
  \subsection{整数}
  \begin{definition}
    整数是形如$a-b$的表达式,且视$a-b=c-d$当且仅当$a+d=b+c$。
  \end{definition}
  此处利用等价关系将$\Z$视为$\N^2$的商构造,需要验证等价关系的自反性,对称性和传递性。前而者显然,传递性要求在
  \begin{alignat*}{2}
    a-b&=c-d, \quad a+d &= b+c,\\
    c-d&=e-f, \quad c+f &= d+e
  \end{alignat*}
  的假设下证明
  \[ a+f = b+e. \]
  将两式相加并消去即可。
  \begin{definition}
    整数的和定义为
    \[ \pare{a-b}+\pare{c-d} = \pare{a+c}-\pare{b+d}. \]
  \end{definition}
  \begin{definition}
    整数的积定义为
    \[ \pare{a-b}\times\pare{c-d} = \pare{ac+bd}-\pare{ad+bc}. \]
  \end{definition}
  \begin{theorem}
    上述二定义在等价类内相容,即结果与代表元的选取无关。
  \end{theorem}
  \begin{proof}
    只证乘法的部分。设$a-b=a'-b'$,即
    \[ a+b' = a'+b. \]
    便需要证
    \[ ac+bd+a'd+b'c=a'c+b'd+ad+bc. \]
    合并有
    \[ c\pare{a+b'}+d\pare{a'+b} = c\pare{a'+b}+d\pare{a+b'}. \]
    由假设知成立。
  \end{proof}
  可以将$n-0$与自然数$n$对应,从而保持上述加法与乘法的结构,此之谓同构。
  \begin{definition}
    定义$\pare{a-b}$的负为$\pare{b-a}$,记作$-\pare{a-b}$。
  \end{definition}
  同样容易证明其于等价类内相容。
  \begin{definition}
    定义负数为正自然数所对应整数的负。
  \end{definition}
  \begin{theorem}[三歧性(trichofomy)]
    任意整数成立如下三个命题之一:
    \begin{aenum}
      \item $x$为零;
      \item $x$对应正的自然数;
      \item $x$对一个正自然数的负。
    \end{aenum}
  \end{theorem}
  \begin{proof}
    对$x=a-b$中$a$和$b$的大小关系分类即可。
    \par
    为了证明三者中仅有一成立,分若干类讨论。当$x$为零时显然不能为正自然数。若其为负自然数则对一正的$n$,有$0-0=0-n$,从而$n=0$,矛盾。若同时为正负则相似可证矛盾。
  \end{proof}
  事实上也可以通过假定三歧性来定义整数,这是大陆教科书的做法,导致了运算验证上的极大混乱。
  \begin{proposition}
    \label{prp:xyyx}
    对于整数,有
    \begin{aenum}
      \item $x+y = y+x$;
      \item $\pare{x+y}+z = x+\pare{y+z}$;
      \item $x+0 = 0+x = x$;
      \item $x+\pare{-x}=\pare{-x}+x=0$;
      \item $xy=yx$;
      \item $\pare{xy}z=x\pare{yz}$;
      \item $x1=1x=x$;
      \item $x\pare{y+z} = xy + xz$;
      \item $\pare{y+z}x = yx+zy$。
    \end{aenum}
  \end{proposition}
  \begin{proof}
    设$x=a-b$后完全展开消去即可。
  \end{proof}
  这一定理表明整数构成以交换环。
  \begin{definition}
    定义整数的减法为$a-b=a+\pare{-b}$。
  \end{definition}
  对于等价类相容的检验,可以略去,因为减法直接借助了经过验证的加法与取负的定义。也容易验证对于自然数的$a$和$b$,等价类$a-b$与差$a-b=\pare{a-0}+\pare{0-b}$相同。此外,还易证一个数减去它自身将得到零。
  \begin{proposition}
    \[ \pare{-1}\times a = -a. \]
  \end{proposition}
  \begin{proposition}
    若整数$ab=0$,则$a=0$或$b=0$。
  \end{proposition}
  \begin{proof}
    借助上一命题,在两侧乘$-1$将$a$与$b$均强制转化为正数后调用\lref{mn0}。
  \end{proof}
  \begin{collary}
    若$ac=bc$且$c\ne 0$,则$a=b$。
  \end{collary}
  \begin{proof}
    即$\pare{a-b}c=0$。
  \end{proof}
  \begin{lemma}[整数的序]
    \begin{aenum}
      \item $a>b$当且仅当$a-b$为正;
      \item 若$a>b$则$a+c>b+c$;
      \item 若$a>b$且$c$为正,则$ac>bc$;
      \item 若$a>b$则$-a<-b$;
      \item 若$a>b$且$b>c$,则$a>c$;
      \item $a>b$,$b>a$,$a=b$有且仅有一成立。
    \end{aenum}
  \end{lemma}
  \begin{proof}
    借助\pref{xyyx}易得。最后一条可借助$a-b$的正负三歧性。
  \end{proof}
  \subsection{有理数}
  \begin{definition}
    有理数是形如$a/b$的表达式,其中$a$和$b$为整数且$b\ne 0$。两个有理数相等当且仅当$ad=bc$。
  \end{definition}
  \begin{definition}
    有理数的和定义为
    \[ \pare{a/b}+\pare{c/d} = \pare{ad+bc}/\pare{bd}. \]
  \end{definition}
  \begin{definition}
    有理数的积定义为
    \[ \pare{a/b}\times\pare{c/d} = \pare{ac}/\pare{bd}. \]
  \end{definition}
  \begin{definition}
    有理数的负定义为
    \[ -\pare{a/b}=\pare{-a}/b. \]
  \end{definition}
  \begin{theorem}
    上述三定义在等价类内相容。
  \end{theorem}
  \begin{proof}
    只证加法的部分,设$ab'=a'b$,需证
    \[ \pare{ad+bc}\pare{b'd} = \pare{a'd+b'c}\pare{bd}. \]
    展开有
    \[ adb'd + bcb'd = a'dbd + b'cbd. \]
    由假设知成立。
  \end{proof}
  注意到$a/1$与整数$a$可同构。
  \begin{definition}
    有理数的倒数定义为
    \[ \pare{a/b}^{-1} = \pare{b/a}. \]
  \end{definition}
  容易验证,倒数也是等价类相容的。
  \begin{theorem}
    对于有理数,有
    \begin{aenum}
      \item $x+y = y+x$;
      \item $\pare{x+y}+z = x+\pare{y+z}$;
      \item $x+0 = 0+x = x$;
      \item $x+\pare{-x}=\pare{-x}+x=0$;
      \item $xy=yx$;
      \item $\pare{xy}z=x\pare{yz}$;
      \item $x1=1x=x$;
      \item $x\pare{y+z} = xy + xz$;
      \item $\pare{y+z}x = yx+zy$。
    \end{aenum}
  \end{theorem}
  \begin{proof}
    同样设$x=a/b$完全展开后消去。
  \end{proof}
  \begin{definition}
    定义有理数的商
    \[ x/y=x\times y^{-1}. \]
  \end{definition}
  \begin{definition}
    一个有理数$x$称为正数,如果对于某两个正整数$a$和$b$有$x=a/b$。称其为负数,如果它是一个正数的负。
  \end{definition}
  \begin{theorem}[三歧性(trichofomy)]
    任意有理数成立如下三个命题之一:
    \begin{aenum}
      \item $x$为零;
      \item $x$是正的有理数;
      \item $x$是负的有理数。
    \end{aenum}
  \end{theorem}
  \begin{proof}
    对$x=a/b$中$a$和$b$的正负分类即可。
    \par
    为了证明三者中仅有一成立,分若干类讨论。当$x$为零时可证其非正且非负。若$x$同时为正负,则展开后借助整数的三歧性可得矛盾。
  \end{proof}
  \begin{definition}
    $x>y$当且仅当$x-y$是正的有理数,$x<y$当且仅当$x-y$是负的。
  \end{definition}
  \begin{theorem}
    设$x$,$y$,$z$均为有理数,则
    \begin{aenum}
      \item $x=y$,$x>y$与$x<y$有且仅有一成立;
      \item $x<y$当且仅当$y>x$;
      \item 若$x<y$,$y<z$,则$x<z$;
      \item 若$x<z$,则$x+z<y+z$;
      \item 若$x<y$且$z$为正,则$xz<yz$。
    \end{aenum}
  \end{theorem}
  \subsection{实数}
  \subsection{实数的Dedekind构造}
  \subsection{实数作为有理数的Cauchy序列}
  \section{逻辑}
  \subsection{归纳定义原理}
  \subsection{无限集与选择公理}
  \subsection{良序集}
  \subsection{极大原理}
  \subsection{良序原理与选择公理}

%ContentEnds
 
\ifx\allfiles\undefined %如果位置放错,可能出现意外中断
\end{document}
\fi
  %Ch2.GeneralPointSetTopology.tex
\ifx\allfiles\undefined
\documentclass{ctexrep}
% Mathematics Include

\usepackage{amsmath}
\usepackage{amssymb}
\usepackage{amsthm}
\usepackage{amsfonts}
\usepackage{mathrsfs}
\usepackage{enumitem}
\usepackage{braket}
\usepackage{hyperref}
\usepackage[all, pdf]{xy}

% Physics Include
\usepackage{amsmath}
\usepackage{physics}
\usepackage{siunitx}
\usepackage[makeroom]{cancel}
\usepackage{pstricks}
\usepackage{pstricks-add}
\psset{algebraic=true}

\usepackage[version=4]{mhchem}
\usepackage{array,booktabs}
\usepackage{longtable}
\usepackage{mathtools}
\usepackage[normalem]{ulem}
\usepackage{multicol}

% Mathematics Head

\newcommand{\pare}[1]{\left(#1\right)}
\newcommand{\blr}[1]{\left[#1\right)}
\newcommand{\lbr}[1]{\left(#1\right]}
\newcommand{\brac}[1]{\left[#1\right]}
\newcommand{\curb}[1]{\left\{#1\right\}}
% \newcommand{\abs}[1]{\left|\, #1 \,\right|}
\newcommand{\rec}[1]{\frac{1}{#1}}
\newcommand{\N}{\mathbb{N}}
\newcommand{\bC}{\mathbb{C}}
\newcommand{\Q}{\mathbb{Q}}
\newcommand{\Z}{\mathbb{Z}}
\newcommand{\R}{\mathbb{R}}
\newcommand{\unk}{\mathcal{X}}
\newcommand{\bu}[3]{#1_{#2}^{\pare{#3}}}
\newcommand{\dref}[1]{定义\ref{def:#1}}
\newcommand{\tref}[1]{定理\ref{thm:#1}}
\newcommand{\lref}[1]{引理\ref{lem:#1}}
\newcommand{\cref}[1]{推论\ref{coll:#1}}
\newcommand{\pref}[1]{命题\ref{prp:#1}}
\newcommand{\eref}[1]{例\ref{ex:#1}}
\newcommand{\func}[3]{#1:\, #2 \rightarrow #3}
\newcommand{\overbar}[1]{\mkern 1.5mu\overline{\mkern-1.5mu#1\mkern-1.5mu}\mkern 1.5mu}
\newcommand{\clo}[1]{\overbar{#1}}
\newcommand{\supi}[2]{\overbar{\int_{#1}^{#2}}}
\newcommand{\infi}[2]{\underbar{\int_{#1}^{#2}}}
\newcommand{\setf}{\mathscr}
\newcommand{\bool}{\mathrm{bool}}
\newcommand{\inc}{++}
\newcommand{\defeq}{:=}
\newcommand{\ntuple}{$n$元组}
\newcommand{\card}[1]{\#\pare{#1}}
\newcommand{\setcond}[2]{\curb{#1 \, \left| \, #2 \right.}}
\newcommand{\setcondl}[2]{\curb{\left. #1 \, \right| \, #2}}
\newcommand{\bv}[1]{\mathbf{#1}}
\newcommand{\bfa}{\bv{a}}
\newcommand{\bfb}{\bv{b}}
\newcommand{\bfx}{\bv{x}}
\newcommand{\bfy}{\bv{y}}
\newcommand{\bfe}{\bv{e}}
\newcommand{\bfF}{\bv{F}}
\newcommand{\bff}{\bv{f}}
\newcommand{\bfG}{\bv{G}}
\newcommand{\bfH}{\bv{H}}
\newcommand{\bfg}{\bv{g}}
\newcommand{\bfh}{\bv{h}}
\newcommand{\bfr}{\bv{r}}
\newcommand{\bfk}{\bv{k}}
\newcommand{\bfu}{\bv{u}}
\newcommand{\bfv}{\bv{v}}
\newcommand{\oo}[1]{o\pare{#1}}
\newcommand{\OO}[1]{O\pare{#1}}
% \newcommand{\norm}[1]{\left\| #1 \right\|}
\newcommand{\DD}{\mathbf{D}}
\newcommand{\comp}{\circ}
\newcommand{\const}{\mathrm{const}}
\newcommand{\dist}[2]{d\pare{#1,#2}}
\newcommand{\len}{\ell}
\newcommand{\siga}{$\sigma$-代数}
\newcommand{\cara}{Carath\'{e}odory}
\newcommand{\Gd}{G_\delta}
\newcommand{\Fs}{F_\sigma}
\newcommand{\mmani}{$m$-维流形}
\newcommand{\open}[1]{\mathcal{#1}}
\newcommand{\half}{\frac{1}{2}}
\newcommand{\maxo}[1]{\text{max}\curb{#1}}
\newcommand{\mino}[1]{\text{min}\curb{#1}}
\newcommand{\epsclo}{$\epsilon$-接近}
\newcommand{\close}[1]{$#1$-接近}
\newcommand{\cinf}{$C^\infty$}
\newcommand{\cuno}{$C^1$}
\newcommand{\Int}{\text{Int}\,}
\newcommand{\Ext}{\text{Ext}\,}
\newcommand{\funcf}{\mathcal}
\newcommand{\DDu}{\overbar{\DD}}
\newcommand{\DDl}{\underbar{\DD}}
\newcommand{\Diff}[1]{\mathrm{Diff}_{#1}\,}
\newcommand{\Av}[1]{\mathrm{Av}_{#1}\,}
\newcommand{\Lip}[1]{Lipschitz-$#1$}
\newcommand{\sgn}{\mathrm{sgn}}
\newcommand{\eset}{\varnothing}
\newcommand{\cT}{\mathcal{T}}
\newcommand{\cS}{\mathcal{S}}
\newcommand{\cG}{\mathcal{G}}
\newcommand{\cF}{\mathcal{F}}
\newcommand{\cC}{\mathcal{C}}
\newcommand{\cB}{\mathcal{B}}
\newcommand{\inter}[1]{\mathring{#1}}
\newcommand{\forest}[3]{对于{#1},存在{#2},使得{#3}}
\newcommand{\tuno}{$T_1$公理}
\newcommand{\isom}{\overset{\sim}{=}}
\newcommand{\diam}{\mathrm{diam}\,}
\newcommand{\ord}[1]{\abs{#1}}
\newcommand{\sbm}[1]{\overbar{#1}}
\newcommand{\inv}[1]{#1^{-1}}
\newcommand{\restr}[2]{#1|_{#2}}
\newcommand{\divs}{|}
\newcommand{\ndivs}{\nmid}
\newcommand{\modeq}[1]{\overbar{#1}}
\newcommand{\ggen}[1]{\langle#1\rangle}
\newcommand{\ggencond}{\braket}

\newcommand{\hd}{H\"{o}lder}

\renewcommand{\proofname}{证明}

\newenvironment{cenum}{\begin{enumerate}\itemsep0em}{\end{enumerate}}

\newtheorem{definition}{定义}[section]
\newtheorem{lemma}{引理}[section]
\newtheorem{theorem}{定理}[section]
\newtheorem{collary}{推论}[section]
\newtheorem{corollary}{推论}[section]
\newtheorem{proposition}{命题}[section]
\newtheorem{axiom}{公理}[section]
\newtheorem{ex}{例}[section]
\newtheorem{reflection}{反射}[section]
\newcommand{\refl}[1]{\vspace{0.5em}\par\noindent\fbox{%
    \parbox{0.9\textwidth}{%
    \begin{reflection}
        #1
    \end{reflection}
    }%
}\vspace{0.5em}\par}
\newcommand{\rref}[1]{反射\ref{refl:#1}}
\newcommand{\tbref}[1]{表\ref{table:#1}}
\allowdisplaybreaks

\newenvironment{aenum}{\begin{enumerate}[label=\textnormal{(\alph*)}]}{\end{enumerate}}

% Physics Head

\DeclareSIUnit\dyne{dynes}

\newcommand{\ddel}[1]{\frac{\partial}{\partial #1}}
\newcommand{\ddelon}[2]{\frac{\partial #1}{\partial #2}}
\newcommand{\dddel}[1]{\frac{\partial^2}{\partial^2 #1}}
\newcommand{\ddt}{\ddel{t}}
\newcommand{\ddT}{\ddel{T}}
\newcommand{\ddV}{\ddel{V}}
\newcommand{\ddr}{\ddel{r}}
\newcommand{\ddth}{\ddel{\theta}}
\newcommand{\ddph}{\ddel{\phi}}
\newcommand{\dddt}{\dddel{t}}
\newcommand{\dddr}{\dddel{t}}
\newcommand{\dddth}{\dddel{\theta}}
\newcommand{\dddph}{\dddel{\phi}}
\newcommand{\rd}[1]{\mathrm{d} #1}
\newcommand{\dt}{\rd{t}}
\newcommand{\dy}{\rd{y}}
\newcommand{\dx}{\rd{x}}
\newcommand{\edd}[1]{\frac{\mathrm{d}}{\mathrm{d} #1}}
\newcommand{\eddd}[1]{\frac{\mathrm{d}^2}{\mathrm{d}^2 #1}}
\newcommand{\eddon}[2]{\frac{\mathrm{d} #1}{\mathrm{d} #2}}
\newcommand{\edddon}[2]{\frac{\mathrm{d}^2 #1}{\mathrm{d}^2 #2}}
\newcommand{\edt}{\edd{t}}
\newcommand{\edton}[1]{\eddon{#1}{t}}
\newcommand{\edT}{\edd{T}}
\newcommand{\edr}{\edd{r}}
\newcommand{\edl}{\edd{l}}
\newcommand{\edx}{\edd{x}}
\newcommand{\edth}{\edd{\theta}}
\newcommand{\eddton}[1]{\edddon{#1}{t}}
\newcommand{\eddzon}[1]{\edddon{#1}{z}}
\newcommand{\vect}[1]{\boldsymbol{#1}}
\newcommand{\alp}{\frac{1}{\sqrt{2}}}
\newcommand{\alpi}{\frac{i}{\sqrt{2}}}
\newcommand{\expc}[1]{\langle#1\rangle}
\newcommand{\bkn}[1]{\bra{#1}\ket{#1}}
\newcommand{\bk}[2]{\bra{#1}\ket{#2}}
\newcommand{\bik}[3]{\bra{#1} #2 \ket{#3}}
\newcommand{\vari}[1]{\sigma_{#1}}
\newcommand{\intc}[2]{\left[#1, #2\right]}
\newcommand{\sch}{Schr\"{o}dinger}
\newcommand{\moment}{\boldsymbol{p}}
\newcommand{\coor}{\boldsymbol{x}}
\newcommand{\lapc}{\nabla^2}
% \newcommand{\rec}[1]{\frac{1}{#1}}
\newcommand{\vva}{\boldsymbol{a}}
\newcommand{\vvb}{\boldsymbol{b}}
\newcommand{\vc}{\boldsymbol{c}}
\newcommand{\vd}{\boldsymbol{d}}
\newcommand{\ve}{\boldsymbol{e}}
\newcommand{\vf}{\boldsymbol{f}}
\newcommand{\vg}{\boldsymbol{g}}
\newcommand{\vh}{\boldsymbol{h}}
\newcommand{\vi}{\boldsymbol{i}}
\newcommand{\vj}{\boldsymbol{j}}
\newcommand{\vk}{\boldsymbol{k}}
\newcommand{\vl}{\boldsymbol{l}}
\newcommand{\vm}{\boldsymbol{m}}
\newcommand{\vn}{\boldsymbol{n}}
\newcommand{\vo}{\boldsymbol{o}}
\newcommand{\vp}{\boldsymbol{p}}
\newcommand{\vq}{\boldsymbol{q}}
\newcommand{\vr}{\boldsymbol{r}}
\newcommand{\vs}{\boldsymbol{s}}
\newcommand{\vt}{\boldsymbol{t}}
\newcommand{\vvu}{\boldsymbol{u}}
\newcommand{\vv}{\boldsymbol{v}}
\newcommand{\vw}{\boldsymbol{w}}
\newcommand{\vx}{\boldsymbol{x}}
\newcommand{\vy}{\boldsymbol{y}}
\newcommand{\vz}{\boldsymbol{z}}
\newcommand{\vA}{\boldsymbol{A}}
\newcommand{\vB}{\boldsymbol{B}}
\newcommand{\vC}{\boldsymbol{C}}
\newcommand{\vD}{\boldsymbol{D}}
\newcommand{\vE}{\boldsymbol{E}}
\newcommand{\vF}{\boldsymbol{F}}
\newcommand{\vG}{\boldsymbol{G}}
\newcommand{\vH}{\boldsymbol{H}}
\newcommand{\vI}{\boldsymbol{I}}
\newcommand{\vJ}{\boldsymbol{J}}
\newcommand{\vK}{\boldsymbol{K}}
\newcommand{\vL}{\boldsymbol{L}}
\newcommand{\vM}{\boldsymbol{M}}
\newcommand{\vN}{\boldsymbol{N}}
\newcommand{\vO}{\boldsymbol{O}}
\newcommand{\vP}{\boldsymbol{P}}
\newcommand{\vQ}{\boldsymbol{Q}}
\newcommand{\vR}{\boldsymbol{R}}
\newcommand{\vS}{\boldsymbol{S}}
\newcommand{\vT}{\boldsymbol{T}}
\newcommand{\vU}{\boldsymbol{U}}
\newcommand{\vV}{\boldsymbol{V}}
\newcommand{\vW}{\boldsymbol{W}}
\newcommand{\vX}{\boldsymbol{X}}
\newcommand{\vY}{\boldsymbol{Y}}
\newcommand{\vZ}{\boldsymbol{Z}}
\newcommand{\vzero}{\boldsymbol{0}}
\newcommand{\vomega}{\boldsymbol{\omega}}
%\newcommand{\half}{\frac{1}{2}}
\newcommand{\thalf}{\frac{3}{2}}
\newcommand{\rot}{\nabla\times}
\newcommand{\divg}{\nabla\cdot}
\newcommand{\cE}{\mathcal{E}}
\newcommand{\conclu}[1]{\vspace{1em}\par\noindent\fbox{\parbox{0.9\textwidth}{#1}}\vspace{1em}}
\newcommand{\subentrynote}{$\bullet$}
\newcommand{\keypoint}[1]{\par\subentrynote\quad #1 \par}
\newcommand{\fconclu}{\boxed}
\newcommand{\pair}[2]{#1 \, #2}
\newcommand{\intn}[2]{\int #1 \,\mathrm{d} #2}
\newcommand{\intu}[3]{\int_0^{#1} #2 \,\mathrm{d} #3}
\newcommand{\intiu}[3]{\int_{-\infty}^{#1} #2 \, \rd{} #3}
\newcommand{\intui}[2]{\int_0^{\infty} #1 \,\mathrm{d} #2}
\newcommand{\intii}[2]{\int_{-\infty}^{\infty} #1 \,\mathrm{d} #2}
\newcommand{\intt}[2]{\int_0^\infty #1 \, \rd{} #2}
\newcommand{\intr}[2]{\int_{-\infty}^{\infty} #1 \, \rd{} #2}
\newcommand{\intbi}[3]{\int_{#1}^{\infty} #2 \, \rd{} #3}
\newcommand{\intab}[4]{\int_{#1}^{#2} #3 \, \rd{} #4}
\newcommand{\bfactor}[1]{e^{-#1/k_BT}}
\newcommand{\pbfactor}[1]{e^{#1/k_BT}}
\newcommand{\dn}[2]{#1^{\pare{#2}}}
\newcommand{\prodg}[1]{\pare{#1}^\times}

\newcommand{\notion}{\emph}
\newcommand{\iP}{\mathcal{P}}
\newcommand{\eiP}{e^{-\iP}}
\newcommand{\iF}{\mathcal{F}}
\newcommand{\eiF}{e^{-\iF}}
\newcommand{\iG}{\mathcal{G}}

\newcommand{\rc}{r\cos\theta}
\newcommand{\rs}{r\sin\theta}
\newcommand{\sn}{\mathrm{sn}}
\newcommand{\cn}{\mathrm{cn}}
\newcommand{\rdn}{\mathrm{dn}}

\newcommand{\hankel}{H_p^{\pare{1}}}
\newcommand{\hankell}{H_p^{\pare{2}}}
\newcommand{\hhankel}{H_n^{\pare{1}}}
\newcommand{\hhankell}{H_n^{\pare{2}}}
\newcommand{\ber}{\text{ber}\,}
\newcommand{\bei}{\text{bei}\,}
\newcommand{\kker}{\text{ker}\,}
\newcommand{\kei}{\text{kei}\,}
\newcommand{\Ai}{\text{Ai}}
\newcommand{\Bi}{\text{Bi}}

\newcommand{\re}{\text{Re}\,}

\newcommand{\Fp}{F_\phi}
\newcommand{\Ep}{E_\phi}
\newcommand{\Fx}{F_x}
\newcommand{\FF}{\mathbf{F}}
\newcommand{\Ex}{E_x}

%\newcommand{\erf}{\mathrm{erf}}
\newcommand{\erfi}{\mathrm{erfi}}
\newcommand{\erfc}{\mathrm{erfc}}
\newcommand{\ehxs}[1]{e^{-\frac{#1^2}{2}}}
\newcommand{\dcol}[2]{\[ \left.#1 \hspace{1em}\right\vert\hspace{1em} #2 \]}
\newcommand{\titlegamma}{\texorpdfstring{$\Gamma$}{Gamma}}
\newcommand{\titleB}{\texorpdfstring{$B$}{B}}

% Computer Science Head
\usepackage{listings}
\usepackage{color}

\definecolor{dkgreen}{rgb}{0,0.6,0}
\definecolor{gray}{rgb}{0.5,0.5,0.5}
\definecolor{mauve}{rgb}{0.58,0,0.82}

\lstset{frame=tb,
  language=Java,
  aboveskip=3mm,
  belowskip=3mm,
  showstringspaces=false,
  columns=flexible,
  basicstyle={\small\ttfamily},
  numbers=none,
  numberstyle=\tiny\color{gray},
  keywordstyle=\color{blue},
  commentstyle=\color{dkgreen},
  stringstyle=\color{mauve},
  breaklines=true,
  breakatwhitespace=true,
  tabsize=3
}
\lstset{language=Java}
\newcommand{\snp}[1]{\lstinline!#1!}
\newcommand{\term}[2]{\textbf{#1(#2)}}
\begin{document}
\fi

%Content

\chapter{普通点集拓扑}
  \section{拓扑空间与连续函数}
  \subsection{拓扑空间}
  \begin{definition}
  集合$X$上的一个拓扑$\cT$谓$X$的一满足如下条件的子集族:
  \begin{cenum}
    \item $\curb{\eset, X} \in \cT$;
    \item $\cT$中元素的任意并仍在$\cT$中;
    \item $\cT$中元素的有限交仍在$\cT$中。
  \end{cenum}
  \end{definition} 
  \begin{definition}
  $X$的所有子集构成的拓扑谓离散拓扑。
  \end{definition}
  \begin{definition}
  由$X$和$\eset$构成的拓扑谓密着拓扑。
  \end{definition}
  \begin{definition}
  由$X$本身与所有满足$X-U$为有限集的$U$构成的拓扑谓有限补拓扑。
  \end{definition}
  \begin{definition}
  $\cT'\supset\cT$则$\cT'$细于$\cT$,反之则谓粗于。
  \end{definition}
  如果把开集比做石子,把石子打碎就得到更细的拓扑。
  \subsection{拓扑的基}
  \begin{definition}
  \label{def:tb}
  基$\cB$谓满足如下条件的子集族:
  \begin{cenum}
    \item 对任意$x\in X$,存在$B\in\cB$满足$x\in B$;
    \item 对任意$x\in B_1\cap B_2$,存在$B$满足$x\in B$且$B\subset B_1\cap B_2$。
  \end{cenum}
  \end{definition}
  注意此定义不针对具体的拓扑。
  \begin{ex}
  平面上的圆域和矩形域构成的集族都构成基。
  \end{ex}
  \begin{definition}
  满足\dref{tb}的$\cB$生成的拓扑为所有满足对$x\in U$,存在$x\in B\subset U$的$U$的集族。
  \end{definition}
  可以直接验证上述定义构成一个拓扑。对所有$x$取对应的$x\in B_x$后将诸$B_x$并起,可得等价的表述
  \begin{theorem}
  若$\cB$为$\cT$的基,则$\cT$为$\cB$中元素并的族。
  \end{theorem}
  \begin{theorem}
  设$\cC$为开集族,若对于任意开集$U$中任意$x$,存在$C\in\cC$满足$x\in C\subset U$,则$\cC$为$\cT$的基。
  \end{theorem}
  \begin{proof}
  容易验证$\cC$为基。再分别证$\cC\subset \curb{U}$与$\curb{\cup C}\supset \curb{U}$。
  \end{proof}
  \begin{theorem}
  \label{thm:critfiner}
  设$\cB$于$\cB'$分别生成$\cT$与$\cT'$,则$\cT'$细于$\cT$当且仅当对任意$x\in B$存在$x\in B'\subset B$。
  \end{theorem}
  \begin{proof}
  强行带入定义,即任意$U$均在$\cT'$内即可。
  \end{proof}
  \begin{definition}
  $\R$上的$\pare{a,b}$生成的拓扑谓标准拓扑。
  \end{definition}
  \begin{definition}
  $\R$上$\blr{a,b}$生成的拓扑谓下限拓扑,记作$\R_\ell$。
  \end{definition}
  \begin{definition}
  $\R$上$\pare{a,b}$与$\pare{a,b} - \curb{\rec{n}}$生成的拓扑谓K-拓扑,记作$\R_K$。
  \end{definition}
  \begin{lemma}
  $\R_\ell$与$\R_K$严格细于标准拓扑,但它们之间不可比较。
  \end{lemma}
  \begin{proof}
  $\R_K$\emph{严格}细于的证明只需考虑$x=0$与$B=\pare{-1,1}-\curb{1/n}$,同一个集合可证$\R_\ell$不细于$R_K$。
  \end{proof}
  \begin{definition}
  子基$\cS$谓满足$\cup S=X$的集族。
  \end{definition}
  \begin{definition}
  子基生成的拓扑谓$\cS$中有限交的所有并。
  \end{definition}
  可以直接验证$\curb{\cap S}$为一个基,故其确实生成一拓扑。
  \subsection{序拓扑}
  \begin{definition}
  具有全序关系的$X$上的序拓扑谓所有$\pare{a,b}$,$\lbr{a,\max X}$,$\blr{\min X, b}$生成的拓扑。
  \end{definition}
  \begin{ex}
  $\Z_+$上的序拓扑是离散拓扑。然而$X=\curb{1,2}\times\Z_+$的字典序拓扑下单点集$1\times1$并非开集。
  \end{ex}
  \begin{definition}
  全序集$X$中$a$决定的射线谓开射线$\pare{a,+\infty}$,$\pare{-\infty,a}$,$\blr{a,+\infty}$,$\lbr{-\infty,a}$。
  \end{definition}
  所有开射线构成$X$的序拓扑的子基。
  \subsection{积拓扑}
  \begin{definition}
  $X\times Y$上的积拓扑谓所有$U\times V$的集族$\cB$生成的拓扑,其中$U$与$V$为$X$与$Y$中的开集。
  \end{definition}
  \begin{theorem}
  若$\cB$与$\cC$分别为$X$与$Y$的基,则$\cB\times\cC$为$X\times Y$的基。
  \end{theorem}
  \begin{definition}
  投射$\pi_1\pare{x,y}=x$,$\pi_2\pare{x,y}=y$。
  \end{definition}
  \begin{theorem}
  如下的$\cS$构成$X\times Y$的一子基,其中$U$和$V$分别为$X$与$Y$中的开集。
  \[ \cS = \curb{\pi_1^{-1}\pare{U}}\cup\curb{\pi_2^{-1}\pare{V}}. \]
  \end{theorem}
  \subsection{子空间拓扑}
  \begin{definition}
  对$X$的子集$Y$定义子空间拓扑,其中$U$为$X$中的开集。
  \[ \cT_Y = \curb{Y\cap U}. \]
  \end{definition}
  \begin{theorem}
  若$\cB$为$X$的一个基,则
  \[ \cB_Y = \setcond{B\cap Y}{B \in \cB} \]
  谓$Y$的子空间拓扑的一个基。
  \end{theorem}
  \begin{lemma}
  若$Y$为$X$中开集而$U$为$Y$中开集,则$U$为$X$中开集。
  \end{lemma}
  \begin{theorem}
  若$A\subset X$,$B\subset Y$,则$A\times B$的积拓扑与其自$X\times Y$继承的子空间拓扑相符。
  \end{theorem}
  然而,对于序拓扑无类似结论。
  \begin{ex}
  考虑$X=\R$而$Y=\brac{0,1}$,$Y$上的序拓扑与子空间拓扑相符。
  \end{ex}
  \begin{ex}
  考虑$X=\R$而$Y=\blr{0,1}\cup\curb{2}$,子空间拓扑中$\curb{2}$为开集,二者不符。
  \end{ex}
  \begin{ex}
  考虑$X=\R^2$而$Y=\brac{0,1}\times\brac{0,1}$,则$\half\times\lbr{\half,1}$为子空间拓扑的开集但不是序拓扑的开集。
  \end{ex}
  \begin{definition}
  子集$Y$称为凸的,如果对$Y$中$a<b$皆有$\pare{a,b}\subset Y$。
  \end{definition}
  \begin{theorem}
  设$X$为全序集,$Y$为凸子集,则子空间拓扑与序拓扑一致。
  \end{theorem}
  \begin{proof}
  借助开射线构造子基后证明其相互包含即可。
  \end{proof}
  \subsection{闭集与极限点}
  \begin{definition}
    若$X-A$为开集,则$A$为闭集。
  \end{definition}
  \begin{ex}
  $\R$中$\brac{a,b}$为闭集,$\R^2$中$\R_+^2$为闭集,有限补拓扑中$X$、$\eset$、有限集为闭集。
  \end{ex}
  \begin{ex}
  离散拓扑每一个集合都是开集也都是闭集,$Y=\brac{0,1}\cup\pare{2,3}$中两个分量都同时是开集和闭集。
  \end{ex}
  \begin{definition}
  对于拓扑空间$X$,成立
  \begin{cenum}
    \item $\eset$、$X$都是闭集;
    \item 闭集的任意交仍为闭集;
    \item 闭集的有限并仍为闭集。
  \end{cenum}
  \end{definition} 
  \begin{theorem}
    $A$为$X$的子空间$Y$的闭集当且仅当有闭集$C$满足$A=Y\cap C$。
  \end{theorem}
  \begin{theorem}
    $A$是$Y$的闭集,$Y$是$X$的闭集,则$A$是$X$的闭集。
  \end{theorem}
  \subsubsection{Hausdorff空间}
  \begin{definition}
    集合的内部$\inter{A}$是包含于其内的所有开集的并,闭包$\clo{A}$是其外所有闭集的交。
  \end{definition}
  显然开集的内部是本身,闭集的闭包也是本身。注意$\pare{0,1}$在其本身中的闭包和在$\R$中的闭包不同,所称闭包都是指父空间闭包。
  \begin{theorem}
  $Y$中$\clo{A}^Y = \clo{A} \cap X$。
  \end{theorem}
  \begin{definition}
    两集合相交,如果它们的交非空。
  \end{definition}
  \begin{definition}
    含有$x$的开集称为其邻域。
  \end{definition}
  \begin{theorem}
  \label{thm:sublpoint}
    $x\in \clo{A}$当且仅当每一个邻域与$A$相交。
  \end{theorem}
  \begin{proof}
    如果存在反例$U$,则$X-U$会成为包含$A$的闭集。
  \end{proof}
  \begin{collary}
    $x\in \clo{A}$当且仅当含有$x$的每一个基元素与$A$相交。
  \end{collary}
  \begin{ex}
    $A=\lbr{0,1}$,$\clo{A}=\brac{0,1}$。$A=\Q$,$\clo{A}=\R$。$\clo{\curb{1/n}} = \curb{1/n}\cup\curb{0}$。
  \end{ex}
  \subsubsection{极限点}
  \begin{definition}
    若$x$的任何一个邻域包含$A$中其他点,则$x$为$A$的极限点。
  \end{definition}
  \begin{ex}
    $A=\lbr{0,1}$,$\brac{0,1}$中的点均为其极限点。$A=\Q$,$\R$中的点均为其极限点。$A=\curb{1/n}$,$0$为其极限点。
  \end{ex}
  \begin{theorem}
  $\clo{A} = A\cup A'$,其中$A'$为极限点集合。
  \end{theorem}
  \begin{proof}
    参考\tref{sublpoint}。
  \end{proof}
  \begin{collary}
  $A$为闭集当且仅当$A'\subset A$。
  \end{collary}
  \subsubsection{Hausdorff空间}
  \begin{definition}
    如果\forest{$x$的任意邻域$U$}{$N$}{当$n>N$,$x_n\in U$},则$\curb{x_n}$收敛到点$x$。
  \end{definition}
  $\R^2$和$\R$中的序列最多收敛至一点,然而其他拓扑空间不一定。
  \begin{definition}
    若$X$中任意两不同点存在无交邻域,则称$X$为一Hausdorff空间(Hausdorff space)。
  \end{definition}
  \begin{theorem}
    Hausdorff空间中有限集为闭集。
  \end{theorem}
  \begin{proof}
    只证单点集。由于隔离邻域的存在,易见其他点都不在闭包内。
  \end{proof}
  比Hausdorff条件更弱的,有\tuno 。
  \begin{definition}
    若$X$中有限集为闭,则$X$满足\tuno 。
  \end{definition}
  \begin{theorem}
    若$X$满足\tuno ,则$x$为$A$的极限点当且仅当$x$的任意邻域与$A$有无限交点。
  \end{theorem}
  \begin{proof}
    如果有一个邻域只有有限交点,挖掉还是开集,但不再与$A$相交。
  \end{proof}
  \begin{theorem}
    若$X$为Hausdorff空间,则$X$中的序列最多收敛至一点。
  \end{theorem}
  \begin{proof}
    如果有两个点,在定义中取隔离邻域即可。
  \end{proof}
  \begin{theorem}
  每一个具有序拓扑的全序集,两个Hausdorff空间的积,Hausdorff空间的子空间是Hausdorff空间。
  \end{theorem}
  \begin{proof}
    全序集可选取中间元分割,中间元不存在的直接射线可分割。
  \end{proof}
  \subsection{连续函数}
  \subsubsection{函数的连续性}
  \begin{definition}
    函数$\func{f}{X}{Y}$称为连续的,如果开集的原像为开集。
  \end{definition}
  为了证明函数连续,只需要证明基的原像为开集即可。
  \begin{ex}
  \label{ex:epsd}
    上述定义等价于$\epsilon-\delta$定义。
  \end{ex}
  \begin{proof}
    如果$\epsilon-\delta$定义成立,则$f\pare{x}$的$\epsilon$-邻域的原像包含$x$的$\delta$-邻域,故任意开集的原像均为开集。如果拓扑定义成立,则显而易见。
  \end{proof}
  \begin{ex}
    $\func{f}{\R}{\R_\ell}$的$f\pare{x}=x$不是连续函数,但其逆连续。
  \end{ex}
  \begin{theorem}
  \label{thm:contineq}
    对于$\func{f}{X}{Y}$,下列条件等价:
    \begin{cenum}
      \item $f$连续;
      \item 对$X$的任意子集$A$有$f\pare{\clo{A}}\subset\clo{f\pare{A}}$;
      \item 对$Y$中任意闭集$B$有$f^{-1}\pare{B}$为闭集;
      \item 对任意$x$与$f\pare{x}$的邻域$V$,存在$x$的邻域$U$满足$f\pare{U}\subset{V}$。
    \end{cenum}
  \end{theorem}
  \begin{proof}
    $1\Rightarrow 2$:若$y$在$f\pare{A}$外一开集内,则原像为$f\pare{A}$外一开集。$2\Rightarrow 3$:闭集$f\pare{A} = \clo{f\pare{A}}\supset f\pare{\clo{A}}$,故$A=\clo{A}$。$3\Rightarrow 1$与$1\Rightarrow 4 \Rightarrow1$显然。
  \end{proof}
  \subsubsection{同胚}
  \begin{definition}
    如果一个一一映射和它的逆都连续,则称之为同胚。
  \end{definition}
  \begin{definition}
    如果$X$的性质于与之同胚的$Y$都成立,则称之为拓扑性质。
  \end{definition}
  \begin{definition}
    映入子空间的同胚称为嵌入。
  \end{definition}
  \begin{ex}
    $F\pare{x}=x/\pare{1-x^2}$与$G\pare{y}=2y/\pare{1+\pare{1+4y^2}^{1/2}}$为$\pare{-1,1}$与$\R$间同胚。
  \end{ex}
  \begin{ex}
    $\blr{0,1}$弯曲到圆周的映射连续而非同胚。其扩张连续而非嵌入。
  \end{ex}
  \subsubsection{构造连续函数}
  \begin{theorem}
    下列函数皆连续:
    \begin{cenum}
      \item 常值函数;
      \item 子空间到父空间的内射;
      \item 连续函数的复合;
      \item 连续函数限制定义域到一子空间的结果;
      \item 连续函数限制或扩大值域至包含像集的子空间或父空间的结果;
      \item 若$X$可写为开集的并,且$f$在每个分量上连续。
    \end{cenum}
  \end{theorem}
  \begin{theorem}[黏结引理]
    设$X=A\cup B$且二者为闭集,并且$\func{f}{A}{Y}$与$\func{g}{B}{Y}$连续且在$A\cap B$上相等,则$h$连续,
    \[ h\pare{x}=\begin{cases}f\pare{x}, x\in A, \\ g\pare{x}, x\in B.\end{cases} \]
  \end{theorem}
  \begin{proof}
    由\tref{contineq},注意闭集被映回闭集即可。
  \end{proof}
  \begin{ex}
    对$x\ge 0$,$h\pare{x}=x$,$x\le 0$,$h\pare{x}=x/2$,则$h$连续。
  \end{ex}
  \begin{theorem}
    $\func{f}{A}{X\times Y}$连续的充分必要条件为$f_X$与$f_Y$连续。
  \end{theorem}
  \begin{proof}
    注意连续的拓扑定义等价于对基连续即可。
  \end{proof}
  \begin{ex}
    向量场连续当且仅当分量连续。
  \end{ex}
  \subsection{积拓扑}
  \begin{definition}
    $X$的元素的$J$-串为$\func{\vx}{J}{X}$,其全体记作$X^J$。
  \end{definition}
  例如,$\R^3\isom \R^{\curb{1,2,3}}$。
  \begin{definition}
    $A_j$的笛卡尔积$\prod A_j$为各取一元构成之$J$-串的集合。
  \end{definition}
  \begin{definition}
    基由$\prod U_\alpha$构成$\prod X_\alpha$的称为箱拓扑。
  \end{definition}
  \begin{definition}
    子基由$\curb{\pi^{-1}\pare{U_{\alpha}}}$构成的称为积拓扑。
  \end{definition}
  \begin{theorem}
    箱拓扑的基由所有$\prod U_\alpha$构成,积拓扑的基由$\prod U_\alpha$构成但$U_\alpha$中只有有限个非$X_\alpha$。
  \end{theorem}
  \begin{theorem}
    $\prod B_\alpha$构成箱拓扑的积,$\prod B_\alpha$中若$B_\alpha$中只有有限个非$X_\alpha$则构成积拓扑的基。
  \end{theorem}
  \begin{ex}
    $\R^n$的积拓扑与箱拓扑一致。
  \end{ex}
  \begin{theorem}
    $\prod A_\alpha$在两种拓扑下都是$\prod X_\alpha$的同种拓扑的子空间。
  \end{theorem}
  \begin{theorem}
    若每个$X_\alpha$都是Hausdorff的,则两拓扑下$\prod X_\alpha$都如此。
  \end{theorem}
  \begin{theorem}
    在$\prod X_\alpha$的两种拓扑下都有$\prod \clo{A_\alpha} = \clo{\prod A_\alpha}$。
  \end{theorem}
  \begin{proof}
    若$x$在$\prod \clo{A_\alpha}$内,则诸$\prod U_\alpha$均有$\prod A_\alpha$的元素,故$x$在$\clo{\prod A_\alpha}$内。若$x$在$\prod \clo{A_\alpha}$外$U_\alpha$内,则$\pi^{-1}\pare{U_\alpha}$包含$x$且为开集,故在$\clo{\prod A_\alpha}$外。
  \end{proof}
  \begin{theorem}
    积拓扑下$\func{f}{A}{\prod X_\alpha}$连续当且仅当各个分量连续。
  \end{theorem}
  \begin{proof}
    注意连续的拓扑定义等价于对基成立即可。
  \end{proof}  
  \begin{ex}
    对箱拓扑下$\R$的可数无限积$\R^\omega$,$f\pare{t}=\pare{t,t,t,\cdots}$不连续。注意$\pare{-1,1}\times\pare{-1/2,1/2}\times{-1/3,1/3}$被映回$0$即可。
  \end{ex}
  \subsection{度量拓扑}
  \begin{definition}
    集合$X$的一个度量$d$是一个函数$\func{d}{X\times X}{\R}$,满足正定、对称与三角不等式。
  \end{definition}
  \begin{definition}
    以全体$\epsilon$-球为基的拓扑称为度量拓扑。
  \end{definition}
  容易验证全体$\epsilon$-球构成基。由这一定义,开集可视作满足任意$y\in U$都有某$B\pare{x,\epsilon}\subset U$的集合$U$。
  \begin{ex}
    若$x=y$,$d\pare{x,y}=1$,否则$d\pare{x,y}=0$诱导出离散拓扑。
  \end{ex}
  \begin{ex}
    $d\pare{x,y}=\abs{x-y}$诱导$\R$上的序拓扑。
  \end{ex}
  \begin{definition}
    若$X$的拓扑由某度量诱导,则称$X$为度量空间。
  \end{definition}
  \begin{definition}
    度量空间的子集$A$为有界的,若$d\pare{a_1,a_2}$一致有界。$A$的直径谓$\diam A = \sup\curb{d\pare{a_1,a_2}}$。
  \end{definition}
  \begin{theorem}
    由度量$d$诱导的度量
    \[ \sbm{d}\pare{x,y} = \min\curb{d\pare{x,y}, 1} \]
    谓标准有界度量。
  \end{theorem}
  \begin{definition}
    分类验证三角不等式即可。
  \end{definition}
  \begin{definition}
    对$\R^n$中的点,$d\pare{\vx,\vy} = \norm{\vx-\vy} = \pare{\sum\pare{x_i-y_i}^2}^{1/2}$诱导欧氏度量,$\rho\pare{\vx,\vy} = \max\curb{\abs{x_i-y_i}}$诱导平方度量。
  \end{definition}
  欧式度量的三角不等式是熟知的结论。平方度量由
  \[ d_3 = \abs{x_k-z_k} \le \abs{x_k-y_k} + \abs{y_k-z_k} \le d_1 + d_2 \]
  验证三角不等式。注意同理可证若$X$上度量$d_1$和$Y$上度量$d_2$可以生成$X\times Y$上一度量$d_3 = \max\curb{d_1,d_2}$。
  \par
  欧式度量和平方度量的基元素分别为圆域和方域。由\tref{critfiner}立得
  \begin{theorem}
    度量拓扑$\cT'$细于$\cT$当且仅当对于任意$x$与$\epsilon$,存在$\epsilon'$满足
    \[ B'\pare{x,\epsilon'} \subset B\pare{x, \epsilon}. \]
  \end{theorem}
  \begin{theorem}
    欧氏度量和平方度量诱导$\R^n$上的积拓扑。
  \end{theorem}
  \begin{proof}
    直接验证不难,但由$\rho \le r \le \sqrt{n} \rho$可立得欧式与平方拓扑等价。
  \end{proof}
  \begin{definition}
    对$\R^J$中的点定义
    \[ \rho\pare{\vx,\vy} = \sup\setcond{\sbm{d}\pare{x_\alpha,y_\alpha}}{\alpha\in J}, \]
    可得一致度量,诱导出一致拓扑。
  \end{definition}
  \begin{theorem}
    一致拓扑细于积拓扑,粗于箱拓扑。$J$为无限集则两两不同。
  \end{theorem}
  \begin{proof}
    玩弄基元素的大小可证其粗细。$J$无限时,$\pare{-1,1}^J$在一致拓扑下为开,积拓扑下非开。$\prod \pare{-1/n,1/n}$在箱拓扑下为开,一致拓扑下非开。
  \end{proof}
  \begin{theorem}
    对$\R$的可数无限积$\R^\omega$定义
    \[ D\pare{\vx, \vy} = \sup\curb{\frac{\sbm{d}\pare{x_i,y_i}}{i}}, \]
    可诱导$\R^\omega$上的积拓扑。
  \end{theorem}
  \begin{proof}
    设$\R^\omega$的某基$B$的分量在$j$后均为$\R$,则某$B\pare{x, \epsilon/j}$包含其内。反之也可以选择这样的基包含于$B\pare{x, 1/j}$内。
  \end{proof}
  类似证明可仿照得到
  \begin{theorem}
    可度量化空间的可数积仍可度量化。
  \end{theorem}
  \begin{ex}
    在$X\times Y=\R^2$上定义$d=\min\curb{y_2-y_1,1}$,如果两点共$x$,否则$d = 1+\pare{x_1-x_2}$,则$d$诱导字典序拓扑。
  \end{ex}
  \begin{ex}
    易见度量空间的子空间仍为度量空间,且子空间的度量直接限制定义域可得。
  \end{ex}
  \subsection{连续函数与度量拓扑}
  \begin{theorem}
    度量空间到度量空间的$\func{f}{X}{Y}$的连续性等价于$\epsilon$-$\delta$条件。
  \end{theorem}
  \begin{proof}
    仿照\eref{epsd}可得。
  \end{proof}
  \begin{lemma}[序列引理]
    若$A$中有收敛于$x$的序列,则$x\in\clo{A}$。若$X$为度量空间,逆命题成立。
  \end{lemma}
  \begin{theorem}
    度量空间之间的$\func{f}{X}{Y}$连续的充要条件谓$x_n\rightarrow x$等价于$f\pare{x_n}\rightarrow f\pare{x}$。
  \end{theorem}
  \begin{proof}
    若拓扑条件成立,则$f\pare{x}$的小邻域原像都会包含$x$的邻域,故包含$\curb{x_n}_{n\ge N}$。若序列条件成立,结合序列引理与\tref{contineq}即可。
  \end{proof}
  注意上述定理对满足下列条件的空间也可以直接适用。
  \begin{definition}
    如果$x$有邻域$\curb{U_n}$满足任意邻域$U$都有某$U_n$含于其内,则称$X$在$x$处有可数基。如果处处都有则称$X$满足第一可数性公理。
  \end{definition}
  \begin{lemma}
    加减乘除是其定义域内的连续函数。
  \end{lemma}
  \begin{theorem}
    连续函数加减乘的结果连续,恒非零的除亦连续。
  \end{theorem}
  \begin{definition}
    若$\curb{f_n}$关于度量$d\pare{f,g}=\sup\curb{\abs{f-g}}$收敛于$f$,则称其一致收敛。
  \end{definition}
  \begin{theorem}
    一致收敛的连续函数列收敛于连续函数。
  \end{theorem}
  \begin{proof}
    对给定的$\epsilon$,存在$\delta$和$N$使得当$\abs{x-y}<\delta$,诸变差皆小于$\delta$。
    \[ \abs{f\pare{x}-f\pare{y}} \le \abs{f\pare{x}-f_N\pare{x}} +  \abs{f_N\pare{x}-f_N\pare{y}} +  \abs{f\pare{y}-f_N\pare{y}}.\qedhere\]
  \end{proof}
  \begin{collary}
    若$x_n\rightarrow x$而$\curb{f_n}$一致收敛于$f$,则$f_n\pare{x_n}\rightarrow f\pare{x}$。
  \end{collary}
  \begin{ex}
    箱拓扑的$\R^\omega$不满足序列引理因此不可度量化。$0\in\clo{\R_+^\omega}$但$\prod\pare{-x_{ii},x_{ii}}$排斥所有$x_i$。
  \end{ex}
  \begin{ex}
    不可数个$\R$的积空间不可度量化。
  \end{ex}
  \begin{proof}
    考虑$\R^J$由那些知有有限个零分量的$\curb{0,1}$序列的子空间,易见$0$在其内。然而,能有幸为零的分量指标仅有可数个,故存在恒$1$的指标。
  \end{proof}
  \subsection{商拓扑}
  \begin{definition}
    满射$\func{p}{X}{Y}$称为商映射,如果$U$是$Y$的开集当且仅当$\inv{p}\pare{U}$是$X$的开集。
  \end{definition}
  易见开集也可以改为闭集。
  \begin{definition}
    $X$的子集$C$为饱和的,如果它是纤维的并。
  \end{definition}
  商映射等价于饱和开集映射到开集。易见开映射和闭映射(把开集映射到开集或者把闭集映射到闭集)都是商映射。
  \begin{ex}
    $\brac{0,1}\cup\brac{2,3}$到$\brac{0,2}$的黏贴映射是闭映射但不是开映射。
  \end{ex}
  \begin{ex}
    $\func{\pi_1}{\R\times\R}{\R}$是开映射但不是闭映射,因为$\curb{y=1/x}$被映射到开集。
  \end{ex}
  \begin{ex}
  \label{ex:qnotoc}
    $\pi_1$在$A=\R\times0 \cup \blr{0,\infty} \times \R$上的限制是商映射,但不是开映射或者闭映射。$A-\lbr{-\infty,0}\times 0$是开集,但是被映射到闭集。$\curb{y=\pm\tan x}$图像左侧是闭集,但被映射到开集。
  \end{ex}
  \begin{definition}
    满射$\func{p}{X}{A}$的像$A$上存在一拓扑使得$X$为商映射,此拓扑谓商拓扑。
  \end{definition}
  \begin{ex}
    $y=\sgn\pare{x}$在点集上可以诱导一个商拓扑$\curb{\curb{\curb{-1},\curb{1}},0}$。
  \end{ex}
  \begin{definition}
    $X^*$为$X$的分拆,则$\func{\pi}{X}{X^*}$诱导的商拓扑使$X^*$为商空间。
  \end{definition}
  \begin{ex}
    将单位圆盘将圆周视为等价类,则商空间同胚于球面。
  \end{ex}
  \begin{ex}
    将矩形四角和对边上对应点视为等价类,则商空间同胚于环面。
  \end{ex}
  由\eref{qnotoc}知商映射在子空间的限制未必是商映射,但仍然有
  \begin{theorem}
    设商映射$\func{p}{X}{Y}$与饱和子空间$A$,则$p$在其上的限制$\func{q}{A}{p\pare{A}}$仍为商映射,如果
    \begin{cenum}
      \item $A$为开集或闭集;
      \item 或者$p$为开映射或闭映射。
    \end{cenum}
  \end{theorem}
  \begin{proof}
    先验证,如果$V\subset p\pare{A}$,则$\inv{q}\pare{V} = \inv{p}\pare{V}$。如果$U\subset X$,则$p\pare{U\cap A}=p\pare{U}\cap p\pare{A}$。都有$\inv{q}\pare{V}$是开的$\Rightarrow V$在$p\pare{A}$中为开。
  \end{proof}
  商映射的复合仍为商映射,但乘积不一定,Hausdorff空间的商空间也不一定是Hausdorff空间。
  \begin{theorem}
    商映射$p$与纤维上的映射$g$诱导$f$满足$f\comp p=g$。$f$连续当且仅当$g$连续,$f$为商映射当且仅当$g$为商映射。
  \end{theorem}
  \centerline{\xymatrix{
      X\ar[d]_p\ar[dr]^g \\
      Y\ar@{.>}[r]_f & Z
    }}{}
  \begin{proof}
    若$p$和$g$为商映射,证明$\inv{f}\pare{V}$为开集$\Rightarrow V$为开集即可。
  \end{proof}
  \begin{corollary}
    设$\func{g}{X}{Z}$为连续满射,$X^*$为各纤维的集,取商拓扑,则
    \begin{cenum}
      \item $g$诱导的$\func{f}{X^*}{Z}$一一连续,其为同胚当且仅当$g$为商映射;
      \item 若$Z$为Hausdorff空间,则$X^*$为Hausdorff空间。
    \end{cenum}
  \end{corollary}
  \centerline{\xymatrix{
      X\ar[d]_p\ar[dr]^g \\
      X^*\ar[r]_f & Z
    }}{}
  \begin{proof}
    注意一一的商映射等价于同胚。
  \end{proof}
  \begin{ex}
    设$X=\brac{0,1}\times\curb{1,2,\cdots}$,$Z=x\times \pare{x/n}$其中$x\in\brac{0,1}$,则$g\pare{x\times n} = x\times\pare{x/n}$诱导出$X^*$为将$X$诸左端点粘合的空间,但$\func{f}{X^*}{Z}$不是同胚。
  \end{ex}
  \begin{proof}
    考虑$x_n=\pare{1/n}\times n$,则$\curb{x_n}$为闭集但是$z_n=\pare{1/n}\times{1/n^2}$不是,因此$g$不是商映射。
  \end{proof}
  \begin{ex}
    设$\func{p}{X}{X^*}$是将$\R$的$\Z_+$粘合为$b$形成的商空间,$\func{i}{\Q}{\Q}$为恒等映射,则$p\times i$不是商映射。
  \end{ex}
  \begin{proof}
    假设$U_n$为$n\times\pare{\sqrt{2}/n}$附加其上方和下方的条带,$U=\cup U_n$,则$U$饱和但$p\times i\pare{U}$不是开集,因为某$I_b\times I_\delta$的原像包含条带的缝隙。
  \end{proof}
  \section{连通性与紧致性}
  \subsection{连通空间}
  \begin{definition}
    拓扑空间$X$的一个分割,谓其一对无交非空开集其并为$X$。
  \end{definition}
  \begin{lemma}
    若$Y$是$X$的子空间,则其分割的分量彼此不包含对方的极限点。若存在一对并为$Y$的非空集合彼此不包含对方极限点,则亦构成分割。
  \end{lemma}
  \begin{proof}
    注意分量既开又闭,故极限点自含。若存在这样的一对,则$A$中任意元素都存在小邻域在$B$外,故在$A$内,故$A$为开集。
  \end{proof}
  \begin{ex}
    密着拓扑是连通的。
  \end{ex}
  \begin{ex}
    $\R$的子空间$\blr{-1,0}\cup\lbr{0,1}$不是连通的。
  \end{ex}
  \begin{ex}
    $\Q$不是连通的。
  \end{ex}
  \begin{ex}
    $\curb{y=1/x}$其中$x>0$和$\curb{y=0}$作为$\R^2$的子空间不是连通的。
  \end{ex}
  \begin{lemma}
    连通子空间包含在分割的二者中一个内。
  \end{lemma}
  \begin{theorem}
    含有一个公共点的连通子空间族的并是连通的。
  \end{theorem}
  \begin{theorem}
    $A$为连通子空间,则$A\subset B \subset \clo{A}$的$B$是连通的。
  \end{theorem}
  \begin{proof}
    若$\clo{A}$被分拆为$C\cup D$,则$A\subset C$而$D$为一个与$A$无交的邻域。
  \end{proof}
  \begin{theorem}
    连通空间的连续映射的像是连通的。
  \end{theorem}
  \begin{theorem}
    有限多个连通空间的积是连通的。
  \end{theorem}
  \begin{proof}
    注意每个十字形是含有公共点的连通空间的并,再将十字形并起。
  \end{proof}
  \begin{ex}
    箱拓扑的$\R^\omega$不连通,后者可分为有界序列和无界序列两开集。
  \end{ex}
  \begin{ex}
    积拓扑的$\R^\omega$连通,因为$\R^\omega=\clo{\R^\infty}$,而$\R^\infty\isom\cap\pare{\R^n+\pare{0,0,\cdots}}$,被并的元素均连通且具有原点为公共点。
  \end{ex}
  \subsection{实直线上的连通子空间}
  \begin{definition}
    若$L$是多于一个元素的全序集,且$L$具有上确界性质,且$x<y\Rightarrow$ 存在$x<z<y$,则$L$谓线性连续统。
  \end{definition}
  \begin{theorem}
    若$L$为序拓扑的线性连续统,则$L$及其区间和射线都连通。
  \end{theorem}
  \begin{proof}
    $L$的凸子集$Y$若分拆为$A$和$B$,则其中的(不妨设)$a<b$有$\brac{a,b}$被分割为$A_0\cup B_0$。$\inf B_0 \in B_0$或$\inf B_0\in A_0$都会导致矛盾。
  \end{proof}
  \begin{corollary}
    $\R$及其区间和射线都是连通的。
  \end{corollary}
  \begin{theorem}[介值定理]
    连通空间到序拓扑的全序集的映射$\func{f}{X}{Y}$,任何$f\pare{a}$与$f\pare{b}$之间的$r$存在$c$满足$f\pare{c}=r$。
  \end{theorem}
  \begin{ex}
    有序矩形是连通的,只需验证上确界性质。分$\sup \pi_1 \pare{A}$在$\pi_1 \pare{A}$内或外取$b\times c$或$b\times 0$即可。
  \end{ex}
  \begin{ex}
    良序集$X$有$X\times\blr{0,1}$关于字典序为线性连续统。
  \end{ex}
  \begin{definition}
    $X$中$x$到$y$到一条道路是连续的$\func{f}{\brac{a,b}}{X}$满足$f\pare{a}=x$与$f\pare{b}=y$。若$X$中任意两点之间都存在道路,则称之道路连通的。
  \end{definition}
  显然道路连通蕴含连通。
  \begin{ex}
    $\R^n$中的球是道路连通的,$f\pare{t}=\pare{1-t}\vx+t\vy$是一条道路。
  \end{ex}
  \begin{ex}
    $\R^n-\curb{\vzero}$是连通的。
  \end{ex}
  \begin{ex}
    单位球面是连通的,因为它可以从球由$\func{f}{\vx}{x/\norm{x}}$得到。
  \end{ex}
  \begin{ex}
    有序矩形$I_o^2$连通而非道路连通。在每个被映射到竖线的$\brac{a_i,b_i}$中选取有理数,只能得到可数竖线。
  \end{ex}
  \begin{ex}
    $S=\curb{x\times \sin\pare{1/x}}$的闭包$\clo{S}$连通而非道路连通。任何$S$到$0\times\brac{-1,1}$的路径都必然震荡$t_n \times \pare{-1}^n$,故无法收敛,不可能连续。
  \end{ex}
  \subsection{紧致空间}
  \begin{definition}
    $X$的子集族$\setf{C}$称为具有有限交性质(finite intersection property),如果$\setf{C}$的任意有限子族交非空。
  \end{definition}
  \begin{theorem}
    \label{thm:finiteinters}
    $X$是紧致的当且仅当$X$中具有有限交性质的每一个闭集族$\setf{C}$,其交非空。
  \end{theorem}
  \begin{proof}
    这些集合的补是一堆开集,这些开集中的任意有限个都不能覆盖$X$,但$X$是紧致的,所以它们合起来也不能覆盖$X$。
  \end{proof}
  \subsection{实直线上的紧致子空间}
  \begin{theorem}
    \label{thm:uncountableR}
    非空紧致Hausforff空间$X$,若无孤立点则不可数。
  \end{theorem}
  \begin{proof}
    对于$X$的任意元素$x$,由Hausdorff性质皆可以选取一非空开集$V$,满足$x \notin \clo{V}$。\par
    假设有$\func{f}{\Z_+}{X}$,则可以选取$V_1$其闭包不包含$x$,且可选取$V_2 \subset V_1$其闭包不包含$x_2$,以此类推。考虑
    \[ \clo{V}_1 \supset \clo{V}_2 \supset \cdots, \]
    由$x$的紧致性与\tref{finiteinters},知其交非空故有元素$x$在诸$x_n$之外。
  \end{proof}
  \subsection{极限点紧致性}
  \begin{definition}
    度量空间内的映射$f$,若
    \[ \dist{f\pare{x}}{f\pare{y}} < \dist{x}{y}, \]
    则称$f$为收紧映射(shrinking map)。
  \end{definition}
    \begin{definition}
    度量空间内的映射$f$,若
    \[ \dist{f\pare{x}}{f\pare{y}} \le \alpha\dist{x}{y}, \]
    其中$\alpha < 1$,则称$f$为压缩映射(contraction map)。
  \end{definition}
  \begin{theorem}
    \label{thm:fixp0}
    若$X$为完备度量空间,则压缩映射存在不动点。
  \end{theorem}


%ContentEnds
 
\ifx\allfiles\undefined %如果位置放错,可能出现意外中断
\end{document}
\fi
  %Ch1.GeneralSetTheory.tex
\ifx\allfiles\undefined
\documentclass{ctexrep}
% Mathematics Include

\usepackage{amsmath}
\usepackage{amssymb}
\usepackage{amsthm}
\usepackage{amsfonts}
\usepackage{mathrsfs}
\usepackage{enumitem}
\usepackage{braket}
\usepackage{hyperref}
\usepackage[all, pdf]{xy}

% Physics Include
\usepackage{amsmath}
\usepackage{physics}
\usepackage{siunitx}
\usepackage[makeroom]{cancel}
\usepackage{pstricks}
\usepackage{pstricks-add}
\psset{algebraic=true}

\usepackage[version=4]{mhchem}
\usepackage{array,booktabs}
\usepackage{longtable}
\usepackage{mathtools}
\usepackage[normalem]{ulem}
\usepackage{multicol}

% Mathematics Head

\newcommand{\pare}[1]{\left(#1\right)}
\newcommand{\blr}[1]{\left[#1\right)}
\newcommand{\lbr}[1]{\left(#1\right]}
\newcommand{\brac}[1]{\left[#1\right]}
\newcommand{\curb}[1]{\left\{#1\right\}}
% \newcommand{\abs}[1]{\left|\, #1 \,\right|}
\newcommand{\rec}[1]{\frac{1}{#1}}
\newcommand{\N}{\mathbb{N}}
\newcommand{\bC}{\mathbb{C}}
\newcommand{\Q}{\mathbb{Q}}
\newcommand{\Z}{\mathbb{Z}}
\newcommand{\R}{\mathbb{R}}
\newcommand{\unk}{\mathcal{X}}
\newcommand{\bu}[3]{#1_{#2}^{\pare{#3}}}
\newcommand{\dref}[1]{定义\ref{def:#1}}
\newcommand{\tref}[1]{定理\ref{thm:#1}}
\newcommand{\lref}[1]{引理\ref{lem:#1}}
\newcommand{\cref}[1]{推论\ref{coll:#1}}
\newcommand{\pref}[1]{命题\ref{prp:#1}}
\newcommand{\eref}[1]{例\ref{ex:#1}}
\newcommand{\func}[3]{#1:\, #2 \rightarrow #3}
\newcommand{\overbar}[1]{\mkern 1.5mu\overline{\mkern-1.5mu#1\mkern-1.5mu}\mkern 1.5mu}
\newcommand{\clo}[1]{\overbar{#1}}
\newcommand{\supi}[2]{\overbar{\int_{#1}^{#2}}}
\newcommand{\infi}[2]{\underbar{\int_{#1}^{#2}}}
\newcommand{\setf}{\mathscr}
\newcommand{\bool}{\mathrm{bool}}
\newcommand{\inc}{++}
\newcommand{\defeq}{:=}
\newcommand{\ntuple}{$n$元组}
\newcommand{\card}[1]{\#\pare{#1}}
\newcommand{\setcond}[2]{\curb{#1 \, \left| \, #2 \right.}}
\newcommand{\setcondl}[2]{\curb{\left. #1 \, \right| \, #2}}
\newcommand{\bv}[1]{\mathbf{#1}}
\newcommand{\bfa}{\bv{a}}
\newcommand{\bfb}{\bv{b}}
\newcommand{\bfx}{\bv{x}}
\newcommand{\bfy}{\bv{y}}
\newcommand{\bfe}{\bv{e}}
\newcommand{\bfF}{\bv{F}}
\newcommand{\bff}{\bv{f}}
\newcommand{\bfG}{\bv{G}}
\newcommand{\bfH}{\bv{H}}
\newcommand{\bfg}{\bv{g}}
\newcommand{\bfh}{\bv{h}}
\newcommand{\bfr}{\bv{r}}
\newcommand{\bfk}{\bv{k}}
\newcommand{\bfu}{\bv{u}}
\newcommand{\bfv}{\bv{v}}
\newcommand{\oo}[1]{o\pare{#1}}
\newcommand{\OO}[1]{O\pare{#1}}
% \newcommand{\norm}[1]{\left\| #1 \right\|}
\newcommand{\DD}{\mathbf{D}}
\newcommand{\comp}{\circ}
\newcommand{\const}{\mathrm{const}}
\newcommand{\dist}[2]{d\pare{#1,#2}}
\newcommand{\len}{\ell}
\newcommand{\siga}{$\sigma$-代数}
\newcommand{\cara}{Carath\'{e}odory}
\newcommand{\Gd}{G_\delta}
\newcommand{\Fs}{F_\sigma}
\newcommand{\mmani}{$m$-维流形}
\newcommand{\open}[1]{\mathcal{#1}}
\newcommand{\half}{\frac{1}{2}}
\newcommand{\maxo}[1]{\text{max}\curb{#1}}
\newcommand{\mino}[1]{\text{min}\curb{#1}}
\newcommand{\epsclo}{$\epsilon$-接近}
\newcommand{\close}[1]{$#1$-接近}
\newcommand{\cinf}{$C^\infty$}
\newcommand{\cuno}{$C^1$}
\newcommand{\Int}{\text{Int}\,}
\newcommand{\Ext}{\text{Ext}\,}
\newcommand{\funcf}{\mathcal}
\newcommand{\DDu}{\overbar{\DD}}
\newcommand{\DDl}{\underbar{\DD}}
\newcommand{\Diff}[1]{\mathrm{Diff}_{#1}\,}
\newcommand{\Av}[1]{\mathrm{Av}_{#1}\,}
\newcommand{\Lip}[1]{Lipschitz-$#1$}
\newcommand{\sgn}{\mathrm{sgn}}
\newcommand{\eset}{\varnothing}
\newcommand{\cT}{\mathcal{T}}
\newcommand{\cS}{\mathcal{S}}
\newcommand{\cG}{\mathcal{G}}
\newcommand{\cF}{\mathcal{F}}
\newcommand{\cC}{\mathcal{C}}
\newcommand{\cB}{\mathcal{B}}
\newcommand{\inter}[1]{\mathring{#1}}
\newcommand{\forest}[3]{对于{#1},存在{#2},使得{#3}}
\newcommand{\tuno}{$T_1$公理}
\newcommand{\isom}{\overset{\sim}{=}}
\newcommand{\diam}{\mathrm{diam}\,}
\newcommand{\ord}[1]{\abs{#1}}
\newcommand{\sbm}[1]{\overbar{#1}}
\newcommand{\inv}[1]{#1^{-1}}
\newcommand{\restr}[2]{#1|_{#2}}
\newcommand{\divs}{|}
\newcommand{\ndivs}{\nmid}
\newcommand{\modeq}[1]{\overbar{#1}}
\newcommand{\ggen}[1]{\langle#1\rangle}
\newcommand{\ggencond}{\braket}

\newcommand{\hd}{H\"{o}lder}

\renewcommand{\proofname}{证明}

\newenvironment{cenum}{\begin{enumerate}\itemsep0em}{\end{enumerate}}

\newtheorem{definition}{定义}[section]
\newtheorem{lemma}{引理}[section]
\newtheorem{theorem}{定理}[section]
\newtheorem{collary}{推论}[section]
\newtheorem{corollary}{推论}[section]
\newtheorem{proposition}{命题}[section]
\newtheorem{axiom}{公理}[section]
\newtheorem{ex}{例}[section]
\newtheorem{reflection}{反射}[section]
\newcommand{\refl}[1]{\vspace{0.5em}\par\noindent\fbox{%
    \parbox{0.9\textwidth}{%
    \begin{reflection}
        #1
    \end{reflection}
    }%
}\vspace{0.5em}\par}
\newcommand{\rref}[1]{反射\ref{refl:#1}}
\newcommand{\tbref}[1]{表\ref{table:#1}}
\allowdisplaybreaks

\newenvironment{aenum}{\begin{enumerate}[label=\textnormal{(\alph*)}]}{\end{enumerate}}

% Physics Head

\DeclareSIUnit\dyne{dynes}

\newcommand{\ddel}[1]{\frac{\partial}{\partial #1}}
\newcommand{\ddelon}[2]{\frac{\partial #1}{\partial #2}}
\newcommand{\dddel}[1]{\frac{\partial^2}{\partial^2 #1}}
\newcommand{\ddt}{\ddel{t}}
\newcommand{\ddT}{\ddel{T}}
\newcommand{\ddV}{\ddel{V}}
\newcommand{\ddr}{\ddel{r}}
\newcommand{\ddth}{\ddel{\theta}}
\newcommand{\ddph}{\ddel{\phi}}
\newcommand{\dddt}{\dddel{t}}
\newcommand{\dddr}{\dddel{t}}
\newcommand{\dddth}{\dddel{\theta}}
\newcommand{\dddph}{\dddel{\phi}}
\newcommand{\rd}[1]{\mathrm{d} #1}
\newcommand{\dt}{\rd{t}}
\newcommand{\dy}{\rd{y}}
\newcommand{\dx}{\rd{x}}
\newcommand{\edd}[1]{\frac{\mathrm{d}}{\mathrm{d} #1}}
\newcommand{\eddd}[1]{\frac{\mathrm{d}^2}{\mathrm{d}^2 #1}}
\newcommand{\eddon}[2]{\frac{\mathrm{d} #1}{\mathrm{d} #2}}
\newcommand{\edddon}[2]{\frac{\mathrm{d}^2 #1}{\mathrm{d}^2 #2}}
\newcommand{\edt}{\edd{t}}
\newcommand{\edton}[1]{\eddon{#1}{t}}
\newcommand{\edT}{\edd{T}}
\newcommand{\edr}{\edd{r}}
\newcommand{\edl}{\edd{l}}
\newcommand{\edx}{\edd{x}}
\newcommand{\edth}{\edd{\theta}}
\newcommand{\eddton}[1]{\edddon{#1}{t}}
\newcommand{\eddzon}[1]{\edddon{#1}{z}}
\newcommand{\vect}[1]{\boldsymbol{#1}}
\newcommand{\alp}{\frac{1}{\sqrt{2}}}
\newcommand{\alpi}{\frac{i}{\sqrt{2}}}
\newcommand{\expc}[1]{\langle#1\rangle}
\newcommand{\bkn}[1]{\bra{#1}\ket{#1}}
\newcommand{\bk}[2]{\bra{#1}\ket{#2}}
\newcommand{\bik}[3]{\bra{#1} #2 \ket{#3}}
\newcommand{\vari}[1]{\sigma_{#1}}
\newcommand{\intc}[2]{\left[#1, #2\right]}
\newcommand{\sch}{Schr\"{o}dinger}
\newcommand{\moment}{\boldsymbol{p}}
\newcommand{\coor}{\boldsymbol{x}}
\newcommand{\lapc}{\nabla^2}
% \newcommand{\rec}[1]{\frac{1}{#1}}
\newcommand{\vva}{\boldsymbol{a}}
\newcommand{\vvb}{\boldsymbol{b}}
\newcommand{\vc}{\boldsymbol{c}}
\newcommand{\vd}{\boldsymbol{d}}
\newcommand{\ve}{\boldsymbol{e}}
\newcommand{\vf}{\boldsymbol{f}}
\newcommand{\vg}{\boldsymbol{g}}
\newcommand{\vh}{\boldsymbol{h}}
\newcommand{\vi}{\boldsymbol{i}}
\newcommand{\vj}{\boldsymbol{j}}
\newcommand{\vk}{\boldsymbol{k}}
\newcommand{\vl}{\boldsymbol{l}}
\newcommand{\vm}{\boldsymbol{m}}
\newcommand{\vn}{\boldsymbol{n}}
\newcommand{\vo}{\boldsymbol{o}}
\newcommand{\vp}{\boldsymbol{p}}
\newcommand{\vq}{\boldsymbol{q}}
\newcommand{\vr}{\boldsymbol{r}}
\newcommand{\vs}{\boldsymbol{s}}
\newcommand{\vt}{\boldsymbol{t}}
\newcommand{\vvu}{\boldsymbol{u}}
\newcommand{\vv}{\boldsymbol{v}}
\newcommand{\vw}{\boldsymbol{w}}
\newcommand{\vx}{\boldsymbol{x}}
\newcommand{\vy}{\boldsymbol{y}}
\newcommand{\vz}{\boldsymbol{z}}
\newcommand{\vA}{\boldsymbol{A}}
\newcommand{\vB}{\boldsymbol{B}}
\newcommand{\vC}{\boldsymbol{C}}
\newcommand{\vD}{\boldsymbol{D}}
\newcommand{\vE}{\boldsymbol{E}}
\newcommand{\vF}{\boldsymbol{F}}
\newcommand{\vG}{\boldsymbol{G}}
\newcommand{\vH}{\boldsymbol{H}}
\newcommand{\vI}{\boldsymbol{I}}
\newcommand{\vJ}{\boldsymbol{J}}
\newcommand{\vK}{\boldsymbol{K}}
\newcommand{\vL}{\boldsymbol{L}}
\newcommand{\vM}{\boldsymbol{M}}
\newcommand{\vN}{\boldsymbol{N}}
\newcommand{\vO}{\boldsymbol{O}}
\newcommand{\vP}{\boldsymbol{P}}
\newcommand{\vQ}{\boldsymbol{Q}}
\newcommand{\vR}{\boldsymbol{R}}
\newcommand{\vS}{\boldsymbol{S}}
\newcommand{\vT}{\boldsymbol{T}}
\newcommand{\vU}{\boldsymbol{U}}
\newcommand{\vV}{\boldsymbol{V}}
\newcommand{\vW}{\boldsymbol{W}}
\newcommand{\vX}{\boldsymbol{X}}
\newcommand{\vY}{\boldsymbol{Y}}
\newcommand{\vZ}{\boldsymbol{Z}}
\newcommand{\vzero}{\boldsymbol{0}}
\newcommand{\vomega}{\boldsymbol{\omega}}
%\newcommand{\half}{\frac{1}{2}}
\newcommand{\thalf}{\frac{3}{2}}
\newcommand{\rot}{\nabla\times}
\newcommand{\divg}{\nabla\cdot}
\newcommand{\cE}{\mathcal{E}}
\newcommand{\conclu}[1]{\vspace{1em}\par\noindent\fbox{\parbox{0.9\textwidth}{#1}}\vspace{1em}}
\newcommand{\subentrynote}{$\bullet$}
\newcommand{\keypoint}[1]{\par\subentrynote\quad #1 \par}
\newcommand{\fconclu}{\boxed}
\newcommand{\pair}[2]{#1 \, #2}
\newcommand{\intn}[2]{\int #1 \,\mathrm{d} #2}
\newcommand{\intu}[3]{\int_0^{#1} #2 \,\mathrm{d} #3}
\newcommand{\intiu}[3]{\int_{-\infty}^{#1} #2 \, \rd{} #3}
\newcommand{\intui}[2]{\int_0^{\infty} #1 \,\mathrm{d} #2}
\newcommand{\intii}[2]{\int_{-\infty}^{\infty} #1 \,\mathrm{d} #2}
\newcommand{\intt}[2]{\int_0^\infty #1 \, \rd{} #2}
\newcommand{\intr}[2]{\int_{-\infty}^{\infty} #1 \, \rd{} #2}
\newcommand{\intbi}[3]{\int_{#1}^{\infty} #2 \, \rd{} #3}
\newcommand{\intab}[4]{\int_{#1}^{#2} #3 \, \rd{} #4}
\newcommand{\bfactor}[1]{e^{-#1/k_BT}}
\newcommand{\pbfactor}[1]{e^{#1/k_BT}}
\newcommand{\dn}[2]{#1^{\pare{#2}}}
\newcommand{\prodg}[1]{\pare{#1}^\times}

\newcommand{\notion}{\emph}
\newcommand{\iP}{\mathcal{P}}
\newcommand{\eiP}{e^{-\iP}}
\newcommand{\iF}{\mathcal{F}}
\newcommand{\eiF}{e^{-\iF}}
\newcommand{\iG}{\mathcal{G}}

\newcommand{\rc}{r\cos\theta}
\newcommand{\rs}{r\sin\theta}
\newcommand{\sn}{\mathrm{sn}}
\newcommand{\cn}{\mathrm{cn}}
\newcommand{\rdn}{\mathrm{dn}}

\newcommand{\hankel}{H_p^{\pare{1}}}
\newcommand{\hankell}{H_p^{\pare{2}}}
\newcommand{\hhankel}{H_n^{\pare{1}}}
\newcommand{\hhankell}{H_n^{\pare{2}}}
\newcommand{\ber}{\text{ber}\,}
\newcommand{\bei}{\text{bei}\,}
\newcommand{\kker}{\text{ker}\,}
\newcommand{\kei}{\text{kei}\,}
\newcommand{\Ai}{\text{Ai}}
\newcommand{\Bi}{\text{Bi}}

\newcommand{\re}{\text{Re}\,}

\newcommand{\Fp}{F_\phi}
\newcommand{\Ep}{E_\phi}
\newcommand{\Fx}{F_x}
\newcommand{\FF}{\mathbf{F}}
\newcommand{\Ex}{E_x}

%\newcommand{\erf}{\mathrm{erf}}
\newcommand{\erfi}{\mathrm{erfi}}
\newcommand{\erfc}{\mathrm{erfc}}
\newcommand{\ehxs}[1]{e^{-\frac{#1^2}{2}}}
\newcommand{\dcol}[2]{\[ \left.#1 \hspace{1em}\right\vert\hspace{1em} #2 \]}
\newcommand{\titlegamma}{\texorpdfstring{$\Gamma$}{Gamma}}
\newcommand{\titleB}{\texorpdfstring{$B$}{B}}

% Computer Science Head
\usepackage{listings}
\usepackage{color}

\definecolor{dkgreen}{rgb}{0,0.6,0}
\definecolor{gray}{rgb}{0.5,0.5,0.5}
\definecolor{mauve}{rgb}{0.58,0,0.82}

\lstset{frame=tb,
  language=Java,
  aboveskip=3mm,
  belowskip=3mm,
  showstringspaces=false,
  columns=flexible,
  basicstyle={\small\ttfamily},
  numbers=none,
  numberstyle=\tiny\color{gray},
  keywordstyle=\color{blue},
  commentstyle=\color{dkgreen},
  stringstyle=\color{mauve},
  breaklines=true,
  breakatwhitespace=true,
  tabsize=3
}
\lstset{language=Java}
\newcommand{\snp}[1]{\lstinline!#1!}
\newcommand{\term}[2]{\textbf{#1(#2)}}
\begin{document}
\fi

%Content

\chapter{普通微积分}
  \section{微分}
  \subsection{向量值函数的微分}
  \begin{theorem}
    \label{thm:vfab}
    设连续的$\bff$在$\pare{a,b}$内可微并将$\brac{a,b}$映入$\R^k$,则必有$x\in\pare{a,b}$满足$\abs{\bff\pare{b}-\bff\pare{a}}\le\pare{b-a}\abs{\bff'\pare{x}}$。
  \end{theorem}


%ContentEnds
 
\ifx\allfiles\undefined %如果位置放错,可能出现意外中断
\end{document}
\fi
  %Ch1.GeneralSetTheory.tex
\ifx\allfiles\undefined
\documentclass{ctexrep}
% Mathematics Include

\usepackage{amsmath}
\usepackage{amssymb}
\usepackage{amsthm}
\usepackage{amsfonts}
\usepackage{mathrsfs}
\usepackage{enumitem}
\usepackage{braket}
\usepackage{hyperref}
\usepackage[all, pdf]{xy}

% Physics Include
\usepackage{amsmath}
\usepackage{physics}
\usepackage{siunitx}
\usepackage[makeroom]{cancel}
\usepackage{pstricks}
\usepackage{pstricks-add}
\psset{algebraic=true}

\usepackage[version=4]{mhchem}
\usepackage{array,booktabs}
\usepackage{longtable}
\usepackage{mathtools}
\usepackage[normalem]{ulem}
\usepackage{multicol}

% Mathematics Head

\newcommand{\pare}[1]{\left(#1\right)}
\newcommand{\blr}[1]{\left[#1\right)}
\newcommand{\lbr}[1]{\left(#1\right]}
\newcommand{\brac}[1]{\left[#1\right]}
\newcommand{\curb}[1]{\left\{#1\right\}}
% \newcommand{\abs}[1]{\left|\, #1 \,\right|}
\newcommand{\rec}[1]{\frac{1}{#1}}
\newcommand{\N}{\mathbb{N}}
\newcommand{\bC}{\mathbb{C}}
\newcommand{\Q}{\mathbb{Q}}
\newcommand{\Z}{\mathbb{Z}}
\newcommand{\R}{\mathbb{R}}
\newcommand{\unk}{\mathcal{X}}
\newcommand{\bu}[3]{#1_{#2}^{\pare{#3}}}
\newcommand{\dref}[1]{定义\ref{def:#1}}
\newcommand{\tref}[1]{定理\ref{thm:#1}}
\newcommand{\lref}[1]{引理\ref{lem:#1}}
\newcommand{\cref}[1]{推论\ref{coll:#1}}
\newcommand{\pref}[1]{命题\ref{prp:#1}}
\newcommand{\eref}[1]{例\ref{ex:#1}}
\newcommand{\func}[3]{#1:\, #2 \rightarrow #3}
\newcommand{\overbar}[1]{\mkern 1.5mu\overline{\mkern-1.5mu#1\mkern-1.5mu}\mkern 1.5mu}
\newcommand{\clo}[1]{\overbar{#1}}
\newcommand{\supi}[2]{\overbar{\int_{#1}^{#2}}}
\newcommand{\infi}[2]{\underbar{\int_{#1}^{#2}}}
\newcommand{\setf}{\mathscr}
\newcommand{\bool}{\mathrm{bool}}
\newcommand{\inc}{++}
\newcommand{\defeq}{:=}
\newcommand{\ntuple}{$n$元组}
\newcommand{\card}[1]{\#\pare{#1}}
\newcommand{\setcond}[2]{\curb{#1 \, \left| \, #2 \right.}}
\newcommand{\setcondl}[2]{\curb{\left. #1 \, \right| \, #2}}
\newcommand{\bv}[1]{\mathbf{#1}}
\newcommand{\bfa}{\bv{a}}
\newcommand{\bfb}{\bv{b}}
\newcommand{\bfx}{\bv{x}}
\newcommand{\bfy}{\bv{y}}
\newcommand{\bfe}{\bv{e}}
\newcommand{\bfF}{\bv{F}}
\newcommand{\bff}{\bv{f}}
\newcommand{\bfG}{\bv{G}}
\newcommand{\bfH}{\bv{H}}
\newcommand{\bfg}{\bv{g}}
\newcommand{\bfh}{\bv{h}}
\newcommand{\bfr}{\bv{r}}
\newcommand{\bfk}{\bv{k}}
\newcommand{\bfu}{\bv{u}}
\newcommand{\bfv}{\bv{v}}
\newcommand{\oo}[1]{o\pare{#1}}
\newcommand{\OO}[1]{O\pare{#1}}
% \newcommand{\norm}[1]{\left\| #1 \right\|}
\newcommand{\DD}{\mathbf{D}}
\newcommand{\comp}{\circ}
\newcommand{\const}{\mathrm{const}}
\newcommand{\dist}[2]{d\pare{#1,#2}}
\newcommand{\len}{\ell}
\newcommand{\siga}{$\sigma$-代数}
\newcommand{\cara}{Carath\'{e}odory}
\newcommand{\Gd}{G_\delta}
\newcommand{\Fs}{F_\sigma}
\newcommand{\mmani}{$m$-维流形}
\newcommand{\open}[1]{\mathcal{#1}}
\newcommand{\half}{\frac{1}{2}}
\newcommand{\maxo}[1]{\text{max}\curb{#1}}
\newcommand{\mino}[1]{\text{min}\curb{#1}}
\newcommand{\epsclo}{$\epsilon$-接近}
\newcommand{\close}[1]{$#1$-接近}
\newcommand{\cinf}{$C^\infty$}
\newcommand{\cuno}{$C^1$}
\newcommand{\Int}{\text{Int}\,}
\newcommand{\Ext}{\text{Ext}\,}
\newcommand{\funcf}{\mathcal}
\newcommand{\DDu}{\overbar{\DD}}
\newcommand{\DDl}{\underbar{\DD}}
\newcommand{\Diff}[1]{\mathrm{Diff}_{#1}\,}
\newcommand{\Av}[1]{\mathrm{Av}_{#1}\,}
\newcommand{\Lip}[1]{Lipschitz-$#1$}
\newcommand{\sgn}{\mathrm{sgn}}
\newcommand{\eset}{\varnothing}
\newcommand{\cT}{\mathcal{T}}
\newcommand{\cS}{\mathcal{S}}
\newcommand{\cG}{\mathcal{G}}
\newcommand{\cF}{\mathcal{F}}
\newcommand{\cC}{\mathcal{C}}
\newcommand{\cB}{\mathcal{B}}
\newcommand{\inter}[1]{\mathring{#1}}
\newcommand{\forest}[3]{对于{#1},存在{#2},使得{#3}}
\newcommand{\tuno}{$T_1$公理}
\newcommand{\isom}{\overset{\sim}{=}}
\newcommand{\diam}{\mathrm{diam}\,}
\newcommand{\ord}[1]{\abs{#1}}
\newcommand{\sbm}[1]{\overbar{#1}}
\newcommand{\inv}[1]{#1^{-1}}
\newcommand{\restr}[2]{#1|_{#2}}
\newcommand{\divs}{|}
\newcommand{\ndivs}{\nmid}
\newcommand{\modeq}[1]{\overbar{#1}}
\newcommand{\ggen}[1]{\langle#1\rangle}
\newcommand{\ggencond}{\braket}

\newcommand{\hd}{H\"{o}lder}

\renewcommand{\proofname}{证明}

\newenvironment{cenum}{\begin{enumerate}\itemsep0em}{\end{enumerate}}

\newtheorem{definition}{定义}[section]
\newtheorem{lemma}{引理}[section]
\newtheorem{theorem}{定理}[section]
\newtheorem{collary}{推论}[section]
\newtheorem{corollary}{推论}[section]
\newtheorem{proposition}{命题}[section]
\newtheorem{axiom}{公理}[section]
\newtheorem{ex}{例}[section]
\newtheorem{reflection}{反射}[section]
\newcommand{\refl}[1]{\vspace{0.5em}\par\noindent\fbox{%
    \parbox{0.9\textwidth}{%
    \begin{reflection}
        #1
    \end{reflection}
    }%
}\vspace{0.5em}\par}
\newcommand{\rref}[1]{反射\ref{refl:#1}}
\newcommand{\tbref}[1]{表\ref{table:#1}}
\allowdisplaybreaks

\newenvironment{aenum}{\begin{enumerate}[label=\textnormal{(\alph*)}]}{\end{enumerate}}

% Physics Head

\DeclareSIUnit\dyne{dynes}

\newcommand{\ddel}[1]{\frac{\partial}{\partial #1}}
\newcommand{\ddelon}[2]{\frac{\partial #1}{\partial #2}}
\newcommand{\dddel}[1]{\frac{\partial^2}{\partial^2 #1}}
\newcommand{\ddt}{\ddel{t}}
\newcommand{\ddT}{\ddel{T}}
\newcommand{\ddV}{\ddel{V}}
\newcommand{\ddr}{\ddel{r}}
\newcommand{\ddth}{\ddel{\theta}}
\newcommand{\ddph}{\ddel{\phi}}
\newcommand{\dddt}{\dddel{t}}
\newcommand{\dddr}{\dddel{t}}
\newcommand{\dddth}{\dddel{\theta}}
\newcommand{\dddph}{\dddel{\phi}}
\newcommand{\rd}[1]{\mathrm{d} #1}
\newcommand{\dt}{\rd{t}}
\newcommand{\dy}{\rd{y}}
\newcommand{\dx}{\rd{x}}
\newcommand{\edd}[1]{\frac{\mathrm{d}}{\mathrm{d} #1}}
\newcommand{\eddd}[1]{\frac{\mathrm{d}^2}{\mathrm{d}^2 #1}}
\newcommand{\eddon}[2]{\frac{\mathrm{d} #1}{\mathrm{d} #2}}
\newcommand{\edddon}[2]{\frac{\mathrm{d}^2 #1}{\mathrm{d}^2 #2}}
\newcommand{\edt}{\edd{t}}
\newcommand{\edton}[1]{\eddon{#1}{t}}
\newcommand{\edT}{\edd{T}}
\newcommand{\edr}{\edd{r}}
\newcommand{\edl}{\edd{l}}
\newcommand{\edx}{\edd{x}}
\newcommand{\edth}{\edd{\theta}}
\newcommand{\eddton}[1]{\edddon{#1}{t}}
\newcommand{\eddzon}[1]{\edddon{#1}{z}}
\newcommand{\vect}[1]{\boldsymbol{#1}}
\newcommand{\alp}{\frac{1}{\sqrt{2}}}
\newcommand{\alpi}{\frac{i}{\sqrt{2}}}
\newcommand{\expc}[1]{\langle#1\rangle}
\newcommand{\bkn}[1]{\bra{#1}\ket{#1}}
\newcommand{\bk}[2]{\bra{#1}\ket{#2}}
\newcommand{\bik}[3]{\bra{#1} #2 \ket{#3}}
\newcommand{\vari}[1]{\sigma_{#1}}
\newcommand{\intc}[2]{\left[#1, #2\right]}
\newcommand{\sch}{Schr\"{o}dinger}
\newcommand{\moment}{\boldsymbol{p}}
\newcommand{\coor}{\boldsymbol{x}}
\newcommand{\lapc}{\nabla^2}
% \newcommand{\rec}[1]{\frac{1}{#1}}
\newcommand{\vva}{\boldsymbol{a}}
\newcommand{\vvb}{\boldsymbol{b}}
\newcommand{\vc}{\boldsymbol{c}}
\newcommand{\vd}{\boldsymbol{d}}
\newcommand{\ve}{\boldsymbol{e}}
\newcommand{\vf}{\boldsymbol{f}}
\newcommand{\vg}{\boldsymbol{g}}
\newcommand{\vh}{\boldsymbol{h}}
\newcommand{\vi}{\boldsymbol{i}}
\newcommand{\vj}{\boldsymbol{j}}
\newcommand{\vk}{\boldsymbol{k}}
\newcommand{\vl}{\boldsymbol{l}}
\newcommand{\vm}{\boldsymbol{m}}
\newcommand{\vn}{\boldsymbol{n}}
\newcommand{\vo}{\boldsymbol{o}}
\newcommand{\vp}{\boldsymbol{p}}
\newcommand{\vq}{\boldsymbol{q}}
\newcommand{\vr}{\boldsymbol{r}}
\newcommand{\vs}{\boldsymbol{s}}
\newcommand{\vt}{\boldsymbol{t}}
\newcommand{\vvu}{\boldsymbol{u}}
\newcommand{\vv}{\boldsymbol{v}}
\newcommand{\vw}{\boldsymbol{w}}
\newcommand{\vx}{\boldsymbol{x}}
\newcommand{\vy}{\boldsymbol{y}}
\newcommand{\vz}{\boldsymbol{z}}
\newcommand{\vA}{\boldsymbol{A}}
\newcommand{\vB}{\boldsymbol{B}}
\newcommand{\vC}{\boldsymbol{C}}
\newcommand{\vD}{\boldsymbol{D}}
\newcommand{\vE}{\boldsymbol{E}}
\newcommand{\vF}{\boldsymbol{F}}
\newcommand{\vG}{\boldsymbol{G}}
\newcommand{\vH}{\boldsymbol{H}}
\newcommand{\vI}{\boldsymbol{I}}
\newcommand{\vJ}{\boldsymbol{J}}
\newcommand{\vK}{\boldsymbol{K}}
\newcommand{\vL}{\boldsymbol{L}}
\newcommand{\vM}{\boldsymbol{M}}
\newcommand{\vN}{\boldsymbol{N}}
\newcommand{\vO}{\boldsymbol{O}}
\newcommand{\vP}{\boldsymbol{P}}
\newcommand{\vQ}{\boldsymbol{Q}}
\newcommand{\vR}{\boldsymbol{R}}
\newcommand{\vS}{\boldsymbol{S}}
\newcommand{\vT}{\boldsymbol{T}}
\newcommand{\vU}{\boldsymbol{U}}
\newcommand{\vV}{\boldsymbol{V}}
\newcommand{\vW}{\boldsymbol{W}}
\newcommand{\vX}{\boldsymbol{X}}
\newcommand{\vY}{\boldsymbol{Y}}
\newcommand{\vZ}{\boldsymbol{Z}}
\newcommand{\vzero}{\boldsymbol{0}}
\newcommand{\vomega}{\boldsymbol{\omega}}
%\newcommand{\half}{\frac{1}{2}}
\newcommand{\thalf}{\frac{3}{2}}
\newcommand{\rot}{\nabla\times}
\newcommand{\divg}{\nabla\cdot}
\newcommand{\cE}{\mathcal{E}}
\newcommand{\conclu}[1]{\vspace{1em}\par\noindent\fbox{\parbox{0.9\textwidth}{#1}}\vspace{1em}}
\newcommand{\subentrynote}{$\bullet$}
\newcommand{\keypoint}[1]{\par\subentrynote\quad #1 \par}
\newcommand{\fconclu}{\boxed}
\newcommand{\pair}[2]{#1 \, #2}
\newcommand{\intn}[2]{\int #1 \,\mathrm{d} #2}
\newcommand{\intu}[3]{\int_0^{#1} #2 \,\mathrm{d} #3}
\newcommand{\intiu}[3]{\int_{-\infty}^{#1} #2 \, \rd{} #3}
\newcommand{\intui}[2]{\int_0^{\infty} #1 \,\mathrm{d} #2}
\newcommand{\intii}[2]{\int_{-\infty}^{\infty} #1 \,\mathrm{d} #2}
\newcommand{\intt}[2]{\int_0^\infty #1 \, \rd{} #2}
\newcommand{\intr}[2]{\int_{-\infty}^{\infty} #1 \, \rd{} #2}
\newcommand{\intbi}[3]{\int_{#1}^{\infty} #2 \, \rd{} #3}
\newcommand{\intab}[4]{\int_{#1}^{#2} #3 \, \rd{} #4}
\newcommand{\bfactor}[1]{e^{-#1/k_BT}}
\newcommand{\pbfactor}[1]{e^{#1/k_BT}}
\newcommand{\dn}[2]{#1^{\pare{#2}}}
\newcommand{\prodg}[1]{\pare{#1}^\times}

\newcommand{\notion}{\emph}
\newcommand{\iP}{\mathcal{P}}
\newcommand{\eiP}{e^{-\iP}}
\newcommand{\iF}{\mathcal{F}}
\newcommand{\eiF}{e^{-\iF}}
\newcommand{\iG}{\mathcal{G}}

\newcommand{\rc}{r\cos\theta}
\newcommand{\rs}{r\sin\theta}
\newcommand{\sn}{\mathrm{sn}}
\newcommand{\cn}{\mathrm{cn}}
\newcommand{\rdn}{\mathrm{dn}}

\newcommand{\hankel}{H_p^{\pare{1}}}
\newcommand{\hankell}{H_p^{\pare{2}}}
\newcommand{\hhankel}{H_n^{\pare{1}}}
\newcommand{\hhankell}{H_n^{\pare{2}}}
\newcommand{\ber}{\text{ber}\,}
\newcommand{\bei}{\text{bei}\,}
\newcommand{\kker}{\text{ker}\,}
\newcommand{\kei}{\text{kei}\,}
\newcommand{\Ai}{\text{Ai}}
\newcommand{\Bi}{\text{Bi}}

\newcommand{\re}{\text{Re}\,}

\newcommand{\Fp}{F_\phi}
\newcommand{\Ep}{E_\phi}
\newcommand{\Fx}{F_x}
\newcommand{\FF}{\mathbf{F}}
\newcommand{\Ex}{E_x}

%\newcommand{\erf}{\mathrm{erf}}
\newcommand{\erfi}{\mathrm{erfi}}
\newcommand{\erfc}{\mathrm{erfc}}
\newcommand{\ehxs}[1]{e^{-\frac{#1^2}{2}}}
\newcommand{\dcol}[2]{\[ \left.#1 \hspace{1em}\right\vert\hspace{1em} #2 \]}
\newcommand{\titlegamma}{\texorpdfstring{$\Gamma$}{Gamma}}
\newcommand{\titleB}{\texorpdfstring{$B$}{B}}

% Computer Science Head
\usepackage{listings}
\usepackage{color}

\definecolor{dkgreen}{rgb}{0,0.6,0}
\definecolor{gray}{rgb}{0.5,0.5,0.5}
\definecolor{mauve}{rgb}{0.58,0,0.82}

\lstset{frame=tb,
  language=Java,
  aboveskip=3mm,
  belowskip=3mm,
  showstringspaces=false,
  columns=flexible,
  basicstyle={\small\ttfamily},
  numbers=none,
  numberstyle=\tiny\color{gray},
  keywordstyle=\color{blue},
  commentstyle=\color{dkgreen},
  stringstyle=\color{mauve},
  breaklines=true,
  breakatwhitespace=true,
  tabsize=3
}
\lstset{language=Java}
\newcommand{\snp}[1]{\lstinline!#1!}
\newcommand{\term}[2]{\textbf{#1(#2)}}
\begin{document}
\fi

%Content

\chapter{普通线性代数}
  \section{算子作为函数的像}
  下一定理表明,与一可逆算子足够接近的算子仍然可逆。
  \begin{theorem}
    \label{thm:opr-0}
    设$A$为可逆算子,$\norm{B-A}<1/\norm{A^{-1}}$,则$B$可逆。
  \end{theorem}
  \begin{theorem}
    \label{thm:opr-1}
    线性算子的求逆是连续映射。
  \end{theorem}

%ContentEnds
 
\ifx\allfiles\undefined %如果位置放错,可能出现意外中断
\end{document}
\fi
  %Ch5.GeneralFunctionTheory.tex
\ifx\allfiles\undefined
\documentclass{ctexrep}
% Mathematics Include

\usepackage{amsmath}
\usepackage{amssymb}
\usepackage{amsthm}
\usepackage{amsfonts}
\usepackage{mathrsfs}
\usepackage{enumitem}
\usepackage{braket}
\usepackage{hyperref}
\usepackage[all, pdf]{xy}

% Physics Include
\usepackage{amsmath}
\usepackage{physics}
\usepackage{siunitx}
\usepackage[makeroom]{cancel}
\usepackage{pstricks}
\usepackage{pstricks-add}
\psset{algebraic=true}

\usepackage[version=4]{mhchem}
\usepackage{array,booktabs}
\usepackage{longtable}
\usepackage{mathtools}
\usepackage[normalem]{ulem}
\usepackage{multicol}

% Mathematics Head

\newcommand{\pare}[1]{\left(#1\right)}
\newcommand{\blr}[1]{\left[#1\right)}
\newcommand{\lbr}[1]{\left(#1\right]}
\newcommand{\brac}[1]{\left[#1\right]}
\newcommand{\curb}[1]{\left\{#1\right\}}
% \newcommand{\abs}[1]{\left|\, #1 \,\right|}
\newcommand{\rec}[1]{\frac{1}{#1}}
\newcommand{\N}{\mathbb{N}}
\newcommand{\bC}{\mathbb{C}}
\newcommand{\Q}{\mathbb{Q}}
\newcommand{\Z}{\mathbb{Z}}
\newcommand{\R}{\mathbb{R}}
\newcommand{\unk}{\mathcal{X}}
\newcommand{\bu}[3]{#1_{#2}^{\pare{#3}}}
\newcommand{\dref}[1]{定义\ref{def:#1}}
\newcommand{\tref}[1]{定理\ref{thm:#1}}
\newcommand{\lref}[1]{引理\ref{lem:#1}}
\newcommand{\cref}[1]{推论\ref{coll:#1}}
\newcommand{\pref}[1]{命题\ref{prp:#1}}
\newcommand{\eref}[1]{例\ref{ex:#1}}
\newcommand{\func}[3]{#1:\, #2 \rightarrow #3}
\newcommand{\overbar}[1]{\mkern 1.5mu\overline{\mkern-1.5mu#1\mkern-1.5mu}\mkern 1.5mu}
\newcommand{\clo}[1]{\overbar{#1}}
\newcommand{\supi}[2]{\overbar{\int_{#1}^{#2}}}
\newcommand{\infi}[2]{\underbar{\int_{#1}^{#2}}}
\newcommand{\setf}{\mathscr}
\newcommand{\bool}{\mathrm{bool}}
\newcommand{\inc}{++}
\newcommand{\defeq}{:=}
\newcommand{\ntuple}{$n$元组}
\newcommand{\card}[1]{\#\pare{#1}}
\newcommand{\setcond}[2]{\curb{#1 \, \left| \, #2 \right.}}
\newcommand{\setcondl}[2]{\curb{\left. #1 \, \right| \, #2}}
\newcommand{\bv}[1]{\mathbf{#1}}
\newcommand{\bfa}{\bv{a}}
\newcommand{\bfb}{\bv{b}}
\newcommand{\bfx}{\bv{x}}
\newcommand{\bfy}{\bv{y}}
\newcommand{\bfe}{\bv{e}}
\newcommand{\bfF}{\bv{F}}
\newcommand{\bff}{\bv{f}}
\newcommand{\bfG}{\bv{G}}
\newcommand{\bfH}{\bv{H}}
\newcommand{\bfg}{\bv{g}}
\newcommand{\bfh}{\bv{h}}
\newcommand{\bfr}{\bv{r}}
\newcommand{\bfk}{\bv{k}}
\newcommand{\bfu}{\bv{u}}
\newcommand{\bfv}{\bv{v}}
\newcommand{\oo}[1]{o\pare{#1}}
\newcommand{\OO}[1]{O\pare{#1}}
% \newcommand{\norm}[1]{\left\| #1 \right\|}
\newcommand{\DD}{\mathbf{D}}
\newcommand{\comp}{\circ}
\newcommand{\const}{\mathrm{const}}
\newcommand{\dist}[2]{d\pare{#1,#2}}
\newcommand{\len}{\ell}
\newcommand{\siga}{$\sigma$-代数}
\newcommand{\cara}{Carath\'{e}odory}
\newcommand{\Gd}{G_\delta}
\newcommand{\Fs}{F_\sigma}
\newcommand{\mmani}{$m$-维流形}
\newcommand{\open}[1]{\mathcal{#1}}
\newcommand{\half}{\frac{1}{2}}
\newcommand{\maxo}[1]{\text{max}\curb{#1}}
\newcommand{\mino}[1]{\text{min}\curb{#1}}
\newcommand{\epsclo}{$\epsilon$-接近}
\newcommand{\close}[1]{$#1$-接近}
\newcommand{\cinf}{$C^\infty$}
\newcommand{\cuno}{$C^1$}
\newcommand{\Int}{\text{Int}\,}
\newcommand{\Ext}{\text{Ext}\,}
\newcommand{\funcf}{\mathcal}
\newcommand{\DDu}{\overbar{\DD}}
\newcommand{\DDl}{\underbar{\DD}}
\newcommand{\Diff}[1]{\mathrm{Diff}_{#1}\,}
\newcommand{\Av}[1]{\mathrm{Av}_{#1}\,}
\newcommand{\Lip}[1]{Lipschitz-$#1$}
\newcommand{\sgn}{\mathrm{sgn}}
\newcommand{\eset}{\varnothing}
\newcommand{\cT}{\mathcal{T}}
\newcommand{\cS}{\mathcal{S}}
\newcommand{\cG}{\mathcal{G}}
\newcommand{\cF}{\mathcal{F}}
\newcommand{\cC}{\mathcal{C}}
\newcommand{\cB}{\mathcal{B}}
\newcommand{\inter}[1]{\mathring{#1}}
\newcommand{\forest}[3]{对于{#1},存在{#2},使得{#3}}
\newcommand{\tuno}{$T_1$公理}
\newcommand{\isom}{\overset{\sim}{=}}
\newcommand{\diam}{\mathrm{diam}\,}
\newcommand{\ord}[1]{\abs{#1}}
\newcommand{\sbm}[1]{\overbar{#1}}
\newcommand{\inv}[1]{#1^{-1}}
\newcommand{\restr}[2]{#1|_{#2}}
\newcommand{\divs}{|}
\newcommand{\ndivs}{\nmid}
\newcommand{\modeq}[1]{\overbar{#1}}
\newcommand{\ggen}[1]{\langle#1\rangle}
\newcommand{\ggencond}{\braket}

\newcommand{\hd}{H\"{o}lder}

\renewcommand{\proofname}{证明}

\newenvironment{cenum}{\begin{enumerate}\itemsep0em}{\end{enumerate}}

\newtheorem{definition}{定义}[section]
\newtheorem{lemma}{引理}[section]
\newtheorem{theorem}{定理}[section]
\newtheorem{collary}{推论}[section]
\newtheorem{corollary}{推论}[section]
\newtheorem{proposition}{命题}[section]
\newtheorem{axiom}{公理}[section]
\newtheorem{ex}{例}[section]
\newtheorem{reflection}{反射}[section]
\newcommand{\refl}[1]{\vspace{0.5em}\par\noindent\fbox{%
    \parbox{0.9\textwidth}{%
    \begin{reflection}
        #1
    \end{reflection}
    }%
}\vspace{0.5em}\par}
\newcommand{\rref}[1]{反射\ref{refl:#1}}
\newcommand{\tbref}[1]{表\ref{table:#1}}
\allowdisplaybreaks

\newenvironment{aenum}{\begin{enumerate}[label=\textnormal{(\alph*)}]}{\end{enumerate}}

% Physics Head

\DeclareSIUnit\dyne{dynes}

\newcommand{\ddel}[1]{\frac{\partial}{\partial #1}}
\newcommand{\ddelon}[2]{\frac{\partial #1}{\partial #2}}
\newcommand{\dddel}[1]{\frac{\partial^2}{\partial^2 #1}}
\newcommand{\ddt}{\ddel{t}}
\newcommand{\ddT}{\ddel{T}}
\newcommand{\ddV}{\ddel{V}}
\newcommand{\ddr}{\ddel{r}}
\newcommand{\ddth}{\ddel{\theta}}
\newcommand{\ddph}{\ddel{\phi}}
\newcommand{\dddt}{\dddel{t}}
\newcommand{\dddr}{\dddel{t}}
\newcommand{\dddth}{\dddel{\theta}}
\newcommand{\dddph}{\dddel{\phi}}
\newcommand{\rd}[1]{\mathrm{d} #1}
\newcommand{\dt}{\rd{t}}
\newcommand{\dy}{\rd{y}}
\newcommand{\dx}{\rd{x}}
\newcommand{\edd}[1]{\frac{\mathrm{d}}{\mathrm{d} #1}}
\newcommand{\eddd}[1]{\frac{\mathrm{d}^2}{\mathrm{d}^2 #1}}
\newcommand{\eddon}[2]{\frac{\mathrm{d} #1}{\mathrm{d} #2}}
\newcommand{\edddon}[2]{\frac{\mathrm{d}^2 #1}{\mathrm{d}^2 #2}}
\newcommand{\edt}{\edd{t}}
\newcommand{\edton}[1]{\eddon{#1}{t}}
\newcommand{\edT}{\edd{T}}
\newcommand{\edr}{\edd{r}}
\newcommand{\edl}{\edd{l}}
\newcommand{\edx}{\edd{x}}
\newcommand{\edth}{\edd{\theta}}
\newcommand{\eddton}[1]{\edddon{#1}{t}}
\newcommand{\eddzon}[1]{\edddon{#1}{z}}
\newcommand{\vect}[1]{\boldsymbol{#1}}
\newcommand{\alp}{\frac{1}{\sqrt{2}}}
\newcommand{\alpi}{\frac{i}{\sqrt{2}}}
\newcommand{\expc}[1]{\langle#1\rangle}
\newcommand{\bkn}[1]{\bra{#1}\ket{#1}}
\newcommand{\bk}[2]{\bra{#1}\ket{#2}}
\newcommand{\bik}[3]{\bra{#1} #2 \ket{#3}}
\newcommand{\vari}[1]{\sigma_{#1}}
\newcommand{\intc}[2]{\left[#1, #2\right]}
\newcommand{\sch}{Schr\"{o}dinger}
\newcommand{\moment}{\boldsymbol{p}}
\newcommand{\coor}{\boldsymbol{x}}
\newcommand{\lapc}{\nabla^2}
% \newcommand{\rec}[1]{\frac{1}{#1}}
\newcommand{\vva}{\boldsymbol{a}}
\newcommand{\vvb}{\boldsymbol{b}}
\newcommand{\vc}{\boldsymbol{c}}
\newcommand{\vd}{\boldsymbol{d}}
\newcommand{\ve}{\boldsymbol{e}}
\newcommand{\vf}{\boldsymbol{f}}
\newcommand{\vg}{\boldsymbol{g}}
\newcommand{\vh}{\boldsymbol{h}}
\newcommand{\vi}{\boldsymbol{i}}
\newcommand{\vj}{\boldsymbol{j}}
\newcommand{\vk}{\boldsymbol{k}}
\newcommand{\vl}{\boldsymbol{l}}
\newcommand{\vm}{\boldsymbol{m}}
\newcommand{\vn}{\boldsymbol{n}}
\newcommand{\vo}{\boldsymbol{o}}
\newcommand{\vp}{\boldsymbol{p}}
\newcommand{\vq}{\boldsymbol{q}}
\newcommand{\vr}{\boldsymbol{r}}
\newcommand{\vs}{\boldsymbol{s}}
\newcommand{\vt}{\boldsymbol{t}}
\newcommand{\vvu}{\boldsymbol{u}}
\newcommand{\vv}{\boldsymbol{v}}
\newcommand{\vw}{\boldsymbol{w}}
\newcommand{\vx}{\boldsymbol{x}}
\newcommand{\vy}{\boldsymbol{y}}
\newcommand{\vz}{\boldsymbol{z}}
\newcommand{\vA}{\boldsymbol{A}}
\newcommand{\vB}{\boldsymbol{B}}
\newcommand{\vC}{\boldsymbol{C}}
\newcommand{\vD}{\boldsymbol{D}}
\newcommand{\vE}{\boldsymbol{E}}
\newcommand{\vF}{\boldsymbol{F}}
\newcommand{\vG}{\boldsymbol{G}}
\newcommand{\vH}{\boldsymbol{H}}
\newcommand{\vI}{\boldsymbol{I}}
\newcommand{\vJ}{\boldsymbol{J}}
\newcommand{\vK}{\boldsymbol{K}}
\newcommand{\vL}{\boldsymbol{L}}
\newcommand{\vM}{\boldsymbol{M}}
\newcommand{\vN}{\boldsymbol{N}}
\newcommand{\vO}{\boldsymbol{O}}
\newcommand{\vP}{\boldsymbol{P}}
\newcommand{\vQ}{\boldsymbol{Q}}
\newcommand{\vR}{\boldsymbol{R}}
\newcommand{\vS}{\boldsymbol{S}}
\newcommand{\vT}{\boldsymbol{T}}
\newcommand{\vU}{\boldsymbol{U}}
\newcommand{\vV}{\boldsymbol{V}}
\newcommand{\vW}{\boldsymbol{W}}
\newcommand{\vX}{\boldsymbol{X}}
\newcommand{\vY}{\boldsymbol{Y}}
\newcommand{\vZ}{\boldsymbol{Z}}
\newcommand{\vzero}{\boldsymbol{0}}
\newcommand{\vomega}{\boldsymbol{\omega}}
%\newcommand{\half}{\frac{1}{2}}
\newcommand{\thalf}{\frac{3}{2}}
\newcommand{\rot}{\nabla\times}
\newcommand{\divg}{\nabla\cdot}
\newcommand{\cE}{\mathcal{E}}
\newcommand{\conclu}[1]{\vspace{1em}\par\noindent\fbox{\parbox{0.9\textwidth}{#1}}\vspace{1em}}
\newcommand{\subentrynote}{$\bullet$}
\newcommand{\keypoint}[1]{\par\subentrynote\quad #1 \par}
\newcommand{\fconclu}{\boxed}
\newcommand{\pair}[2]{#1 \, #2}
\newcommand{\intn}[2]{\int #1 \,\mathrm{d} #2}
\newcommand{\intu}[3]{\int_0^{#1} #2 \,\mathrm{d} #3}
\newcommand{\intiu}[3]{\int_{-\infty}^{#1} #2 \, \rd{} #3}
\newcommand{\intui}[2]{\int_0^{\infty} #1 \,\mathrm{d} #2}
\newcommand{\intii}[2]{\int_{-\infty}^{\infty} #1 \,\mathrm{d} #2}
\newcommand{\intt}[2]{\int_0^\infty #1 \, \rd{} #2}
\newcommand{\intr}[2]{\int_{-\infty}^{\infty} #1 \, \rd{} #2}
\newcommand{\intbi}[3]{\int_{#1}^{\infty} #2 \, \rd{} #3}
\newcommand{\intab}[4]{\int_{#1}^{#2} #3 \, \rd{} #4}
\newcommand{\bfactor}[1]{e^{-#1/k_BT}}
\newcommand{\pbfactor}[1]{e^{#1/k_BT}}
\newcommand{\dn}[2]{#1^{\pare{#2}}}
\newcommand{\prodg}[1]{\pare{#1}^\times}

\newcommand{\notion}{\emph}
\newcommand{\iP}{\mathcal{P}}
\newcommand{\eiP}{e^{-\iP}}
\newcommand{\iF}{\mathcal{F}}
\newcommand{\eiF}{e^{-\iF}}
\newcommand{\iG}{\mathcal{G}}

\newcommand{\rc}{r\cos\theta}
\newcommand{\rs}{r\sin\theta}
\newcommand{\sn}{\mathrm{sn}}
\newcommand{\cn}{\mathrm{cn}}
\newcommand{\rdn}{\mathrm{dn}}

\newcommand{\hankel}{H_p^{\pare{1}}}
\newcommand{\hankell}{H_p^{\pare{2}}}
\newcommand{\hhankel}{H_n^{\pare{1}}}
\newcommand{\hhankell}{H_n^{\pare{2}}}
\newcommand{\ber}{\text{ber}\,}
\newcommand{\bei}{\text{bei}\,}
\newcommand{\kker}{\text{ker}\,}
\newcommand{\kei}{\text{kei}\,}
\newcommand{\Ai}{\text{Ai}}
\newcommand{\Bi}{\text{Bi}}

\newcommand{\re}{\text{Re}\,}

\newcommand{\Fp}{F_\phi}
\newcommand{\Ep}{E_\phi}
\newcommand{\Fx}{F_x}
\newcommand{\FF}{\mathbf{F}}
\newcommand{\Ex}{E_x}

%\newcommand{\erf}{\mathrm{erf}}
\newcommand{\erfi}{\mathrm{erfi}}
\newcommand{\erfc}{\mathrm{erfc}}
\newcommand{\ehxs}[1]{e^{-\frac{#1^2}{2}}}
\newcommand{\dcol}[2]{\[ \left.#1 \hspace{1em}\right\vert\hspace{1em} #2 \]}
\newcommand{\titlegamma}{\texorpdfstring{$\Gamma$}{Gamma}}
\newcommand{\titleB}{\texorpdfstring{$B$}{B}}

% Computer Science Head
\usepackage{listings}
\usepackage{color}

\definecolor{dkgreen}{rgb}{0,0.6,0}
\definecolor{gray}{rgb}{0.5,0.5,0.5}
\definecolor{mauve}{rgb}{0.58,0,0.82}

\lstset{frame=tb,
  language=Java,
  aboveskip=3mm,
  belowskip=3mm,
  showstringspaces=false,
  columns=flexible,
  basicstyle={\small\ttfamily},
  numbers=none,
  numberstyle=\tiny\color{gray},
  keywordstyle=\color{blue},
  commentstyle=\color{dkgreen},
  stringstyle=\color{mauve},
  breaklines=true,
  breakatwhitespace=true,
  tabsize=3
}
\lstset{language=Java}
\newcommand{\snp}[1]{\lstinline!#1!}
\newcommand{\term}[2]{\textbf{#1(#2)}}
\begin{document}
\fi

%Content

\chapter{普通函数理论}
  \section{函数序列}
  \section{特殊函数}
  \section{多元函数}
  \subsection{微分法}
  \begin{definition}
    设$E$是$\R^n$中的开集,$\bv{f}$将$E$映入$\R^m$,$\bv{x}\in E$。若存在将$\R^n$映入$\R^m$的线性变换$A$,使得
    \[ \lim_{\bv{h}\to 0} \frac{\abs{\bv{f}\pare{\bv{x}+\bv{h}}}-\bv{f}\pare{\bv{x}}-A\bv{h}}{\abs{\bv{h}}} = 0, \]
    则称$\bv{f}$在$\bv{x}$处可微。记作
    \[ \bv{f}'\pare{x}=A. \]
  \end{definition}
  \begin{theorem}
    上述定义的$A$是唯一的。
  \end{theorem}
  \begin{proof}
    设$B=A_1-A_2$,则
    \[ \abs{B\bfh}\le\abs{\bff\pare{\bfx+\bfh}-\bff\pare{\bfx}-A_1\bfh}+\abs{\bff\pare{\bfx+\bfh}-\bff\pare{\bfx}-A_2\bfh}. \]
    故$\abs{B\bfh}/\abs{\bfh}\to 0$。因此$B=0$。
  \end{proof}
  例如线性变换的导数显然是其自身。
  \begin{theorem}
    若$\bfF\pare{\bfx} = \bfg\pare{\bff\pare{x}}$,且$\bff$与$\bfg$在对应点可微,则
    \[ \bfF'\pare{\bfx_0} = \bfg'\pare{\bff\pare{\bfx_0}}\bff'\pare{\bfx_0}. \]
  \end{theorem}
  \begin{proof}
    设$\bfx_0$的增量为$\bfh$,$\bff\pare{\bfx_0}$的增量为$\bfk$。
    \begin{align*}
      \bfF\pare{\bfx_0+\bfh}-\bfF\pare{\bfx_0}-BA\bfh &= \bfg\pare{\bfy_0+\bfk} - \bfg\pare{\bfy_0}-BA\bfh \\
      &= B\pare{\bfk-A\bfh}+\oo{\bfk} \\
      &= B\oo{\bfh} + \oo{\bfk}.
    \end{align*}
    再注意$\bfk = A \bfh + \oo{\bfh}$便知其趋向零。
  \end{proof}
  \begin{definition}
    设$E$是$\R^n$中的开集,$\bv{f}$将$E$映入$\R^m$。$\curb{e_n}$是定义域的标准基,$\curb{u_m}$是值域的标准基,以此定义$\bff$的分量。若极限
    \[ \pare{\DD_jf_i}\pare{\bfx} = \lim_{t\to 0}\frac{f_i\pare{\bfx+t\bfe_j}-f_i\pare{\bfx}}{t} \]
    存在,则称之为偏导数。
  \end{definition}
  \begin{theorem}
    \label{thm:adij}
    若$\bff$在$\bfx$处可微,则各偏导数存在,且
    \[ A_{ij} = \DD_jf_i. \]
  \end{theorem}
  \begin{proof}
    注意到方向导数的存在为可导所蕴含即可。
  \end{proof}
  注意$ij$的顺序,曲线的$A$为列矩阵,即切向量。
  \begin{definition}若$f$是$\R^n$上的标量函数,则$A$为行矩阵,称为$f$的梯度,记作$\nabla f$。
  \end{definition}
  设$\gamma$为一曲线,$f$标量函数,$g=f \comp \gamma$,于是$g'\pare{t} = \pare{\nabla f}\gamma'$。若$\gamma$为指向$\bfu$的直线,则$g'\pare{t}=\bfu\cdot\nabla f$,称为$f$的方向导数。当$\bfu$与$\nabla f$同向时\footnote{此处将行向量与列向量等同,或者说协变矢量与逆变矢量等同。},其具有最大值。
  \begin{theorem}
    若在凸集上$\norm{\bff'}\le M$,则$\abs{\bff\pare{\bfb}-\bff\pare{\bfa}} \le M \abs{\bfb-\bfa}$。
  \end{theorem}
  \begin{proof}
    设$\gamma\pare{t} = \pare{1-t}\bfa+t\bfb$,$\bfg=\bff\comp\gamma$,因此$\bfg'=\bff'\comp\pare{\bfb-\bfa}$,再注意$\bff'$的上界并调用\tref{vfab}。
  \end{proof}
  \begin{collary}
    若$\bff'=0$,则$\bff=\const$。
  \end{collary}
  \begin{definition}
    若$\bff$可微且$\bff'$连续,则称$\bff$连续可微。
  \end{definition}
  \begin{theorem}
    当且仅当$\func{\bff}{E\subset\R^n}{\R^m}$偏导数存在且连续时,$\bff$连续可微。
  \end{theorem}
  \begin{proof}
    若$\bff$连续可微,则借助\tref{adij},并注意
    \[ \abs{\pare{\DD_jf_i}\pare{\bfy}-\pare{\DD_jf_i}\pare{\bfx}}\le\abs{\brac{\bff'\pare{\bfy}-\bff'\pare{\bfx}}\bfe_j} \]
    即可。
    \par
    若$\bff$在$\bfx$的各偏导数存在,则可以仅考虑$\bff$的一个分量$f$。将$\bfx$的无穷小位移$\bfh$分解为各个方向的和,借助中值定理将增量转化为"稍微偏离"$\bfx$处的偏导。再借助连续性,从而$\bff'$可以写为$\DD_jf$的列向量,且各分量由题设连续。
  \end{proof}
  \subsection{反函数定理}
  \begin{theorem}
    若连续可微的$\bff$将开集$E\subset \R^n$映入$\R^n$,且对于某个$\bfa$,$\bff'\pare{\bfa}$可逆,$\bff\pare{\bfa} = \bfb$。则存在$\bfa$和$\bfb$的邻域$U$和$V$,使$\bff$为双射。
  \end{theorem}
  \begin{proof}
    为了求出$\bff$的反函数,采用Newton法,令
    \[ \varphi_\bfy\pare{\bfx} = \bfx + A^{-1}\pare{\bfy-\bff\pare{\bfx}}. \]
    注意$\varphi'=A^{-1}\pare{A-\bff'}$,因此可以选择足够小的邻域使$\norm{\varphi'}$足够小。由\tref{fixp0},其存在不动点,故$\bff$可逆。
    \par
    $\bff$的一一性已证,下证其为开映射。设$\bff\pare{\bfx_0}=\bfy_0$确凿,对于足够接近$\bfy_0$的$\bfy$,从任何$\bfx_0$附近的$\bfx$出发试图寻找其原像,皆有
    \[ \abs{\varphi\pare{\bfx}-\bfx_0}\le\abs{\varphi\pare{\bfx}-\varphi\pare{\bfx_0}}+\abs{\varphi\pare{\bfx_0}-\bfx_0}.\]
    前一绝对值的大小由$\norm{\varphi'}$限制,后一绝对值的大小由$\bfy$的偏移量限制。因此迭代后的不动点仍在$\bfx$附近,故仍在$U$内,所以$\bfy$仍在$V$内。
  \end{proof}
  \begin{theorem}
    前开定理的局域反函数$\bfg$亦连续可微。
  \end{theorem}
  \begin{proof}
    由前证以及\tref{opr-0},可以选取足够小的邻域$U$使得$\bff'$在此邻域内与$\bff'\pare{\bfa}$足够接近故可逆。还可以使得在此邻域内,$\bfh = \OO{\bfk}$。注意
    \[ \bfg\pare{\bfy+\bfk}-\bfg\pare{\bfy}-T\bfk=-T\brac{\bff\pare{\bfx+\bfh}-\bff\pare{\bfx}-\bff'\pare{\bfx}\bfh}, \]
    其中$T=\bff'^{-1}$。因此余项$\bfv=\OO{\bfu}$。再注意到求逆是连续映射即可。
  \end{proof}
  \begin{theorem}
    若连续可微的$\bff$将开集$E\subset \R^n$映入$\R^n$,$\bff'$逐点可逆,则$\bff$为开映射。
  \end{theorem}
  \subsection{隐函数定理}
  \begin{definition}
    $\pare{\bfx,\bfy}$为
    \[ \pare{x_1,\cdots,x_n,y_1,\cdots,y_m} \in \R^{n+m}. \]
  \end{definition}
  每个线性变换$\func{A}{\R^{n+m}}{\R^n}$都可以分解成$A_x$和$A_y$两部分。
  \begin{theorem}
    对于上述$A$,若$A_x$可逆,则对于每个$\bfk\in\R^m$,有唯一的$\bfh\in\R^n$满足$A\pare{\bfh,\bfk}=0$。
  \end{theorem}
  \begin{proof}
    \[ \bfh = -\pare{A_x}^{-1}A_y\bfk. \qedhere \]
  \end{proof}
  \begin{theorem}
    设$\func{\bff}{\pare{X,Y}}{Z}$是开集$E\subset\R^{n+m}$到$\R^n$内的连续可微映射,且在某点$\pare{\bfa,\bfb}$为零。令$A=\bff'\pare{\bfa,\bfb}$且$A_x$可逆,则存在$\pare{\bfa,\bfb}$的邻域$U$和$\bfb$的邻域$W$,$W$内的任意$\bfy$有唯一$\bfx$使$\bff=0$。
  \end{theorem}
  \begin{proof}
    令$\bfF\pare{\bfx, \bfy}= \pare{\bff\pare{\bfx,\bfy},\bfy}$。因此
    \[ \bfF\pare{\bfx+\bfh, \bfy+\bfk}-\bfF\pare{\bfx, \bfy}=\pare{A\pare{\bfh,\bfk},\bfk}+\oo{\bfh,\bfk}. \]
    若右侧为零,则$\bfk=0$,$\bfh=0$。因此$\bfF'$可逆,故反函数定理可用于$\bfF$。因此,\emph{存在$\pare{0,\bfb}$的邻域$V$和$\pare{\bfa,\bfb}$的邻域$U$,使得$\bfF$是一一的}。在$V$中投影出$\pare{0,\bfy}$,注意其为开集即可。
  \end{proof}
  \begin{theorem}
    在前开命题中设$\bfx=\bfg\pare{\bfy}$,有$\bfg$为连续可微映射且$\bfg'\pare{\bfb}=-\pare{A_x}^{-1}A_y$。
  \end{theorem}
  \begin{proof}
    令$\bfG$为$\bfF$的局域反函数,则$\bfG$连续可微,故作为其限制的$\bfg$亦然。欲求$\bfg'\pare{\bfb}$,注意到
    \[ \bff\pare{\bfg\pare{\bfb},\bfb}'=0, \]
    即$A\pare{\bfg'\pare{\bfb},I}=0$,$A_x\bfg'\pare{\bfb}+A_y=0$。
  \end{proof}
  \subsection{秩定理}
  \begin{theorem}
  设$\func{\bfF}{E\subset\R^n}{\R^m}$将开集$E$映入$\R^m$,对于任意$\bfx$有$\bfF'$的秩为$r$。对于$\bfa\in E$,$A=\bfF'$的像空间为$Y_1$,$P$是到其上的投影,$Y_2$是$Y_1$的正交空间。
  \par
  存在$\bfa$的邻域$U$和$\R^n$中的开集$V$,存在连续可微双射$\func{\bfH}{V}{U}$满足
  \[ \bfF\pare{\bfH\pare{\bfx}} = A\bfx+\varphi\pare{A\bfx}, \]
  且$\bfH$的逆亦连续可微。式中$\varphi$映入$Y_2$。
  \end{theorem}
  \begin{proof}
    若$r=0$(这里并没有使用归纳法的打算),存在$\bfa$的邻域使得$\bfF$为常量且$Y_1$为零空间。故可以取$\bfH=I$,$V=U$,$\varphi\pare{0}=\bfF\pare{\bfa}$。
    \par
    对于$r>0$,设$S$为$A$在$Y_1$上的逆,定义
    \[ \func{\bfG}{E}{U\subset\R^n} = \bfx+SP\brac{\bfF\pare{\bfx}-A\bfx}. \]
    由$\bfG'=I$知存在邻域使其可逆,设其逆为$\bfH$。将$\bfG$经$A$映射,即
    \[ A\bfG = ASP\bfF = P\bfF. \]
    这是因为$ASPA=A$。如果$\bfx = \bfH\pare{\bfv}$,则有$P\bfF\comp\bfH=A$。因此,
    \[  \Phi\pare{\bfv}=\bfF\bfH\pare{\bfv}-A\bfv \]
    是到$Y_2$的连续可微映射。下尚需证存在连续可微的映射$\varphi\pare{A\bfv} = \Phi\pare{\bfv}$。
    \par
    先证$\Phi$的值仅仅取决于$A\bfv$。考虑两$\bfv$的差$\bfh$,只需证明当$A\bfh=0$,有$\pare{\bfF\bfH}'\bfh=0$。由构造,$\pare{\bfF\bfH}'\bfh$到$Y_1$上投影仍为$A\bfh$,且二算子像空间皆$r$维,知$A\bfh$可决定$\pare{\bfF\bfH}'$。当前者为零,后者亦然。
    \par
    再证$\varphi$连续可微。注意$\varphi$的定义域是诸$A\bfv$,故对于$\varphi\pare{\bfu}$,相应的$\bfv=S\bfu+\bfa$,其中$\bfa$是$A$的零空间中任意元素。对于任意$\bfu$,可以寻找其邻域使得存在$\bfa$让$\bfv$留在$V$内。于是$\varphi\pare{\bfu}=\Phi\pare{\bfa+S\bfu}$连续可微。
  \end{proof}
  注意$A\bfx$映射到$Y_1$,因此$\bfF$在此点的值仅仅取决于其射影,因而可将其视为$r$维曲面,而其水平集可以视为$X$中的$n-r$维曲面。例如将汤勺映为棒棒糖的映射,勺柄的二维平面被映射为糖棒的一维线段,其$\bfF'$的秩固然为$1$。而其水平集,即糖棒上一点对应的原像则是$2-1=1$维的垂直于汤勺柄的线段。

%ContentEnds
 
\ifx\allfiles\undefined %如果位置放错,可能出现意外中断
\end{document}
\fi
  %Ch6.PointSetTopology.tex
\ifx\allfiles\undefined
\documentclass{ctexrep}
% Mathematics Include

\usepackage{amsmath}
\usepackage{amssymb}
\usepackage{amsthm}
\usepackage{amsfonts}
\usepackage{mathrsfs}
\usepackage{enumitem}
\usepackage{braket}
\usepackage{hyperref}
\usepackage[all, pdf]{xy}

% Physics Include
\usepackage{amsmath}
\usepackage{physics}
\usepackage{siunitx}
\usepackage[makeroom]{cancel}
\usepackage{pstricks}
\usepackage{pstricks-add}
\psset{algebraic=true}

\usepackage[version=4]{mhchem}
\usepackage{array,booktabs}
\usepackage{longtable}
\usepackage{mathtools}
\usepackage[normalem]{ulem}
\usepackage{multicol}

% Mathematics Head

\newcommand{\pare}[1]{\left(#1\right)}
\newcommand{\blr}[1]{\left[#1\right)}
\newcommand{\lbr}[1]{\left(#1\right]}
\newcommand{\brac}[1]{\left[#1\right]}
\newcommand{\curb}[1]{\left\{#1\right\}}
% \newcommand{\abs}[1]{\left|\, #1 \,\right|}
\newcommand{\rec}[1]{\frac{1}{#1}}
\newcommand{\N}{\mathbb{N}}
\newcommand{\bC}{\mathbb{C}}
\newcommand{\Q}{\mathbb{Q}}
\newcommand{\Z}{\mathbb{Z}}
\newcommand{\R}{\mathbb{R}}
\newcommand{\unk}{\mathcal{X}}
\newcommand{\bu}[3]{#1_{#2}^{\pare{#3}}}
\newcommand{\dref}[1]{定义\ref{def:#1}}
\newcommand{\tref}[1]{定理\ref{thm:#1}}
\newcommand{\lref}[1]{引理\ref{lem:#1}}
\newcommand{\cref}[1]{推论\ref{coll:#1}}
\newcommand{\pref}[1]{命题\ref{prp:#1}}
\newcommand{\eref}[1]{例\ref{ex:#1}}
\newcommand{\func}[3]{#1:\, #2 \rightarrow #3}
\newcommand{\overbar}[1]{\mkern 1.5mu\overline{\mkern-1.5mu#1\mkern-1.5mu}\mkern 1.5mu}
\newcommand{\clo}[1]{\overbar{#1}}
\newcommand{\supi}[2]{\overbar{\int_{#1}^{#2}}}
\newcommand{\infi}[2]{\underbar{\int_{#1}^{#2}}}
\newcommand{\setf}{\mathscr}
\newcommand{\bool}{\mathrm{bool}}
\newcommand{\inc}{++}
\newcommand{\defeq}{:=}
\newcommand{\ntuple}{$n$元组}
\newcommand{\card}[1]{\#\pare{#1}}
\newcommand{\setcond}[2]{\curb{#1 \, \left| \, #2 \right.}}
\newcommand{\setcondl}[2]{\curb{\left. #1 \, \right| \, #2}}
\newcommand{\bv}[1]{\mathbf{#1}}
\newcommand{\bfa}{\bv{a}}
\newcommand{\bfb}{\bv{b}}
\newcommand{\bfx}{\bv{x}}
\newcommand{\bfy}{\bv{y}}
\newcommand{\bfe}{\bv{e}}
\newcommand{\bfF}{\bv{F}}
\newcommand{\bff}{\bv{f}}
\newcommand{\bfG}{\bv{G}}
\newcommand{\bfH}{\bv{H}}
\newcommand{\bfg}{\bv{g}}
\newcommand{\bfh}{\bv{h}}
\newcommand{\bfr}{\bv{r}}
\newcommand{\bfk}{\bv{k}}
\newcommand{\bfu}{\bv{u}}
\newcommand{\bfv}{\bv{v}}
\newcommand{\oo}[1]{o\pare{#1}}
\newcommand{\OO}[1]{O\pare{#1}}
% \newcommand{\norm}[1]{\left\| #1 \right\|}
\newcommand{\DD}{\mathbf{D}}
\newcommand{\comp}{\circ}
\newcommand{\const}{\mathrm{const}}
\newcommand{\dist}[2]{d\pare{#1,#2}}
\newcommand{\len}{\ell}
\newcommand{\siga}{$\sigma$-代数}
\newcommand{\cara}{Carath\'{e}odory}
\newcommand{\Gd}{G_\delta}
\newcommand{\Fs}{F_\sigma}
\newcommand{\mmani}{$m$-维流形}
\newcommand{\open}[1]{\mathcal{#1}}
\newcommand{\half}{\frac{1}{2}}
\newcommand{\maxo}[1]{\text{max}\curb{#1}}
\newcommand{\mino}[1]{\text{min}\curb{#1}}
\newcommand{\epsclo}{$\epsilon$-接近}
\newcommand{\close}[1]{$#1$-接近}
\newcommand{\cinf}{$C^\infty$}
\newcommand{\cuno}{$C^1$}
\newcommand{\Int}{\text{Int}\,}
\newcommand{\Ext}{\text{Ext}\,}
\newcommand{\funcf}{\mathcal}
\newcommand{\DDu}{\overbar{\DD}}
\newcommand{\DDl}{\underbar{\DD}}
\newcommand{\Diff}[1]{\mathrm{Diff}_{#1}\,}
\newcommand{\Av}[1]{\mathrm{Av}_{#1}\,}
\newcommand{\Lip}[1]{Lipschitz-$#1$}
\newcommand{\sgn}{\mathrm{sgn}}
\newcommand{\eset}{\varnothing}
\newcommand{\cT}{\mathcal{T}}
\newcommand{\cS}{\mathcal{S}}
\newcommand{\cG}{\mathcal{G}}
\newcommand{\cF}{\mathcal{F}}
\newcommand{\cC}{\mathcal{C}}
\newcommand{\cB}{\mathcal{B}}
\newcommand{\inter}[1]{\mathring{#1}}
\newcommand{\forest}[3]{对于{#1},存在{#2},使得{#3}}
\newcommand{\tuno}{$T_1$公理}
\newcommand{\isom}{\overset{\sim}{=}}
\newcommand{\diam}{\mathrm{diam}\,}
\newcommand{\ord}[1]{\abs{#1}}
\newcommand{\sbm}[1]{\overbar{#1}}
\newcommand{\inv}[1]{#1^{-1}}
\newcommand{\restr}[2]{#1|_{#2}}
\newcommand{\divs}{|}
\newcommand{\ndivs}{\nmid}
\newcommand{\modeq}[1]{\overbar{#1}}
\newcommand{\ggen}[1]{\langle#1\rangle}
\newcommand{\ggencond}{\braket}

\newcommand{\hd}{H\"{o}lder}

\renewcommand{\proofname}{证明}

\newenvironment{cenum}{\begin{enumerate}\itemsep0em}{\end{enumerate}}

\newtheorem{definition}{定义}[section]
\newtheorem{lemma}{引理}[section]
\newtheorem{theorem}{定理}[section]
\newtheorem{collary}{推论}[section]
\newtheorem{corollary}{推论}[section]
\newtheorem{proposition}{命题}[section]
\newtheorem{axiom}{公理}[section]
\newtheorem{ex}{例}[section]
\newtheorem{reflection}{反射}[section]
\newcommand{\refl}[1]{\vspace{0.5em}\par\noindent\fbox{%
    \parbox{0.9\textwidth}{%
    \begin{reflection}
        #1
    \end{reflection}
    }%
}\vspace{0.5em}\par}
\newcommand{\rref}[1]{反射\ref{refl:#1}}
\newcommand{\tbref}[1]{表\ref{table:#1}}
\allowdisplaybreaks

\newenvironment{aenum}{\begin{enumerate}[label=\textnormal{(\alph*)}]}{\end{enumerate}}

% Physics Head

\DeclareSIUnit\dyne{dynes}

\newcommand{\ddel}[1]{\frac{\partial}{\partial #1}}
\newcommand{\ddelon}[2]{\frac{\partial #1}{\partial #2}}
\newcommand{\dddel}[1]{\frac{\partial^2}{\partial^2 #1}}
\newcommand{\ddt}{\ddel{t}}
\newcommand{\ddT}{\ddel{T}}
\newcommand{\ddV}{\ddel{V}}
\newcommand{\ddr}{\ddel{r}}
\newcommand{\ddth}{\ddel{\theta}}
\newcommand{\ddph}{\ddel{\phi}}
\newcommand{\dddt}{\dddel{t}}
\newcommand{\dddr}{\dddel{t}}
\newcommand{\dddth}{\dddel{\theta}}
\newcommand{\dddph}{\dddel{\phi}}
\newcommand{\rd}[1]{\mathrm{d} #1}
\newcommand{\dt}{\rd{t}}
\newcommand{\dy}{\rd{y}}
\newcommand{\dx}{\rd{x}}
\newcommand{\edd}[1]{\frac{\mathrm{d}}{\mathrm{d} #1}}
\newcommand{\eddd}[1]{\frac{\mathrm{d}^2}{\mathrm{d}^2 #1}}
\newcommand{\eddon}[2]{\frac{\mathrm{d} #1}{\mathrm{d} #2}}
\newcommand{\edddon}[2]{\frac{\mathrm{d}^2 #1}{\mathrm{d}^2 #2}}
\newcommand{\edt}{\edd{t}}
\newcommand{\edton}[1]{\eddon{#1}{t}}
\newcommand{\edT}{\edd{T}}
\newcommand{\edr}{\edd{r}}
\newcommand{\edl}{\edd{l}}
\newcommand{\edx}{\edd{x}}
\newcommand{\edth}{\edd{\theta}}
\newcommand{\eddton}[1]{\edddon{#1}{t}}
\newcommand{\eddzon}[1]{\edddon{#1}{z}}
\newcommand{\vect}[1]{\boldsymbol{#1}}
\newcommand{\alp}{\frac{1}{\sqrt{2}}}
\newcommand{\alpi}{\frac{i}{\sqrt{2}}}
\newcommand{\expc}[1]{\langle#1\rangle}
\newcommand{\bkn}[1]{\bra{#1}\ket{#1}}
\newcommand{\bk}[2]{\bra{#1}\ket{#2}}
\newcommand{\bik}[3]{\bra{#1} #2 \ket{#3}}
\newcommand{\vari}[1]{\sigma_{#1}}
\newcommand{\intc}[2]{\left[#1, #2\right]}
\newcommand{\sch}{Schr\"{o}dinger}
\newcommand{\moment}{\boldsymbol{p}}
\newcommand{\coor}{\boldsymbol{x}}
\newcommand{\lapc}{\nabla^2}
% \newcommand{\rec}[1]{\frac{1}{#1}}
\newcommand{\vva}{\boldsymbol{a}}
\newcommand{\vvb}{\boldsymbol{b}}
\newcommand{\vc}{\boldsymbol{c}}
\newcommand{\vd}{\boldsymbol{d}}
\newcommand{\ve}{\boldsymbol{e}}
\newcommand{\vf}{\boldsymbol{f}}
\newcommand{\vg}{\boldsymbol{g}}
\newcommand{\vh}{\boldsymbol{h}}
\newcommand{\vi}{\boldsymbol{i}}
\newcommand{\vj}{\boldsymbol{j}}
\newcommand{\vk}{\boldsymbol{k}}
\newcommand{\vl}{\boldsymbol{l}}
\newcommand{\vm}{\boldsymbol{m}}
\newcommand{\vn}{\boldsymbol{n}}
\newcommand{\vo}{\boldsymbol{o}}
\newcommand{\vp}{\boldsymbol{p}}
\newcommand{\vq}{\boldsymbol{q}}
\newcommand{\vr}{\boldsymbol{r}}
\newcommand{\vs}{\boldsymbol{s}}
\newcommand{\vt}{\boldsymbol{t}}
\newcommand{\vvu}{\boldsymbol{u}}
\newcommand{\vv}{\boldsymbol{v}}
\newcommand{\vw}{\boldsymbol{w}}
\newcommand{\vx}{\boldsymbol{x}}
\newcommand{\vy}{\boldsymbol{y}}
\newcommand{\vz}{\boldsymbol{z}}
\newcommand{\vA}{\boldsymbol{A}}
\newcommand{\vB}{\boldsymbol{B}}
\newcommand{\vC}{\boldsymbol{C}}
\newcommand{\vD}{\boldsymbol{D}}
\newcommand{\vE}{\boldsymbol{E}}
\newcommand{\vF}{\boldsymbol{F}}
\newcommand{\vG}{\boldsymbol{G}}
\newcommand{\vH}{\boldsymbol{H}}
\newcommand{\vI}{\boldsymbol{I}}
\newcommand{\vJ}{\boldsymbol{J}}
\newcommand{\vK}{\boldsymbol{K}}
\newcommand{\vL}{\boldsymbol{L}}
\newcommand{\vM}{\boldsymbol{M}}
\newcommand{\vN}{\boldsymbol{N}}
\newcommand{\vO}{\boldsymbol{O}}
\newcommand{\vP}{\boldsymbol{P}}
\newcommand{\vQ}{\boldsymbol{Q}}
\newcommand{\vR}{\boldsymbol{R}}
\newcommand{\vS}{\boldsymbol{S}}
\newcommand{\vT}{\boldsymbol{T}}
\newcommand{\vU}{\boldsymbol{U}}
\newcommand{\vV}{\boldsymbol{V}}
\newcommand{\vW}{\boldsymbol{W}}
\newcommand{\vX}{\boldsymbol{X}}
\newcommand{\vY}{\boldsymbol{Y}}
\newcommand{\vZ}{\boldsymbol{Z}}
\newcommand{\vzero}{\boldsymbol{0}}
\newcommand{\vomega}{\boldsymbol{\omega}}
%\newcommand{\half}{\frac{1}{2}}
\newcommand{\thalf}{\frac{3}{2}}
\newcommand{\rot}{\nabla\times}
\newcommand{\divg}{\nabla\cdot}
\newcommand{\cE}{\mathcal{E}}
\newcommand{\conclu}[1]{\vspace{1em}\par\noindent\fbox{\parbox{0.9\textwidth}{#1}}\vspace{1em}}
\newcommand{\subentrynote}{$\bullet$}
\newcommand{\keypoint}[1]{\par\subentrynote\quad #1 \par}
\newcommand{\fconclu}{\boxed}
\newcommand{\pair}[2]{#1 \, #2}
\newcommand{\intn}[2]{\int #1 \,\mathrm{d} #2}
\newcommand{\intu}[3]{\int_0^{#1} #2 \,\mathrm{d} #3}
\newcommand{\intiu}[3]{\int_{-\infty}^{#1} #2 \, \rd{} #3}
\newcommand{\intui}[2]{\int_0^{\infty} #1 \,\mathrm{d} #2}
\newcommand{\intii}[2]{\int_{-\infty}^{\infty} #1 \,\mathrm{d} #2}
\newcommand{\intt}[2]{\int_0^\infty #1 \, \rd{} #2}
\newcommand{\intr}[2]{\int_{-\infty}^{\infty} #1 \, \rd{} #2}
\newcommand{\intbi}[3]{\int_{#1}^{\infty} #2 \, \rd{} #3}
\newcommand{\intab}[4]{\int_{#1}^{#2} #3 \, \rd{} #4}
\newcommand{\bfactor}[1]{e^{-#1/k_BT}}
\newcommand{\pbfactor}[1]{e^{#1/k_BT}}
\newcommand{\dn}[2]{#1^{\pare{#2}}}
\newcommand{\prodg}[1]{\pare{#1}^\times}

\newcommand{\notion}{\emph}
\newcommand{\iP}{\mathcal{P}}
\newcommand{\eiP}{e^{-\iP}}
\newcommand{\iF}{\mathcal{F}}
\newcommand{\eiF}{e^{-\iF}}
\newcommand{\iG}{\mathcal{G}}

\newcommand{\rc}{r\cos\theta}
\newcommand{\rs}{r\sin\theta}
\newcommand{\sn}{\mathrm{sn}}
\newcommand{\cn}{\mathrm{cn}}
\newcommand{\rdn}{\mathrm{dn}}

\newcommand{\hankel}{H_p^{\pare{1}}}
\newcommand{\hankell}{H_p^{\pare{2}}}
\newcommand{\hhankel}{H_n^{\pare{1}}}
\newcommand{\hhankell}{H_n^{\pare{2}}}
\newcommand{\ber}{\text{ber}\,}
\newcommand{\bei}{\text{bei}\,}
\newcommand{\kker}{\text{ker}\,}
\newcommand{\kei}{\text{kei}\,}
\newcommand{\Ai}{\text{Ai}}
\newcommand{\Bi}{\text{Bi}}

\newcommand{\re}{\text{Re}\,}

\newcommand{\Fp}{F_\phi}
\newcommand{\Ep}{E_\phi}
\newcommand{\Fx}{F_x}
\newcommand{\FF}{\mathbf{F}}
\newcommand{\Ex}{E_x}

%\newcommand{\erf}{\mathrm{erf}}
\newcommand{\erfi}{\mathrm{erfi}}
\newcommand{\erfc}{\mathrm{erfc}}
\newcommand{\ehxs}[1]{e^{-\frac{#1^2}{2}}}
\newcommand{\dcol}[2]{\[ \left.#1 \hspace{1em}\right\vert\hspace{1em} #2 \]}
\newcommand{\titlegamma}{\texorpdfstring{$\Gamma$}{Gamma}}
\newcommand{\titleB}{\texorpdfstring{$B$}{B}}

% Computer Science Head
\usepackage{listings}
\usepackage{color}

\definecolor{dkgreen}{rgb}{0,0.6,0}
\definecolor{gray}{rgb}{0.5,0.5,0.5}
\definecolor{mauve}{rgb}{0.58,0,0.82}

\lstset{frame=tb,
  language=Java,
  aboveskip=3mm,
  belowskip=3mm,
  showstringspaces=false,
  columns=flexible,
  basicstyle={\small\ttfamily},
  numbers=none,
  numberstyle=\tiny\color{gray},
  keywordstyle=\color{blue},
  commentstyle=\color{dkgreen},
  stringstyle=\color{mauve},
  breaklines=true,
  breakatwhitespace=true,
  tabsize=3
}
\lstset{language=Java}
\newcommand{\snp}[1]{\lstinline!#1!}
\newcommand{\term}[2]{\textbf{#1(#2)}}
\begin{document}
\fi

%Content

\chapter{点集拓扑}
  \section{可数性公理与分离公理}
  \subsection{流形的嵌入}
  \begin{definition}
    一个\mmani 为一个具有可数基的Hausdorff空间$X$,其每一点$x$都有一邻域同胚于$\R^m$中的开子集。
  \end{definition}
  曲线为1-维流形,曲面为2-维流形。
  \begin{definition}
    $\func{\phi}{X}{\R}$的支撑为使其非零的定义域子集的闭包。
  \end{definition}
  \begin{definition}
    设$\curb{U_1,\cdots,U_n}$为$X$的加标有限开覆盖. 连续函数加标族$\func{\phi}{X}{\brac{0,1}}$称为由$\curb{U_i}$控制的单位分拆(partition of unity),如果$\phi_i$的支撑在$U_i$内且$\sum \phi_i\pare{x} = 1$。
  \end{definition}
  \begin{theorem}[有限单位分拆的存在性(existence of finite partitions of unity)]
    正规空间的有限开覆盖存在单位分拆。
  \end{theorem}
  \begin{proof}
    首先注意,通过正规性条件,闭集$A=X-\pare{U_2\cup \cdots}$内可以选择满足$\clo{V}_1 \subset U_1$的开集$V_1$。归纳可将诸$U_k$均缩小为$V_k$且含其闭包在内。
    \par
    再将$V_k$同样缩小为覆盖$X$的$W_k$,借助Urysohn定理对诸$i$存在$\psi\pare{W}=1$且$\psi\pare{X-U}=0$的$\psi$,注意$\phi_i=\psi_i/\sum \psi_i$满足条件即可。
  \end{proof}
  \begin{theorem}
    若$X$为\mmani ,则存在$N$使之得嵌入$\R^N$。
  \end{theorem}
  \begin{proof}
    设有限开覆盖$\curb{U_i}$可各由$\bfg_i$嵌入$\R^m$,又设$\phi_i$为相应单位分拆。似可径以$\pare{\bfg_1,\cdots,\bfg_n}$映入$\R^{mn}$,然而诸$\bfg$虽在$U$内连续,在$X$上则会"突然"归零,故不可采,应采用$\pare{\phi_1\bfg_1,\cdots}$将之平滑化。此则不得保证单射性,盖不等之$\bfg$可由不等之$\phi$相乘后相等。是故复以
    \[ \func{F}{X}{\R^{n+mn}} = \pare{\phi_1,\cdots,\phi_n,\phi_1\bfg_1,\cdots,\phi_n\bfg_n} \]
    为之,易证为单射。
  \end{proof}

%ContentEnds
 
\ifx\allfiles\undefined %如果位置放错,可能出现意外中断
\end{document}
\fi
  %Ch7.AnalysisOnManifolds.tex
\ifx\allfiles\undefined
\documentclass{ctexrep}
% Mathematics Include

\usepackage{amsmath}
\usepackage{amssymb}
\usepackage{amsthm}
\usepackage{amsfonts}
\usepackage{mathrsfs}
\usepackage{enumitem}
\usepackage{braket}
\usepackage{hyperref}
\usepackage[all, pdf]{xy}

% Physics Include
\usepackage{amsmath}
\usepackage{physics}
\usepackage{siunitx}
\usepackage[makeroom]{cancel}
\usepackage{pstricks}
\usepackage{pstricks-add}
\psset{algebraic=true}

\usepackage[version=4]{mhchem}
\usepackage{array,booktabs}
\usepackage{longtable}
\usepackage{mathtools}
\usepackage[normalem]{ulem}
\usepackage{multicol}

% Mathematics Head

\newcommand{\pare}[1]{\left(#1\right)}
\newcommand{\blr}[1]{\left[#1\right)}
\newcommand{\lbr}[1]{\left(#1\right]}
\newcommand{\brac}[1]{\left[#1\right]}
\newcommand{\curb}[1]{\left\{#1\right\}}
% \newcommand{\abs}[1]{\left|\, #1 \,\right|}
\newcommand{\rec}[1]{\frac{1}{#1}}
\newcommand{\N}{\mathbb{N}}
\newcommand{\bC}{\mathbb{C}}
\newcommand{\Q}{\mathbb{Q}}
\newcommand{\Z}{\mathbb{Z}}
\newcommand{\R}{\mathbb{R}}
\newcommand{\unk}{\mathcal{X}}
\newcommand{\bu}[3]{#1_{#2}^{\pare{#3}}}
\newcommand{\dref}[1]{定义\ref{def:#1}}
\newcommand{\tref}[1]{定理\ref{thm:#1}}
\newcommand{\lref}[1]{引理\ref{lem:#1}}
\newcommand{\cref}[1]{推论\ref{coll:#1}}
\newcommand{\pref}[1]{命题\ref{prp:#1}}
\newcommand{\eref}[1]{例\ref{ex:#1}}
\newcommand{\func}[3]{#1:\, #2 \rightarrow #3}
\newcommand{\overbar}[1]{\mkern 1.5mu\overline{\mkern-1.5mu#1\mkern-1.5mu}\mkern 1.5mu}
\newcommand{\clo}[1]{\overbar{#1}}
\newcommand{\supi}[2]{\overbar{\int_{#1}^{#2}}}
\newcommand{\infi}[2]{\underbar{\int_{#1}^{#2}}}
\newcommand{\setf}{\mathscr}
\newcommand{\bool}{\mathrm{bool}}
\newcommand{\inc}{++}
\newcommand{\defeq}{:=}
\newcommand{\ntuple}{$n$元组}
\newcommand{\card}[1]{\#\pare{#1}}
\newcommand{\setcond}[2]{\curb{#1 \, \left| \, #2 \right.}}
\newcommand{\setcondl}[2]{\curb{\left. #1 \, \right| \, #2}}
\newcommand{\bv}[1]{\mathbf{#1}}
\newcommand{\bfa}{\bv{a}}
\newcommand{\bfb}{\bv{b}}
\newcommand{\bfx}{\bv{x}}
\newcommand{\bfy}{\bv{y}}
\newcommand{\bfe}{\bv{e}}
\newcommand{\bfF}{\bv{F}}
\newcommand{\bff}{\bv{f}}
\newcommand{\bfG}{\bv{G}}
\newcommand{\bfH}{\bv{H}}
\newcommand{\bfg}{\bv{g}}
\newcommand{\bfh}{\bv{h}}
\newcommand{\bfr}{\bv{r}}
\newcommand{\bfk}{\bv{k}}
\newcommand{\bfu}{\bv{u}}
\newcommand{\bfv}{\bv{v}}
\newcommand{\oo}[1]{o\pare{#1}}
\newcommand{\OO}[1]{O\pare{#1}}
% \newcommand{\norm}[1]{\left\| #1 \right\|}
\newcommand{\DD}{\mathbf{D}}
\newcommand{\comp}{\circ}
\newcommand{\const}{\mathrm{const}}
\newcommand{\dist}[2]{d\pare{#1,#2}}
\newcommand{\len}{\ell}
\newcommand{\siga}{$\sigma$-代数}
\newcommand{\cara}{Carath\'{e}odory}
\newcommand{\Gd}{G_\delta}
\newcommand{\Fs}{F_\sigma}
\newcommand{\mmani}{$m$-维流形}
\newcommand{\open}[1]{\mathcal{#1}}
\newcommand{\half}{\frac{1}{2}}
\newcommand{\maxo}[1]{\text{max}\curb{#1}}
\newcommand{\mino}[1]{\text{min}\curb{#1}}
\newcommand{\epsclo}{$\epsilon$-接近}
\newcommand{\close}[1]{$#1$-接近}
\newcommand{\cinf}{$C^\infty$}
\newcommand{\cuno}{$C^1$}
\newcommand{\Int}{\text{Int}\,}
\newcommand{\Ext}{\text{Ext}\,}
\newcommand{\funcf}{\mathcal}
\newcommand{\DDu}{\overbar{\DD}}
\newcommand{\DDl}{\underbar{\DD}}
\newcommand{\Diff}[1]{\mathrm{Diff}_{#1}\,}
\newcommand{\Av}[1]{\mathrm{Av}_{#1}\,}
\newcommand{\Lip}[1]{Lipschitz-$#1$}
\newcommand{\sgn}{\mathrm{sgn}}
\newcommand{\eset}{\varnothing}
\newcommand{\cT}{\mathcal{T}}
\newcommand{\cS}{\mathcal{S}}
\newcommand{\cG}{\mathcal{G}}
\newcommand{\cF}{\mathcal{F}}
\newcommand{\cC}{\mathcal{C}}
\newcommand{\cB}{\mathcal{B}}
\newcommand{\inter}[1]{\mathring{#1}}
\newcommand{\forest}[3]{对于{#1},存在{#2},使得{#3}}
\newcommand{\tuno}{$T_1$公理}
\newcommand{\isom}{\overset{\sim}{=}}
\newcommand{\diam}{\mathrm{diam}\,}
\newcommand{\ord}[1]{\abs{#1}}
\newcommand{\sbm}[1]{\overbar{#1}}
\newcommand{\inv}[1]{#1^{-1}}
\newcommand{\restr}[2]{#1|_{#2}}
\newcommand{\divs}{|}
\newcommand{\ndivs}{\nmid}
\newcommand{\modeq}[1]{\overbar{#1}}
\newcommand{\ggen}[1]{\langle#1\rangle}
\newcommand{\ggencond}{\braket}

\newcommand{\hd}{H\"{o}lder}

\renewcommand{\proofname}{证明}

\newenvironment{cenum}{\begin{enumerate}\itemsep0em}{\end{enumerate}}

\newtheorem{definition}{定义}[section]
\newtheorem{lemma}{引理}[section]
\newtheorem{theorem}{定理}[section]
\newtheorem{collary}{推论}[section]
\newtheorem{corollary}{推论}[section]
\newtheorem{proposition}{命题}[section]
\newtheorem{axiom}{公理}[section]
\newtheorem{ex}{例}[section]
\newtheorem{reflection}{反射}[section]
\newcommand{\refl}[1]{\vspace{0.5em}\par\noindent\fbox{%
    \parbox{0.9\textwidth}{%
    \begin{reflection}
        #1
    \end{reflection}
    }%
}\vspace{0.5em}\par}
\newcommand{\rref}[1]{反射\ref{refl:#1}}
\newcommand{\tbref}[1]{表\ref{table:#1}}
\allowdisplaybreaks

\newenvironment{aenum}{\begin{enumerate}[label=\textnormal{(\alph*)}]}{\end{enumerate}}

% Physics Head

\DeclareSIUnit\dyne{dynes}

\newcommand{\ddel}[1]{\frac{\partial}{\partial #1}}
\newcommand{\ddelon}[2]{\frac{\partial #1}{\partial #2}}
\newcommand{\dddel}[1]{\frac{\partial^2}{\partial^2 #1}}
\newcommand{\ddt}{\ddel{t}}
\newcommand{\ddT}{\ddel{T}}
\newcommand{\ddV}{\ddel{V}}
\newcommand{\ddr}{\ddel{r}}
\newcommand{\ddth}{\ddel{\theta}}
\newcommand{\ddph}{\ddel{\phi}}
\newcommand{\dddt}{\dddel{t}}
\newcommand{\dddr}{\dddel{t}}
\newcommand{\dddth}{\dddel{\theta}}
\newcommand{\dddph}{\dddel{\phi}}
\newcommand{\rd}[1]{\mathrm{d} #1}
\newcommand{\dt}{\rd{t}}
\newcommand{\dy}{\rd{y}}
\newcommand{\dx}{\rd{x}}
\newcommand{\edd}[1]{\frac{\mathrm{d}}{\mathrm{d} #1}}
\newcommand{\eddd}[1]{\frac{\mathrm{d}^2}{\mathrm{d}^2 #1}}
\newcommand{\eddon}[2]{\frac{\mathrm{d} #1}{\mathrm{d} #2}}
\newcommand{\edddon}[2]{\frac{\mathrm{d}^2 #1}{\mathrm{d}^2 #2}}
\newcommand{\edt}{\edd{t}}
\newcommand{\edton}[1]{\eddon{#1}{t}}
\newcommand{\edT}{\edd{T}}
\newcommand{\edr}{\edd{r}}
\newcommand{\edl}{\edd{l}}
\newcommand{\edx}{\edd{x}}
\newcommand{\edth}{\edd{\theta}}
\newcommand{\eddton}[1]{\edddon{#1}{t}}
\newcommand{\eddzon}[1]{\edddon{#1}{z}}
\newcommand{\vect}[1]{\boldsymbol{#1}}
\newcommand{\alp}{\frac{1}{\sqrt{2}}}
\newcommand{\alpi}{\frac{i}{\sqrt{2}}}
\newcommand{\expc}[1]{\langle#1\rangle}
\newcommand{\bkn}[1]{\bra{#1}\ket{#1}}
\newcommand{\bk}[2]{\bra{#1}\ket{#2}}
\newcommand{\bik}[3]{\bra{#1} #2 \ket{#3}}
\newcommand{\vari}[1]{\sigma_{#1}}
\newcommand{\intc}[2]{\left[#1, #2\right]}
\newcommand{\sch}{Schr\"{o}dinger}
\newcommand{\moment}{\boldsymbol{p}}
\newcommand{\coor}{\boldsymbol{x}}
\newcommand{\lapc}{\nabla^2}
% \newcommand{\rec}[1]{\frac{1}{#1}}
\newcommand{\vva}{\boldsymbol{a}}
\newcommand{\vvb}{\boldsymbol{b}}
\newcommand{\vc}{\boldsymbol{c}}
\newcommand{\vd}{\boldsymbol{d}}
\newcommand{\ve}{\boldsymbol{e}}
\newcommand{\vf}{\boldsymbol{f}}
\newcommand{\vg}{\boldsymbol{g}}
\newcommand{\vh}{\boldsymbol{h}}
\newcommand{\vi}{\boldsymbol{i}}
\newcommand{\vj}{\boldsymbol{j}}
\newcommand{\vk}{\boldsymbol{k}}
\newcommand{\vl}{\boldsymbol{l}}
\newcommand{\vm}{\boldsymbol{m}}
\newcommand{\vn}{\boldsymbol{n}}
\newcommand{\vo}{\boldsymbol{o}}
\newcommand{\vp}{\boldsymbol{p}}
\newcommand{\vq}{\boldsymbol{q}}
\newcommand{\vr}{\boldsymbol{r}}
\newcommand{\vs}{\boldsymbol{s}}
\newcommand{\vt}{\boldsymbol{t}}
\newcommand{\vvu}{\boldsymbol{u}}
\newcommand{\vv}{\boldsymbol{v}}
\newcommand{\vw}{\boldsymbol{w}}
\newcommand{\vx}{\boldsymbol{x}}
\newcommand{\vy}{\boldsymbol{y}}
\newcommand{\vz}{\boldsymbol{z}}
\newcommand{\vA}{\boldsymbol{A}}
\newcommand{\vB}{\boldsymbol{B}}
\newcommand{\vC}{\boldsymbol{C}}
\newcommand{\vD}{\boldsymbol{D}}
\newcommand{\vE}{\boldsymbol{E}}
\newcommand{\vF}{\boldsymbol{F}}
\newcommand{\vG}{\boldsymbol{G}}
\newcommand{\vH}{\boldsymbol{H}}
\newcommand{\vI}{\boldsymbol{I}}
\newcommand{\vJ}{\boldsymbol{J}}
\newcommand{\vK}{\boldsymbol{K}}
\newcommand{\vL}{\boldsymbol{L}}
\newcommand{\vM}{\boldsymbol{M}}
\newcommand{\vN}{\boldsymbol{N}}
\newcommand{\vO}{\boldsymbol{O}}
\newcommand{\vP}{\boldsymbol{P}}
\newcommand{\vQ}{\boldsymbol{Q}}
\newcommand{\vR}{\boldsymbol{R}}
\newcommand{\vS}{\boldsymbol{S}}
\newcommand{\vT}{\boldsymbol{T}}
\newcommand{\vU}{\boldsymbol{U}}
\newcommand{\vV}{\boldsymbol{V}}
\newcommand{\vW}{\boldsymbol{W}}
\newcommand{\vX}{\boldsymbol{X}}
\newcommand{\vY}{\boldsymbol{Y}}
\newcommand{\vZ}{\boldsymbol{Z}}
\newcommand{\vzero}{\boldsymbol{0}}
\newcommand{\vomega}{\boldsymbol{\omega}}
%\newcommand{\half}{\frac{1}{2}}
\newcommand{\thalf}{\frac{3}{2}}
\newcommand{\rot}{\nabla\times}
\newcommand{\divg}{\nabla\cdot}
\newcommand{\cE}{\mathcal{E}}
\newcommand{\conclu}[1]{\vspace{1em}\par\noindent\fbox{\parbox{0.9\textwidth}{#1}}\vspace{1em}}
\newcommand{\subentrynote}{$\bullet$}
\newcommand{\keypoint}[1]{\par\subentrynote\quad #1 \par}
\newcommand{\fconclu}{\boxed}
\newcommand{\pair}[2]{#1 \, #2}
\newcommand{\intn}[2]{\int #1 \,\mathrm{d} #2}
\newcommand{\intu}[3]{\int_0^{#1} #2 \,\mathrm{d} #3}
\newcommand{\intiu}[3]{\int_{-\infty}^{#1} #2 \, \rd{} #3}
\newcommand{\intui}[2]{\int_0^{\infty} #1 \,\mathrm{d} #2}
\newcommand{\intii}[2]{\int_{-\infty}^{\infty} #1 \,\mathrm{d} #2}
\newcommand{\intt}[2]{\int_0^\infty #1 \, \rd{} #2}
\newcommand{\intr}[2]{\int_{-\infty}^{\infty} #1 \, \rd{} #2}
\newcommand{\intbi}[3]{\int_{#1}^{\infty} #2 \, \rd{} #3}
\newcommand{\intab}[4]{\int_{#1}^{#2} #3 \, \rd{} #4}
\newcommand{\bfactor}[1]{e^{-#1/k_BT}}
\newcommand{\pbfactor}[1]{e^{#1/k_BT}}
\newcommand{\dn}[2]{#1^{\pare{#2}}}
\newcommand{\prodg}[1]{\pare{#1}^\times}

\newcommand{\notion}{\emph}
\newcommand{\iP}{\mathcal{P}}
\newcommand{\eiP}{e^{-\iP}}
\newcommand{\iF}{\mathcal{F}}
\newcommand{\eiF}{e^{-\iF}}
\newcommand{\iG}{\mathcal{G}}

\newcommand{\rc}{r\cos\theta}
\newcommand{\rs}{r\sin\theta}
\newcommand{\sn}{\mathrm{sn}}
\newcommand{\cn}{\mathrm{cn}}
\newcommand{\rdn}{\mathrm{dn}}

\newcommand{\hankel}{H_p^{\pare{1}}}
\newcommand{\hankell}{H_p^{\pare{2}}}
\newcommand{\hhankel}{H_n^{\pare{1}}}
\newcommand{\hhankell}{H_n^{\pare{2}}}
\newcommand{\ber}{\text{ber}\,}
\newcommand{\bei}{\text{bei}\,}
\newcommand{\kker}{\text{ker}\,}
\newcommand{\kei}{\text{kei}\,}
\newcommand{\Ai}{\text{Ai}}
\newcommand{\Bi}{\text{Bi}}

\newcommand{\re}{\text{Re}\,}

\newcommand{\Fp}{F_\phi}
\newcommand{\Ep}{E_\phi}
\newcommand{\Fx}{F_x}
\newcommand{\FF}{\mathbf{F}}
\newcommand{\Ex}{E_x}

%\newcommand{\erf}{\mathrm{erf}}
\newcommand{\erfi}{\mathrm{erfi}}
\newcommand{\erfc}{\mathrm{erfc}}
\newcommand{\ehxs}[1]{e^{-\frac{#1^2}{2}}}
\newcommand{\dcol}[2]{\[ \left.#1 \hspace{1em}\right\vert\hspace{1em} #2 \]}
\newcommand{\titlegamma}{\texorpdfstring{$\Gamma$}{Gamma}}
\newcommand{\titleB}{\texorpdfstring{$B$}{B}}

% Computer Science Head
\usepackage{listings}
\usepackage{color}

\definecolor{dkgreen}{rgb}{0,0.6,0}
\definecolor{gray}{rgb}{0.5,0.5,0.5}
\definecolor{mauve}{rgb}{0.58,0,0.82}

\lstset{frame=tb,
  language=Java,
  aboveskip=3mm,
  belowskip=3mm,
  showstringspaces=false,
  columns=flexible,
  basicstyle={\small\ttfamily},
  numbers=none,
  numberstyle=\tiny\color{gray},
  keywordstyle=\color{blue},
  commentstyle=\color{dkgreen},
  stringstyle=\color{mauve},
  breaklines=true,
  breakatwhitespace=true,
  tabsize=3
}
\lstset{language=Java}
\newcommand{\snp}[1]{\lstinline!#1!}
\newcommand{\term}[2]{\textbf{#1(#2)}}
\begin{document}
\fi

%Content

\chapter{流形上的分析}
  \section{重积分的换元}
  \subsection{单位分解}
  \begin{lemma}
    对$\R^n$中的矩形,存在\cinf 的函数恰以之为支撑。
  \end{lemma}
  \begin{proof}
    设$f\pare{x}=e^{-1/x}\chi_{\R^+}$,则$f\pare{x}f\pare{1-x}$为\cinf 且以$\brac{0,1}$为支撑。
  \end{proof}
  \begin{lemma}
    设$\setf{A}$是$\R^n$的一族开集,其并为$A$。存在矩形的可数族$\curb{Q_i}$覆盖之,而诸矩形在诸集内,且局部有限。
  \end{lemma}
  \begin{proof}
    取覆盖$A$的严格递增紧子集列$\curb{D_i}$并设其差分为$\curb{B_i}$,知为紧致,故可以有限多矩形覆盖至且诸矩形在$D_{i-2}$外。易知此矩形族满足条件。
  \end{proof}
  \begin{definition}
    $\func{\phi}{\R^n}{\R}$的支撑为使其非零的定义域子集的闭包。
  \end{definition}
  \begin{theorem}[单位分解的存在性]
    在前开引理的条件下,存在诸矩形控制的\cinf 可数单位分拆。
  \end{theorem}
  \begin{proof}
    参考前二引理,注意由局部有限性,各点处均有邻域使可数分拆仅为有限和,故和收敛且为\cinf ,故可加和后归一。
  \end{proof}
  \begin{ex}
    将$f\pare{x}=\chi_{\brac{-2\pi,2\pi}}\pare{1+\cos x}/2$逐次移动$\pi$,可以得到$\R$的\cuno 单位分解。
  \end{ex}
  \begin{lemma}
    $\func{f}{A}{\R}$连续且在紧集$C\subset A$外为零,则$\int_A f=\int_C f$。
  \end{lemma}
  \begin{proof}
    存在性由$C$的有界性和极值定理推出$f$的有界性可得。取覆盖$A$的严格增紧集列$C_i$,其亦覆盖$C$,故$C$在某$C_M$内。
    \[ \int_C f = \lim \int_{C_N} f = \int_A f. \qedhere \]
  \end{proof}
  \begin{theorem}
    $\func{f}{A}{\R}$连续,$\curb{\varphi_i}$为$A$的具有紧支撑的单位分解,则$\int_A f$存在当且仅当
    \[ \sum \brac{\int_A \varphi_i \abs{f}} \]
    收敛,此时
    \[ \int_A f = \sum \brac{\int_A \varphi_i f}. \]
  \end{theorem}


%ContentEnds
 
\ifx\allfiles\undefined %如果位置放错,可能出现意外中断
\end{document}
\fi
  %Ch8.MeasureTheory.tex
\ifx\allfiles\undefined
\documentclass{ctexrep}
% Mathematics Include

\usepackage{amsmath}
\usepackage{amssymb}
\usepackage{amsthm}
\usepackage{amsfonts}
\usepackage{mathrsfs}
\usepackage{enumitem}
\usepackage{braket}
\usepackage{hyperref}
\usepackage[all, pdf]{xy}

% Physics Include
\usepackage{amsmath}
\usepackage{physics}
\usepackage{siunitx}
\usepackage[makeroom]{cancel}
\usepackage{pstricks}
\usepackage{pstricks-add}
\psset{algebraic=true}

\usepackage[version=4]{mhchem}
\usepackage{array,booktabs}
\usepackage{longtable}
\usepackage{mathtools}
\usepackage[normalem]{ulem}
\usepackage{multicol}

% Mathematics Head

\newcommand{\pare}[1]{\left(#1\right)}
\newcommand{\blr}[1]{\left[#1\right)}
\newcommand{\lbr}[1]{\left(#1\right]}
\newcommand{\brac}[1]{\left[#1\right]}
\newcommand{\curb}[1]{\left\{#1\right\}}
% \newcommand{\abs}[1]{\left|\, #1 \,\right|}
\newcommand{\rec}[1]{\frac{1}{#1}}
\newcommand{\N}{\mathbb{N}}
\newcommand{\bC}{\mathbb{C}}
\newcommand{\Q}{\mathbb{Q}}
\newcommand{\Z}{\mathbb{Z}}
\newcommand{\R}{\mathbb{R}}
\newcommand{\unk}{\mathcal{X}}
\newcommand{\bu}[3]{#1_{#2}^{\pare{#3}}}
\newcommand{\dref}[1]{定义\ref{def:#1}}
\newcommand{\tref}[1]{定理\ref{thm:#1}}
\newcommand{\lref}[1]{引理\ref{lem:#1}}
\newcommand{\cref}[1]{推论\ref{coll:#1}}
\newcommand{\pref}[1]{命题\ref{prp:#1}}
\newcommand{\eref}[1]{例\ref{ex:#1}}
\newcommand{\func}[3]{#1:\, #2 \rightarrow #3}
\newcommand{\overbar}[1]{\mkern 1.5mu\overline{\mkern-1.5mu#1\mkern-1.5mu}\mkern 1.5mu}
\newcommand{\clo}[1]{\overbar{#1}}
\newcommand{\supi}[2]{\overbar{\int_{#1}^{#2}}}
\newcommand{\infi}[2]{\underbar{\int_{#1}^{#2}}}
\newcommand{\setf}{\mathscr}
\newcommand{\bool}{\mathrm{bool}}
\newcommand{\inc}{++}
\newcommand{\defeq}{:=}
\newcommand{\ntuple}{$n$元组}
\newcommand{\card}[1]{\#\pare{#1}}
\newcommand{\setcond}[2]{\curb{#1 \, \left| \, #2 \right.}}
\newcommand{\setcondl}[2]{\curb{\left. #1 \, \right| \, #2}}
\newcommand{\bv}[1]{\mathbf{#1}}
\newcommand{\bfa}{\bv{a}}
\newcommand{\bfb}{\bv{b}}
\newcommand{\bfx}{\bv{x}}
\newcommand{\bfy}{\bv{y}}
\newcommand{\bfe}{\bv{e}}
\newcommand{\bfF}{\bv{F}}
\newcommand{\bff}{\bv{f}}
\newcommand{\bfG}{\bv{G}}
\newcommand{\bfH}{\bv{H}}
\newcommand{\bfg}{\bv{g}}
\newcommand{\bfh}{\bv{h}}
\newcommand{\bfr}{\bv{r}}
\newcommand{\bfk}{\bv{k}}
\newcommand{\bfu}{\bv{u}}
\newcommand{\bfv}{\bv{v}}
\newcommand{\oo}[1]{o\pare{#1}}
\newcommand{\OO}[1]{O\pare{#1}}
% \newcommand{\norm}[1]{\left\| #1 \right\|}
\newcommand{\DD}{\mathbf{D}}
\newcommand{\comp}{\circ}
\newcommand{\const}{\mathrm{const}}
\newcommand{\dist}[2]{d\pare{#1,#2}}
\newcommand{\len}{\ell}
\newcommand{\siga}{$\sigma$-代数}
\newcommand{\cara}{Carath\'{e}odory}
\newcommand{\Gd}{G_\delta}
\newcommand{\Fs}{F_\sigma}
\newcommand{\mmani}{$m$-维流形}
\newcommand{\open}[1]{\mathcal{#1}}
\newcommand{\half}{\frac{1}{2}}
\newcommand{\maxo}[1]{\text{max}\curb{#1}}
\newcommand{\mino}[1]{\text{min}\curb{#1}}
\newcommand{\epsclo}{$\epsilon$-接近}
\newcommand{\close}[1]{$#1$-接近}
\newcommand{\cinf}{$C^\infty$}
\newcommand{\cuno}{$C^1$}
\newcommand{\Int}{\text{Int}\,}
\newcommand{\Ext}{\text{Ext}\,}
\newcommand{\funcf}{\mathcal}
\newcommand{\DDu}{\overbar{\DD}}
\newcommand{\DDl}{\underbar{\DD}}
\newcommand{\Diff}[1]{\mathrm{Diff}_{#1}\,}
\newcommand{\Av}[1]{\mathrm{Av}_{#1}\,}
\newcommand{\Lip}[1]{Lipschitz-$#1$}
\newcommand{\sgn}{\mathrm{sgn}}
\newcommand{\eset}{\varnothing}
\newcommand{\cT}{\mathcal{T}}
\newcommand{\cS}{\mathcal{S}}
\newcommand{\cG}{\mathcal{G}}
\newcommand{\cF}{\mathcal{F}}
\newcommand{\cC}{\mathcal{C}}
\newcommand{\cB}{\mathcal{B}}
\newcommand{\inter}[1]{\mathring{#1}}
\newcommand{\forest}[3]{对于{#1},存在{#2},使得{#3}}
\newcommand{\tuno}{$T_1$公理}
\newcommand{\isom}{\overset{\sim}{=}}
\newcommand{\diam}{\mathrm{diam}\,}
\newcommand{\ord}[1]{\abs{#1}}
\newcommand{\sbm}[1]{\overbar{#1}}
\newcommand{\inv}[1]{#1^{-1}}
\newcommand{\restr}[2]{#1|_{#2}}
\newcommand{\divs}{|}
\newcommand{\ndivs}{\nmid}
\newcommand{\modeq}[1]{\overbar{#1}}
\newcommand{\ggen}[1]{\langle#1\rangle}
\newcommand{\ggencond}{\braket}

\newcommand{\hd}{H\"{o}lder}

\renewcommand{\proofname}{证明}

\newenvironment{cenum}{\begin{enumerate}\itemsep0em}{\end{enumerate}}

\newtheorem{definition}{定义}[section]
\newtheorem{lemma}{引理}[section]
\newtheorem{theorem}{定理}[section]
\newtheorem{collary}{推论}[section]
\newtheorem{corollary}{推论}[section]
\newtheorem{proposition}{命题}[section]
\newtheorem{axiom}{公理}[section]
\newtheorem{ex}{例}[section]
\newtheorem{reflection}{反射}[section]
\newcommand{\refl}[1]{\vspace{0.5em}\par\noindent\fbox{%
    \parbox{0.9\textwidth}{%
    \begin{reflection}
        #1
    \end{reflection}
    }%
}\vspace{0.5em}\par}
\newcommand{\rref}[1]{反射\ref{refl:#1}}
\newcommand{\tbref}[1]{表\ref{table:#1}}
\allowdisplaybreaks

\newenvironment{aenum}{\begin{enumerate}[label=\textnormal{(\alph*)}]}{\end{enumerate}}

% Physics Head

\DeclareSIUnit\dyne{dynes}

\newcommand{\ddel}[1]{\frac{\partial}{\partial #1}}
\newcommand{\ddelon}[2]{\frac{\partial #1}{\partial #2}}
\newcommand{\dddel}[1]{\frac{\partial^2}{\partial^2 #1}}
\newcommand{\ddt}{\ddel{t}}
\newcommand{\ddT}{\ddel{T}}
\newcommand{\ddV}{\ddel{V}}
\newcommand{\ddr}{\ddel{r}}
\newcommand{\ddth}{\ddel{\theta}}
\newcommand{\ddph}{\ddel{\phi}}
\newcommand{\dddt}{\dddel{t}}
\newcommand{\dddr}{\dddel{t}}
\newcommand{\dddth}{\dddel{\theta}}
\newcommand{\dddph}{\dddel{\phi}}
\newcommand{\rd}[1]{\mathrm{d} #1}
\newcommand{\dt}{\rd{t}}
\newcommand{\dy}{\rd{y}}
\newcommand{\dx}{\rd{x}}
\newcommand{\edd}[1]{\frac{\mathrm{d}}{\mathrm{d} #1}}
\newcommand{\eddd}[1]{\frac{\mathrm{d}^2}{\mathrm{d}^2 #1}}
\newcommand{\eddon}[2]{\frac{\mathrm{d} #1}{\mathrm{d} #2}}
\newcommand{\edddon}[2]{\frac{\mathrm{d}^2 #1}{\mathrm{d}^2 #2}}
\newcommand{\edt}{\edd{t}}
\newcommand{\edton}[1]{\eddon{#1}{t}}
\newcommand{\edT}{\edd{T}}
\newcommand{\edr}{\edd{r}}
\newcommand{\edl}{\edd{l}}
\newcommand{\edx}{\edd{x}}
\newcommand{\edth}{\edd{\theta}}
\newcommand{\eddton}[1]{\edddon{#1}{t}}
\newcommand{\eddzon}[1]{\edddon{#1}{z}}
\newcommand{\vect}[1]{\boldsymbol{#1}}
\newcommand{\alp}{\frac{1}{\sqrt{2}}}
\newcommand{\alpi}{\frac{i}{\sqrt{2}}}
\newcommand{\expc}[1]{\langle#1\rangle}
\newcommand{\bkn}[1]{\bra{#1}\ket{#1}}
\newcommand{\bk}[2]{\bra{#1}\ket{#2}}
\newcommand{\bik}[3]{\bra{#1} #2 \ket{#3}}
\newcommand{\vari}[1]{\sigma_{#1}}
\newcommand{\intc}[2]{\left[#1, #2\right]}
\newcommand{\sch}{Schr\"{o}dinger}
\newcommand{\moment}{\boldsymbol{p}}
\newcommand{\coor}{\boldsymbol{x}}
\newcommand{\lapc}{\nabla^2}
% \newcommand{\rec}[1]{\frac{1}{#1}}
\newcommand{\vva}{\boldsymbol{a}}
\newcommand{\vvb}{\boldsymbol{b}}
\newcommand{\vc}{\boldsymbol{c}}
\newcommand{\vd}{\boldsymbol{d}}
\newcommand{\ve}{\boldsymbol{e}}
\newcommand{\vf}{\boldsymbol{f}}
\newcommand{\vg}{\boldsymbol{g}}
\newcommand{\vh}{\boldsymbol{h}}
\newcommand{\vi}{\boldsymbol{i}}
\newcommand{\vj}{\boldsymbol{j}}
\newcommand{\vk}{\boldsymbol{k}}
\newcommand{\vl}{\boldsymbol{l}}
\newcommand{\vm}{\boldsymbol{m}}
\newcommand{\vn}{\boldsymbol{n}}
\newcommand{\vo}{\boldsymbol{o}}
\newcommand{\vp}{\boldsymbol{p}}
\newcommand{\vq}{\boldsymbol{q}}
\newcommand{\vr}{\boldsymbol{r}}
\newcommand{\vs}{\boldsymbol{s}}
\newcommand{\vt}{\boldsymbol{t}}
\newcommand{\vvu}{\boldsymbol{u}}
\newcommand{\vv}{\boldsymbol{v}}
\newcommand{\vw}{\boldsymbol{w}}
\newcommand{\vx}{\boldsymbol{x}}
\newcommand{\vy}{\boldsymbol{y}}
\newcommand{\vz}{\boldsymbol{z}}
\newcommand{\vA}{\boldsymbol{A}}
\newcommand{\vB}{\boldsymbol{B}}
\newcommand{\vC}{\boldsymbol{C}}
\newcommand{\vD}{\boldsymbol{D}}
\newcommand{\vE}{\boldsymbol{E}}
\newcommand{\vF}{\boldsymbol{F}}
\newcommand{\vG}{\boldsymbol{G}}
\newcommand{\vH}{\boldsymbol{H}}
\newcommand{\vI}{\boldsymbol{I}}
\newcommand{\vJ}{\boldsymbol{J}}
\newcommand{\vK}{\boldsymbol{K}}
\newcommand{\vL}{\boldsymbol{L}}
\newcommand{\vM}{\boldsymbol{M}}
\newcommand{\vN}{\boldsymbol{N}}
\newcommand{\vO}{\boldsymbol{O}}
\newcommand{\vP}{\boldsymbol{P}}
\newcommand{\vQ}{\boldsymbol{Q}}
\newcommand{\vR}{\boldsymbol{R}}
\newcommand{\vS}{\boldsymbol{S}}
\newcommand{\vT}{\boldsymbol{T}}
\newcommand{\vU}{\boldsymbol{U}}
\newcommand{\vV}{\boldsymbol{V}}
\newcommand{\vW}{\boldsymbol{W}}
\newcommand{\vX}{\boldsymbol{X}}
\newcommand{\vY}{\boldsymbol{Y}}
\newcommand{\vZ}{\boldsymbol{Z}}
\newcommand{\vzero}{\boldsymbol{0}}
\newcommand{\vomega}{\boldsymbol{\omega}}
%\newcommand{\half}{\frac{1}{2}}
\newcommand{\thalf}{\frac{3}{2}}
\newcommand{\rot}{\nabla\times}
\newcommand{\divg}{\nabla\cdot}
\newcommand{\cE}{\mathcal{E}}
\newcommand{\conclu}[1]{\vspace{1em}\par\noindent\fbox{\parbox{0.9\textwidth}{#1}}\vspace{1em}}
\newcommand{\subentrynote}{$\bullet$}
\newcommand{\keypoint}[1]{\par\subentrynote\quad #1 \par}
\newcommand{\fconclu}{\boxed}
\newcommand{\pair}[2]{#1 \, #2}
\newcommand{\intn}[2]{\int #1 \,\mathrm{d} #2}
\newcommand{\intu}[3]{\int_0^{#1} #2 \,\mathrm{d} #3}
\newcommand{\intiu}[3]{\int_{-\infty}^{#1} #2 \, \rd{} #3}
\newcommand{\intui}[2]{\int_0^{\infty} #1 \,\mathrm{d} #2}
\newcommand{\intii}[2]{\int_{-\infty}^{\infty} #1 \,\mathrm{d} #2}
\newcommand{\intt}[2]{\int_0^\infty #1 \, \rd{} #2}
\newcommand{\intr}[2]{\int_{-\infty}^{\infty} #1 \, \rd{} #2}
\newcommand{\intbi}[3]{\int_{#1}^{\infty} #2 \, \rd{} #3}
\newcommand{\intab}[4]{\int_{#1}^{#2} #3 \, \rd{} #4}
\newcommand{\bfactor}[1]{e^{-#1/k_BT}}
\newcommand{\pbfactor}[1]{e^{#1/k_BT}}
\newcommand{\dn}[2]{#1^{\pare{#2}}}
\newcommand{\prodg}[1]{\pare{#1}^\times}

\newcommand{\notion}{\emph}
\newcommand{\iP}{\mathcal{P}}
\newcommand{\eiP}{e^{-\iP}}
\newcommand{\iF}{\mathcal{F}}
\newcommand{\eiF}{e^{-\iF}}
\newcommand{\iG}{\mathcal{G}}

\newcommand{\rc}{r\cos\theta}
\newcommand{\rs}{r\sin\theta}
\newcommand{\sn}{\mathrm{sn}}
\newcommand{\cn}{\mathrm{cn}}
\newcommand{\rdn}{\mathrm{dn}}

\newcommand{\hankel}{H_p^{\pare{1}}}
\newcommand{\hankell}{H_p^{\pare{2}}}
\newcommand{\hhankel}{H_n^{\pare{1}}}
\newcommand{\hhankell}{H_n^{\pare{2}}}
\newcommand{\ber}{\text{ber}\,}
\newcommand{\bei}{\text{bei}\,}
\newcommand{\kker}{\text{ker}\,}
\newcommand{\kei}{\text{kei}\,}
\newcommand{\Ai}{\text{Ai}}
\newcommand{\Bi}{\text{Bi}}

\newcommand{\re}{\text{Re}\,}

\newcommand{\Fp}{F_\phi}
\newcommand{\Ep}{E_\phi}
\newcommand{\Fx}{F_x}
\newcommand{\FF}{\mathbf{F}}
\newcommand{\Ex}{E_x}

%\newcommand{\erf}{\mathrm{erf}}
\newcommand{\erfi}{\mathrm{erfi}}
\newcommand{\erfc}{\mathrm{erfc}}
\newcommand{\ehxs}[1]{e^{-\frac{#1^2}{2}}}
\newcommand{\dcol}[2]{\[ \left.#1 \hspace{1em}\right\vert\hspace{1em} #2 \]}
\newcommand{\titlegamma}{\texorpdfstring{$\Gamma$}{Gamma}}
\newcommand{\titleB}{\texorpdfstring{$B$}{B}}

% Computer Science Head
\usepackage{listings}
\usepackage{color}

\definecolor{dkgreen}{rgb}{0,0.6,0}
\definecolor{gray}{rgb}{0.5,0.5,0.5}
\definecolor{mauve}{rgb}{0.58,0,0.82}

\lstset{frame=tb,
  language=Java,
  aboveskip=3mm,
  belowskip=3mm,
  showstringspaces=false,
  columns=flexible,
  basicstyle={\small\ttfamily},
  numbers=none,
  numberstyle=\tiny\color{gray},
  keywordstyle=\color{blue},
  commentstyle=\color{dkgreen},
  stringstyle=\color{mauve},
  breaklines=true,
  breakatwhitespace=true,
  tabsize=3
}
\lstset{language=Java}
\newcommand{\snp}[1]{\lstinline!#1!}
\newcommand{\term}[2]{\textbf{#1(#2)}}
\begin{document}
\fi

%Content

\chapter{测度论}
  \section{Lebesgue测度}
  \subsection{引论}
  \subsubsection*{Lebesgue测度的性质}
  我们期望Lebesgue测度具有如下一些性质。
  \paragraph{区间的测度为其长度}非空区间是可测集,且
  \[ m\pare{I}=\len\pare{I}. \]
  \paragraph{测度是平移不变的}若$E$为Lebesgue可测集且$y$为一数,则
  \[ m\pare{E+y}=m\pare{E}. \]
  \paragraph{无交集的可数并的测度可加}$若E_k$为可数个无交可测集,则
  \[ m\pare{\bigcup E_k} = \sum m\pare{E_k}. \]
  且Lebesgue可测集全体构成一\siga。
  \begin{definition}
    一集族构成代数 ,如果其元素的补,有限交与有限并皆封闭。
  \end{definition}
  \begin{definition}
    一集族构成\siga ,如果其元素的补,可数交与可数并皆封闭。
  \end{definition}
  在全体集合上定义满足条件的测度是不可能的,甚至仅仅满足前两个条件而具有有限可加性都是不能指望的。但在定义Lebesgue测度前,仍可先构造对任意集合都适用的外测度,满足前二条件,而第三条件替换为无论诸$E_k$无交与否,皆有
  \[ m^*\pare{\bigcup E_k} \le \sum m^*\pare{E_k}. \]
  \subsection{Lebesgue外测度}
  定义无界区间的长度为$\infty$。对于任意集合,定义外测度
  \[ m^*\pare{A} = \inf\sum\len\pare{I_k}. \]
  其中$\curb{I_k}$为$A$的区间覆盖。立即可得空集外测度为零且外测度具有单调性,即若$A \subset B$则
  \[ m^*\pare{A} \le m^*\pare{B}. \]
  可以由此证明,可数集的测度为零。
  \begin{proposition}
    区间的测度为其长度。
  \end{proposition}
  \begin{proof}
    考虑有界闭区间$\brac{a,b}$,易证$m*\le\pare{b-a}$。另一方向的不等号需要
    \[ \sum\len\pare{I_k} \ge b-a. \]
    由紧致性只需要对有限开覆盖证明
    \[ \sum^n \len\pare{I_k} \ge b-a. \]
    选取包含$a$的区间1,若右端点在$\pare{a,b}$内则选取另一包含其右端点的区间2,重复这一过程直到右端点在$\pare{a,b}$外,则上述不等式成立。
    \par
    对于任意有界区间,选取其闭区间的上下逼近并注意外测度的单调性即可。对于无界区间,易得其测度为$\infty$。
  \end{proof}
  \begin{proposition}
    Lebesgue外测度是平移不变的。
  \end{proposition}
  \begin{proof}
    注意区间的平移不变即可。
  \end{proof}
  \begin{proposition}
    对任意$\curb{E_k}$,有
    \[ m^*\pare{\bigcup E_k} \le \sum m^*\pare{E_k}. \]
  \end{proposition}
  \begin{proof}
    对$E_k$取误差不超过$2^{-k}\epsilon$的覆盖区间,加和即可。
  \end{proof}
  \subsection{Lebesgue可测集的\siga}
  \subsubsection{\cara 可测}
  \begin{definition}
    若对于任意集合$A$,都有
    \[ m^*\pare{A} = m^*\pare{A\cap E}+m^*\pare{A\cap\complement E}, \]
    则称$E$可测。
  \end{definition}
  鉴于外测度的次可加性,上述条件可弱化为
  \[ m^*\pare{A} \ge m^*\pare{A\cap E}+m^*\pare{A\cap\complement E}. \]
  此外还应注意到,对于无交集,若其中任一可测,立刻有
  \begin{align*}
  m*\pare{A\cup B} &= m^*\pare{\brac{A\cup B}\cap A}+m^*\pare{\brac{A\cup B}\cap \complement A}\\ &= m^*\pare{A}+m^*\pare{B}.
  \end{align*}
  故有可加性。此外,可测集的补仍为可测集。
  \begin{theorem}
    零测集为可测集。
  \end{theorem}
  \begin{proof}
    代入弱化后的条件,注意外测度的单调性即可。
  \end{proof}
  \begin{theorem}
    可测集的有限并可测。故可测集构成代数。
  \end{theorem}
  \begin{proof}
    只证二可测集的并可测。借助二集可测的\cara 条件,有
    \begin{align*}
      m^*\pare{A} &= m^*\pare{A\cap E_1}+m^*\pare{A\cap\complement E_1\cap E_2} + m^*\pare{A\cap\complement E_1\cap \complement E_2} \\
      &\ge m^*\pare{A\cap\brac{E_1\cup E_2}} + m^*\pare{A\cap\complement\brac{E_1\cup E_2}}.\qedhere
    \end{align*}
  \end{proof}
  \begin{theorem}
    无交可测集的有限并满足
    \[ m^*\pare{A\cap\bigcup E_k} = \sum m^*\pare{A\cap E_k}. \]
  \end{theorem}
  \begin{proof}
    注意到\cara 条件的
    \[ A\cap\bigcup^n E_k \cap E_n = A \cap E_n \]
    以及
    \[ A\cap\bigcup^n E_k \cap \complement E_n = A\cap\bigcup^{n-1} E_n, \]
    归纳即可。
  \end{proof}
  \begin{collary}
    可测集的测度有限可加。
  \end{collary}
  \begin{theorem}
    可测集的可数并可测。故可测集构成\siga 。
  \end{theorem}
  \begin{proof}
    不妨设诸集无交。设其并为$E$,则根据前开命题及单调性,有
    \[ m^*\pare{A} \ge \sum^n m^*\pare{A\cap E_k} + m^*\pare{A\cap\complement E}. \]
    让$n\to\infty$,借助次可加性即可。
  \end{proof}
  \begin{theorem}
    区间是可测集。
  \end{theorem}
  \begin{proof}
    只证$I=\pare{a,\infty}$型区间可测。不妨设$a$不在$A$内且将之分割为$A\cap \complement I = A_1$与$A \cap I = A_2$。对于$A$的任意覆盖$\curb{I_k}$均同样割裂之,有
    \[ m^*\pare{A_1} + m^*\pare{A_2} \le \sum \len\pare{I_k}, \]
    故满足弱化后条件。
  \end{proof}
  \begin{definition}
    开集的可数交为$\Gd$型集。
  \end{definition}
  \begin{definition}
    闭集的可数并为$\Fs$型集。
  \end{definition}
  注意$\R$中开集为区间的并,故$\Gd$型(以及$\Fs$型)集可测。
  \begin{definition}
    包含开集的最小\siga 称为Borel \siga ,其元素称为Borel集。
  \end{definition}
  \begin{theorem}
    $\R$中可测集包含Borel \siga 。区间,开集,闭集,$\Gd$与$\Fs$型集可测。
  \end{theorem}
  \begin{proposition}
    可测集平移后可测。
  \end{proposition}
  \begin{proof}
    在\cara 条件中将$E$的平移转化为$A$的平移,注意外测度的平移不变即可。
  \end{proof}
  \subsection{Lebesgue可测集的内外逼近}
  \begin{lemma}
    对$A\subset B$,有
    \[ m^*\pare{B-A} = m^*\pare{B} - m^*\pare{A}. \]
  \end{lemma}
  \begin{proof}
  注意由\cara 条件,
  \[ m^*\pare{B} = m^*\pare{B \cap A} + m^*\pare{B-A}.\qedhere \]
  \end{proof}
  \begin{theorem}
    下列条件与$E$的可测性等价。
    \begin{aenum}
      \item 对$\epsilon>0$,存在包含$E$的开集$\open{O}$满足$m^*\pare{\open{O}-E}<\epsilon$;
      \item 存在包含$E$的$\Gd$型集满足$m^*\pare{G-E}=0$;
      \item 对$\epsilon>0$,存在$E$内的闭集$F$满足$m^*\pare{E-F}<\epsilon$;
      \item 存在$E$内的$\Fs$型集满足$m^*\pare{E-F}=0$。
    \end{aenum}
  \end{theorem}
  \begin{proof}
    只证前二者。后二者取补可得。
    \par
    设$E$可测,则存在区间并任意逼近其外测度,取$\open{O}$为区间并即可。有
    \[ m^*\pare{\open{O}-E} = m^*\pare{\open{O}} - m^*\pare{E} < \epsilon. \]
    对于无界$E$,分为可数个有界部分即可。不断缩小$\epsilon$,可得所求$\Gd$型集。鉴于零测集可测, 又$E=G\cap\complement\pare{G-E}$,知$E$可测。
  \end{proof}
  注意到对于任意集合$E$都存在开集使$m^*\pare{\open{O}}-m^*\pare{E}$任意小,然而外测度的减性仅对可测集成立。
  \begin{theorem}
    \label{thm:lt1}
    对有限测度的$E\subset \R$,存在有限多个区间的并$\open{O}$满足$m*\pare{E-\open{O}} + m^*\pare{\open{O}-E} < \epsilon$。
  \end{theorem}
  \begin{proof}
    取开集$U$为$E$的$\epsilon/2$外逼近,写$U$为区间并,选取其中有限个以$\epsilon/2$逼近之,注意到两差均小于$\epsilon/2$即可。
  \end{proof}
  \subsection{Lebesgue测度的其他性质}
  \begin{definition}
    对可测集定义其Lebesgue测度为外测度。
  \end{definition}
  \begin{theorem}
    Lebesgue测度是可数可加的。
  \end{theorem}
  \begin{proof}
    $m\pare{\cup}\le\sum m$由次可加性可得,由有限可加性和单调性又有$m\pare{\cup}\ge\sum^n m$,让右侧$n\to\infty$即可。
  \end{proof}
  \begin{theorem}
    $\R$中可测集包含Borel \siga 。区间测度为长度,且平移不变,可数可加。
  \end{theorem}
  \begin{definition}
    一个可数集族称为升链,如果$E_k\subset E_{k+1}$,相似定义降链。
  \end{definition}
  \begin{theorem}
    Lebesgue测度满足
    \begin{aenum}
      \item 若$\curb{A_k}$为升链,则
      \[ m\pare{\bigcup A_k} = \lim_{k\to\infty}m\pare{A_k}. \]
      \item 若$\curb{B_k}$为降链且$m\pare{B_1}<\infty$,则
      \[ m\pare{\bigcap B_k} = \lim_{k\to\infty}m\pare{B_k}. \]
    \end{aenum}
  \end{theorem}
  \begin{proof}
    不妨设诸$A_k$测度有限,则构造$A_k$的差得到等价的无交序列,后应用可数可加性即可。
    \par
    对于$B$则关于$B_1$取补后构造等价无交序列,借助减性即可。
  \end{proof}
  \begin{definition}
    称一性质在$E$上几乎处处成立,如果它在除一零测集外成立。
  \end{definition}
  \begin{lemma}[Borel-Cantelli]
    若$\curb{E_k}$测度和有限,则几乎任意$x\in\R$最多属于有限多个$E_k$。
  \end{lemma}
  \begin{proof}
    \[ m\pare{\bigcap\bigcup_{k=n}E_k}=\lim_{n\to\infty}m\pare{\bigcup_{k=n}E_k} = 0.\qedhere \]
  \end{proof}
  \subsection{不可测集}
  \begin{lemma}
    设$E\subset\R$有界且存在可数无限有界实数集$\Lambda$其元素使诸$\lambda+E$无交,则$m\pare{E}=0$。
  \end{lemma}
  \begin{proof}
    注意平移不变性与可数可加性,以及有界性即可。
  \end{proof}
  \begin{definition}
    定义二实数有理等价,若其差为有理数。
  \end{definition}
  \begin{theorem}[Vitali]
    任意正测度的实数集$E$存在一不可测子集。
  \end{theorem}
  \begin{proof}
    不妨设$E$有界,取$E$内有理等价类的代表元集$\mathcal{C}$,有上述引理知$m\pare{\mathcal{C}}=0$。再选取$\Lambda$为$\Q$足够大的子集,使诸$\lambda+E$可覆盖$E$,矛盾。
  \end{proof}
  \begin{theorem}
    存在$\R$的无交子集$A$与$B$满足
    \[ m^*\pare{A\cup B} < m^*\pare{A} + m^*\pare{B}. \]
  \end{theorem}
  \subsection{Cantor集与Cantor-Lebesgue函数}
  \begin{definition}
    定义Cantor集为$I=\brac{0,1}$不断挖去各连通分量之三等分之中间部分的结果。令诸$C_k$为每一步的结果,$\mathbf{C}=\cap C_k$。
  \end{definition}
  \begin{theorem}
    Cantor集不可数,且$m\pare{\mathbf{C}}=0$。
  \end{theorem}
  \begin{proof}
    易证其可测且测度为零。参考\tref{uncountableR}的证明过程知不可数。
  \end{proof}
  定义Cantor-Lebesgue函数$\varphi$函数如下。对$\open{O}_k=\brac{0,1}-C_k$的$2^k-1$个连通分量分别赋值
  \[ \curb{1/2^k,2/2^k,3/2^k,\cdots,\pare{2^k-1}/2^k}. \]
  令$\varphi\pare{0}=0$且
  \[ \varphi\pare{x} = \sup\setcond{\varphi\pare{t}}{t\in\blr{0,x}}. \]
  \begin{theorem}
    $\varphi$连续单调递增且在$\open{O}$内导数为零,并将$\brac{0,1}$映满$\brac{0,1}$。
  \end{theorem}
  \begin{proof}
    注意$\varphi$在$x\in\mathbf{C}$附近的跳跃不超过其两侧$\open{O}$的跳跃,而随$k$增大其可任意小。故其连续,由介值定理知映满。
  \end{proof}
  \begin{theorem}
    \label{thm:cantorl}
    连续严格递增映射$\psi\pare{x}=\varphi\pare{x}+x$满足:
    \begin{aenum}
      \item 将零测$\mathbf{C}$映为一正测集;
      \item 将一可测$E\subset\mathbf{C}$映为不可测集。
    \end{aenum}
  \end{theorem}
  \begin{proof}
    注意到$\brac{0,2}=\psi\pare{\open{O}}+\psi\pare{\mathbf{C}}$且开集与闭集映射后仍为开集与闭集,故仍可测。将$\open{O}$分解成区间,映射后区间长度不变即知$m\pare{\psi\pare{\open{O}}}=1$。
    \par
    因此,$m\pare{\psi\pare{\mathbf{C}}}=1$而含有不可测集,其原像为零测可测集。
  \end{proof}
  \begin{lemma}
    严格递增映射存在连续逆。
  \end{lemma}
  \begin{lemma}
    连续映射$f$的Borel集像的原像为Borel集。
  \end{lemma}
  \begin{proof}
    注意$f^{-1}\pare{\complement U}=\complement f^{-1}\pare{U}$,$f^{-1}\pare{A\cap B}=f^{-1}\pare{A} \cap f^{-1}\pare{B}$。
  \end{proof}
  \begin{theorem}
    存在非Borel集的可测集。
  \end{theorem}
  \begin{proof}
    Borel集经严格增映射后仍为Borel集,可测集映射后可能不可测。
  \end{proof}
  \section{可测函数}
  \subsection{可测函数的和、积与复合}
  \begin{proposition}
    对于在可测集上定义的函数$f$,下列命题等价。
    \begin{enumerate}
      \item 对任意$c$,$f\pare{x}>c$的$x$可测;
      \item 对任意$c$,$f\pare{x}\ge c$的$x$可测;
      \item 对任意$c$,$f\pare{x}<c$的$x$可测;
      \item 对任意$c$,$f\pare{x}\le c$的$x$可测;
    \end{enumerate}
  \end{proposition}
  \begin{proof}
    只证前二。将$f\pare{x}\ge c$的$x$视为诸$f\pare{x}>c-1/k$的交,而$f\pare{x}>c$视为诸$f\pare{x}\ge c+1/k$的并。
  \end{proof}
  \begin{definition}
    可测集上定义的函数$f$称为可测的,若其满足前开命题之一。
  \end{definition}
  \begin{proposition}
    可测集上定义的$f$为可测当且仅当开集的原像均可测。
  \end{proposition}
  \begin{proof}
    注意开集可写为区间并,而$\pare{a,b}=\pare{-\infty,b}\cap\pare{a,+\infty}$。
  \end{proof}
  \begin{proposition}
    可测集上定义的连续函数可测。
  \end{proposition}
  \begin{proposition}
    区间上定义的单调函数可测。
  \end{proposition}
  \begin{proposition}
    设$\func{f}{E}{\clo{\R}}$。
    \begin{enumerate}
      \item 若$f$可测而$g$与$f$几乎处处相等,则$g$可测;
      \item 设$D$为可测子集,$f$可测当且仅当在$D$和$E-D$上可测。
    \end{enumerate}
  \end{proposition}
  \begin{theorem}
    $f$和$g$为几乎处处有界的可测函数,则$ \alpha f + \beta g $与$ fg $可测。
  \end{theorem}
  \begin{proof}
    只证$f+g$和$fg$可测。$f+g<c$,则存在$q\in\Q$满足$f<q<c-g$,将诸可数个$q$并起即可。又注意
    \[ fg=\half\brac{\pare{f+g}^2-f^2-g^2} \]
    以及可测函数的平方可测即可。
  \end{proof}
  \begin{ex}
    由\tref{cantorl}可知,可测函数的复合$\chi_E\comp\psi^{-1}>0$的原像$\psi\pare{E}$不可测。
  \end{ex}
  \begin{theorem}
    设$f$连续可测而$g$可测,则$f\comp g$可测。
  \end{theorem}
  \begin{proof}
    注意$\pare{f\comp g}^{-1}\pare{\open{O}} = g^{-1}\pare{f^{-1}\pare{\open{O}}}$即可。
  \end{proof}
  由是立得$\abs{\pare{f}}$与$\abs{\pare{f}}^p$可测。
  \begin{proposition}
    $\maxo{f_1,\cdots,f_n}$与$\mino{f_1,\cdots,f_n}$可测。
  \end{proposition}
  由是立得诸
  \[ \abs{f} = \maxo{f,-f},\quad f^+=\maxo{f,0},\quad f^-=\maxo{-f,0} \]
  可测。故$f$可写为可测函数之差$f=f^+-f^-$。
  \subsection{可测函数的极限与逼近}
  \begin{definition}
    称$\curb{f_n}$一致收敛于$f$,若对于充分大的$n$有$\norm{f-f_n}<\epsilon$。
  \end{definition}
  \begin{proposition}
    若可测函数列$\curb{f_n}$逐点收敛于$f$,则$f$可测。
  \end{proposition}
  \begin{proof}
    若$f\pare{x}<c$,对于充分大的$N$有$f_{N:}\pare{x}<c$,并起诸$N$即可。
  \end{proof}
  \begin{definition}
    简单函数为仅取有限多个值的可测函数。
  \end{definition}
  注意简单函数$\varphi$均可写为
  \[ \varphi = \sum^n c_k\cdot \chi_{E_k}. \]
  \begin{lemma}[简单函数逼近]
    可测函数存在\epsclo 的上下逼近$\varphi_\epsilon$与$\psi_\epsilon$。
  \end{lemma}
  \begin{proof}
    将可测函数的值域分割为若干$\epsilon$小区间即可。
  \end{proof}
  \begin{theorem}[简单函数逼近]
    可测函数存在满足$\abs{\varphi_n}<\abs{\func{f}{E}{\clo{\R}}}$的逼近。若$f$恒正,则存在诸$\varphi_n$递增。
  \end{theorem}
  \begin{proof}
    设$f$恒正。在第$n$步截断$f$的值域至$n$后作$1/n$逼近即可。取$\varphi_n=\maxo{\varphi_1,\cdots,\varphi_n}$可得递增序列。
    \par
    一般情形将$f$写为$f^+-f^-$即可。
  \end{proof}
  \subsection{Littlewood的三大原理}
  三大原理谓
  \begin{enumerate}
    \item 每个\emph{可测}集都\emph{几乎}是区间的并;(\tref{lt1})
    \item 每个\emph{可测}函数都\emph{几乎}是连续的;(\tref{Lusin})
    \item 每个\emph{可测}函数的逐点收敛序列都\emph{几乎}是一致收敛的。(\tref{Egoroff})
  \end{enumerate}
  \begin{lemma}
    对有限测度的$E$上定义的逐点收敛可测函数列$\curb{f_n}\to f$,存在充分大的$N$使$f_{N:}$在任意逼近$E$的集合上任意逼近$f$。
  \end{lemma}
  \begin{proof}
    注意由逐点收敛,诸$N$的$A$为升列且并为$E$即可。
  \end{proof}
  \begin{theorem}[Egoroff定理]
    \label{thm:Egoroff}
    有限测度的$E$上定义的逐点收敛可测函数列$\curb{f_n}\to f$在一\epsclo $E$的闭集$F$上一致收敛。
  \end{theorem}
  \begin{proof}
    据上引理,对任意$n$取$A_n$与$E$为\close{\epsilon/2^{n+1}}而$f_{N:}$与$f$为\close{1/n},由是其交$A$与$E$为\close{\epsilon}且一致收敛。再取闭集逼近$A$即可。
  \end{proof}
  \begin{proposition}
    对在$E$上定义的简单函数,存在连续函数在任意逼近$E$的集合上与之相等。
  \end{proposition}
  \begin{proof}
    对诸$E_k$选取闭集逼近之,后调用Urysohn引理。
  \end{proof}
  \begin{theorem}[Lusin定理]
    \label{thm:Lusin}
    对可测函数,前开命题成立。
  \end{theorem}
  \begin{proof}
    由简单函数逼近之,后以连续函数逼近之,再选取一致收敛的闭集。
  \end{proof}


%ContentEnds
 
\ifx\allfiles\undefined %如果位置放错,可能出现意外中断
\end{document}
\fi
  %Ch9.IntegrationTheory.tex
\ifx\allfiles\undefined
\documentclass{ctexrep}
% Mathematics Include

\usepackage{amsmath}
\usepackage{amssymb}
\usepackage{amsthm}
\usepackage{amsfonts}
\usepackage{mathrsfs}
\usepackage{enumitem}
\usepackage{braket}
\usepackage{hyperref}
\usepackage[all, pdf]{xy}

% Physics Include
\usepackage{amsmath}
\usepackage{physics}
\usepackage{siunitx}
\usepackage[makeroom]{cancel}
\usepackage{pstricks}
\usepackage{pstricks-add}
\psset{algebraic=true}

\usepackage[version=4]{mhchem}
\usepackage{array,booktabs}
\usepackage{longtable}
\usepackage{mathtools}
\usepackage[normalem]{ulem}
\usepackage{multicol}

% Mathematics Head

\newcommand{\pare}[1]{\left(#1\right)}
\newcommand{\blr}[1]{\left[#1\right)}
\newcommand{\lbr}[1]{\left(#1\right]}
\newcommand{\brac}[1]{\left[#1\right]}
\newcommand{\curb}[1]{\left\{#1\right\}}
% \newcommand{\abs}[1]{\left|\, #1 \,\right|}
\newcommand{\rec}[1]{\frac{1}{#1}}
\newcommand{\N}{\mathbb{N}}
\newcommand{\bC}{\mathbb{C}}
\newcommand{\Q}{\mathbb{Q}}
\newcommand{\Z}{\mathbb{Z}}
\newcommand{\R}{\mathbb{R}}
\newcommand{\unk}{\mathcal{X}}
\newcommand{\bu}[3]{#1_{#2}^{\pare{#3}}}
\newcommand{\dref}[1]{定义\ref{def:#1}}
\newcommand{\tref}[1]{定理\ref{thm:#1}}
\newcommand{\lref}[1]{引理\ref{lem:#1}}
\newcommand{\cref}[1]{推论\ref{coll:#1}}
\newcommand{\pref}[1]{命题\ref{prp:#1}}
\newcommand{\eref}[1]{例\ref{ex:#1}}
\newcommand{\func}[3]{#1:\, #2 \rightarrow #3}
\newcommand{\overbar}[1]{\mkern 1.5mu\overline{\mkern-1.5mu#1\mkern-1.5mu}\mkern 1.5mu}
\newcommand{\clo}[1]{\overbar{#1}}
\newcommand{\supi}[2]{\overbar{\int_{#1}^{#2}}}
\newcommand{\infi}[2]{\underbar{\int_{#1}^{#2}}}
\newcommand{\setf}{\mathscr}
\newcommand{\bool}{\mathrm{bool}}
\newcommand{\inc}{++}
\newcommand{\defeq}{:=}
\newcommand{\ntuple}{$n$元组}
\newcommand{\card}[1]{\#\pare{#1}}
\newcommand{\setcond}[2]{\curb{#1 \, \left| \, #2 \right.}}
\newcommand{\setcondl}[2]{\curb{\left. #1 \, \right| \, #2}}
\newcommand{\bv}[1]{\mathbf{#1}}
\newcommand{\bfa}{\bv{a}}
\newcommand{\bfb}{\bv{b}}
\newcommand{\bfx}{\bv{x}}
\newcommand{\bfy}{\bv{y}}
\newcommand{\bfe}{\bv{e}}
\newcommand{\bfF}{\bv{F}}
\newcommand{\bff}{\bv{f}}
\newcommand{\bfG}{\bv{G}}
\newcommand{\bfH}{\bv{H}}
\newcommand{\bfg}{\bv{g}}
\newcommand{\bfh}{\bv{h}}
\newcommand{\bfr}{\bv{r}}
\newcommand{\bfk}{\bv{k}}
\newcommand{\bfu}{\bv{u}}
\newcommand{\bfv}{\bv{v}}
\newcommand{\oo}[1]{o\pare{#1}}
\newcommand{\OO}[1]{O\pare{#1}}
% \newcommand{\norm}[1]{\left\| #1 \right\|}
\newcommand{\DD}{\mathbf{D}}
\newcommand{\comp}{\circ}
\newcommand{\const}{\mathrm{const}}
\newcommand{\dist}[2]{d\pare{#1,#2}}
\newcommand{\len}{\ell}
\newcommand{\siga}{$\sigma$-代数}
\newcommand{\cara}{Carath\'{e}odory}
\newcommand{\Gd}{G_\delta}
\newcommand{\Fs}{F_\sigma}
\newcommand{\mmani}{$m$-维流形}
\newcommand{\open}[1]{\mathcal{#1}}
\newcommand{\half}{\frac{1}{2}}
\newcommand{\maxo}[1]{\text{max}\curb{#1}}
\newcommand{\mino}[1]{\text{min}\curb{#1}}
\newcommand{\epsclo}{$\epsilon$-接近}
\newcommand{\close}[1]{$#1$-接近}
\newcommand{\cinf}{$C^\infty$}
\newcommand{\cuno}{$C^1$}
\newcommand{\Int}{\text{Int}\,}
\newcommand{\Ext}{\text{Ext}\,}
\newcommand{\funcf}{\mathcal}
\newcommand{\DDu}{\overbar{\DD}}
\newcommand{\DDl}{\underbar{\DD}}
\newcommand{\Diff}[1]{\mathrm{Diff}_{#1}\,}
\newcommand{\Av}[1]{\mathrm{Av}_{#1}\,}
\newcommand{\Lip}[1]{Lipschitz-$#1$}
\newcommand{\sgn}{\mathrm{sgn}}
\newcommand{\eset}{\varnothing}
\newcommand{\cT}{\mathcal{T}}
\newcommand{\cS}{\mathcal{S}}
\newcommand{\cG}{\mathcal{G}}
\newcommand{\cF}{\mathcal{F}}
\newcommand{\cC}{\mathcal{C}}
\newcommand{\cB}{\mathcal{B}}
\newcommand{\inter}[1]{\mathring{#1}}
\newcommand{\forest}[3]{对于{#1},存在{#2},使得{#3}}
\newcommand{\tuno}{$T_1$公理}
\newcommand{\isom}{\overset{\sim}{=}}
\newcommand{\diam}{\mathrm{diam}\,}
\newcommand{\ord}[1]{\abs{#1}}
\newcommand{\sbm}[1]{\overbar{#1}}
\newcommand{\inv}[1]{#1^{-1}}
\newcommand{\restr}[2]{#1|_{#2}}
\newcommand{\divs}{|}
\newcommand{\ndivs}{\nmid}
\newcommand{\modeq}[1]{\overbar{#1}}
\newcommand{\ggen}[1]{\langle#1\rangle}
\newcommand{\ggencond}{\braket}

\newcommand{\hd}{H\"{o}lder}

\renewcommand{\proofname}{证明}

\newenvironment{cenum}{\begin{enumerate}\itemsep0em}{\end{enumerate}}

\newtheorem{definition}{定义}[section]
\newtheorem{lemma}{引理}[section]
\newtheorem{theorem}{定理}[section]
\newtheorem{collary}{推论}[section]
\newtheorem{corollary}{推论}[section]
\newtheorem{proposition}{命题}[section]
\newtheorem{axiom}{公理}[section]
\newtheorem{ex}{例}[section]
\newtheorem{reflection}{反射}[section]
\newcommand{\refl}[1]{\vspace{0.5em}\par\noindent\fbox{%
    \parbox{0.9\textwidth}{%
    \begin{reflection}
        #1
    \end{reflection}
    }%
}\vspace{0.5em}\par}
\newcommand{\rref}[1]{反射\ref{refl:#1}}
\newcommand{\tbref}[1]{表\ref{table:#1}}
\allowdisplaybreaks

\newenvironment{aenum}{\begin{enumerate}[label=\textnormal{(\alph*)}]}{\end{enumerate}}

% Physics Head

\DeclareSIUnit\dyne{dynes}

\newcommand{\ddel}[1]{\frac{\partial}{\partial #1}}
\newcommand{\ddelon}[2]{\frac{\partial #1}{\partial #2}}
\newcommand{\dddel}[1]{\frac{\partial^2}{\partial^2 #1}}
\newcommand{\ddt}{\ddel{t}}
\newcommand{\ddT}{\ddel{T}}
\newcommand{\ddV}{\ddel{V}}
\newcommand{\ddr}{\ddel{r}}
\newcommand{\ddth}{\ddel{\theta}}
\newcommand{\ddph}{\ddel{\phi}}
\newcommand{\dddt}{\dddel{t}}
\newcommand{\dddr}{\dddel{t}}
\newcommand{\dddth}{\dddel{\theta}}
\newcommand{\dddph}{\dddel{\phi}}
\newcommand{\rd}[1]{\mathrm{d} #1}
\newcommand{\dt}{\rd{t}}
\newcommand{\dy}{\rd{y}}
\newcommand{\dx}{\rd{x}}
\newcommand{\edd}[1]{\frac{\mathrm{d}}{\mathrm{d} #1}}
\newcommand{\eddd}[1]{\frac{\mathrm{d}^2}{\mathrm{d}^2 #1}}
\newcommand{\eddon}[2]{\frac{\mathrm{d} #1}{\mathrm{d} #2}}
\newcommand{\edddon}[2]{\frac{\mathrm{d}^2 #1}{\mathrm{d}^2 #2}}
\newcommand{\edt}{\edd{t}}
\newcommand{\edton}[1]{\eddon{#1}{t}}
\newcommand{\edT}{\edd{T}}
\newcommand{\edr}{\edd{r}}
\newcommand{\edl}{\edd{l}}
\newcommand{\edx}{\edd{x}}
\newcommand{\edth}{\edd{\theta}}
\newcommand{\eddton}[1]{\edddon{#1}{t}}
\newcommand{\eddzon}[1]{\edddon{#1}{z}}
\newcommand{\vect}[1]{\boldsymbol{#1}}
\newcommand{\alp}{\frac{1}{\sqrt{2}}}
\newcommand{\alpi}{\frac{i}{\sqrt{2}}}
\newcommand{\expc}[1]{\langle#1\rangle}
\newcommand{\bkn}[1]{\bra{#1}\ket{#1}}
\newcommand{\bk}[2]{\bra{#1}\ket{#2}}
\newcommand{\bik}[3]{\bra{#1} #2 \ket{#3}}
\newcommand{\vari}[1]{\sigma_{#1}}
\newcommand{\intc}[2]{\left[#1, #2\right]}
\newcommand{\sch}{Schr\"{o}dinger}
\newcommand{\moment}{\boldsymbol{p}}
\newcommand{\coor}{\boldsymbol{x}}
\newcommand{\lapc}{\nabla^2}
% \newcommand{\rec}[1]{\frac{1}{#1}}
\newcommand{\vva}{\boldsymbol{a}}
\newcommand{\vvb}{\boldsymbol{b}}
\newcommand{\vc}{\boldsymbol{c}}
\newcommand{\vd}{\boldsymbol{d}}
\newcommand{\ve}{\boldsymbol{e}}
\newcommand{\vf}{\boldsymbol{f}}
\newcommand{\vg}{\boldsymbol{g}}
\newcommand{\vh}{\boldsymbol{h}}
\newcommand{\vi}{\boldsymbol{i}}
\newcommand{\vj}{\boldsymbol{j}}
\newcommand{\vk}{\boldsymbol{k}}
\newcommand{\vl}{\boldsymbol{l}}
\newcommand{\vm}{\boldsymbol{m}}
\newcommand{\vn}{\boldsymbol{n}}
\newcommand{\vo}{\boldsymbol{o}}
\newcommand{\vp}{\boldsymbol{p}}
\newcommand{\vq}{\boldsymbol{q}}
\newcommand{\vr}{\boldsymbol{r}}
\newcommand{\vs}{\boldsymbol{s}}
\newcommand{\vt}{\boldsymbol{t}}
\newcommand{\vvu}{\boldsymbol{u}}
\newcommand{\vv}{\boldsymbol{v}}
\newcommand{\vw}{\boldsymbol{w}}
\newcommand{\vx}{\boldsymbol{x}}
\newcommand{\vy}{\boldsymbol{y}}
\newcommand{\vz}{\boldsymbol{z}}
\newcommand{\vA}{\boldsymbol{A}}
\newcommand{\vB}{\boldsymbol{B}}
\newcommand{\vC}{\boldsymbol{C}}
\newcommand{\vD}{\boldsymbol{D}}
\newcommand{\vE}{\boldsymbol{E}}
\newcommand{\vF}{\boldsymbol{F}}
\newcommand{\vG}{\boldsymbol{G}}
\newcommand{\vH}{\boldsymbol{H}}
\newcommand{\vI}{\boldsymbol{I}}
\newcommand{\vJ}{\boldsymbol{J}}
\newcommand{\vK}{\boldsymbol{K}}
\newcommand{\vL}{\boldsymbol{L}}
\newcommand{\vM}{\boldsymbol{M}}
\newcommand{\vN}{\boldsymbol{N}}
\newcommand{\vO}{\boldsymbol{O}}
\newcommand{\vP}{\boldsymbol{P}}
\newcommand{\vQ}{\boldsymbol{Q}}
\newcommand{\vR}{\boldsymbol{R}}
\newcommand{\vS}{\boldsymbol{S}}
\newcommand{\vT}{\boldsymbol{T}}
\newcommand{\vU}{\boldsymbol{U}}
\newcommand{\vV}{\boldsymbol{V}}
\newcommand{\vW}{\boldsymbol{W}}
\newcommand{\vX}{\boldsymbol{X}}
\newcommand{\vY}{\boldsymbol{Y}}
\newcommand{\vZ}{\boldsymbol{Z}}
\newcommand{\vzero}{\boldsymbol{0}}
\newcommand{\vomega}{\boldsymbol{\omega}}
%\newcommand{\half}{\frac{1}{2}}
\newcommand{\thalf}{\frac{3}{2}}
\newcommand{\rot}{\nabla\times}
\newcommand{\divg}{\nabla\cdot}
\newcommand{\cE}{\mathcal{E}}
\newcommand{\conclu}[1]{\vspace{1em}\par\noindent\fbox{\parbox{0.9\textwidth}{#1}}\vspace{1em}}
\newcommand{\subentrynote}{$\bullet$}
\newcommand{\keypoint}[1]{\par\subentrynote\quad #1 \par}
\newcommand{\fconclu}{\boxed}
\newcommand{\pair}[2]{#1 \, #2}
\newcommand{\intn}[2]{\int #1 \,\mathrm{d} #2}
\newcommand{\intu}[3]{\int_0^{#1} #2 \,\mathrm{d} #3}
\newcommand{\intiu}[3]{\int_{-\infty}^{#1} #2 \, \rd{} #3}
\newcommand{\intui}[2]{\int_0^{\infty} #1 \,\mathrm{d} #2}
\newcommand{\intii}[2]{\int_{-\infty}^{\infty} #1 \,\mathrm{d} #2}
\newcommand{\intt}[2]{\int_0^\infty #1 \, \rd{} #2}
\newcommand{\intr}[2]{\int_{-\infty}^{\infty} #1 \, \rd{} #2}
\newcommand{\intbi}[3]{\int_{#1}^{\infty} #2 \, \rd{} #3}
\newcommand{\intab}[4]{\int_{#1}^{#2} #3 \, \rd{} #4}
\newcommand{\bfactor}[1]{e^{-#1/k_BT}}
\newcommand{\pbfactor}[1]{e^{#1/k_BT}}
\newcommand{\dn}[2]{#1^{\pare{#2}}}
\newcommand{\prodg}[1]{\pare{#1}^\times}

\newcommand{\notion}{\emph}
\newcommand{\iP}{\mathcal{P}}
\newcommand{\eiP}{e^{-\iP}}
\newcommand{\iF}{\mathcal{F}}
\newcommand{\eiF}{e^{-\iF}}
\newcommand{\iG}{\mathcal{G}}

\newcommand{\rc}{r\cos\theta}
\newcommand{\rs}{r\sin\theta}
\newcommand{\sn}{\mathrm{sn}}
\newcommand{\cn}{\mathrm{cn}}
\newcommand{\rdn}{\mathrm{dn}}

\newcommand{\hankel}{H_p^{\pare{1}}}
\newcommand{\hankell}{H_p^{\pare{2}}}
\newcommand{\hhankel}{H_n^{\pare{1}}}
\newcommand{\hhankell}{H_n^{\pare{2}}}
\newcommand{\ber}{\text{ber}\,}
\newcommand{\bei}{\text{bei}\,}
\newcommand{\kker}{\text{ker}\,}
\newcommand{\kei}{\text{kei}\,}
\newcommand{\Ai}{\text{Ai}}
\newcommand{\Bi}{\text{Bi}}

\newcommand{\re}{\text{Re}\,}

\newcommand{\Fp}{F_\phi}
\newcommand{\Ep}{E_\phi}
\newcommand{\Fx}{F_x}
\newcommand{\FF}{\mathbf{F}}
\newcommand{\Ex}{E_x}

%\newcommand{\erf}{\mathrm{erf}}
\newcommand{\erfi}{\mathrm{erfi}}
\newcommand{\erfc}{\mathrm{erfc}}
\newcommand{\ehxs}[1]{e^{-\frac{#1^2}{2}}}
\newcommand{\dcol}[2]{\[ \left.#1 \hspace{1em}\right\vert\hspace{1em} #2 \]}
\newcommand{\titlegamma}{\texorpdfstring{$\Gamma$}{Gamma}}
\newcommand{\titleB}{\texorpdfstring{$B$}{B}}

% Computer Science Head
\usepackage{listings}
\usepackage{color}

\definecolor{dkgreen}{rgb}{0,0.6,0}
\definecolor{gray}{rgb}{0.5,0.5,0.5}
\definecolor{mauve}{rgb}{0.58,0,0.82}

\lstset{frame=tb,
  language=Java,
  aboveskip=3mm,
  belowskip=3mm,
  showstringspaces=false,
  columns=flexible,
  basicstyle={\small\ttfamily},
  numbers=none,
  numberstyle=\tiny\color{gray},
  keywordstyle=\color{blue},
  commentstyle=\color{dkgreen},
  stringstyle=\color{mauve},
  breaklines=true,
  breakatwhitespace=true,
  tabsize=3
}
\lstset{language=Java}
\newcommand{\snp}[1]{\lstinline!#1!}
\newcommand{\term}[2]{\textbf{#1(#2)}}
\begin{document}
\fi

%Content

\chapter{积分论}
  \section{Lebesgue积分}
  \subsection{Riemann积分}
  Riemann积分的定义如前不赘,唯注意下例。
  \begin{ex}
    对于Dirichlet函数
    \[
      f\pare{x}=
      \begin{cases}
        1, \quad x \in \Q, \\
        0, \quad x \in \brac{0,1}-\Q.
      \end{cases}
    \]
    虽可写为可数个简单函数之和,亦知其非Riemann可积。
  \end{ex}
  \subsection{有界函数在有限测度集上的Lebesgue积分}
  \begin{definition}
    对于有限测度集$E$上的简单函数$\psi$,定义其积分如
    \[ \int_E\psi = \sum a_i\cdot m\pare{E_i}. \]
    表达式中诸$a_i$不等。
  \end{definition}
  \begin{lemma}
    纵表达式中$a_i$简并,亦无改其积分值。
  \end{lemma}
  \begin{proposition}[积分的线性与单调性]
    对于简单函数$\varphi$与$\psi$,有
    \[ \int\pare{\alpha\varphi+\beta\psi} = \alpha\int\varphi + \beta\int\psi. \]
    以及若$\varphi<\psi$,则
    \[ \int\varphi < \int\psi. \]
  \end{proposition}
  \begin{proof}
    将$\varphi$与$\psi$共用一组$E_i$展开即可。
  \end{proof}
  此时已足够推断阶梯函数的Riemann与Lebesgue积分相符。
  \begin{definition}
    对有限测度集上的有界函数$f$,定义其Lebesgue上积分为全体$\varphi>f$之简单函数的Lebesgue积分的下界。相似定义Lebesgue下积分。
  \end{definition}
  \begin{definition}
    前开$f$若Lebesgue上下积分相等,则称之其Lebesgue积分。
  \end{definition}
  \begin{theorem}
    Lebesgue积分兼容Riemann积分。
  \end{theorem}
  \begin{proof}
    注意到阶梯函数含于简单函数即可。
  \end{proof}
  \begin{ex}
    注意Dirichlet函数$f=\chi_\Q$,故$\int f = m\pare{\Q}=0$。
  \end{ex}
  \begin{theorem}
    \label{thm:intable}
    有限测度集上定义的有界函数可积。
  \end{theorem}
  \begin{proof}
    注意其存在简单函数的上下逼近即可。
  \end{proof}
  \begin{proposition}[积分的线性与单调性]
    对于有限测度集上的可测函数$f$与$g$,有
    \[ \int\pare{\alpha f+\beta g} = \alpha\int f + \beta\int g. \]
    以及若$f<g$,则
    \[ \int f < \int g. \]
  \end{proposition}
  \begin{proof}
    只证$\alpha=\beta=1$的情况,目标积分不超二Lebesgue上积分之和而不低于二Legesgue下积分之和,再注意上下积分之和即积分之和。
    \par
    单调性考虑$\int\pare{f-g}$即可。
  \end{proof}
  \begin{collary}
    对无交可测集$A$与$B$,有
    \[ \int_{A\cup B}f=\int_A f+\int_B f. \]
  \end{collary}
  \begin{collary}
    对有限测度集上的有界函数$f$,有
    \[ \abs{\int f} \le \int \abs{f}. \]
  \end{collary}
  \begin{proof}
    注意$-\abs{f}\le f\le \abs{f}$即可。
  \end{proof}
  \begin{proposition}
    若有限测度集上的有界函数列$\curb{f}$一致收敛于$f$,则
    \[ \lim_{n\to\infty}\int f_n = \int \lim_{n\to\infty} f_n. \]
  \end{proposition}
  \begin{proof}
    注意$\norm{f-f_n}$可以任意小,借助前开推论即可。
  \end{proof}
  \begin{ex}
    考虑$f_n$定义为$f\pare{0}=0$,$f\pare{1/n}=n$,$f\pare{2/n}=0$并线性连接,则其除逐点收敛于零外满足前开所有条件,而积分后序列非零。
  \end{ex}
  \begin{theorem}[有界收敛定理]
    若有限测度集上的各点一致有界函数列$\curb{f}$逐点收敛于$f$,则
    \[ \lim_{n\to\infty}\int f_n = \int \lim_{n\to\infty} f_n. \]
  \end{theorem}
  \begin{proof}
    由Egoroff定理,$f_n$在任意接近$E$的闭集上一致收敛。故定义域的残余部分的积分任意小。
  \end{proof}
  \subsection{非负函数的Lebesgue测度}
  \begin{definition}
    定义$f$的支撑为使之非零的定义域部分\footnote{这和拓扑学上定义为其闭包不同。}。
  \end{definition}
  \begin{definition}
    设$f$为$E$上的非负可测函数,定义其积分
    \[ \int_E f = \sup \setcondl{\int_E h}{0\le h \le f}. \]
    其中$h$为有限测度集上定义的有界可测函数。
  \end{definition}
  \begin{proposition}[Chebychev不等式]
    设$f$非负可测,对$\lambda>0$,有
    \[ m\pare{f\ge \lambda} \le \rec{\lambda}\int_E f. \]
  \end{proposition}
  \begin{proof}
    取$g=\lambda\chi_{f\ge\lambda}$,并注意$0\le g\le f$。
  \end{proof}
  \begin{proposition}
    \label{prp:ae0}
    设$f$非负可测,则$\int f = 0$当且仅当$f$几乎处处为零。
  \end{proposition}
  \begin{proof}
    由前不等式,诸$m\pare{f\le1/n}=0$,并起即可。
  \end{proof}
  \begin{proposition}[积分的线性与单调性]
    对于非负可测函数$f$与$g$,有
    \[ \int\pare{\alpha f+\beta g} = \alpha\int f + \beta\int g. \]
    以及若$f<g$,则
    \[ \int f < \int g. \]
  \end{proposition}
  \begin{proof}
    易证$\int f + \int g \le \int \pare{f+g}$。反向的不等式则注意取$h=\mino{f,l}$,$k=l-h$,则$h$与$k$有界可测且
    \[ \int l = \int\pare{h+k} \le \int f + \int g. \]
    左侧取上界即可。单调性亦左侧取上界可证。
  \end{proof}
  \begin{theorem}[积分区间的可加性]
    $f$非负可测而$A$与$B$为无交可测集,则
    \[ \int_{A\cup B} f = \int_A f + \int_B f. \]
  \end{theorem}
  \begin{lemma}[Fatou引理]
    非负可测函数列$\curb{f_n}$几乎处处逐点收敛于$f$,则
    \[ \int_E f \le \lim\inf\int_E f_n. \]
  \end{lemma}
  \begin{proof}
    除开一零测集,可设其处处收敛。对任意$h$,设$h_n=\mino{h,f_n}$,故$h_n\to h$且由有界收敛定理
    \[ \lim_{n\to\infty} \int_E h_n = \int_E h. \]
    再注意$h_n\le f_n$,$\lim\int h_n \le \lim \inf \int f_n$即可。
  \end{proof}
  \begin{ex}
    令$E=\blr{0,1}$且$f_n=n\chi_{\pare{0,1/n}}$,则$f_n$极限的积分与积分的极限分别为$0$和$1$。再如$\chi_{\pare{n,n+1}}$逐点收敛至$0$但显然积分与极限不可互换。
  \end{ex}
  \begin{theorem}[单调收敛定理]
    在Fatou引理的条件下,若$\curb{f_n}$递增,则
    \[ \lim_{n\to\infty}\int f_n = \int f. \]
  \end{theorem}
  \begin{proof}
    由积分的单调性知
    \[ \lim\sup\int f_n \le \int f. \qedhere \]
  \end{proof}
  \begin{collary}
    非负可测函数和$\sum u_n$几乎处处逐点收敛于$f$,则
    \[ \int f = \sum \int u_n. \]
  \end{collary}
  \begin{definition}
    积分有限的可测函数称为可积函数。
  \end{definition}
  \begin{proposition}
    可积函数几乎处处有限。
  \end{proposition}
  \begin{proof}
    注意对任意$n$,有
    \[ m\pare{f \ge n} \le \rec{n}\int f. \]
  \end{proof}
  \begin{lemma}[Beppo Levi引理]
    非负可测函数列$\curb{f_n}$诸积分一致有界,则$f_n$逐点收敛于一几乎处处有界的可积函数。
  \end{lemma}
  \begin{proof}
    递增数列收敛于一广义实数,故定义$f\pare{x}=\lim f_n\pare{x}$,复用前开命题与有界收敛定理。
  \end{proof}
  \subsection{一般Lebesgue积分}
  注意$f=f^+-f^-$且$\abs{f} = f^++f^-$。
  \begin{proposition}
    对可测函数$f$,$f^+$与$f^-$可积当且仅当$\abs{f}$可积。
  \end{proposition}
  \begin{definition}
    若$\abs{f}$可积则称可测函数$f$可积且定义
    \[ \int f = \int f^+ - \int f^-. \]
  \end{definition}
  \begin{proposition}
    \label{prp:Ehole}
    若$f$可积,则$\abs{f}$几乎处处有限且对零测集$E_0$,
    \[ \int_E f = \int_{E-E_0} f. \]
  \end{proposition}
  \begin{proof}
    前开命题知几乎处处有限。再注意对非负函数有相同成立即可。
  \end{proof}
  \begin{proposition}[比较审敛法]
    若$\abs{f}$处处小于一可积函数,则$f$可积且
    \[ \abs{\int f} \le \int \abs{f}. \]
  \end{proposition}
  \begin{proof}
    可积性易证。再由实数的三角不等式,
    \[ \abs{\int f^+ - \int f^-} \le \int f^+ + \int f^- \le \int \abs{f}. \]
  \end{proof}
  注意由\pref{Ehole},两可积函数若某处值无限,则积分可径直挖去该点而无需定义在该点的值。
  \begin{proposition}[积分的线性与单调性]
    对于可积函数$f$与$g$,有
    \[ \int\pare{\alpha f+\beta g} = \alpha\int f + \beta\int g. \]
    以及若$f<g$,则
    \[ \int f < \int g. \]
  \end{proposition}
  \begin{proof}
    可积性由$\abs{f+g}\le\abs{f}+\abs{g}$得,其余易证。
  \end{proof}
  \begin{collary}
    对无交可测集$A$与$B$,有
    \[ \int_{A\cup B}f=\int_A f+\int_B f. \]
  \end{collary}
  \begin{theorem}[Lebesgue控制收敛定理]
    逐点收敛于$f$的可测函数列$\curb{f_n}$满足$\abs{f_n}\le g$,则有$f$可积且
    \[ \lim_{n\to\infty}\int f_n = \int f. \]
  \end{theorem}
  \begin{proof}
    注意到由Fatou引理,
    \[ \int \pare{g+f} \le \lim \inf \int \pare{g+f_n},  \]
    以及
    \[ \int \pare{g-f} \le \lim \inf \int \pare{g-f_n}. \qedhere \]
  \end{proof}
  \begin{theorem}[一般的Lebesgue控制收敛定理]
    逐点收敛于$f$的可测函数列$\curb{f_n}$满足$\abs{f_n}\le g_n$,若$\curb{g_n}$几乎处处收敛于$g$,且
    \[ \lim_{n\to\infty}\int g_n = \int g, \]
    则有$f$可积且
    \[ \lim_{n\to\infty}\int f_n = \int f. \]
  \end{theorem}
  \begin{proof}
    证法同上。
  \end{proof}
  \subsection{积分的可数可加性与连续性}
  \begin{theorem}[积分的可数可加性]
    设$f$可积而$\curb{E_n}$为无交可测集族,其并为$E$,则
    \[ \int f = \sum \int_n f. \]
  \end{theorem}
  \begin{proof}
    对$f_n = f \chi_{E_1\cap\cdots\cap E_n}$应用控制收敛定理。
  \end{proof}
  \begin{theorem}[积分的连续性]
    $f$为$E$上的可积函数,则
    \begin{aenum}
      \item 若$\curb{A_k}$为升链,则
      \[ \int_{\cup A_n} f = \lim_{n\to\infty} \int_{A_n} f. \]
      \item 若$\curb{B_k}$为降链,则
      \[ \int_{\cap B_n} f = \lim_{n\to\infty} \int_{B_n} f. \]
    \end{aenum}
  \end{theorem}
  \subsection{一致可积性}
  \begin{lemma}
    有限测度集可以被划分为有限个测度小于$\delta$的无交集。
  \end{lemma}
  \begin{proof}
    注意$m\pare{E-\brac{-n,n}}$迟早小于$\delta$后划分$\brac{-n,n}$即可。
  \end{proof}
  \begin{proposition}
    \label{prp:previtali}
    $f$在$E$上可积,则对于任意小的$\epsilon$,存在$\delta$使得对任意满足$m\pare{A}<\delta$的子集$A$有
    \[ \int_A\abs{f}<\epsilon. \]
    反之,若$E$测度有限而对任意小的$\epsilon$,存在上述的$\delta$,则$f$可积。
  \end{proposition}
  \begin{proof}
    仅考虑正的$f$。正向结论可由定义以有界函数逼近$f$并注意有界性推得。反向结论则选取一对$\epsilon$与$\delta$,并由前引理将$E$写为有限个小集的并。
  \end{proof}
  \begin{definition}
    $E$上的可测函数族称为一致可积,若对于任意小的$\epsilon$,存在$\delta$使得对任意$m\pare{A}<\delta$以及其中的$f$,有
    \[ \int_A \abs{f} < \epsilon. \]
  \end{definition}
  \begin{ex}
    设$g$可积,所有满足$\abs{f}<g$的可测函数为一致可积。
  \end{ex}
  \begin{proposition}
    有限个可积函数构成的族是一致可积的。
  \end{proposition}
  \begin{proposition}
    若有限测度的$E$上一致可积的$\curb{f_n}$几乎处处逐点收敛于$f$,则$f$可积。
  \end{proposition}
  \begin{proof}
    由\pref{previtali},诸$f_n$的积分一致有界,由Fatou引理
    \[ \int \abs{f} \le \lim\inf \int \abs{f_n}. \qedhere \]
  \end{proof}
  \begin{theorem}[Vitali收敛定理]
    若有限测度的$E$上一致可积的$\curb{f_n}$几乎处处逐点收敛于$f$,则
    \[ \lim_{n\to\infty}\int f_n = \int f. \]
  \end{theorem}
  \begin{proof}
    由Egoroff定理,选取任意逼近$E$的$A$使得$\curb{f_n}$一致收敛,则
    \[ \abs{\int_E f_n - \int_E f} \le \int_{E-A}\abs{f_n-f} + \int_A\abs{f_n} + \int_A\abs{f}. \]
  第一项积分由一致收敛任意小,后二项由一致可积与Fatou引理任意小。
  \end{proof}
  \begin{theorem}
    有限测度集上几乎处处收敛于零的非负可测函数列$\curb{h_n}$,当且仅当其一致可积时有
    \[ \lim_{n\to\infty}\int h_n = 0. \]
  \end{theorem}
  \begin{proof}
    只证极限为零推出一致可积。对任意$\epsilon>0$,可以选取足够大的$N$使
    \[ \int h_{N:} < \epsilon, \]
    再注意有限个$h_{:N}$一致可积即可。
  \end{proof}
  \section{进一步的主题}
  \subsection{一致可积性与测度紧密型}
  \begin{ex}
    对无限测度的$E$,考虑$f=\chi_{\brac{n,n+1}}$知Vitali定理不适用。
  \end{ex}
  \begin{proposition}
    设$f$可积,则在一有限测度集$E_0$外其积分任意小。
  \end{proposition}
  \begin{proof}
    有定义知存在有限测度上有界的函数其积分任意逼近$f$。
  \end{proof}
  \begin{definition}
    $E$上的可测函数族$\funcf{F}$称为紧密的,如果存在一有限测度集$E_0$使其全体在其外的积分一致任意小。
  \end{definition}
  \begin{theorem}[Vitali收敛定理]
    若$E$上一致可积且紧密的$\curb{f_n}$几乎处处逐点收敛于$f$,则
    \[ \lim_{n\to\infty}\int f_n = \int f. \]
  \end{theorem}
  \begin{proof}
    选取$E_0$使其外的积分任意小,其内的积分调用前开Vitali定理。
  \end{proof}
  \begin{collary}
    $E$上几乎处处收敛于零的非负可测函数列$\curb{h_n}$,当且仅当其一致可积且紧密时有
    \[ \lim_{n\to\infty}\int h_n = 0. \]
  \end{collary}
  \subsection{依测度收敛}
  \begin{definition}
    $E$上几乎处处有限的可测函数列$\curb{f_n}$称为依测度收敛于可测的$f$,如果对任意$\eta$,
    \[ \lim_{n\to\infty}m\setcond{x}{\abs{f_n\pare{x}-f\pare{x}}>\eta} = 0. \]
  \end{definition}
  \begin{proposition}
    有限测度$E$上的逐点收敛是一致收敛。
  \end{proposition}
  \begin{proof}
    由Egoroff定理选取逼近$E$的闭集上的一致收敛即可。
  \end{proof}
  \begin{ex}
    考虑下述诸区间上的特征函数,虽依测度收敛却非逐点收敛。
    \[ \brac{0,1},\brac{0,1/2},\brac{1/2,1},\brac{0,1/3},\brac{1/3,2/3},\brac{2/3,1},\brac{0,1/4}\cdots. \]
  \end{ex}
  \begin{theorem}[Riesz]
    依测度收敛的函数列存在几乎处处逐点收敛的子列。
  \end{theorem}
  \begin{proof}
    由依测度收敛知存在子列使$m\pare{\abs{f_{n_k}-f}>1/k}<1/2^k$,再由Borel-Cantelli引理知几乎每个$x$都收敛于$f$。
  \end{proof}
  \begin{collary}
    \label{coll:mcon2int}
    非负可积函数列$\curb{f_n}$满足
    \[ \lim_{n\to\infty} \int f = 0 \]
    当且仅当其依测度收敛于零且一致可积且紧密。
  \end{collary}
  \begin{proof}
    由前开推论与Chebyshev不等式知其依测度收敛。
    \par
    反之,假设积分不收敛于零,则存在一子列之积分漂浮于一实数之上,此子列存在一几乎处处逐点收敛之子列,由Vitali定理知矛盾。
  \end{proof}
  \subsection{Riemann与Lebesgue可积性的特征}
  \begin{lemma}
    设$\curb{\varphi_n}$与$\curb{\psi_n}$分别为可积函数的升列与降列且夹挤$f$,若
    \[ \lim_{n\to\infty}\int\brac{\psi_n-\varphi_n}=0, \]
    则$\curb{\varphi_n}$与$\curb{\psi_n}$几乎处处逐点收敛于$f$且$f$可积且三者积分相等。
  \end{lemma}
  \begin{proof}
    由单调收敛定理,$\int\pare{\psi-\varphi}\to 0$。由\pref{ae0}知其几乎处处为零,从而几乎处处逐点收敛于$f$,进而其可测。三积分相等易证。
  \end{proof}
  \begin{theorem}
    有限测度集上的有界$f$,其可积当且仅当其可测。
  \end{theorem}
  \begin{proof}
    若假设可测,由\tref{intable}知可积。
    \par
    若已知可积,则由定义知存在$f$的上下简单函数逼近且积分差为零,取诸$\maxo{\varphi_i}$与$\mino{\psi_i}$可设其分别为升降列,再调用前开引理。
  \end{proof}
  \begin{theorem}[Lebesgue]
    紧区间上的有界函数$f$为Riemann可积当且仅当其非连续点为零测集。
  \end{theorem}
  \begin{proof}
    假定Riemann可积,则存在一列加细的划分$\curb{P_n}$,对应上下逼近$\curb{\varphi_n}$与$\curb{\psi_n}$且由前开引理几乎处处收敛于$f$。在除开$P_\infty$的点处,对$\epsilon$取足够大的$N$即可使此处$\psi_n-\varphi_n<\epsilon$,从而此点的$\delta$邻域内变差任意小。
    \par
    反之,假定$f$不连续点为零测集。对加细至稠密的划分列$P_n$以及$P_\infty$及不连续点以外的点,选取足够的大$n$使$P_n$的间隙小于$\delta$,则$f$在诸间隙内的变差小$\epsilon$,故上下逼近可互相接近而积分相等。
  \end{proof}
  \section{微分与积分}
  \subsection{单调函数的连续性}
  \begin{theorem}
    单调函数最多仅有可数个不连续点。
  \end{theorem}
  \begin{proposition}
    对开区间内的可数集,存在增函数仅在此可数集上不连续。
  \end{proposition}
  \begin{proof}
    取$f$如下,在任意$E-C$的点有足够小的开区间不包含$q_1,\cdots ,q_n$。
    \[ f\pare{x} = \sum_{q_n\le x} 1/2^n. \qedhere \]
  \end{proof}
  \subsection{单调函数的可微性}
  \begin{definition}
    非退化紧区间集$\setf{F}$称为$E$的Vitali覆盖,如果对于任意点$x$和$\epsilon>0$,存在长度小于$\epsilon$的区间覆盖$x$。
  \end{definition}
  \begin{lemma}[Vitali覆盖引理]
    设$E$为有限外测度集,$\setf{F}$是其Vitali覆盖,则其无交有限子集任意可接近$E$。
  \end{lemma}
  \begin{proof}
    若存在有限子集覆盖之则证毕。反之,依"在剩余无交区间内选取区间长度过半者"之程式选取"下一区间"而得无交可数族,则任意有限子集外的区间与可数族内一区间有交。将后者扩大5倍即可覆盖之。故其有限子集外者扩大5倍后便可覆盖$E$。
  \end{proof}
  \begin{definition}
    定义上导数
    \[ \DDu f\pare{x} = \lim_{h\to 0}\brac{\sup_{0<\abs{t}\le h} \frac{f\pare{x+t}-f\pare{x}}{t}}. \]
    相似定义下导数。若二者相等则称可导。
  \end{definition}
  \begin{lemma}
    设$f$为紧区间上的增函数,则对任意正数$\alpha$,
    \[ m^*\pare{\DDu f \ge \alpha} \le \rec{\alpha}\brac{f\pare{b}-f\pare{a}}. \]
    特别地,$m^*\pare{\DDu f = \infty} = 0$。
  \end{lemma}
  \begin{proof}
    注意诸$\Delta f \ge \alpha^-\pare{d-c}$的区间$\brac{c,d}$构成其Vitali覆盖。
  \end{proof}
  \begin{theorem}[Lebesgue]
    开区间上的单调函数几乎处处可导。
  \end{theorem}
  \begin{proof}
    设$E$中上导数大于$\alpha$而下导数小于$\beta$,则诸$\Delta f \le \beta \pare{d-c}$的$\brac{c,d}$构成$E$的Vitali覆盖。$\sum^n \Delta f \le \beta m^*\pare{E}$。再由前开引理,
    \[ m^*\pare{E}\le \rec{\alpha} \sum^n \Delta f. \qedhere \]
  \end{proof}
  \begin{definition}
    紧区间上的可积函数$f$,两侧水平延伸其值,对正数$h$定义差分与平均分别为
    \[ \Diff{h}f\pare{x}=\frac{f\pare{x+h}-f\pare{h}}{h},\quad \Av{h}f\pare{x} = \rec{h}\int_x^{x+h}f. \]
  \end{definition}
  \begin{collary}
    \label{coll:diff2int}
    紧区间上的增函数,其导数可积且
    \[ \int f' \le f\pare{b} - f\pare{a}. \]
  \end{collary}
  \begin{proof}
    由Fatou引理,
    \[ \int f' \le \lim \inf \int \Diff{} f. \qedhere \]
  \end{proof}
  参考Cantor-Lebesgue函数知等号可严格成立。
  \subsection{有界变差函数:Jordan定理}
  \begin{definition}
    定义变差为
    \[ V\pare{f,P} = \sum \abs{\Delta f}, \]
    全变差为$TV = \sup V$。若全变差有界,则称之有界变差。
  \end{definition}
  \begin{ex}
    增函数,\Lip{1}的函数是有界变差的。$x\cos\pare{\pi/2x}$则不是。
  \end{ex}
  \begin{lemma}
    有界变差函数可以写为如下二增函数之差:
    \begin{equation}
      \label{eq:TVd}
      f\pare{x} = \brac{f\pare{x} + TV\pare{x}} - TV\pare{x}.
    \end{equation}
  \end{lemma}
  \begin{theorem}[Jordan]
    紧区间上的函数有界变差当且仅当其为增函数之差。
  \end{theorem}
  \begin{proof}
    $f=g-h$称为其Jordan分解,只需注意
    \[ V\pare{f,P} = \sum \abs{\Delta f} \le \sum \abs{\Delta g} + \sum \abs{\Delta h}. \qedhere \]
  \end{proof}
  \begin{collary}
    紧区间上的有界变差函数几乎处处可微且导数可积。
  \end{collary}
  \subsection{绝对连续函数}
  \begin{definition}
    对任意$\epsilon>0$,存在$\delta>0$使$\sum \pare{b_k-a_k}$之一切区间族上$\sum \abs{\Delta f} < \epsilon$,则称$f$绝对连续。
  \end{definition}
  \begin{ex}
    Cantor-Lebesgue函数虽连续却非绝对连续。
  \end{ex}
  \begin{proposition}
    \Lip{1}的函数绝对连续。
  \end{proposition}
  \begin{theorem}
    紧区间上的绝对连续函数可写为绝对连续的增函数之差。
  \end{theorem}
  \begin{proof}
    注意绝对连续的$f$的全变差为绝对连续即可。
  \end{proof}
  \begin{theorem}
    \label{thm:abs2unint}
    紧区间上的连续函数绝对连续当且仅当$\lbr{0,1}$的差分一致可积。
  \end{theorem}
  \begin{proof}
    设其差分一致可积,注意$\Delta \Av{h}f = \int \Diff{h} f$以及$\lim \Av{h} f = f$。\par
    反之只证$f$为非负绝对连续增函数的情形,注意只需证"任意小的区间集上的积分任意小"即可,再藉$\int \Diff{h}f = 1/h\cdot\int \brac{f\pare{u+t}-f\pare{v+t}}$。
  \end{proof}
  可以发现如下的包含关系
  \[ \funcf{F}_{Lip} \subset \funcf{F}_{AC} \subset \funcf{F}_{BV}. \]
  且各族内的函数都可以如\eqref{eq:TVd}写成族内二增函数之差。
  \subsection{积分下的微分}
  \begin{theorem}
    \label{thm:int2dif}
    紧区间上的绝对连续函数几乎处处可微且
    \[ \int_a^b f' = f\pare{b} - f\pare{a}. \]
  \end{theorem}
  \begin{proof}
    注意
    \[ \lim_{h\to 0}\int_a^b \Diff{h} f = \lim_{h\to 0} \brac{\Av{h} f\pare{b} - \Av{h} f\pare{a}}. \]
    右侧为所求,左侧由\tref{abs2unint}知一致可积后调用Vitali定理。
  \end{proof}
  \begin{theorem}
    紧区间上的函数一致连续当且仅当其为一不定积分。
  \end{theorem}
  \begin{proof}
    假设$f=\int g$,注意由\pref{previtali},小测度上$g$的积分可任意小。
  \end{proof}
  \begin{collary}
    紧区间上单调的$f$为绝对连续当且仅当
    \[ \int_a^b f' = f\pare{b}-f\pare{a}. \]
  \end{collary}
  \begin{proof}
    由\cref{diff2int},
    \[ \int_a^x f' \le f\pare{x}-f\pare{a},\quad \int_x^b f' \le f\pare{b}-f\pare{x}. \]
    故二者均为相等,从而$f=\int f'$。
  \end{proof}
  \begin{lemma}
    紧区间上的$f$几乎处处为零当且仅当任意区间上积分为零。
  \end{lemma}
  \begin{proof}
    注意任意区间上积分为零得出任意$\Gd$型集上积分为零即可。
  \end{proof}
  \begin{theorem}
    紧区间上的可积函数几乎处处有
    \[ \DD \int_a^x f = f\pare{x}. \]
  \end{theorem}
  \begin{proof}
    借助前开命题,注意对任意$\brac{x_1,x_2}$,有
    \[ \int_{x_1}^{x_2}\brac{F'-f} = F\pare{x_2}-F\pare{x_1} - \int_{x_1}^{x_2} f = 0. \qedhere \]
  \end{proof}
  并非所有函数解有\tref{int2dif}的适用。借助前开命题,考虑下述的分解
  \[ f = \pare{f-\int f'} + \int f', \]
  前者导数几乎处处为零,后者为一绝对连续函数,此谓其Lebesgue分解。
  \subsection{凸函数}
  \begin{definition}
    满足下式者谓凸函数,其中$a+b=1$。
    \[ \varphi\pare{a x_1 + b x_2} \le a \varphi\pare{x_1} + b \varphi\pare{x_2}. \]
  \end{definition}
  在上式中令
  \[ a=\frac{x_2-x}{x_2-x_1},\quad b=\frac{x-x_1}{x_2-x_1}, \]
  可得对于任意$x_1<x<x_2$,
  \[ \frac{\varphi\pare{x}-\varphi\pare{x_1}}{x-x_1} \le \frac{\varphi\pare{x_2}-\varphi\pare{x}}{x_2-x}. \]
  \begin{proposition}
    若$\varphi$可微而$\varphi'$为增函数,则$\varphi$为凸函数。
  \end{proposition}
  \begin{lemma}[弦斜率]
    凸函数上顺次三点$p_1$,$p$,$p_2$,有$k_{p_1p}<k_{p_1p_2}<k_{pp_2}$。
  \end{lemma}
  \begin{lemma}
    凸函数$\varphi$各点左右导数存在,且若$u<v$则
    \[ \varphi'\pare{u^-} \le \varphi'\pare{u^+}\le\frac{\varphi\pare{v}-\varphi\pare{u}}{v-u}\le\varphi'\pare{v^-}\le\varphi'\pare{u^+}. \]
  \end{lemma}
  \begin{collary}
    开区间上的凸函数是Lipschitz的,在任意紧区间上绝对连续。
  \end{collary}
  \begin{theorem}
    凸函数几乎处处可导且导函数为增函数。
  \end{theorem}
  \begin{theorem}[Jensen不等式]
    对凸函数$\varphi$与可积函数$f$,设$\varphi\comp f$可积,有
    \[ \varphi\pare{\int_0^1 f} \le \int_0^1 \varphi\comp f. \]
  \end{theorem}
  \begin{proof}
    令$\alpha=\int f$,则
    \[ \int_0^1 \varphi\comp f \ge \int_0^1 \brac{m\pare{f-\alpha}+\varphi\pare{\alpha}} = \varphi\pare{\alpha}. \qedhere \]
  \end{proof}
  将上开不等式用于加和为$1$的诸$\alpha_n$,
  \[ \sum \alpha_n \log x_n \le \log \sum \alpha_n x_n \]
  可得算术-几何不等式。
  
%ContentEnds
 
\ifx\allfiles\undefined %如果位置放错,可能出现意外中断
\end{document}
\fi
  %Ch10.LpSpace
\ifx\allfiles\undefined
\documentclass{ctexrep}
% Mathematics Include

\usepackage{amsmath}
\usepackage{amssymb}
\usepackage{amsthm}
\usepackage{amsfonts}
\usepackage{mathrsfs}
\usepackage{enumitem}
\usepackage{braket}
\usepackage{hyperref}
\usepackage[all, pdf]{xy}

% Physics Include
\usepackage{amsmath}
\usepackage{physics}
\usepackage{siunitx}
\usepackage[makeroom]{cancel}
\usepackage{pstricks}
\usepackage{pstricks-add}
\psset{algebraic=true}

\usepackage[version=4]{mhchem}
\usepackage{array,booktabs}
\usepackage{longtable}
\usepackage{mathtools}
\usepackage[normalem]{ulem}
\usepackage{multicol}

% Mathematics Head

\newcommand{\pare}[1]{\left(#1\right)}
\newcommand{\blr}[1]{\left[#1\right)}
\newcommand{\lbr}[1]{\left(#1\right]}
\newcommand{\brac}[1]{\left[#1\right]}
\newcommand{\curb}[1]{\left\{#1\right\}}
% \newcommand{\abs}[1]{\left|\, #1 \,\right|}
\newcommand{\rec}[1]{\frac{1}{#1}}
\newcommand{\N}{\mathbb{N}}
\newcommand{\bC}{\mathbb{C}}
\newcommand{\Q}{\mathbb{Q}}
\newcommand{\Z}{\mathbb{Z}}
\newcommand{\R}{\mathbb{R}}
\newcommand{\unk}{\mathcal{X}}
\newcommand{\bu}[3]{#1_{#2}^{\pare{#3}}}
\newcommand{\dref}[1]{定义\ref{def:#1}}
\newcommand{\tref}[1]{定理\ref{thm:#1}}
\newcommand{\lref}[1]{引理\ref{lem:#1}}
\newcommand{\cref}[1]{推论\ref{coll:#1}}
\newcommand{\pref}[1]{命题\ref{prp:#1}}
\newcommand{\eref}[1]{例\ref{ex:#1}}
\newcommand{\func}[3]{#1:\, #2 \rightarrow #3}
\newcommand{\overbar}[1]{\mkern 1.5mu\overline{\mkern-1.5mu#1\mkern-1.5mu}\mkern 1.5mu}
\newcommand{\clo}[1]{\overbar{#1}}
\newcommand{\supi}[2]{\overbar{\int_{#1}^{#2}}}
\newcommand{\infi}[2]{\underbar{\int_{#1}^{#2}}}
\newcommand{\setf}{\mathscr}
\newcommand{\bool}{\mathrm{bool}}
\newcommand{\inc}{++}
\newcommand{\defeq}{:=}
\newcommand{\ntuple}{$n$元组}
\newcommand{\card}[1]{\#\pare{#1}}
\newcommand{\setcond}[2]{\curb{#1 \, \left| \, #2 \right.}}
\newcommand{\setcondl}[2]{\curb{\left. #1 \, \right| \, #2}}
\newcommand{\bv}[1]{\mathbf{#1}}
\newcommand{\bfa}{\bv{a}}
\newcommand{\bfb}{\bv{b}}
\newcommand{\bfx}{\bv{x}}
\newcommand{\bfy}{\bv{y}}
\newcommand{\bfe}{\bv{e}}
\newcommand{\bfF}{\bv{F}}
\newcommand{\bff}{\bv{f}}
\newcommand{\bfG}{\bv{G}}
\newcommand{\bfH}{\bv{H}}
\newcommand{\bfg}{\bv{g}}
\newcommand{\bfh}{\bv{h}}
\newcommand{\bfr}{\bv{r}}
\newcommand{\bfk}{\bv{k}}
\newcommand{\bfu}{\bv{u}}
\newcommand{\bfv}{\bv{v}}
\newcommand{\oo}[1]{o\pare{#1}}
\newcommand{\OO}[1]{O\pare{#1}}
% \newcommand{\norm}[1]{\left\| #1 \right\|}
\newcommand{\DD}{\mathbf{D}}
\newcommand{\comp}{\circ}
\newcommand{\const}{\mathrm{const}}
\newcommand{\dist}[2]{d\pare{#1,#2}}
\newcommand{\len}{\ell}
\newcommand{\siga}{$\sigma$-代数}
\newcommand{\cara}{Carath\'{e}odory}
\newcommand{\Gd}{G_\delta}
\newcommand{\Fs}{F_\sigma}
\newcommand{\mmani}{$m$-维流形}
\newcommand{\open}[1]{\mathcal{#1}}
\newcommand{\half}{\frac{1}{2}}
\newcommand{\maxo}[1]{\text{max}\curb{#1}}
\newcommand{\mino}[1]{\text{min}\curb{#1}}
\newcommand{\epsclo}{$\epsilon$-接近}
\newcommand{\close}[1]{$#1$-接近}
\newcommand{\cinf}{$C^\infty$}
\newcommand{\cuno}{$C^1$}
\newcommand{\Int}{\text{Int}\,}
\newcommand{\Ext}{\text{Ext}\,}
\newcommand{\funcf}{\mathcal}
\newcommand{\DDu}{\overbar{\DD}}
\newcommand{\DDl}{\underbar{\DD}}
\newcommand{\Diff}[1]{\mathrm{Diff}_{#1}\,}
\newcommand{\Av}[1]{\mathrm{Av}_{#1}\,}
\newcommand{\Lip}[1]{Lipschitz-$#1$}
\newcommand{\sgn}{\mathrm{sgn}}
\newcommand{\eset}{\varnothing}
\newcommand{\cT}{\mathcal{T}}
\newcommand{\cS}{\mathcal{S}}
\newcommand{\cG}{\mathcal{G}}
\newcommand{\cF}{\mathcal{F}}
\newcommand{\cC}{\mathcal{C}}
\newcommand{\cB}{\mathcal{B}}
\newcommand{\inter}[1]{\mathring{#1}}
\newcommand{\forest}[3]{对于{#1},存在{#2},使得{#3}}
\newcommand{\tuno}{$T_1$公理}
\newcommand{\isom}{\overset{\sim}{=}}
\newcommand{\diam}{\mathrm{diam}\,}
\newcommand{\ord}[1]{\abs{#1}}
\newcommand{\sbm}[1]{\overbar{#1}}
\newcommand{\inv}[1]{#1^{-1}}
\newcommand{\restr}[2]{#1|_{#2}}
\newcommand{\divs}{|}
\newcommand{\ndivs}{\nmid}
\newcommand{\modeq}[1]{\overbar{#1}}
\newcommand{\ggen}[1]{\langle#1\rangle}
\newcommand{\ggencond}{\braket}

\newcommand{\hd}{H\"{o}lder}

\renewcommand{\proofname}{证明}

\newenvironment{cenum}{\begin{enumerate}\itemsep0em}{\end{enumerate}}

\newtheorem{definition}{定义}[section]
\newtheorem{lemma}{引理}[section]
\newtheorem{theorem}{定理}[section]
\newtheorem{collary}{推论}[section]
\newtheorem{corollary}{推论}[section]
\newtheorem{proposition}{命题}[section]
\newtheorem{axiom}{公理}[section]
\newtheorem{ex}{例}[section]
\newtheorem{reflection}{反射}[section]
\newcommand{\refl}[1]{\vspace{0.5em}\par\noindent\fbox{%
    \parbox{0.9\textwidth}{%
    \begin{reflection}
        #1
    \end{reflection}
    }%
}\vspace{0.5em}\par}
\newcommand{\rref}[1]{反射\ref{refl:#1}}
\newcommand{\tbref}[1]{表\ref{table:#1}}
\allowdisplaybreaks

\newenvironment{aenum}{\begin{enumerate}[label=\textnormal{(\alph*)}]}{\end{enumerate}}

% Physics Head

\DeclareSIUnit\dyne{dynes}

\newcommand{\ddel}[1]{\frac{\partial}{\partial #1}}
\newcommand{\ddelon}[2]{\frac{\partial #1}{\partial #2}}
\newcommand{\dddel}[1]{\frac{\partial^2}{\partial^2 #1}}
\newcommand{\ddt}{\ddel{t}}
\newcommand{\ddT}{\ddel{T}}
\newcommand{\ddV}{\ddel{V}}
\newcommand{\ddr}{\ddel{r}}
\newcommand{\ddth}{\ddel{\theta}}
\newcommand{\ddph}{\ddel{\phi}}
\newcommand{\dddt}{\dddel{t}}
\newcommand{\dddr}{\dddel{t}}
\newcommand{\dddth}{\dddel{\theta}}
\newcommand{\dddph}{\dddel{\phi}}
\newcommand{\rd}[1]{\mathrm{d} #1}
\newcommand{\dt}{\rd{t}}
\newcommand{\dy}{\rd{y}}
\newcommand{\dx}{\rd{x}}
\newcommand{\edd}[1]{\frac{\mathrm{d}}{\mathrm{d} #1}}
\newcommand{\eddd}[1]{\frac{\mathrm{d}^2}{\mathrm{d}^2 #1}}
\newcommand{\eddon}[2]{\frac{\mathrm{d} #1}{\mathrm{d} #2}}
\newcommand{\edddon}[2]{\frac{\mathrm{d}^2 #1}{\mathrm{d}^2 #2}}
\newcommand{\edt}{\edd{t}}
\newcommand{\edton}[1]{\eddon{#1}{t}}
\newcommand{\edT}{\edd{T}}
\newcommand{\edr}{\edd{r}}
\newcommand{\edl}{\edd{l}}
\newcommand{\edx}{\edd{x}}
\newcommand{\edth}{\edd{\theta}}
\newcommand{\eddton}[1]{\edddon{#1}{t}}
\newcommand{\eddzon}[1]{\edddon{#1}{z}}
\newcommand{\vect}[1]{\boldsymbol{#1}}
\newcommand{\alp}{\frac{1}{\sqrt{2}}}
\newcommand{\alpi}{\frac{i}{\sqrt{2}}}
\newcommand{\expc}[1]{\langle#1\rangle}
\newcommand{\bkn}[1]{\bra{#1}\ket{#1}}
\newcommand{\bk}[2]{\bra{#1}\ket{#2}}
\newcommand{\bik}[3]{\bra{#1} #2 \ket{#3}}
\newcommand{\vari}[1]{\sigma_{#1}}
\newcommand{\intc}[2]{\left[#1, #2\right]}
\newcommand{\sch}{Schr\"{o}dinger}
\newcommand{\moment}{\boldsymbol{p}}
\newcommand{\coor}{\boldsymbol{x}}
\newcommand{\lapc}{\nabla^2}
% \newcommand{\rec}[1]{\frac{1}{#1}}
\newcommand{\vva}{\boldsymbol{a}}
\newcommand{\vvb}{\boldsymbol{b}}
\newcommand{\vc}{\boldsymbol{c}}
\newcommand{\vd}{\boldsymbol{d}}
\newcommand{\ve}{\boldsymbol{e}}
\newcommand{\vf}{\boldsymbol{f}}
\newcommand{\vg}{\boldsymbol{g}}
\newcommand{\vh}{\boldsymbol{h}}
\newcommand{\vi}{\boldsymbol{i}}
\newcommand{\vj}{\boldsymbol{j}}
\newcommand{\vk}{\boldsymbol{k}}
\newcommand{\vl}{\boldsymbol{l}}
\newcommand{\vm}{\boldsymbol{m}}
\newcommand{\vn}{\boldsymbol{n}}
\newcommand{\vo}{\boldsymbol{o}}
\newcommand{\vp}{\boldsymbol{p}}
\newcommand{\vq}{\boldsymbol{q}}
\newcommand{\vr}{\boldsymbol{r}}
\newcommand{\vs}{\boldsymbol{s}}
\newcommand{\vt}{\boldsymbol{t}}
\newcommand{\vvu}{\boldsymbol{u}}
\newcommand{\vv}{\boldsymbol{v}}
\newcommand{\vw}{\boldsymbol{w}}
\newcommand{\vx}{\boldsymbol{x}}
\newcommand{\vy}{\boldsymbol{y}}
\newcommand{\vz}{\boldsymbol{z}}
\newcommand{\vA}{\boldsymbol{A}}
\newcommand{\vB}{\boldsymbol{B}}
\newcommand{\vC}{\boldsymbol{C}}
\newcommand{\vD}{\boldsymbol{D}}
\newcommand{\vE}{\boldsymbol{E}}
\newcommand{\vF}{\boldsymbol{F}}
\newcommand{\vG}{\boldsymbol{G}}
\newcommand{\vH}{\boldsymbol{H}}
\newcommand{\vI}{\boldsymbol{I}}
\newcommand{\vJ}{\boldsymbol{J}}
\newcommand{\vK}{\boldsymbol{K}}
\newcommand{\vL}{\boldsymbol{L}}
\newcommand{\vM}{\boldsymbol{M}}
\newcommand{\vN}{\boldsymbol{N}}
\newcommand{\vO}{\boldsymbol{O}}
\newcommand{\vP}{\boldsymbol{P}}
\newcommand{\vQ}{\boldsymbol{Q}}
\newcommand{\vR}{\boldsymbol{R}}
\newcommand{\vS}{\boldsymbol{S}}
\newcommand{\vT}{\boldsymbol{T}}
\newcommand{\vU}{\boldsymbol{U}}
\newcommand{\vV}{\boldsymbol{V}}
\newcommand{\vW}{\boldsymbol{W}}
\newcommand{\vX}{\boldsymbol{X}}
\newcommand{\vY}{\boldsymbol{Y}}
\newcommand{\vZ}{\boldsymbol{Z}}
\newcommand{\vzero}{\boldsymbol{0}}
\newcommand{\vomega}{\boldsymbol{\omega}}
%\newcommand{\half}{\frac{1}{2}}
\newcommand{\thalf}{\frac{3}{2}}
\newcommand{\rot}{\nabla\times}
\newcommand{\divg}{\nabla\cdot}
\newcommand{\cE}{\mathcal{E}}
\newcommand{\conclu}[1]{\vspace{1em}\par\noindent\fbox{\parbox{0.9\textwidth}{#1}}\vspace{1em}}
\newcommand{\subentrynote}{$\bullet$}
\newcommand{\keypoint}[1]{\par\subentrynote\quad #1 \par}
\newcommand{\fconclu}{\boxed}
\newcommand{\pair}[2]{#1 \, #2}
\newcommand{\intn}[2]{\int #1 \,\mathrm{d} #2}
\newcommand{\intu}[3]{\int_0^{#1} #2 \,\mathrm{d} #3}
\newcommand{\intiu}[3]{\int_{-\infty}^{#1} #2 \, \rd{} #3}
\newcommand{\intui}[2]{\int_0^{\infty} #1 \,\mathrm{d} #2}
\newcommand{\intii}[2]{\int_{-\infty}^{\infty} #1 \,\mathrm{d} #2}
\newcommand{\intt}[2]{\int_0^\infty #1 \, \rd{} #2}
\newcommand{\intr}[2]{\int_{-\infty}^{\infty} #1 \, \rd{} #2}
\newcommand{\intbi}[3]{\int_{#1}^{\infty} #2 \, \rd{} #3}
\newcommand{\intab}[4]{\int_{#1}^{#2} #3 \, \rd{} #4}
\newcommand{\bfactor}[1]{e^{-#1/k_BT}}
\newcommand{\pbfactor}[1]{e^{#1/k_BT}}
\newcommand{\dn}[2]{#1^{\pare{#2}}}
\newcommand{\prodg}[1]{\pare{#1}^\times}

\newcommand{\notion}{\emph}
\newcommand{\iP}{\mathcal{P}}
\newcommand{\eiP}{e^{-\iP}}
\newcommand{\iF}{\mathcal{F}}
\newcommand{\eiF}{e^{-\iF}}
\newcommand{\iG}{\mathcal{G}}

\newcommand{\rc}{r\cos\theta}
\newcommand{\rs}{r\sin\theta}
\newcommand{\sn}{\mathrm{sn}}
\newcommand{\cn}{\mathrm{cn}}
\newcommand{\rdn}{\mathrm{dn}}

\newcommand{\hankel}{H_p^{\pare{1}}}
\newcommand{\hankell}{H_p^{\pare{2}}}
\newcommand{\hhankel}{H_n^{\pare{1}}}
\newcommand{\hhankell}{H_n^{\pare{2}}}
\newcommand{\ber}{\text{ber}\,}
\newcommand{\bei}{\text{bei}\,}
\newcommand{\kker}{\text{ker}\,}
\newcommand{\kei}{\text{kei}\,}
\newcommand{\Ai}{\text{Ai}}
\newcommand{\Bi}{\text{Bi}}

\newcommand{\re}{\text{Re}\,}

\newcommand{\Fp}{F_\phi}
\newcommand{\Ep}{E_\phi}
\newcommand{\Fx}{F_x}
\newcommand{\FF}{\mathbf{F}}
\newcommand{\Ex}{E_x}

%\newcommand{\erf}{\mathrm{erf}}
\newcommand{\erfi}{\mathrm{erfi}}
\newcommand{\erfc}{\mathrm{erfc}}
\newcommand{\ehxs}[1]{e^{-\frac{#1^2}{2}}}
\newcommand{\dcol}[2]{\[ \left.#1 \hspace{1em}\right\vert\hspace{1em} #2 \]}
\newcommand{\titlegamma}{\texorpdfstring{$\Gamma$}{Gamma}}
\newcommand{\titleB}{\texorpdfstring{$B$}{B}}

% Computer Science Head
\usepackage{listings}
\usepackage{color}

\definecolor{dkgreen}{rgb}{0,0.6,0}
\definecolor{gray}{rgb}{0.5,0.5,0.5}
\definecolor{mauve}{rgb}{0.58,0,0.82}

\lstset{frame=tb,
  language=Java,
  aboveskip=3mm,
  belowskip=3mm,
  showstringspaces=false,
  columns=flexible,
  basicstyle={\small\ttfamily},
  numbers=none,
  numberstyle=\tiny\color{gray},
  keywordstyle=\color{blue},
  commentstyle=\color{dkgreen},
  stringstyle=\color{mauve},
  breaklines=true,
  breakatwhitespace=true,
  tabsize=3
}
\lstset{language=Java}
\newcommand{\snp}[1]{\lstinline!#1!}
\newcommand{\term}[2]{\textbf{#1(#2)}}
\begin{document}
\fi

%Content

\chapter{$L^p$空间}
  \section{$L^p$空间:完备性与逼近}
  \subsection{赋范线性空间}
  \begin{definition}
    二函数称为等价,如果其几乎处处相等。
  \end{definition}
  \begin{definition}
    $E$上满足
    \[ \int_E \abs{f}^p < \infty \]
    之函数等价类全体构成一线性空间,谓$L^p$空间。
  \end{definition}
  由
  \[ \abs{a+b}^p \le 2^p \curb{\abs{a}^p+\abs{b^p}} \]
  知其可构成线性空间。
  \begin{definition}
    若$f$几乎处处满足
    \[ \abs{f\pare{x}} \le M, \]
    谓之本质有界。其等价类全体构成$L^\infty$。
  \end{definition}
  \begin{definition}
    线性空间上一泛函$\norm{\cdot}$称为范数,若
    \begin{align*}
      \norm{f+g} &\le \norm{f}+\norm{g},\\
      \norm{\alpha f} &= \abs{\alpha}\norm{f},\\
      \norm{f} &\ge 0.
    \end{align*}
    最后的等号严格成立当且仅当$f=0$。
  \end{definition}
  \begin{ex}
    易知$L^1$构成一赋范线性空间。
  \end{ex}
  \begin{ex}
    易知$L^\infty$关于$\norm{f}=\inf M$构成一赋范线性空间。
  \end{ex}
  \begin{ex}
    易知$\ell_1$与$\ell_\infty$构成一赋范线性空间。
  \end{ex}
  \begin{ex}
    易知紧区间上的连续函数全体关于$\norm{f}=\max f$构成一赋范线性空间。
  \end{ex}
  \subsection{Young不等式,H\"{o}lder不等式,Minkowski不等式}
  \begin{definition}
    对于$1<p<\infty$以及$L^p$中的$f$,定义
    \[ \norm{f}_p = \brac{\int \abs{f}^p}^{1/p}.  \]
  \end{definition}
  \begin{definition}
    对$p\in\pare{1,\infty}$定义其共轭$q=p/\pare{p-1}$,同在$\pare{1,\infty}$内且
    \[ \rec{p}+\rec{q}=1. \]
  \end{definition}
  \begin{theorem}[Young不等式]
    设$p$,$q$共轭,对正数$a$,$b$有
    \[ ab\le \frac{a^p}{p}+\frac{b^q}{q}. \]
  \end{theorem}
  \begin{proof}
    由Jensen不等式,
    \[ \frac{\log a^p}{p} + \frac{\log b^q}{q} \le \log\pare{\frac{a^p}{p} + \frac{b^q}{q}}. \qedhere \]
  \end{proof}
  \begin{theorem}[H\"{o}lder不等式]
    对于$L^p$之$f$与$L^q$之$g$,有
    \[ \int \abs{f\cdot g} \le \norm{f}_p \cdot\norm{g}_q. \]
  \end{theorem}
  \begin{proof}
    不妨设$\norm{f}=\norm{g}=1$,从而由Young不等式易得
    \[ \int\abs{f\cdot g} \le 1. \qedhere \]
  \end{proof}
  \begin{collary}
    $f$之共轭$f^* = \norm{f}^{1-p}_p\cdot\sgn\pare{f}\cdot\abs{f}^{p-1}$为$L^q$,且
    \[ \int f\cdot f^* = \norm{f}_p,\quad \norm{f^*}_q=1. \]
  \end{collary}
  \begin{theorem}[Minkowski不等式]
    若$f$与$g$均为$L^p$,则$f+g$同且
    \[ \norm{f+g}_p \le \norm{f}_p + \norm{g}_p. \]
  \end{theorem}
  \begin{proof}
    借助前开推论与H\"{o}lder不等式,
    \[ \norm{f+g}_p = \int f\cdot\pare{f+g}^* + \int g\cdot\pare{f+g}^* \le \norm{f}_p + \norm{g}_p. \qedhere \]
  \end{proof}
  \begin{collary}[Cauchy-Schwarz不等式]
  对于$L^2$内的$f$与$g$,
    \[ \int\abs{fg} \le \sqrt{\int f^2}\cdot\sqrt{\int g^2}. \]
  \end{collary}
  \begin{collary}
    若$\funcf{F}$内诸$\norm{f}_p\le M$,则$\funcf{F}$一致可积。
  \end{collary}
  \begin{proof}
    由H\"{o}lder不等式,
    \[ \brac{\int_A \abs{f}} \le \brac{\int_E \abs{f}^p}^{1/p} \brac{m\pare{A}}^{1/q}. \qedhere \]
  \end{proof}
  \begin{collary}
    有限测度上若$p_1<p_2$,则$L^{p_2}\subset L^{p_1}$。且
    \[ \norm{f}_{p_1} \le c\norm{f}_{p_2}, \]
    其中$c=\brac{m\pare{E}}^{\frac{p_2-p_1}{p_1p_2}}$。
  \end{collary}
  \begin{proof}
    令$p=p_2/p_1$,则$f^{p_1}\in L^p$。由H\"{o}lder不等式,
    \[ \int_E \abs{f}^{p_1} \le \norm{f}^{p_1}_{p_2}\brac{m\pare{E}}^{1/q}. \qedhere \]
  \end{proof}
  \begin{ex}
    通常,有限测度集如$\lbr{0,1}$上上述包含关系是严格的。取$-1/p_1<\alpha<-1/p_2$,有$x^\alpha \in L^{p_1} - L^{p_2}$。
  \end{ex}
  \begin{ex}
    在$\pare{0,\infty}$上$f=x^{-1/2}/\pare{1+\abs{\log x}}$仅仅属于$L^2$。
  \end{ex}
  \subsection{$L^p$的完备性}
  \begin{definition}
    称一序列收敛于$f$,如果
    \[ \lim_{n\to\infty}\norm{f-f_n}=0. \]
  \end{definition}
  \begin{definition}
    完备的赋范线性空间称为Banach空间。
  \end{definition}
  \begin{proposition}
    完备空间内的收敛序列均为Cauchy序列,且包含收敛子序列的Cauchy序列收敛。
  \end{proposition}
  \begin{proof}
    后一命题注意
    \[ \norm{f_n-f} \le \norm{f_n-f_{n_k}} + \norm{f_{n_k}-f}. \qedhere  \]
  \end{proof}
  \begin{definition}
    一序列称为快速Cauchy的,如果对于一收敛级数$\sum\epsilon_k$,有
    \[ \norm{f_{k+1}-f_k} \le \epsilon_k^2. \]
  \end{definition}
  \begin{proposition}
    快速Cauchy序列都是Cauchy的,而任一Cauchy序列均有快速子列。
  \end{proposition}
  \begin{proof}
    选取子列满足
    \[ \norm{f_{n_{k+1}}-f_{n_k}} \le \pare{\half}^k. \qedhere \]
  \end{proof}
  \begin{theorem}
    $L^p$内的快速Cauchy序列依范数且几乎处处逐点收敛。
  \end{theorem}
  \begin{proof}
    依定义选取$\sum\epsilon_k$后注意
    \[ m\pare{\abs{f_{k+1}-f_k}^p>\epsilon_k^p} \le \rec{\epsilon_k^p}\int\abs{f_{k+1}-f_k}^p \le \epsilon_k^p. \]
    由Borel-Cantelli引理知$f_n$几乎处处逐点收敛。由Fatou引理,
    \[ \int\abs{f-f_n}^p \le \int\abs{f_{n+k}-f_n}^p \le \brac{\sum_{j=n} \epsilon_j^2}^p. \qedhere \]
  \end{proof}
  \begin{theorem}[Riesz-Fischer]
    $L^p$空间为Banach空间。且Cauchy序列存在子列几乎处处逐点收敛。
  \end{theorem}
  \begin{ex}
    $f_n=n^{1/p}\chi_{\lbr{0,1/n}}$逐点收敛于零,但不依范数收敛。
  \end{ex}
  \begin{theorem}
    对$1\le p<\infty$,逐点收敛的序列依范数收敛当且仅当
    \[ \lim_{n\to\infty}\int\abs{f_n}^p=\int\abs{f}^p. \]
  \end{theorem}
  \begin{proof}
    若依范数收敛,由三角不等式即得结论。反之设极限成立,令
    \[ h_n = \frac{\abs{f_n}^p+\abs{f}^p}{2}-\abs{\frac{f_n-f}{2}}^p. \]
    由凸性知$h_n\ge 0$,且$\lim h_n = \abs{f}^p$逐点收敛,由Fatou引理
    \[ \int \abs{f}^p \le \lim \inf \int h_n = \int \abs{f}^p - \lim\sup\int\abs{\frac{f_n-f}{2}}^p. \qedhere \]
  \end{proof}
  \begin{theorem}
    对$1\le p<\infty$,逐点收敛的序列依范数收敛当且仅当$\curb{\abs{f_n}^p}$一致可积且紧密。
  \end{theorem}
  \begin{proof}
    由\cref{mcon2int},要求$\abs{f_n-f}^p$一致可积且紧密。再注意
    \[ \abs{f_n-f}^p \le 2^p\curb{\abs{f_n}^p+\abs{f}^p}, \quad \abs{f_n}^p \le 2^p\curb{\abs{f_n-f}^p+\abs{f}^p}. \qedhere \]
  \end{proof}
  \subsection{逼近与可分性}
  \begin{definition}
    $L^p$下一函数族称为稠密的,如果其依范数可任意逼近$L^p$。
  \end{definition}
  \begin{proposition}
    简单函数在$L^p$内稠密。
  \end{proposition}
  \begin{proof}
    借助简单函数逼近引理,注意$\abs{\varphi_n-g}^p \le 2^{p+1}\abs{g}^p$后控制收敛。
  \end{proof}
  \begin{proposition}
    对$1\le p<\infty$,阶梯函数在紧区间上的$L^p$稠密。
  \end{proposition}
  \begin{proof}
    注意阶梯函数可在任意小的集合外逼近简单函数即可。而对$p=\infty$,再小的非零测集皆会导致范数不得为零,是故于其不成立。
  \end{proof}
  \begin{definition}
    空间谓可分者,其下存在一可数稠密子集。
  \end{definition}
  \begin{theorem}
    对$1\le p < \infty$,$L^p$可分。
  \end{theorem}
  \begin{proof}
    紧区间内有理阶梯函数稠密,积分可由$\brac{-n,n}$上单调收敛逼近。
  \end{proof}
  \begin{ex}
    紧区间上的$L^\infty$不可分。
  \end{ex}
  \begin{proof}
    不可数特征函数族的区间稍变,逼近不复成立,不可以可数族逼近。
  \end{proof}
  \begin{theorem}
    对$1\le p<\infty$,有界支撑的连续函数在$L^p$中稠密。
  \end{theorem}
  \section{$L^p$空间的对偶与弱收敛}
  \subsection{$L^p$的对偶与表示}
  \begin{definition}
    线性泛函是函数上的线性算子。
  \end{definition}
  \begin{ex}
    $T\pare{f}=\int fg$与$T\pare{f}=\int f \, \mathrm{d} g$均为线性泛函。
  \end{ex}
  \begin{definition}
    所有$\abs{T\pare{f}} \le M \norm{f}$的$M$的下确界记作$\norm{T}$。
  \end{definition}
  由三角不等式知线性泛函连续。同时有
  \[ \norm{T} = \sup\setcond{T\pare{f}}{\norm{f}\le 1}. \]
  \begin{proposition}
    赋范线性空间上的线性算子的空间构成一赋范线性空间。
  \end{proposition}
  \begin{proposition}
    $L^p$上的算子
    \[ T\pare{f}=\int g\cdot f \]
    的范数为$\norm{g}$。
  \end{proposition}
  \begin{proposition}
     在一稠密子集上相等的线性算子相等。
  \end{proposition}
  \begin{lemma}
    可测函数$g$若对$L^p$上的简单函数$f$皆满足
    \[ \abs{\int g\cdot f} \le M \norm{f}, \]
    则$g\in L^q$,且$\norm{g}\le M$。
  \end{lemma}
  \begin{proof}
    对于$p>1$,考虑$g$的下逼近,只证$\int\varphi_n^p\le M^p$即可,再注意$\varphi_n^q\le\abs{g}\varphi_n^{q-1}$,以及$p\pare{q-1}=q$并借助题设。
    \par
    对于$p=1$,需证$M$为一本质上界。考虑$f$为诸特征函数即可。
  \end{proof}
  \begin{theorem}
    紧区间上的线性泛函满足$T\pare{f}=\int g\cdot f$的形式。
  \end{theorem}
  \begin{proof}
    令$\Phi\pare{x}=T\chi_{\blr{a,x}}$,其绝对连续,故$\Phi'=g$积分还原。对阶梯函数,
    \[ T\pare{f} = \int g\cdot f. \]
    控制收敛后知对简单函数均成立之。调用前开命题再注意简单函数稠密。
  \end{proof}
  \begin{theorem}[$L^p$的Riesz表示定理]
    $1\le p<\infty$上的线性泛函有$g$满足
    \[ Tf = \Braket{g|f}, \quad \norm{T} = \norm{f}. \]
  \end{theorem}
  \begin{proof}
    考虑$\brac{-n,n}$上的限制后不断扩大$n$,Fatou后知$g\in L^q$。
  \end{proof}
  \subsection{弱收敛性}
  \begin{ex}
    $\brac{0,1}$上的$1/2^n$-方波在$L^p$内不存在收敛子列。
  \end{ex}
  \begin{definition}
    赋范线性空间上的序列$\curb{f_n}$,若$Tf_n\to Tf$对任意$T$成立,则称之弱收敛。
  \end{definition}
  \begin{proposition}
    $\curb{f_n}$弱收敛于$f$当且仅当对任意$g$成立
    \[ \lim_{n\to\infty} g\cdot f_n = \int g \cdot f. \]
  \end{proposition}
  弱收敛具有唯一性。因为
  \[ \int \pare{f_1-f_2}^*\cdot f_2 = \lim_{n\to\infty} \int \pare{f_1-f_2}^*\cdot f_n = \int \pare{f_1-f_2}^* f_2. \]
  \begin{theorem}
    $L^p$上的弱收敛序列有诸$\norm{f_n}$有界且
    \[ \norm{f} \le \lim \inf \norm{f_n}. \]
  \end{theorem}
  \begin{proof}
    注意到
    \[ \int f^*\cdot f_n \le \norm{f^*}_q \cdot \norm{f_n}_p = \norm{f_n}_p \]
    后Fatou即可。为证明有界,假设$\curb{\norm{f_n}}$无界,选取$\curb{f_n}$的子列$\curb{g_n}$满足$\norm{g_n}\ge n\cdot 3^n$,并再度选取子列$\curb{h_n}$满足$\norm{h_n}/\pare{n\cdot 3^n}\rightarrow \alpha\in\brac{1,+\infty}$。
    \par
    于是$\cF_n = n\cdot 3^n/\norm{h_n}\cdot h_n$满足$\cF_n$弱收敛于$f$且$\norm{\cF_n}=n\cdot 3^n$。定义
    \[ \epsilon_{n+1} = \rec{3^{n+1}}\sgn \int \brac{\sum_k^n \epsilon_k \cdot f_k^*}\cdot f_{n+1}, \]
    则$\norm{\epsilon_k\cdot f_k^*}=1/3^k$,由$L^p$的完备性知$g=\sum^\infty \epsilon_k \cdot f_k^*$收敛于$L^p$内。
    
  \end{proof}


%ContentEnds
 
\ifx\allfiles\undefined %如果位置放错,可能出现意外中断
\end{document}
\fi  

\begin{thebibliography}{9}
\bibitem{br} 
Walter Rudin.
\textit{数学分析原理}. 
机械工业出版社,2004.
\bibitem{ar} 
Walter Rudin.
\textit{实分析与复分析}. 
机械工业出版社, 2006.
\bibitem{mt} 
James R. Munkres.
\textit{拓扑学}. 
机械工业出版社, 2006.
\end{thebibliography}
  
%    \subsubsection*{参考源}
%  \noindent
% :,;\\
%  Munkres:,流形上的分析;\\
%  Zorich:数学分析I,数学分析II;\\
%  Stein:傅立叶分析导论,复分析,实分析,泛函分析;\\
%  Tao:陶哲轩实分析;\\
%  Natonson:实变函数论;\\
%  Folland:实分析;\\
%  Mattuck:分析学导论;\\
%  Apostol:数学分析,解析数论导论;\\
%  Tenenbaum:解析与概率数论导引。\\
      
\end{document}
