%Ch5.GeneralFunctionTheory.tex
\ifx\allfiles\undefined
\documentclass{ctexrep}
\usepackage{amsmath}
\usepackage{amssymb}
\usepackage{amsthm}
\usepackage{amsfonts}
\usepackage{mathrsfs}
\usepackage{enumitem}
\usepackage{braket}
\usepackage{hyperref}


\newcommand{\pare}[1]{\left(#1\right)}
\newcommand{\blr}[1]{\left[#1\right)}
\newcommand{\lbr}[1]{\left(#1\right]}
\newcommand{\brac}[1]{\left[#1\right]}
\newcommand{\curb}[1]{\left\{#1\right\}}
\newcommand{\abs}[1]{\left|\, #1 \,\right|}
\newcommand{\rec}[1]{\frac{1}{#1}}
\newcommand{\N}{\mathbb{N}}
\newcommand{\Q}{\mathbb{Q}}
\newcommand{\Z}{\mathbb{Z}}
\newcommand{\R}{\mathbb{R}}
\newcommand{\unk}{\mathcal{X}}
\newcommand{\bu}[3]{#1_{#2}^{\pare{#3}}}
\newcommand{\dref}[1]{定义\ref{def:#1}}
\newcommand{\tref}[1]{定理\ref{thm:#1}}
\newcommand{\lref}[1]{引理\ref{lem:#1}}
\newcommand{\cref}[1]{推论\ref{coll:#1}}
\newcommand{\pref}[1]{命题\ref{prp:#1}}
\newcommand{\func}[3]{#1:\, #2 \rightarrow #3}
\newcommand{\overbar}[1]{\mkern 1.5mu\overline{\mkern-1.5mu#1\mkern-1.5mu}\mkern 1.5mu}
\newcommand{\clo}[1]{\overbar{#1}}
\newcommand{\supi}[2]{\overbar{\int_{#1}^{#2}}}
\newcommand{\infi}[2]{\underbar{\int_{#1}^{#2}}}
\newcommand{\setf}{\mathscr}
\newcommand{\bool}{\mathrm{bool}}
\newcommand{\inc}{++}
\newcommand{\defeq}{:=}
\newcommand{\ntuple}{$n$元组}
\newcommand{\card}[1]{\#\pare{#1}}
\newcommand{\setcond}[2]{\curb{#1 \, \left| \, #2 \right.}}
\newcommand{\setcondl}[2]{\curb{\left. #1 \, \right| \, #2}}
\newcommand{\bv}[1]{\mathbf{#1}}
\newcommand{\bfa}{\bv{a}}
\newcommand{\bfb}{\bv{b}}
\newcommand{\bfx}{\bv{x}}
\newcommand{\bfy}{\bv{y}}
\newcommand{\bfe}{\bv{e}}
\newcommand{\bfF}{\bv{F}}
\newcommand{\bff}{\bv{f}}
\newcommand{\bfG}{\bv{G}}
\newcommand{\bfH}{\bv{H}}
\newcommand{\bfg}{\bv{g}}
\newcommand{\bfh}{\bv{h}}
\newcommand{\bfr}{\bv{r}}
\newcommand{\bfk}{\bv{k}}
\newcommand{\bfu}{\bv{u}}
\newcommand{\bfv}{\bv{v}}
\newcommand{\oo}[1]{o\pare{#1}}
\newcommand{\OO}[1]{O\pare{#1}}
\newcommand{\norm}[1]{\left\| #1 \right\|}
\newcommand{\DD}{\mathbf{D}}
\newcommand{\comp}{\circ}
\newcommand{\const}{\mathrm{const}}
\newcommand{\dist}[2]{d\pare{#1,#2}}
\newcommand{\len}{\ell}
\newcommand{\siga}{$\sigma$-代数}
\newcommand{\cara}{Carath\'{e}odory}
\newcommand{\Gd}{G_\delta}
\newcommand{\Fs}{F_\sigma}
\newcommand{\mmani}{$m$-维流形}
\newcommand{\open}[1]{\mathcal{#1}}
\newcommand{\half}{\frac{1}{2}}
\newcommand{\maxo}[1]{\text{max}\curb{#1}}
\newcommand{\mino}[1]{\text{min}\curb{#1}}
\newcommand{\epsclo}{$\epsilon$-接近}
\newcommand{\close}[1]{$#1$-接近}
\newcommand{\cinf}{$C^\infty$}
\newcommand{\cuno}{$C^1$}
\newcommand{\Int}{\text{Int}\,}
\newcommand{\Ext}{\text{Ext}\,}
\newcommand{\funcf}{\mathcal}
\newcommand{\DDu}{\overbar{\DD}}
\newcommand{\DDl}{\underbar{\DD}}
\newcommand{\Diff}[1]{\mathrm{Diff}_{#1}\,}
\newcommand{\Av}[1]{\mathrm{Av}_{#1}\,}
\newcommand{\Lip}[1]{Lipschitz-$#1$}
\newcommand{\sgn}[1]{\mathrm{sgn}}
\newcommand{\eset}{\varnothing}
\newcommand{\cT}{\mathcal{T}}
\newcommand{\cS}{\mathcal{S}}
\newcommand{\cG}{\mathcal{G}}
\newcommand{\cF}{\mathcal{F}}
\newcommand{\cC}{\mathcal{C}}
\newcommand{\cB}{\mathcal{B}}

\newcommand{\hd}{H\"{o}lder}

\renewcommand{\proofname}{证明}

\newenvironment{cenum}{\begin{enumerate}\itemsep0em}{\end{enumerate}}

\newtheorem{definition}{定义}[section]
\newtheorem{lemma}{引理}[section]
\newtheorem{theorem}{定理}[section]
\newtheorem{collary}{推论}[section]
\newtheorem{proposition}{命题}[section]
\newtheorem{axiom}{公理}[section]
\newtheorem{ex}{例}[section]

\newenvironment{aenum}{\begin{enumerate}[label=\textnormal{(\alph*)}]}{\end{enumerate}}
\begin{document}
\fi

%Content

\chapter{普通函数理论}
  \section{函数序列}
  \section{特殊函数}
  \section{多元函数}
  \subsection{微分法}
  \begin{definition}
    设$E$是$\R^n$中的开集,$\bv{f}$将$E$映入$\R^m$,$\bv{x}\in E$。若存在将$\R^n$映入$\R^m$的线性变换$A$,使得
    \[ \lim_{\bv{h}\to 0} \frac{\abs{\bv{f}\pare{\bv{x}+\bv{h}}}-\bv{f}\pare{\bv{x}}-A\bv{h}}{\abs{\bv{h}}} = 0, \]
    则称$\bv{f}$在$\bv{x}$处可微。记作
    \[ \bv{f}'\pare{x}=A. \]
  \end{definition}
  \begin{theorem}
    上述定义的$A$是唯一的。
  \end{theorem}
  \begin{proof}
    设$B=A_1-A_2$,则
    \[ \abs{B\bfh}\le\abs{\bff\pare{\bfx+\bfh}-\bff\pare{\bfx}-A_1\bfh}+\abs{\bff\pare{\bfx+\bfh}-\bff\pare{\bfx}-A_2\bfh}. \]
    故$\abs{B\bfh}/\abs{\bfh}\to 0$。因此$B=0$。
  \end{proof}
  例如线性变换的导数显然是其自身。
  \begin{theorem}
    若$\bfF\pare{\bfx} = \bfg\pare{\bff\pare{x}}$,且$\bff$与$\bfg$在对应点可微,则
    \[ \bfF'\pare{\bfx_0} = \bfg'\pare{\bff\pare{\bfx_0}}\bff'\pare{\bfx_0}. \]
  \end{theorem}
  \begin{proof}
    设$\bfx_0$的增量为$\bfh$,$\bff\pare{\bfx_0}$的增量为$\bfk$。
    \begin{align*}
      \bfF\pare{\bfx_0+\bfh}-\bfF\pare{\bfx_0}-BA\bfh &= \bfg\pare{\bfy_0+\bfk} - \bfg\pare{\bfy_0}-BA\bfh \\
      &= B\pare{\bfk-A\bfh}+\oo{\bfk} \\
      &= B\oo{\bfh} + \oo{\bfk}.
    \end{align*}
    再注意$\bfk = A \bfh + \oo{\bfh}$便知其趋向零。
  \end{proof}
  \begin{definition}
    设$E$是$\R^n$中的开集,$\bv{f}$将$E$映入$\R^m$。$\curb{e_n}$是定义域的标准基,$\curb{u_m}$是值域的标准基,以此定义$\bff$的分量。若极限
    \[ \pare{\DD_jf_i}\pare{\bfx} = \lim_{t\to 0}\frac{f_i\pare{\bfx+t\bfe_j}-f_i\pare{\bfx}}{t} \]
    存在,则称之为偏导数。
  \end{definition}
  \begin{theorem}
    \label{thm:adij}
    若$\bff$在$\bfx$处可微,则各偏导数存在,且
    \[ A_{ij} = \DD_jf_i. \]
  \end{theorem}
  \begin{proof}
    注意到方向导数的存在为可导所蕴含即可。
  \end{proof}
  注意$ij$的顺序,曲线的$A$为列矩阵,即切向量。
  \begin{definition}若$f$是$\R^n$上的标量函数,则$A$为行矩阵,称为$f$的梯度,记作$\nabla f$。
  \end{definition}
  设$\gamma$为一曲线,$f$标量函数,$g=f \comp \gamma$,于是$g'\pare{t} = \pare{\nabla f}\gamma'$。若$\gamma$为指向$\bfu$的直线,则$g'\pare{t}=\bfu\cdot\nabla f$,称为$f$的方向导数。当$\bfu$与$\nabla f$同向时\footnote{此处将行向量与列向量等同,或者说协变矢量与逆变矢量等同。},其具有最大值。
  \begin{theorem}
    若在凸集上$\norm{\bff'}\le M$,则$\abs{\bff\pare{\bfb}-\bff\pare{\bfa}} \le M \abs{\bfb-\bfa}$。
  \end{theorem}
  \begin{proof}
    设$\gamma\pare{t} = \pare{1-t}\bfa+t\bfb$,$\bfg=\bff\comp\gamma$,因此$\bfg'=\bff'\comp\pare{\bfb-\bfa}$,再注意$\bff'$的上界并调用\tref{vfab}。
  \end{proof}
  \begin{collary}
    若$\bff'=0$,则$\bff=\const$。
  \end{collary}
  \begin{definition}
    若$\bff$可微且$\bff'$连续,则称$\bff$连续可微。
  \end{definition}
  \begin{theorem}
    当且仅当$\func{\bff}{E\subset\R^n}{\R^m}$偏导数存在且连续时,$\bff$连续可微。
  \end{theorem}
  \begin{proof}
    若$\bff$连续可微,则借助\tref{adij},并注意
    \[ \abs{\pare{\DD_jf_i}\pare{\bfy}-\pare{\DD_jf_i}\pare{\bfx}}\le\abs{\brac{\bff'\pare{\bfy}-\bff'\pare{\bfx}}\bfe_j} \]
    即可。
    \par
    若$\bff$在$\bfx$的各偏导数存在,则可以仅考虑$\bff$的一个分量$f$。将$\bfx$的无穷小位移$\bfh$分解为各个方向的和,借助中值定理将增量转化为"稍微偏离"$\bfx$处的偏导。再借助连续性,从而$\bff'$可以写为$\DD_jf$的列向量,且各分量由题设连续。
  \end{proof}
  \subsection{反函数定理}
  \begin{theorem}
    若连续可微的$\bff$将开集$E\subset \R^n$映入$\R^n$,且对于某个$\bfa$,$\bff'\pare{\bfa}$可逆,$\bff\pare{\bfa} = \bfb$。则存在$\bfa$和$\bfb$的邻域$U$和$V$,使$\bff$为双射。
  \end{theorem}
  \begin{proof}
    为了求出$\bff$的反函数,采用Newton法,令
    \[ \varphi_\bfy\pare{\bfx} = \bfx + A^{-1}\pare{\bfy-\bff\pare{\bfx}}. \]
    注意$\varphi'=A^{-1}\pare{A-\bff'}$,因此可以选择足够小的邻域使$\norm{\varphi'}$足够小。由\tref{fixp0},其存在不动点,故$\bff$可逆。
    \par
    $\bff$的一一性已证,下证其为开映射。设$\bff\pare{\bfx_0}=\bfy_0$确凿,对于足够接近$\bfy_0$的$\bfy$,从任何$\bfx_0$附近的$\bfx$出发试图寻找其原像,皆有
    \[ \abs{\varphi\pare{\bfx}-\bfx_0}\le\abs{\varphi\pare{\bfx}-\varphi\pare{\bfx_0}}+\abs{\varphi\pare{\bfx_0}-\bfx_0}.\]
    前一绝对值的大小由$\norm{\varphi'}$限制,后一绝对值的大小由$\bfy$的偏移量限制。因此迭代后的不动点仍在$\bfx$附近,故仍在$U$内,所以$\bfy$仍在$V$内。
  \end{proof}
  \begin{theorem}
    前开定理的局域反函数$\bfg$亦连续可微。
  \end{theorem}
  \begin{proof}
    由前证以及\tref{opr-0},可以选取足够小的邻域$U$使得$\bff'$在此邻域内与$\bff'\pare{\bfa}$足够接近故可逆。还可以使得在此邻域内,$\bfh = \OO{\bfk}$。注意
    \[ \bfg\pare{\bfy+\bfk}-\bfg\pare{\bfy}-T\bfk=-T\brac{\bff\pare{\bfx+\bfh}-\bff\pare{\bfx}-\bff'\pare{\bfx}\bfh}, \]
    其中$T=\bff'^{-1}$。因此余项$\bfv=\OO{\bfu}$。再注意到求逆是连续映射即可。
  \end{proof}
  \begin{theorem}
    若连续可微的$\bff$将开集$E\subset \R^n$映入$\R^n$,$\bff'$逐点可逆,则$\bff$为开映射。
  \end{theorem}
  \subsection{隐函数定理}
  \begin{definition}
    $\pare{\bfx,\bfy}$为
    \[ \pare{x_1,\cdots,x_n,y_1,\cdots,y_m} \in \R^{n+m}. \]
  \end{definition}
  每个线性变换$\func{A}{\R^{n+m}}{\R^n}$都可以分解成$A_x$和$A_y$两部分。
  \begin{theorem}
    对于上述$A$,若$A_x$可逆,则对于每个$\bfk\in\R^m$,有唯一的$\bfh\in\R^n$满足$A\pare{\bfh,\bfk}=0$。
  \end{theorem}
  \begin{proof}
    \[ \bfh = -\pare{A_x}^{-1}A_y\bfk. \qedhere \]
  \end{proof}
  \begin{theorem}
    设$\func{\bff}{\pare{X,Y}}{Z}$是开集$E\subset\R^{n+m}$到$\R^n$内的连续可微映射,且在某点$\pare{\bfa,\bfb}$为零。令$A=\bff'\pare{\bfa,\bfb}$且$A_x$可逆,则存在$\pare{\bfa,\bfb}$的邻域$U$和$\bfb$的邻域$W$,$W$内的任意$\bfy$有唯一$\bfx$使$\bff=0$。
  \end{theorem}
  \begin{proof}
    令$\bfF\pare{\bfx, \bfy}= \pare{\bff\pare{\bfx,\bfy},\bfy}$。因此
    \[ \bfF\pare{\bfx+\bfh, \bfy+\bfk}-\bfF\pare{\bfx, \bfy}=\pare{A\pare{\bfh,\bfk},\bfk}+\oo{\bfh,\bfk}. \]
    若右侧为零,则$\bfk=0$,$\bfh=0$。因此$\bfF'$可逆,故反函数定理可用于$\bfF$。因此,\emph{存在$\pare{0,\bfb}$的邻域$V$和$\pare{\bfa,\bfb}$的邻域$U$,使得$\bfF$是一一的}。在$V$中投影出$\pare{0,\bfy}$,注意其为开集即可。
  \end{proof}
  \begin{theorem}
    在前开命题中设$\bfx=\bfg\pare{\bfy}$,有$\bfg$为连续可微映射且$\bfg'\pare{\bfb}=-\pare{A_x}^{-1}A_y$。
  \end{theorem}
  \begin{proof}
    令$\bfG$为$\bfF$的局域反函数,则$\bfG$连续可微,故作为其限制的$\bfg$亦然。欲求$\bfg'\pare{\bfb}$,注意到
    \[ \bff\pare{\bfg\pare{\bfb},\bfb}'=0, \]
    即$A\pare{\bfg'\pare{\bfb},I}=0$,$A_x\bfg'\pare{\bfb}+A_y=0$。
  \end{proof}
  \subsection{秩定理}
  \begin{theorem}
  设$\func{\bfF}{E\subset\R^n}{\R^m}$将开集$E$映入$\R^m$,对于任意$\bfx$有$\bfF'$的秩为$r$。对于$\bfa\in E$,$A=\bfF'$的像空间为$Y_1$,$P$是到其上的投影,$Y_2$是$Y_1$的正交空间。
  \par
  存在$\bfa$的邻域$U$和$\R^n$中的开集$V$,存在连续可微双射$\func{\bfH}{V}{U}$满足
  \[ \bfF\pare{\bfH\pare{\bfx}} = A\bfx+\varphi\pare{A\bfx}, \]
  且$\bfH$的逆亦连续可微。式中$\varphi$映入$Y_2$。
  \end{theorem}
  \begin{proof}
    若$r=0$(这里并没有使用归纳法的打算),存在$\bfa$的邻域使得$\bfF$为常量且$Y_1$为零空间。故可以取$\bfH=I$,$V=U$,$\varphi\pare{0}=\bfF\pare{\bfa}$。
    \par
    对于$r>0$,设$S$为$A$在$Y_1$上的逆,定义
    \[ \func{\bfG}{E}{U\subset\R^n} = \bfx+SP\brac{\bfF\pare{\bfx}-A\bfx}. \]
    由$\bfG'=I$知存在邻域使其可逆,设其逆为$\bfH$。将$\bfG$经$A$映射,即
    \[ A\bfG = ASP\bfF = P\bfF. \]
    这是因为$ASPA=A$。如果$\bfx = \bfH\pare{\bfv}$,则有$P\bfF\comp\bfH=A$。因此,
    \[  \Phi\pare{\bfv}=\bfF\bfH\pare{\bfv}-A\bfv \]
    是到$Y_2$的连续可微映射。下尚需证存在连续可微的映射$\varphi\pare{A\bfv} = \Phi\pare{\bfv}$。
    \par
    先证$\Phi$的值仅仅取决于$A\bfv$。考虑两$\bfv$的差$\bfh$,只需证明当$A\bfh=0$,有$\pare{\bfF\bfH}'\bfh=0$。由构造,$\pare{\bfF\bfH}'\bfh$到$Y_1$上投影仍为$A\bfh$,且二算子像空间皆$r$维,知$A\bfh$可决定$\pare{\bfF\bfH}'$。当前者为零,后者亦然。
    \par
    再证$\varphi$连续可微。注意$\varphi$的定义域是诸$A\bfv$,故对于$\varphi\pare{\bfu}$,相应的$\bfv=S\bfu+\bfa$,其中$\bfa$是$A$的零空间中任意元素。对于任意$\bfu$,可以寻找其邻域使得存在$\bfa$让$\bfv$留在$V$内。于是$\varphi\pare{\bfu}=\Phi\pare{\bfa+S\bfu}$连续可微。
  \end{proof}
  注意$A\bfx$映射到$Y_1$,因此$\bfF$在此点的值仅仅取决于其射影,因而可将其视为$r$维曲面,而其水平集可以视为$X$中的$n-r$维曲面。例如将汤勺映为棒棒糖的映射,勺柄的二维平面被映射为糖棒的一维线段,其$\bfF'$的秩固然为$1$。而其水平集,即糖棒上一点对应的原像则是$2-1=1$维的垂直于汤勺柄的线段。

%ContentEnds
 
\ifx\allfiles\undefined %如果位置放错,可能出现意外中断
\end{document}
\fi