%Ch9.IntegrationTheory.tex
\ifx\allfiles\undefined
\documentclass{ctexrep}
\usepackage{amsmath}
\usepackage{amssymb}
\usepackage{amsthm}
\usepackage{amsfonts}
\usepackage{mathrsfs}
\usepackage{enumitem}
\usepackage{braket}
\usepackage{hyperref}


\newcommand{\pare}[1]{\left(#1\right)}
\newcommand{\blr}[1]{\left[#1\right)}
\newcommand{\lbr}[1]{\left(#1\right]}
\newcommand{\brac}[1]{\left[#1\right]}
\newcommand{\curb}[1]{\left\{#1\right\}}
\newcommand{\abs}[1]{\left|\, #1 \,\right|}
\newcommand{\rec}[1]{\frac{1}{#1}}
\newcommand{\N}{\mathbb{N}}
\newcommand{\Q}{\mathbb{Q}}
\newcommand{\Z}{\mathbb{Z}}
\newcommand{\R}{\mathbb{R}}
\newcommand{\unk}{\mathcal{X}}
\newcommand{\bu}[3]{#1_{#2}^{\pare{#3}}}
\newcommand{\dref}[1]{定义\ref{def:#1}}
\newcommand{\tref}[1]{定理\ref{thm:#1}}
\newcommand{\lref}[1]{引理\ref{lem:#1}}
\newcommand{\cref}[1]{推论\ref{coll:#1}}
\newcommand{\pref}[1]{命题\ref{prp:#1}}
\newcommand{\func}[3]{#1:\, #2 \rightarrow #3}
\newcommand{\overbar}[1]{\mkern 1.5mu\overline{\mkern-1.5mu#1\mkern-1.5mu}\mkern 1.5mu}
\newcommand{\clo}[1]{\overbar{#1}}
\newcommand{\supi}[2]{\overbar{\int_{#1}^{#2}}}
\newcommand{\infi}[2]{\underbar{\int_{#1}^{#2}}}
\newcommand{\setf}{\mathscr}
\newcommand{\bool}{\mathrm{bool}}
\newcommand{\inc}{++}
\newcommand{\defeq}{:=}
\newcommand{\ntuple}{$n$元组}
\newcommand{\card}[1]{\#\pare{#1}}
\newcommand{\setcond}[2]{\curb{#1 \, \left| \, #2 \right.}}
\newcommand{\setcondl}[2]{\curb{\left. #1 \, \right| \, #2}}
\newcommand{\bv}[1]{\mathbf{#1}}
\newcommand{\bfa}{\bv{a}}
\newcommand{\bfb}{\bv{b}}
\newcommand{\bfx}{\bv{x}}
\newcommand{\bfy}{\bv{y}}
\newcommand{\bfe}{\bv{e}}
\newcommand{\bfF}{\bv{F}}
\newcommand{\bff}{\bv{f}}
\newcommand{\bfG}{\bv{G}}
\newcommand{\bfH}{\bv{H}}
\newcommand{\bfg}{\bv{g}}
\newcommand{\bfh}{\bv{h}}
\newcommand{\bfr}{\bv{r}}
\newcommand{\bfk}{\bv{k}}
\newcommand{\bfu}{\bv{u}}
\newcommand{\bfv}{\bv{v}}
\newcommand{\oo}[1]{o\pare{#1}}
\newcommand{\OO}[1]{O\pare{#1}}
\newcommand{\norm}[1]{\left\| #1 \right\|}
\newcommand{\DD}{\mathbf{D}}
\newcommand{\comp}{\circ}
\newcommand{\const}{\mathrm{const}}
\newcommand{\dist}[2]{d\pare{#1,#2}}
\newcommand{\len}{\ell}
\newcommand{\siga}{$\sigma$-代数}
\newcommand{\cara}{Carath\'{e}odory}
\newcommand{\Gd}{G_\delta}
\newcommand{\Fs}{F_\sigma}
\newcommand{\mmani}{$m$-维流形}
\newcommand{\open}[1]{\mathcal{#1}}
\newcommand{\half}{\frac{1}{2}}
\newcommand{\maxo}[1]{\text{max}\curb{#1}}
\newcommand{\mino}[1]{\text{min}\curb{#1}}
\newcommand{\epsclo}{$\epsilon$-接近}
\newcommand{\close}[1]{$#1$-接近}
\newcommand{\cinf}{$C^\infty$}
\newcommand{\cuno}{$C^1$}
\newcommand{\Int}{\text{Int}\,}
\newcommand{\Ext}{\text{Ext}\,}
\newcommand{\funcf}{\mathcal}
\newcommand{\DDu}{\overbar{\DD}}
\newcommand{\DDl}{\underbar{\DD}}
\newcommand{\Diff}[1]{\mathrm{Diff}_{#1}\,}
\newcommand{\Av}[1]{\mathrm{Av}_{#1}\,}
\newcommand{\Lip}[1]{Lipschitz-$#1$}
\newcommand{\sgn}[1]{\mathrm{sgn}}
\newcommand{\eset}{\varnothing}
\newcommand{\cT}{\mathcal{T}}
\newcommand{\cS}{\mathcal{S}}
\newcommand{\cG}{\mathcal{G}}
\newcommand{\cF}{\mathcal{F}}
\newcommand{\cC}{\mathcal{C}}
\newcommand{\cB}{\mathcal{B}}

\newcommand{\hd}{H\"{o}lder}

\renewcommand{\proofname}{证明}

\newenvironment{cenum}{\begin{enumerate}\itemsep0em}{\end{enumerate}}

\newtheorem{definition}{定义}[section]
\newtheorem{lemma}{引理}[section]
\newtheorem{theorem}{定理}[section]
\newtheorem{collary}{推论}[section]
\newtheorem{proposition}{命题}[section]
\newtheorem{axiom}{公理}[section]
\newtheorem{ex}{例}[section]

\newenvironment{aenum}{\begin{enumerate}[label=\textnormal{(\alph*)}]}{\end{enumerate}}
\begin{document}
\fi

%Content

\chapter{积分论}
  \section{Lebesgue积分}
  \subsection{Riemann积分}
  Riemann积分的定义如前不赘,唯注意下例。
  \begin{ex}
    对于Dirichlet函数
    \[
      f\pare{x}=
      \begin{cases}
        1, \quad x \in \Q, \\
        0, \quad x \in \brac{0,1}-\Q.
      \end{cases}
    \]
    虽可写为可数个简单函数之和,亦知其非Riemann可积。
  \end{ex}
  \subsection{有界函数在有限测度集上的Lebesgue积分}
  \begin{definition}
    对于有限测度集$E$上的简单函数$\psi$,定义其积分如
    \[ \int_E\psi = \sum a_i\cdot m\pare{E_i}. \]
    表达式中诸$a_i$不等。
  \end{definition}
  \begin{lemma}
    纵表达式中$a_i$简并,亦无改其积分值。
  \end{lemma}
  \begin{proposition}[积分的线性与单调性]
    对于简单函数$\varphi$与$\psi$,有
    \[ \int\pare{\alpha\varphi+\beta\psi} = \alpha\int\varphi + \beta\int\psi. \]
    以及若$\varphi<\psi$,则
    \[ \int\varphi < \int\psi. \]
  \end{proposition}
  \begin{proof}
    将$\varphi$与$\psi$共用一组$E_i$展开即可。
  \end{proof}
  此时已足够推断阶梯函数的Riemann与Lebesgue积分相符。
  \begin{definition}
    对有限测度集上的有界函数$f$,定义其Lebesgue上积分为全体$\varphi>f$之简单函数的Lebesgue积分的下界。相似定义Lebesgue下积分。
  \end{definition}
  \begin{definition}
    前开$f$若Lebesgue上下积分相等,则称之其Lebesgue积分。
  \end{definition}
  \begin{theorem}
    Lebesgue积分兼容Riemann积分。
  \end{theorem}
  \begin{proof}
    注意到阶梯函数含于简单函数即可。
  \end{proof}
  \begin{ex}
    注意Dirichlet函数$f=\chi_\Q$,故$\int f = m\pare{\Q}=0$。
  \end{ex}
  \begin{theorem}
    \label{thm:intable}
    有限测度集上定义的有界函数可积。
  \end{theorem}
  \begin{proof}
    注意其存在简单函数的上下逼近即可。
  \end{proof}
  \begin{proposition}[积分的线性与单调性]
    对于有限测度集上的可测函数$f$与$g$,有
    \[ \int\pare{\alpha f+\beta g} = \alpha\int f + \beta\int g. \]
    以及若$f<g$,则
    \[ \int f < \int g. \]
  \end{proposition}
  \begin{proof}
    只证$\alpha=\beta=1$的情况,目标积分不超二Lebesgue上积分之和而不低于二Legesgue下积分之和,再注意上下积分之和即积分之和。
    \par
    单调性考虑$\int\pare{f-g}$即可。
  \end{proof}
  \begin{collary}
    对无交可测集$A$与$B$,有
    \[ \int_{A\cup B}f=\int_A f+\int_B f. \]
  \end{collary}
  \begin{collary}
    对有限测度集上的有界函数$f$,有
    \[ \abs{\int f} \le \int \abs{f}. \]
  \end{collary}
  \begin{proof}
    注意$-\abs{f}\le f\le \abs{f}$即可。
  \end{proof}
  \begin{proposition}
    若有限测度集上的有界函数列$\curb{f}$一致收敛于$f$,则
    \[ \lim_{n\to\infty}\int f_n = \int \lim_{n\to\infty} f_n. \]
  \end{proposition}
  \begin{proof}
    注意$\norm{f-f_n}$可以任意小,借助前开推论即可。
  \end{proof}
  \begin{ex}
    考虑$f_n$定义为$f\pare{0}=0$,$f\pare{1/n}=n$,$f\pare{2/n}=0$并线性连接,则其除逐点收敛于零外满足前开所有条件,而积分后序列非零。
  \end{ex}
  \begin{theorem}[有界收敛定理]
    若有限测度集上的各点一致有界函数列$\curb{f}$逐点收敛于$f$,则
    \[ \lim_{n\to\infty}\int f_n = \int \lim_{n\to\infty} f_n. \]
  \end{theorem}
  \begin{proof}
    由Egoroff定理,$f_n$在任意接近$E$的闭集上一致收敛。故定义域的残余部分的积分任意小。
  \end{proof}
  \subsection{非负函数的Lebesgue测度}
  \begin{definition}
    定义$f$的支撑为使之非零的定义域部分\footnote{这和拓扑学上定义为其闭包不同。}。
  \end{definition}
  \begin{definition}
    设$f$为$E$上的非负可测函数,定义其积分
    \[ \int_E f = \sup \setcondl{\int_E h}{0\le h \le f}. \]
    其中$h$为有限测度集上定义的有界可测函数。
  \end{definition}
  \begin{proposition}[Chebychev不等式]
    设$f$非负可测,对$\lambda>0$,有
    \[ m\pare{f\ge \lambda} \le \rec{\lambda}\int_E f. \]
  \end{proposition}
  \begin{proof}
    取$g=\lambda\chi_{f\ge\lambda}$,并注意$0\le g\le f$。
  \end{proof}
  \begin{proposition}
    \label{prp:ae0}
    设$f$非负可测,则$\int f = 0$当且仅当$f$几乎处处为零。
  \end{proposition}
  \begin{proof}
    由前不等式,诸$m\pare{f\le1/n}=0$,并起即可。
  \end{proof}
  \begin{proposition}[积分的线性与单调性]
    对于非负可测函数$f$与$g$,有
    \[ \int\pare{\alpha f+\beta g} = \alpha\int f + \beta\int g. \]
    以及若$f<g$,则
    \[ \int f < \int g. \]
  \end{proposition}
  \begin{proof}
    易证$\int f + \int g \le \int \pare{f+g}$。反向的不等式则注意取$h=\mino{f,l}$,$k=l-h$,则$h$与$k$有界可测且
    \[ \int l = \int\pare{h+k} \le \int f + \int g. \]
    左侧取上界即可。单调性亦左侧取上界可证。
  \end{proof}
  \begin{theorem}[积分区间的可加性]
    $f$非负可测而$A$与$B$为无交可测集,则
    \[ \int_{A\cup B} f = \int_A f + \int_B f. \]
  \end{theorem}
  \begin{lemma}[Fatou引理]
    非负可测函数列$\curb{f_n}$几乎处处逐点收敛于$f$,则
    \[ \int_E f \le \lim\inf\int_E f_n. \]
  \end{lemma}
  \begin{proof}
    除开一零测集,可设其处处收敛。对任意$h$,设$h_n=\mino{h,f_n}$,故$h_n\to h$且由有界收敛定理
    \[ \lim_{n\to\infty} \int_E h_n = \int_E h. \]
    再注意$h_n\le f_n$,$\lim\int h_n \le \lim \inf \int f_n$即可。
  \end{proof}
  \begin{ex}
    令$E=\blr{0,1}$且$f_n=n\chi_{\pare{0,1/n}}$,则$f_n$极限的积分与积分的极限分别为$0$和$1$。再如$\chi_{\pare{n,n+1}}$逐点收敛至$0$但显然积分与极限不可互换。
  \end{ex}
  \begin{theorem}[单调收敛定理]
    在Fatou引理的条件下,若$\curb{f_n}$递增,则
    \[ \lim_{n\to\infty}\int f_n = \int f. \]
  \end{theorem}
  \begin{proof}
    由积分的单调性知
    \[ \lim\sup\int f_n \le \int f. \qedhere \]
  \end{proof}
  \begin{collary}
    非负可测函数和$\sum u_n$几乎处处逐点收敛于$f$,则
    \[ \int f = \sum \int u_n. \]
  \end{collary}
  \begin{definition}
    积分有限的可测函数称为可积函数。
  \end{definition}
  \begin{proposition}
    可积函数几乎处处有限。
  \end{proposition}
  \begin{proof}
    注意对任意$n$,有
    \[ m\pare{f \ge n} \le \rec{n}\int f. \]
  \end{proof}
  \begin{lemma}[Beppo Levi引理]
    非负可测函数列$\curb{f_n}$诸积分一致有界,则$f_n$逐点收敛于一几乎处处有界的可积函数。
  \end{lemma}
  \begin{proof}
    递增数列收敛于一广义实数,故定义$f\pare{x}=\lim f_n\pare{x}$,复用前开命题与有界收敛定理。
  \end{proof}
  \subsection{一般Lebesgue积分}
  注意$f=f^+-f^-$且$\abs{f} = f^++f^-$。
  \begin{proposition}
    对可测函数$f$,$f^+$与$f^-$可积当且仅当$\abs{f}$可积。
  \end{proposition}
  \begin{definition}
    若$\abs{f}$可积则称可测函数$f$可积且定义
    \[ \int f = \int f^+ - \int f^-. \]
  \end{definition}
  \begin{proposition}
    \label{prp:Ehole}
    若$f$可积,则$\abs{f}$几乎处处有限且对零测集$E_0$,
    \[ \int_E f = \int_{E-E_0} f. \]
  \end{proposition}
  \begin{proof}
    前开命题知几乎处处有限。再注意对非负函数有相同成立即可。
  \end{proof}
  \begin{proposition}[比较审敛法]
    若$\abs{f}$处处小于一可积函数,则$f$可积且
    \[ \abs{\int f} \le \int \abs{f}. \]
  \end{proposition}
  \begin{proof}
    可积性易证。再由实数的三角不等式,
    \[ \abs{\int f^+ - \int f^-} \le \int f^+ + \int f^- \le \int \abs{f}. \]
  \end{proof}
  注意由\pref{Ehole},两可积函数若某处值无限,则积分可径直挖去该点而无需定义在该点的值。
  \begin{proposition}[积分的线性与单调性]
    对于可积函数$f$与$g$,有
    \[ \int\pare{\alpha f+\beta g} = \alpha\int f + \beta\int g. \]
    以及若$f<g$,则
    \[ \int f < \int g. \]
  \end{proposition}
  \begin{proof}
    可积性由$\abs{f+g}\le\abs{f}+\abs{g}$得,其余易证。
  \end{proof}
  \begin{collary}
    对无交可测集$A$与$B$,有
    \[ \int_{A\cup B}f=\int_A f+\int_B f. \]
  \end{collary}
  \begin{theorem}[Lebesgue控制收敛定理]
    逐点收敛于$f$的可测函数列$\curb{f_n}$满足$\abs{f_n}\le g$,则有$f$可积且
    \[ \lim_{n\to\infty}\int f_n = \int f. \]
  \end{theorem}
  \begin{proof}
    注意到由Fatou引理,
    \[ \int \pare{g+f} \le \lim \inf \int \pare{g+f_n},  \]
    以及
    \[ \int \pare{g-f} \le \lim \inf \int \pare{g-f_n}. \qedhere \]
  \end{proof}
  \begin{theorem}[一般的Lebesgue控制收敛定理]
    逐点收敛于$f$的可测函数列$\curb{f_n}$满足$\abs{f_n}\le g_n$,若$\curb{g_n}$几乎处处收敛于$g$,且
    \[ \lim_{n\to\infty}\int g_n = \int g, \]
    则有$f$可积且
    \[ \lim_{n\to\infty}\int f_n = \int f. \]
  \end{theorem}
  \begin{proof}
    证法同上。
  \end{proof}
  \subsection{积分的可数可加性与连续性}
  \begin{theorem}[积分的可数可加性]
    设$f$可积而$\curb{E_n}$为无交可测集族,其并为$E$,则
    \[ \int f = \sum \int_n f. \]
  \end{theorem}
  \begin{proof}
    对$f_n = f \chi_{E_1\cap\cdots\cap E_n}$应用控制收敛定理。
  \end{proof}
  \begin{theorem}[积分的连续性]
    $f$为$E$上的可积函数,则
    \begin{aenum}
      \item 若$\curb{A_k}$为升链,则
      \[ \int_{\cup A_n} f = \lim_{n\to\infty} \int_{A_n} f. \]
      \item 若$\curb{B_k}$为降链,则
      \[ \int_{\cap B_n} f = \lim_{n\to\infty} \int_{B_n} f. \]
    \end{aenum}
  \end{theorem}
  \subsection{一致可积性}
  \begin{lemma}
    有限测度集可以被划分为有限个测度小于$\delta$的无交集。
  \end{lemma}
  \begin{proof}
    注意$m\pare{E-\brac{-n,n}}$迟早小于$\delta$后划分$\brac{-n,n}$即可。
  \end{proof}
  \begin{proposition}
    \label{prp:previtali}
    $f$在$E$上可积,则对于任意小的$\epsilon$,存在$\delta$使得对任意满足$m\pare{A}<\delta$的子集$A$有
    \[ \int_A\abs{f}<\epsilon. \]
    反之,若$E$测度有限而对任意小的$\epsilon$,存在上述的$\delta$,则$f$可积。
  \end{proposition}
  \begin{proof}
    仅考虑正的$f$。正向结论可由定义以有界函数逼近$f$并注意有界性推得。反向结论则选取一对$\epsilon$与$\delta$,并由前引理将$E$写为有限个小集的并。
  \end{proof}
  \begin{definition}
    $E$上的可测函数族称为一致可积,若对于任意小的$\epsilon$,存在$\delta$使得对任意$m\pare{A}<\delta$以及其中的$f$,有
    \[ \int_A \abs{f} < \epsilon. \]
  \end{definition}
  \begin{ex}
    设$g$可积,所有满足$\abs{f}<g$的可测函数为一致可积。
  \end{ex}
  \begin{proposition}
    有限个可积函数构成的族是一致可积的。
  \end{proposition}
  \begin{proposition}
    若有限测度的$E$上一致可积的$\curb{f_n}$几乎处处逐点收敛于$f$,则$f$可积。
  \end{proposition}
  \begin{proof}
    由\pref{previtali},诸$f_n$的积分一致有界,由Fatou引理
    \[ \int \abs{f} \le \lim\inf \int \abs{f_n}. \qedhere \]
  \end{proof}
  \begin{theorem}[Vitali收敛定理]
    若有限测度的$E$上一致可积的$\curb{f_n}$几乎处处逐点收敛于$f$,则
    \[ \lim_{n\to\infty}\int f_n = \int f. \]
  \end{theorem}
  \begin{proof}
    由Egoroff定理,选取任意逼近$E$的$A$使得$\curb{f_n}$一致收敛,则
    \[ \abs{\int_E f_n - \int_E f} \le \int_{E-A}\abs{f_n-f} + \int_A\abs{f_n} + \int_A\abs{f}. \]
  第一项积分由一致收敛任意小,后二项由一致可积与Fatou引理任意小。
  \end{proof}
  \begin{theorem}
    有限测度集上几乎处处收敛于零的非负可测函数列$\curb{h_n}$,当且仅当其一致可积时有
    \[ \lim_{n\to\infty}\int h_n = 0. \]
  \end{theorem}
  \begin{proof}
    只证极限为零推出一致可积。对任意$\epsilon>0$,可以选取足够大的$N$使
    \[ \int h_{N:} < \epsilon, \]
    再注意有限个$h_{:N}$一致可积即可。
  \end{proof}
  \section{进一步的主题}
  \subsection{一致可积性与测度紧密型}
  \begin{ex}
    对无限测度的$E$,考虑$f=\chi_{\brac{n,n+1}}$知Vitali定理不适用。
  \end{ex}
  \begin{proposition}
    设$f$可积,则在一有限测度集$E_0$外其积分任意小。
  \end{proposition}
  \begin{proof}
    有定义知存在有限测度上有界的函数其积分任意逼近$f$。
  \end{proof}
  \begin{definition}
    $E$上的可测函数族$\funcf{F}$称为紧密的,如果存在一有限测度集$E_0$使其全体在其外的积分一致任意小。
  \end{definition}
  \begin{theorem}[Vitali收敛定理]
    若$E$上一致可积且紧密的$\curb{f_n}$几乎处处逐点收敛于$f$,则
    \[ \lim_{n\to\infty}\int f_n = \int f. \]
  \end{theorem}
  \begin{proof}
    选取$E_0$使其外的积分任意小,其内的积分调用前开Vitali定理。
  \end{proof}
  \begin{collary}
    $E$上几乎处处收敛于零的非负可测函数列$\curb{h_n}$,当且仅当其一致可积且紧密时有
    \[ \lim_{n\to\infty}\int h_n = 0. \]
  \end{collary}
  \subsection{依测度收敛}
  \begin{definition}
    $E$上几乎处处有限的可测函数列$\curb{f_n}$称为依测度收敛于可测的$f$,如果对任意$\eta$,
    \[ \lim_{n\to\infty}m\setcond{x}{\abs{f_n\pare{x}-f\pare{x}}>\eta} = 0. \]
  \end{definition}
  \begin{proposition}
    有限测度$E$上的逐点收敛是一致收敛。
  \end{proposition}
  \begin{proof}
    由Egoroff定理选取逼近$E$的闭集上的一致收敛即可。
  \end{proof}
  \begin{ex}
    考虑下述诸区间上的特征函数,虽依测度收敛却非逐点收敛。
    \[ \brac{0,1},\brac{0,1/2},\brac{1/2,1},\brac{0,1/3},\brac{1/3,2/3},\brac{2/3,1},\brac{0,1/4}\cdots. \]
  \end{ex}
  \begin{theorem}[Riesz]
    依测度收敛的函数列存在几乎处处逐点收敛的子列。
  \end{theorem}
  \begin{proof}
    由依测度收敛知存在子列使$m\pare{\abs{f_{n_k}-f}>1/k}<1/2^k$,再由Borel-Cantelli引理知几乎每个$x$都收敛于$f$。
  \end{proof}
  \begin{collary}
    \label{coll:mcon2int}
    非负可积函数列$\curb{f_n}$满足
    \[ \lim_{n\to\infty} \int f = 0 \]
    当且仅当其依测度收敛于零且一致可积且紧密。
  \end{collary}
  \begin{proof}
    由前开推论与Chebyshev不等式知其依测度收敛。
    \par
    反之,假设积分不收敛于零,则存在一子列之积分漂浮于一实数之上,此子列存在一几乎处处逐点收敛之子列,由Vitali定理知矛盾。
  \end{proof}
  \subsection{Riemann与Lebesgue可积性的特征}
  \begin{lemma}
    设$\curb{\varphi_n}$与$\curb{\psi_n}$分别为可积函数的升列与降列且夹挤$f$,若
    \[ \lim_{n\to\infty}\int\brac{\psi_n-\varphi_n}=0, \]
    则$\curb{\varphi_n}$与$\curb{\psi_n}$几乎处处逐点收敛于$f$且$f$可积且三者积分相等。
  \end{lemma}
  \begin{proof}
    由单调收敛定理,$\int\pare{\psi-\varphi}\to 0$。由\pref{ae0}知其几乎处处为零,从而几乎处处逐点收敛于$f$,进而其可测。三积分相等易证。
  \end{proof}
  \begin{theorem}
    有限测度集上的有界$f$,其可积当且仅当其可测。
  \end{theorem}
  \begin{proof}
    若假设可测,由\tref{intable}知可积。
    \par
    若已知可积,则由定义知存在$f$的上下简单函数逼近且积分差为零,取诸$\maxo{\varphi_i}$与$\mino{\psi_i}$可设其分别为升降列,再调用前开引理。
  \end{proof}
  \begin{theorem}[Lebesgue]
    紧区间上的有界函数$f$为Riemann可积当且仅当其非连续点为零测集。
  \end{theorem}
  \begin{proof}
    假定Riemann可积,则存在一列加细的划分$\curb{P_n}$,对应上下逼近$\curb{\varphi_n}$与$\curb{\psi_n}$且由前开引理几乎处处收敛于$f$。在除开$P_\infty$的点处,对$\epsilon$取足够大的$N$即可使此处$\psi_n-\varphi_n<\epsilon$,从而此点的$\delta$邻域内变差任意小。
    \par
    反之,假定$f$不连续点为零测集。对加细至稠密的划分列$P_n$以及$P_\infty$及不连续点以外的点,选取足够的大$n$使$P_n$的间隙小于$\delta$,则$f$在诸间隙内的变差小$\epsilon$,故上下逼近可互相接近而积分相等。
  \end{proof}
  \section{微分与积分}
  \subsection{单调函数的连续性}
  \begin{theorem}
    单调函数最多仅有可数个不连续点。
  \end{theorem}
  \begin{proposition}
    对开区间内的可数集,存在增函数仅在此可数集上不连续。
  \end{proposition}
  \begin{proof}
    取$f$如下,在任意$E-C$的点有足够小的开区间不包含$q_1,\cdots ,q_n$。
    \[ f\pare{x} = \sum_{q_n\le x} 1/2^n. \qedhere \]
  \end{proof}
  \subsection{单调函数的可微性}
  \begin{definition}
    非退化紧区间集$\setf{F}$称为$E$的Vitali覆盖,如果对于任意点$x$和$\epsilon>0$,存在长度小于$\epsilon$的区间覆盖$x$。
  \end{definition}
  \begin{lemma}[Vitali覆盖引理]
    设$E$为有限外测度集,$\setf{F}$是其Vitali覆盖,则其无交有限子集任意可接近$E$。
  \end{lemma}
  \begin{proof}
    若存在有限子集覆盖之则证毕。反之,依"在剩余无交区间内选取区间长度过半者"之程式选取"下一区间"而得无交可数族,则任意有限子集外的区间与可数族内一区间有交。将后者扩大5倍即可覆盖之。故其有限子集外者扩大5倍后便可覆盖$E$。
  \end{proof}
  \begin{definition}
    定义上导数
    \[ \DDu f\pare{x} = \lim_{h\to 0}\brac{\sup_{0<\abs{t}\le h} \frac{f\pare{x+t}-f\pare{x}}{t}}. \]
    相似定义下导数。若二者相等则称可导。
  \end{definition}
  \begin{lemma}
    设$f$为紧区间上的增函数,则对任意正数$\alpha$,
    \[ m^*\pare{\DDu f \ge \alpha} \le \rec{\alpha}\brac{f\pare{b}-f\pare{a}}. \]
    特别地,$m^*\pare{\DDu f = \infty} = 0$。
  \end{lemma}
  \begin{proof}
    注意诸$\Delta f \ge \alpha^-\pare{d-c}$的区间$\brac{c,d}$构成其Vitali覆盖。
  \end{proof}
  \begin{theorem}[Lebesgue]
    开区间上的单调函数几乎处处可导。
  \end{theorem}
  \begin{proof}
    设$E$中上导数大于$\alpha$而下导数小于$\beta$,则诸$\Delta f \le \beta \pare{d-c}$的$\brac{c,d}$构成$E$的Vitali覆盖。$\sum^n \Delta f \le \beta m^*\pare{E}$。再由前开引理,
    \[ m^*\pare{E}\le \rec{\alpha} \sum^n \Delta f. \qedhere \]
  \end{proof}
  \begin{definition}
    紧区间上的可积函数$f$,两侧水平延伸其值,对正数$h$定义差分与平均分别为
    \[ \Diff{h}f\pare{x}=\frac{f\pare{x+h}-f\pare{h}}{h},\quad \Av{h}f\pare{x} = \rec{h}\int_x^{x+h}f. \]
  \end{definition}
  \begin{collary}
    \label{coll:diff2int}
    紧区间上的增函数,其导数可积且
    \[ \int f' \le f\pare{b} - f\pare{a}. \]
  \end{collary}
  \begin{proof}
    由Fatou引理,
    \[ \int f' \le \lim \inf \int \Diff{} f. \qedhere \]
  \end{proof}
  参考Cantor-Lebesgue函数知等号可严格成立。
  \subsection{有界变差函数:Jordan定理}
  \begin{definition}
    定义变差为
    \[ V\pare{f,P} = \sum \abs{\Delta f}, \]
    全变差为$TV = \sup V$。若全变差有界,则称之有界变差。
  \end{definition}
  \begin{ex}
    增函数,\Lip{1}的函数是有界变差的。$x\cos\pare{\pi/2x}$则不是。
  \end{ex}
  \begin{lemma}
    有界变差函数可以写为如下二增函数之差:
    \begin{equation}
      \label{eq:TVd}
      f\pare{x} = \brac{f\pare{x} + TV\pare{x}} - TV\pare{x}.
    \end{equation}
  \end{lemma}
  \begin{theorem}[Jordan]
    紧区间上的函数有界变差当且仅当其为增函数之差。
  \end{theorem}
  \begin{proof}
    $f=g-h$称为其Jordan分解,只需注意
    \[ V\pare{f,P} = \sum \abs{\Delta f} \le \sum \abs{\Delta g} + \sum \abs{\Delta h}. \qedhere \]
  \end{proof}
  \begin{collary}
    紧区间上的有界变差函数几乎处处可微且导数可积。
  \end{collary}
  \subsection{绝对连续函数}
  \begin{definition}
    对任意$\epsilon>0$,存在$\delta>0$使$\sum \pare{b_k-a_k}$之一切区间族上$\sum \abs{\Delta f} < \epsilon$,则称$f$绝对连续。
  \end{definition}
  \begin{ex}
    Cantor-Lebesgue函数虽连续却非绝对连续。
  \end{ex}
  \begin{proposition}
    \Lip{1}的函数绝对连续。
  \end{proposition}
  \begin{theorem}
    紧区间上的绝对连续函数可写为绝对连续的增函数之差。
  \end{theorem}
  \begin{proof}
    注意绝对连续的$f$的全变差为绝对连续即可。
  \end{proof}
  \begin{theorem}
    \label{thm:abs2unint}
    紧区间上的连续函数绝对连续当且仅当$\lbr{0,1}$的差分一致可积。
  \end{theorem}
  \begin{proof}
    设其差分一致可积,注意$\Delta \Av{h}f = \int \Diff{h} f$以及$\lim \Av{h} f = f$。\par
    反之只证$f$为非负绝对连续增函数的情形,注意只需证"任意小的区间集上的积分任意小"即可,再藉$\int \Diff{h}f = 1/h\cdot\int \brac{f\pare{u+t}-f\pare{v+t}}$。
  \end{proof}
  可以发现如下的包含关系
  \[ \funcf{F}_{Lip} \subset \funcf{F}_{AC} \subset \funcf{F}_{BV}. \]
  且各族内的函数都可以如\eqref{eq:TVd}写成族内二增函数之差。
  \subsection{积分下的微分}
  \begin{theorem}
    \label{thm:int2dif}
    紧区间上的绝对连续函数几乎处处可微且
    \[ \int_a^b f' = f\pare{b} - f\pare{a}. \]
  \end{theorem}
  \begin{proof}
    注意
    \[ \lim_{h\to 0}\int_a^b \Diff{h} f = \lim_{h\to 0} \brac{\Av{h} f\pare{b} - \Av{h} f\pare{a}}. \]
    右侧为所求,左侧由\tref{abs2unint}知一致可积后调用Vitali定理。
  \end{proof}
  \begin{theorem}
    紧区间上的函数一致连续当且仅当其为一不定积分。
  \end{theorem}
  \begin{proof}
    假设$f=\int g$,注意由\pref{previtali},小测度上$g$的积分可任意小。
  \end{proof}
  \begin{collary}
    紧区间上单调的$f$为绝对连续当且仅当
    \[ \int_a^b f' = f\pare{b}-f\pare{a}. \]
  \end{collary}
  \begin{proof}
    由\cref{diff2int},
    \[ \int_a^x f' \le f\pare{x}-f\pare{a},\quad \int_x^b f' \le f\pare{b}-f\pare{x}. \]
    故二者均为相等,从而$f=\int f'$。
  \end{proof}
  \begin{lemma}
    紧区间上的$f$几乎处处为零当且仅当任意区间上积分为零。
  \end{lemma}
  \begin{proof}
    注意任意区间上积分为零得出任意$\Gd$型集上积分为零即可。
  \end{proof}
  \begin{theorem}
    紧区间上的可积函数几乎处处有
    \[ \DD \int_a^x f = f\pare{x}. \]
  \end{theorem}
  \begin{proof}
    借助前开命题,注意对任意$\brac{x_1,x_2}$,有
    \[ \int_{x_1}^{x_2}\brac{F'-f} = F\pare{x_2}-F\pare{x_1} - \int_{x_1}^{x_2} f = 0. \qedhere \]
  \end{proof}
  并非所有函数解有\tref{int2dif}的适用。借助前开命题,考虑下述的分解
  \[ f = \pare{f-\int f'} + \int f', \]
  前者导数几乎处处为零,后者为一绝对连续函数,此谓其Lebesgue分解。
  \subsection{凸函数}
  \begin{definition}
    满足下式者谓凸函数,其中$a+b=1$。
    \[ \varphi\pare{a x_1 + b x_2} \le a \varphi\pare{x_1} + b \varphi\pare{x_2}. \]
  \end{definition}
  在上式中令
  \[ a=\frac{x_2-x}{x_2-x_1},\quad b=\frac{x-x_1}{x_2-x_1}, \]
  可得对于任意$x_1<x<x_2$,
  \[ \frac{\varphi\pare{x}-\varphi\pare{x_1}}{x-x_1} \le \frac{\varphi\pare{x_2}-\varphi\pare{x}}{x_2-x}. \]
  \begin{proposition}
    若$\varphi$可微而$\varphi'$为增函数,则$\varphi$为凸函数。
  \end{proposition}
  \begin{lemma}[弦斜率]
    凸函数上顺次三点$p_1$,$p$,$p_2$,有$k_{p_1p}<k_{p_1p_2}<k_{pp_2}$。
  \end{lemma}
  \begin{lemma}
    凸函数$\varphi$各点左右导数存在,且若$u<v$则
    \[ \varphi'\pare{u^-} \le \varphi'\pare{u^+}\le\frac{\varphi\pare{v}-\varphi\pare{u}}{v-u}\le\varphi'\pare{v^-}\le\varphi'\pare{u^+}. \]
  \end{lemma}
  \begin{collary}
    开区间上的凸函数是Lipschitz的,在任意紧区间上绝对连续。
  \end{collary}
  \begin{theorem}
    凸函数几乎处处可导且导函数为增函数。
  \end{theorem}
  \begin{theorem}[Jensen不等式]
    对凸函数$\varphi$与可积函数$f$,设$\varphi\comp f$可积,有
    \[ \varphi\pare{\int_0^1 f} \le \int_0^1 \varphi\comp f. \]
  \end{theorem}
  \begin{proof}
    令$\alpha=\int f$,则
    \[ \int_0^1 \varphi\comp f \ge \int_0^1 \brac{m\pare{f-\alpha}+\varphi\pare{\alpha}} = \varphi\pare{\alpha}. \qedhere \]
  \end{proof}
  将上开不等式用于加和为$1$的诸$\alpha_n$,
  \[ \sum \alpha_n \log x_n \le \log \sum \alpha_n x_n \]
  可得算术-几何不等式。
  
%ContentEnds
 
\ifx\allfiles\undefined %如果位置放错,可能出现意外中断
\end{document}
\fi