%Ch8.MeasureTheory.tex
\ifx\allfiles\undefined
\documentclass{ctexrep}
% Mathematics Include

\usepackage{amsmath}
\usepackage{amssymb}
\usepackage{amsthm}
\usepackage{amsfonts}
\usepackage{mathrsfs}
\usepackage{enumitem}
\usepackage{braket}
\usepackage{hyperref}
\usepackage[all, pdf]{xy}

% Physics Include
\usepackage{amsmath}
\usepackage{physics}
\usepackage{siunitx}
\usepackage[makeroom]{cancel}
\usepackage{pstricks}
\usepackage{pstricks-add}
\psset{algebraic=true}

\usepackage[version=4]{mhchem}
\usepackage{array,booktabs}
\usepackage{longtable}
\usepackage{mathtools}
\usepackage[normalem]{ulem}
\usepackage{multicol}

% Mathematics Head

\newcommand{\pare}[1]{\left(#1\right)}
\newcommand{\blr}[1]{\left[#1\right)}
\newcommand{\lbr}[1]{\left(#1\right]}
\newcommand{\brac}[1]{\left[#1\right]}
\newcommand{\curb}[1]{\left\{#1\right\}}
% \newcommand{\abs}[1]{\left|\, #1 \,\right|}
\newcommand{\rec}[1]{\frac{1}{#1}}
\newcommand{\N}{\mathbb{N}}
\newcommand{\bC}{\mathbb{C}}
\newcommand{\Q}{\mathbb{Q}}
\newcommand{\Z}{\mathbb{Z}}
\newcommand{\R}{\mathbb{R}}
\newcommand{\unk}{\mathcal{X}}
\newcommand{\bu}[3]{#1_{#2}^{\pare{#3}}}
\newcommand{\dref}[1]{定义\ref{def:#1}}
\newcommand{\tref}[1]{定理\ref{thm:#1}}
\newcommand{\lref}[1]{引理\ref{lem:#1}}
\newcommand{\cref}[1]{推论\ref{coll:#1}}
\newcommand{\pref}[1]{命题\ref{prp:#1}}
\newcommand{\eref}[1]{例\ref{ex:#1}}
\newcommand{\func}[3]{#1:\, #2 \rightarrow #3}
\newcommand{\overbar}[1]{\mkern 1.5mu\overline{\mkern-1.5mu#1\mkern-1.5mu}\mkern 1.5mu}
\newcommand{\clo}[1]{\overbar{#1}}
\newcommand{\supi}[2]{\overbar{\int_{#1}^{#2}}}
\newcommand{\infi}[2]{\underbar{\int_{#1}^{#2}}}
\newcommand{\setf}{\mathscr}
\newcommand{\bool}{\mathrm{bool}}
\newcommand{\inc}{++}
\newcommand{\defeq}{:=}
\newcommand{\ntuple}{$n$元组}
\newcommand{\card}[1]{\#\pare{#1}}
\newcommand{\setcond}[2]{\curb{#1 \, \left| \, #2 \right.}}
\newcommand{\setcondl}[2]{\curb{\left. #1 \, \right| \, #2}}
\newcommand{\bv}[1]{\mathbf{#1}}
\newcommand{\bfa}{\bv{a}}
\newcommand{\bfb}{\bv{b}}
\newcommand{\bfx}{\bv{x}}
\newcommand{\bfy}{\bv{y}}
\newcommand{\bfe}{\bv{e}}
\newcommand{\bfF}{\bv{F}}
\newcommand{\bff}{\bv{f}}
\newcommand{\bfG}{\bv{G}}
\newcommand{\bfH}{\bv{H}}
\newcommand{\bfg}{\bv{g}}
\newcommand{\bfh}{\bv{h}}
\newcommand{\bfr}{\bv{r}}
\newcommand{\bfk}{\bv{k}}
\newcommand{\bfu}{\bv{u}}
\newcommand{\bfv}{\bv{v}}
\newcommand{\oo}[1]{o\pare{#1}}
\newcommand{\OO}[1]{O\pare{#1}}
% \newcommand{\norm}[1]{\left\| #1 \right\|}
\newcommand{\DD}{\mathbf{D}}
\newcommand{\comp}{\circ}
\newcommand{\const}{\mathrm{const}}
\newcommand{\dist}[2]{d\pare{#1,#2}}
\newcommand{\len}{\ell}
\newcommand{\siga}{$\sigma$-代数}
\newcommand{\cara}{Carath\'{e}odory}
\newcommand{\Gd}{G_\delta}
\newcommand{\Fs}{F_\sigma}
\newcommand{\mmani}{$m$-维流形}
\newcommand{\open}[1]{\mathcal{#1}}
\newcommand{\half}{\frac{1}{2}}
\newcommand{\maxo}[1]{\text{max}\curb{#1}}
\newcommand{\mino}[1]{\text{min}\curb{#1}}
\newcommand{\epsclo}{$\epsilon$-接近}
\newcommand{\close}[1]{$#1$-接近}
\newcommand{\cinf}{$C^\infty$}
\newcommand{\cuno}{$C^1$}
\newcommand{\Int}{\text{Int}\,}
\newcommand{\Ext}{\text{Ext}\,}
\newcommand{\funcf}{\mathcal}
\newcommand{\DDu}{\overbar{\DD}}
\newcommand{\DDl}{\underbar{\DD}}
\newcommand{\Diff}[1]{\mathrm{Diff}_{#1}\,}
\newcommand{\Av}[1]{\mathrm{Av}_{#1}\,}
\newcommand{\Lip}[1]{Lipschitz-$#1$}
\newcommand{\sgn}{\mathrm{sgn}}
\newcommand{\eset}{\varnothing}
\newcommand{\cT}{\mathcal{T}}
\newcommand{\cS}{\mathcal{S}}
\newcommand{\cG}{\mathcal{G}}
\newcommand{\cF}{\mathcal{F}}
\newcommand{\cC}{\mathcal{C}}
\newcommand{\cB}{\mathcal{B}}
\newcommand{\inter}[1]{\mathring{#1}}
\newcommand{\forest}[3]{对于{#1},存在{#2},使得{#3}}
\newcommand{\tuno}{$T_1$公理}
\newcommand{\isom}{\overset{\sim}{=}}
\newcommand{\diam}{\mathrm{diam}\,}
\newcommand{\ord}[1]{\abs{#1}}
\newcommand{\sbm}[1]{\overbar{#1}}
\newcommand{\inv}[1]{#1^{-1}}
\newcommand{\restr}[2]{#1|_{#2}}
\newcommand{\divs}{|}
\newcommand{\ndivs}{\nmid}
\newcommand{\modeq}[1]{\overbar{#1}}
\newcommand{\ggen}[1]{\langle#1\rangle}
\newcommand{\ggencond}{\braket}

\newcommand{\hd}{H\"{o}lder}

\renewcommand{\proofname}{证明}

\newenvironment{cenum}{\begin{enumerate}\itemsep0em}{\end{enumerate}}

\newtheorem{definition}{定义}[section]
\newtheorem{lemma}{引理}[section]
\newtheorem{theorem}{定理}[section]
\newtheorem{collary}{推论}[section]
\newtheorem{corollary}{推论}[section]
\newtheorem{proposition}{命题}[section]
\newtheorem{axiom}{公理}[section]
\newtheorem{ex}{例}[section]
\newtheorem{reflection}{反射}[section]
\newcommand{\refl}[1]{\vspace{0.5em}\par\noindent\fbox{%
    \parbox{0.9\textwidth}{%
    \begin{reflection}
        #1
    \end{reflection}
    }%
}\vspace{0.5em}\par}
\newcommand{\rref}[1]{反射\ref{refl:#1}}
\newcommand{\tbref}[1]{表\ref{table:#1}}
\allowdisplaybreaks

\newenvironment{aenum}{\begin{enumerate}[label=\textnormal{(\alph*)}]}{\end{enumerate}}

% Physics Head

\DeclareSIUnit\dyne{dynes}

\newcommand{\ddel}[1]{\frac{\partial}{\partial #1}}
\newcommand{\ddelon}[2]{\frac{\partial #1}{\partial #2}}
\newcommand{\dddel}[1]{\frac{\partial^2}{\partial^2 #1}}
\newcommand{\ddt}{\ddel{t}}
\newcommand{\ddT}{\ddel{T}}
\newcommand{\ddV}{\ddel{V}}
\newcommand{\ddr}{\ddel{r}}
\newcommand{\ddth}{\ddel{\theta}}
\newcommand{\ddph}{\ddel{\phi}}
\newcommand{\dddt}{\dddel{t}}
\newcommand{\dddr}{\dddel{t}}
\newcommand{\dddth}{\dddel{\theta}}
\newcommand{\dddph}{\dddel{\phi}}
\newcommand{\rd}[1]{\mathrm{d} #1}
\newcommand{\dt}{\rd{t}}
\newcommand{\dy}{\rd{y}}
\newcommand{\dx}{\rd{x}}
\newcommand{\edd}[1]{\frac{\mathrm{d}}{\mathrm{d} #1}}
\newcommand{\eddd}[1]{\frac{\mathrm{d}^2}{\mathrm{d}^2 #1}}
\newcommand{\eddon}[2]{\frac{\mathrm{d} #1}{\mathrm{d} #2}}
\newcommand{\edddon}[2]{\frac{\mathrm{d}^2 #1}{\mathrm{d}^2 #2}}
\newcommand{\edt}{\edd{t}}
\newcommand{\edton}[1]{\eddon{#1}{t}}
\newcommand{\edT}{\edd{T}}
\newcommand{\edr}{\edd{r}}
\newcommand{\edl}{\edd{l}}
\newcommand{\edx}{\edd{x}}
\newcommand{\edth}{\edd{\theta}}
\newcommand{\eddton}[1]{\edddon{#1}{t}}
\newcommand{\eddzon}[1]{\edddon{#1}{z}}
\newcommand{\vect}[1]{\boldsymbol{#1}}
\newcommand{\alp}{\frac{1}{\sqrt{2}}}
\newcommand{\alpi}{\frac{i}{\sqrt{2}}}
\newcommand{\expc}[1]{\langle#1\rangle}
\newcommand{\bkn}[1]{\bra{#1}\ket{#1}}
\newcommand{\bk}[2]{\bra{#1}\ket{#2}}
\newcommand{\bik}[3]{\bra{#1} #2 \ket{#3}}
\newcommand{\vari}[1]{\sigma_{#1}}
\newcommand{\intc}[2]{\left[#1, #2\right]}
\newcommand{\sch}{Schr\"{o}dinger}
\newcommand{\moment}{\boldsymbol{p}}
\newcommand{\coor}{\boldsymbol{x}}
\newcommand{\lapc}{\nabla^2}
% \newcommand{\rec}[1]{\frac{1}{#1}}
\newcommand{\vva}{\boldsymbol{a}}
\newcommand{\vvb}{\boldsymbol{b}}
\newcommand{\vc}{\boldsymbol{c}}
\newcommand{\vd}{\boldsymbol{d}}
\newcommand{\ve}{\boldsymbol{e}}
\newcommand{\vf}{\boldsymbol{f}}
\newcommand{\vg}{\boldsymbol{g}}
\newcommand{\vh}{\boldsymbol{h}}
\newcommand{\vi}{\boldsymbol{i}}
\newcommand{\vj}{\boldsymbol{j}}
\newcommand{\vk}{\boldsymbol{k}}
\newcommand{\vl}{\boldsymbol{l}}
\newcommand{\vm}{\boldsymbol{m}}
\newcommand{\vn}{\boldsymbol{n}}
\newcommand{\vo}{\boldsymbol{o}}
\newcommand{\vp}{\boldsymbol{p}}
\newcommand{\vq}{\boldsymbol{q}}
\newcommand{\vr}{\boldsymbol{r}}
\newcommand{\vs}{\boldsymbol{s}}
\newcommand{\vt}{\boldsymbol{t}}
\newcommand{\vvu}{\boldsymbol{u}}
\newcommand{\vv}{\boldsymbol{v}}
\newcommand{\vw}{\boldsymbol{w}}
\newcommand{\vx}{\boldsymbol{x}}
\newcommand{\vy}{\boldsymbol{y}}
\newcommand{\vz}{\boldsymbol{z}}
\newcommand{\vA}{\boldsymbol{A}}
\newcommand{\vB}{\boldsymbol{B}}
\newcommand{\vC}{\boldsymbol{C}}
\newcommand{\vD}{\boldsymbol{D}}
\newcommand{\vE}{\boldsymbol{E}}
\newcommand{\vF}{\boldsymbol{F}}
\newcommand{\vG}{\boldsymbol{G}}
\newcommand{\vH}{\boldsymbol{H}}
\newcommand{\vI}{\boldsymbol{I}}
\newcommand{\vJ}{\boldsymbol{J}}
\newcommand{\vK}{\boldsymbol{K}}
\newcommand{\vL}{\boldsymbol{L}}
\newcommand{\vM}{\boldsymbol{M}}
\newcommand{\vN}{\boldsymbol{N}}
\newcommand{\vO}{\boldsymbol{O}}
\newcommand{\vP}{\boldsymbol{P}}
\newcommand{\vQ}{\boldsymbol{Q}}
\newcommand{\vR}{\boldsymbol{R}}
\newcommand{\vS}{\boldsymbol{S}}
\newcommand{\vT}{\boldsymbol{T}}
\newcommand{\vU}{\boldsymbol{U}}
\newcommand{\vV}{\boldsymbol{V}}
\newcommand{\vW}{\boldsymbol{W}}
\newcommand{\vX}{\boldsymbol{X}}
\newcommand{\vY}{\boldsymbol{Y}}
\newcommand{\vZ}{\boldsymbol{Z}}
\newcommand{\vzero}{\boldsymbol{0}}
\newcommand{\vomega}{\boldsymbol{\omega}}
%\newcommand{\half}{\frac{1}{2}}
\newcommand{\thalf}{\frac{3}{2}}
\newcommand{\rot}{\nabla\times}
\newcommand{\divg}{\nabla\cdot}
\newcommand{\cE}{\mathcal{E}}
\newcommand{\conclu}[1]{\vspace{1em}\par\noindent\fbox{\parbox{0.9\textwidth}{#1}}\vspace{1em}}
\newcommand{\subentrynote}{$\bullet$}
\newcommand{\keypoint}[1]{\par\subentrynote\quad #1 \par}
\newcommand{\fconclu}{\boxed}
\newcommand{\pair}[2]{#1 \, #2}
\newcommand{\intn}[2]{\int #1 \,\mathrm{d} #2}
\newcommand{\intu}[3]{\int_0^{#1} #2 \,\mathrm{d} #3}
\newcommand{\intiu}[3]{\int_{-\infty}^{#1} #2 \, \rd{} #3}
\newcommand{\intui}[2]{\int_0^{\infty} #1 \,\mathrm{d} #2}
\newcommand{\intii}[2]{\int_{-\infty}^{\infty} #1 \,\mathrm{d} #2}
\newcommand{\intt}[2]{\int_0^\infty #1 \, \rd{} #2}
\newcommand{\intr}[2]{\int_{-\infty}^{\infty} #1 \, \rd{} #2}
\newcommand{\intbi}[3]{\int_{#1}^{\infty} #2 \, \rd{} #3}
\newcommand{\intab}[4]{\int_{#1}^{#2} #3 \, \rd{} #4}
\newcommand{\bfactor}[1]{e^{-#1/k_BT}}
\newcommand{\pbfactor}[1]{e^{#1/k_BT}}
\newcommand{\dn}[2]{#1^{\pare{#2}}}
\newcommand{\prodg}[1]{\pare{#1}^\times}

\newcommand{\notion}{\emph}
\newcommand{\iP}{\mathcal{P}}
\newcommand{\eiP}{e^{-\iP}}
\newcommand{\iF}{\mathcal{F}}
\newcommand{\eiF}{e^{-\iF}}
\newcommand{\iG}{\mathcal{G}}

\newcommand{\rc}{r\cos\theta}
\newcommand{\rs}{r\sin\theta}
\newcommand{\sn}{\mathrm{sn}}
\newcommand{\cn}{\mathrm{cn}}
\newcommand{\rdn}{\mathrm{dn}}

\newcommand{\hankel}{H_p^{\pare{1}}}
\newcommand{\hankell}{H_p^{\pare{2}}}
\newcommand{\hhankel}{H_n^{\pare{1}}}
\newcommand{\hhankell}{H_n^{\pare{2}}}
\newcommand{\ber}{\text{ber}\,}
\newcommand{\bei}{\text{bei}\,}
\newcommand{\kker}{\text{ker}\,}
\newcommand{\kei}{\text{kei}\,}
\newcommand{\Ai}{\text{Ai}}
\newcommand{\Bi}{\text{Bi}}

\newcommand{\re}{\text{Re}\,}

\newcommand{\Fp}{F_\phi}
\newcommand{\Ep}{E_\phi}
\newcommand{\Fx}{F_x}
\newcommand{\FF}{\mathbf{F}}
\newcommand{\Ex}{E_x}

%\newcommand{\erf}{\mathrm{erf}}
\newcommand{\erfi}{\mathrm{erfi}}
\newcommand{\erfc}{\mathrm{erfc}}
\newcommand{\ehxs}[1]{e^{-\frac{#1^2}{2}}}
\newcommand{\dcol}[2]{\[ \left.#1 \hspace{1em}\right\vert\hspace{1em} #2 \]}
\newcommand{\titlegamma}{\texorpdfstring{$\Gamma$}{Gamma}}
\newcommand{\titleB}{\texorpdfstring{$B$}{B}}

% Computer Science Head
\usepackage{listings}
\usepackage{color}

\definecolor{dkgreen}{rgb}{0,0.6,0}
\definecolor{gray}{rgb}{0.5,0.5,0.5}
\definecolor{mauve}{rgb}{0.58,0,0.82}

\lstset{frame=tb,
  language=Java,
  aboveskip=3mm,
  belowskip=3mm,
  showstringspaces=false,
  columns=flexible,
  basicstyle={\small\ttfamily},
  numbers=none,
  numberstyle=\tiny\color{gray},
  keywordstyle=\color{blue},
  commentstyle=\color{dkgreen},
  stringstyle=\color{mauve},
  breaklines=true,
  breakatwhitespace=true,
  tabsize=3
}
\lstset{language=Java}
\newcommand{\snp}[1]{\lstinline!#1!}
\newcommand{\term}[2]{\textbf{#1(#2)}}
\begin{document}
\fi

%Content

\chapter{测度论}
  \section{Lebesgue测度}
  \subsection{引论}
  \subsubsection*{Lebesgue测度的性质}
  我们期望Lebesgue测度具有如下一些性质。
  \paragraph{区间的测度为其长度}非空区间是可测集,且
  \[ m\pare{I}=\len\pare{I}. \]
  \paragraph{测度是平移不变的}若$E$为Lebesgue可测集且$y$为一数,则
  \[ m\pare{E+y}=m\pare{E}. \]
  \paragraph{无交集的可数并的测度可加}$若E_k$为可数个无交可测集,则
  \[ m\pare{\bigcup E_k} = \sum m\pare{E_k}. \]
  且Lebesgue可测集全体构成一\siga。
  \begin{definition}
    一集族构成代数 ,如果其元素的补,有限交与有限并皆封闭。
  \end{definition}
  \begin{definition}
    一集族构成\siga ,如果其元素的补,可数交与可数并皆封闭。
  \end{definition}
  在全体集合上定义满足条件的测度是不可能的,甚至仅仅满足前两个条件而具有有限可加性都是不能指望的。但在定义Lebesgue测度前,仍可先构造对任意集合都适用的外测度,满足前二条件,而第三条件替换为无论诸$E_k$无交与否,皆有
  \[ m^*\pare{\bigcup E_k} \le \sum m^*\pare{E_k}. \]
  \subsection{Lebesgue外测度}
  定义无界区间的长度为$\infty$。对于任意集合,定义外测度
  \[ m^*\pare{A} = \inf\sum\len\pare{I_k}. \]
  其中$\curb{I_k}$为$A$的区间覆盖。立即可得空集外测度为零且外测度具有单调性,即若$A \subset B$则
  \[ m^*\pare{A} \le m^*\pare{B}. \]
  可以由此证明,可数集的测度为零。
  \begin{proposition}
    区间的测度为其长度。
  \end{proposition}
  \begin{proof}
    考虑有界闭区间$\brac{a,b}$,易证$m*\le\pare{b-a}$。另一方向的不等号需要
    \[ \sum\len\pare{I_k} \ge b-a. \]
    由紧致性只需要对有限开覆盖证明
    \[ \sum^n \len\pare{I_k} \ge b-a. \]
    选取包含$a$的区间1,若右端点在$\pare{a,b}$内则选取另一包含其右端点的区间2,重复这一过程直到右端点在$\pare{a,b}$外,则上述不等式成立。
    \par
    对于任意有界区间,选取其闭区间的上下逼近并注意外测度的单调性即可。对于无界区间,易得其测度为$\infty$。
  \end{proof}
  \begin{proposition}
    Lebesgue外测度是平移不变的。
  \end{proposition}
  \begin{proof}
    注意区间的平移不变即可。
  \end{proof}
  \begin{proposition}
    对任意$\curb{E_k}$,有
    \[ m^*\pare{\bigcup E_k} \le \sum m^*\pare{E_k}. \]
  \end{proposition}
  \begin{proof}
    对$E_k$取误差不超过$2^{-k}\epsilon$的覆盖区间,加和即可。
  \end{proof}
  \subsection{Lebesgue可测集的\siga}
  \subsubsection{\cara 可测}
  \begin{definition}
    若对于任意集合$A$,都有
    \[ m^*\pare{A} = m^*\pare{A\cap E}+m^*\pare{A\cap\complement E}, \]
    则称$E$可测。
  \end{definition}
  鉴于外测度的次可加性,上述条件可弱化为
  \[ m^*\pare{A} \ge m^*\pare{A\cap E}+m^*\pare{A\cap\complement E}. \]
  此外还应注意到,对于无交集,若其中任一可测,立刻有
  \begin{align*}
  m*\pare{A\cup B} &= m^*\pare{\brac{A\cup B}\cap A}+m^*\pare{\brac{A\cup B}\cap \complement A}\\ &= m^*\pare{A}+m^*\pare{B}.
  \end{align*}
  故有可加性。此外,可测集的补仍为可测集。
  \begin{theorem}
    零测集为可测集。
  \end{theorem}
  \begin{proof}
    代入弱化后的条件,注意外测度的单调性即可。
  \end{proof}
  \begin{theorem}
    可测集的有限并可测。故可测集构成代数。
  \end{theorem}
  \begin{proof}
    只证二可测集的并可测。借助二集可测的\cara 条件,有
    \begin{align*}
      m^*\pare{A} &= m^*\pare{A\cap E_1}+m^*\pare{A\cap\complement E_1\cap E_2} + m^*\pare{A\cap\complement E_1\cap \complement E_2} \\
      &\ge m^*\pare{A\cap\brac{E_1\cup E_2}} + m^*\pare{A\cap\complement\brac{E_1\cup E_2}}.\qedhere
    \end{align*}
  \end{proof}
  \begin{theorem}
    无交可测集的有限并满足
    \[ m^*\pare{A\cap\bigcup E_k} = \sum m^*\pare{A\cap E_k}. \]
  \end{theorem}
  \begin{proof}
    注意到\cara 条件的
    \[ A\cap\bigcup^n E_k \cap E_n = A \cap E_n \]
    以及
    \[ A\cap\bigcup^n E_k \cap \complement E_n = A\cap\bigcup^{n-1} E_n, \]
    归纳即可。
  \end{proof}
  \begin{collary}
    可测集的测度有限可加。
  \end{collary}
  \begin{theorem}
    可测集的可数并可测。故可测集构成\siga 。
  \end{theorem}
  \begin{proof}
    不妨设诸集无交。设其并为$E$,则根据前开命题及单调性,有
    \[ m^*\pare{A} \ge \sum^n m^*\pare{A\cap E_k} + m^*\pare{A\cap\complement E}. \]
    让$n\to\infty$,借助次可加性即可。
  \end{proof}
  \begin{theorem}
    区间是可测集。
  \end{theorem}
  \begin{proof}
    只证$I=\pare{a,\infty}$型区间可测。不妨设$a$不在$A$内且将之分割为$A\cap \complement I = A_1$与$A \cap I = A_2$。对于$A$的任意覆盖$\curb{I_k}$均同样割裂之,有
    \[ m^*\pare{A_1} + m^*\pare{A_2} \le \sum \len\pare{I_k}, \]
    故满足弱化后条件。
  \end{proof}
  \begin{definition}
    开集的可数交为$\Gd$型集。
  \end{definition}
  \begin{definition}
    闭集的可数并为$\Fs$型集。
  \end{definition}
  注意$\R$中开集为区间的并,故$\Gd$型(以及$\Fs$型)集可测。
  \begin{definition}
    包含开集的最小\siga 称为Borel \siga ,其元素称为Borel集。
  \end{definition}
  \begin{theorem}
    $\R$中可测集包含Borel \siga 。区间,开集,闭集,$\Gd$与$\Fs$型集可测。
  \end{theorem}
  \begin{proposition}
    可测集平移后可测。
  \end{proposition}
  \begin{proof}
    在\cara 条件中将$E$的平移转化为$A$的平移,注意外测度的平移不变即可。
  \end{proof}
  \subsection{Lebesgue可测集的内外逼近}
  \begin{lemma}
    对$A\subset B$,有
    \[ m^*\pare{B-A} = m^*\pare{B} - m^*\pare{A}. \]
  \end{lemma}
  \begin{proof}
  注意由\cara 条件,
  \[ m^*\pare{B} = m^*\pare{B \cap A} + m^*\pare{B-A}.\qedhere \]
  \end{proof}
  \begin{theorem}
    下列条件与$E$的可测性等价。
    \begin{aenum}
      \item 对$\epsilon>0$,存在包含$E$的开集$\open{O}$满足$m^*\pare{\open{O}-E}<\epsilon$;
      \item 存在包含$E$的$\Gd$型集满足$m^*\pare{G-E}=0$;
      \item 对$\epsilon>0$,存在$E$内的闭集$F$满足$m^*\pare{E-F}<\epsilon$;
      \item 存在$E$内的$\Fs$型集满足$m^*\pare{E-F}=0$。
    \end{aenum}
  \end{theorem}
  \begin{proof}
    只证前二者。后二者取补可得。
    \par
    设$E$可测,则存在区间并任意逼近其外测度,取$\open{O}$为区间并即可。有
    \[ m^*\pare{\open{O}-E} = m^*\pare{\open{O}} - m^*\pare{E} < \epsilon. \]
    对于无界$E$,分为可数个有界部分即可。不断缩小$\epsilon$,可得所求$\Gd$型集。鉴于零测集可测, 又$E=G\cap\complement\pare{G-E}$,知$E$可测。
  \end{proof}
  注意到对于任意集合$E$都存在开集使$m^*\pare{\open{O}}-m^*\pare{E}$任意小,然而外测度的减性仅对可测集成立。
  \begin{theorem}
    \label{thm:lt1}
    对有限测度的$E\subset \R$,存在有限多个区间的并$\open{O}$满足$m*\pare{E-\open{O}} + m^*\pare{\open{O}-E} < \epsilon$。
  \end{theorem}
  \begin{proof}
    取开集$U$为$E$的$\epsilon/2$外逼近,写$U$为区间并,选取其中有限个以$\epsilon/2$逼近之,注意到两差均小于$\epsilon/2$即可。
  \end{proof}
  \subsection{Lebesgue测度的其他性质}
  \begin{definition}
    对可测集定义其Lebesgue测度为外测度。
  \end{definition}
  \begin{theorem}
    Lebesgue测度是可数可加的。
  \end{theorem}
  \begin{proof}
    $m\pare{\cup}\le\sum m$由次可加性可得,由有限可加性和单调性又有$m\pare{\cup}\ge\sum^n m$,让右侧$n\to\infty$即可。
  \end{proof}
  \begin{theorem}
    $\R$中可测集包含Borel \siga 。区间测度为长度,且平移不变,可数可加。
  \end{theorem}
  \begin{definition}
    一个可数集族称为升链,如果$E_k\subset E_{k+1}$,相似定义降链。
  \end{definition}
  \begin{theorem}
    Lebesgue测度满足
    \begin{aenum}
      \item 若$\curb{A_k}$为升链,则
      \[ m\pare{\bigcup A_k} = \lim_{k\to\infty}m\pare{A_k}. \]
      \item 若$\curb{B_k}$为降链且$m\pare{B_1}<\infty$,则
      \[ m\pare{\bigcap B_k} = \lim_{k\to\infty}m\pare{B_k}. \]
    \end{aenum}
  \end{theorem}
  \begin{proof}
    不妨设诸$A_k$测度有限,则构造$A_k$的差得到等价的无交序列,后应用可数可加性即可。
    \par
    对于$B$则关于$B_1$取补后构造等价无交序列,借助减性即可。
  \end{proof}
  \begin{definition}
    称一性质在$E$上几乎处处成立,如果它在除一零测集外成立。
  \end{definition}
  \begin{lemma}[Borel-Cantelli]
    若$\curb{E_k}$测度和有限,则几乎任意$x\in\R$最多属于有限多个$E_k$。
  \end{lemma}
  \begin{proof}
    \[ m\pare{\bigcap\bigcup_{k=n}E_k}=\lim_{n\to\infty}m\pare{\bigcup_{k=n}E_k} = 0.\qedhere \]
  \end{proof}
  \subsection{不可测集}
  \begin{lemma}
    设$E\subset\R$有界且存在可数无限有界实数集$\Lambda$其元素使诸$\lambda+E$无交,则$m\pare{E}=0$。
  \end{lemma}
  \begin{proof}
    注意平移不变性与可数可加性,以及有界性即可。
  \end{proof}
  \begin{definition}
    定义二实数有理等价,若其差为有理数。
  \end{definition}
  \begin{theorem}[Vitali]
    任意正测度的实数集$E$存在一不可测子集。
  \end{theorem}
  \begin{proof}
    不妨设$E$有界,取$E$内有理等价类的代表元集$\mathcal{C}$,有上述引理知$m\pare{\mathcal{C}}=0$。再选取$\Lambda$为$\Q$足够大的子集,使诸$\lambda+E$可覆盖$E$,矛盾。
  \end{proof}
  \begin{theorem}
    存在$\R$的无交子集$A$与$B$满足
    \[ m^*\pare{A\cup B} < m^*\pare{A} + m^*\pare{B}. \]
  \end{theorem}
  \subsection{Cantor集与Cantor-Lebesgue函数}
  \begin{definition}
    定义Cantor集为$I=\brac{0,1}$不断挖去各连通分量之三等分之中间部分的结果。令诸$C_k$为每一步的结果,$\mathbf{C}=\cap C_k$。
  \end{definition}
  \begin{theorem}
    Cantor集不可数,且$m\pare{\mathbf{C}}=0$。
  \end{theorem}
  \begin{proof}
    易证其可测且测度为零。参考\tref{uncountableR}的证明过程知不可数。
  \end{proof}
  定义Cantor-Lebesgue函数$\varphi$函数如下。对$\open{O}_k=\brac{0,1}-C_k$的$2^k-1$个连通分量分别赋值
  \[ \curb{1/2^k,2/2^k,3/2^k,\cdots,\pare{2^k-1}/2^k}. \]
  令$\varphi\pare{0}=0$且
  \[ \varphi\pare{x} = \sup\setcond{\varphi\pare{t}}{t\in\blr{0,x}}. \]
  \begin{theorem}
    $\varphi$连续单调递增且在$\open{O}$内导数为零,并将$\brac{0,1}$映满$\brac{0,1}$。
  \end{theorem}
  \begin{proof}
    注意$\varphi$在$x\in\mathbf{C}$附近的跳跃不超过其两侧$\open{O}$的跳跃,而随$k$增大其可任意小。故其连续,由介值定理知映满。
  \end{proof}
  \begin{theorem}
    \label{thm:cantorl}
    连续严格递增映射$\psi\pare{x}=\varphi\pare{x}+x$满足:
    \begin{aenum}
      \item 将零测$\mathbf{C}$映为一正测集;
      \item 将一可测$E\subset\mathbf{C}$映为不可测集。
    \end{aenum}
  \end{theorem}
  \begin{proof}
    注意到$\brac{0,2}=\psi\pare{\open{O}}+\psi\pare{\mathbf{C}}$且开集与闭集映射后仍为开集与闭集,故仍可测。将$\open{O}$分解成区间,映射后区间长度不变即知$m\pare{\psi\pare{\open{O}}}=1$。
    \par
    因此,$m\pare{\psi\pare{\mathbf{C}}}=1$而含有不可测集,其原像为零测可测集。
  \end{proof}
  \begin{lemma}
    严格递增映射存在连续逆。
  \end{lemma}
  \begin{lemma}
    连续映射$f$的Borel集像的原像为Borel集。
  \end{lemma}
  \begin{proof}
    注意$f^{-1}\pare{\complement U}=\complement f^{-1}\pare{U}$,$f^{-1}\pare{A\cap B}=f^{-1}\pare{A} \cap f^{-1}\pare{B}$。
  \end{proof}
  \begin{theorem}
    存在非Borel集的可测集。
  \end{theorem}
  \begin{proof}
    Borel集经严格增映射后仍为Borel集,可测集映射后可能不可测。
  \end{proof}
  \section{可测函数}
  \subsection{可测函数的和、积与复合}
  \begin{proposition}
    对于在可测集上定义的函数$f$,下列命题等价。
    \begin{enumerate}
      \item 对任意$c$,$f\pare{x}>c$的$x$可测;
      \item 对任意$c$,$f\pare{x}\ge c$的$x$可测;
      \item 对任意$c$,$f\pare{x}<c$的$x$可测;
      \item 对任意$c$,$f\pare{x}\le c$的$x$可测;
    \end{enumerate}
  \end{proposition}
  \begin{proof}
    只证前二。将$f\pare{x}\ge c$的$x$视为诸$f\pare{x}>c-1/k$的交,而$f\pare{x}>c$视为诸$f\pare{x}\ge c+1/k$的并。
  \end{proof}
  \begin{definition}
    可测集上定义的函数$f$称为可测的,若其满足前开命题之一。
  \end{definition}
  \begin{proposition}
    可测集上定义的$f$为可测当且仅当开集的原像均可测。
  \end{proposition}
  \begin{proof}
    注意开集可写为区间并,而$\pare{a,b}=\pare{-\infty,b}\cap\pare{a,+\infty}$。
  \end{proof}
  \begin{proposition}
    可测集上定义的连续函数可测。
  \end{proposition}
  \begin{proposition}
    区间上定义的单调函数可测。
  \end{proposition}
  \begin{proposition}
    设$\func{f}{E}{\clo{\R}}$。
    \begin{enumerate}
      \item 若$f$可测而$g$与$f$几乎处处相等,则$g$可测;
      \item 设$D$为可测子集,$f$可测当且仅当在$D$和$E-D$上可测。
    \end{enumerate}
  \end{proposition}
  \begin{theorem}
    $f$和$g$为几乎处处有界的可测函数,则$ \alpha f + \beta g $与$ fg $可测。
  \end{theorem}
  \begin{proof}
    只证$f+g$和$fg$可测。$f+g<c$,则存在$q\in\Q$满足$f<q<c-g$,将诸可数个$q$并起即可。又注意
    \[ fg=\half\brac{\pare{f+g}^2-f^2-g^2} \]
    以及可测函数的平方可测即可。
  \end{proof}
  \begin{ex}
    由\tref{cantorl}可知,可测函数的复合$\chi_E\comp\psi^{-1}>0$的原像$\psi\pare{E}$不可测。
  \end{ex}
  \begin{theorem}
    设$f$连续可测而$g$可测,则$f\comp g$可测。
  \end{theorem}
  \begin{proof}
    注意$\pare{f\comp g}^{-1}\pare{\open{O}} = g^{-1}\pare{f^{-1}\pare{\open{O}}}$即可。
  \end{proof}
  由是立得$\abs{\pare{f}}$与$\abs{\pare{f}}^p$可测。
  \begin{proposition}
    $\maxo{f_1,\cdots,f_n}$与$\mino{f_1,\cdots,f_n}$可测。
  \end{proposition}
  由是立得诸
  \[ \abs{f} = \maxo{f,-f},\quad f^+=\maxo{f,0},\quad f^-=\maxo{-f,0} \]
  可测。故$f$可写为可测函数之差$f=f^+-f^-$。
  \subsection{可测函数的极限与逼近}
  \begin{definition}
    称$\curb{f_n}$一致收敛于$f$,若对于充分大的$n$有$\norm{f-f_n}<\epsilon$。
  \end{definition}
  \begin{proposition}
    若可测函数列$\curb{f_n}$逐点收敛于$f$,则$f$可测。
  \end{proposition}
  \begin{proof}
    若$f\pare{x}<c$,对于充分大的$N$有$f_{N:}\pare{x}<c$,并起诸$N$即可。
  \end{proof}
  \begin{definition}
    简单函数为仅取有限多个值的可测函数。
  \end{definition}
  注意简单函数$\varphi$均可写为
  \[ \varphi = \sum^n c_k\cdot \chi_{E_k}. \]
  \begin{lemma}[简单函数逼近]
    可测函数存在\epsclo 的上下逼近$\varphi_\epsilon$与$\psi_\epsilon$。
  \end{lemma}
  \begin{proof}
    将可测函数的值域分割为若干$\epsilon$小区间即可。
  \end{proof}
  \begin{theorem}[简单函数逼近]
    可测函数存在满足$\abs{\varphi_n}<\abs{\func{f}{E}{\clo{\R}}}$的逼近。若$f$恒正,则存在诸$\varphi_n$递增。
  \end{theorem}
  \begin{proof}
    设$f$恒正。在第$n$步截断$f$的值域至$n$后作$1/n$逼近即可。取$\varphi_n=\maxo{\varphi_1,\cdots,\varphi_n}$可得递增序列。
    \par
    一般情形将$f$写为$f^+-f^-$即可。
  \end{proof}
  \subsection{Littlewood的三大原理}
  三大原理谓
  \begin{enumerate}
    \item 每个\emph{可测}集都\emph{几乎}是区间的并;(\tref{lt1})
    \item 每个\emph{可测}函数都\emph{几乎}是连续的;(\tref{Lusin})
    \item 每个\emph{可测}函数的逐点收敛序列都\emph{几乎}是一致收敛的。(\tref{Egoroff})
  \end{enumerate}
  \begin{lemma}
    对有限测度的$E$上定义的逐点收敛可测函数列$\curb{f_n}\to f$,存在充分大的$N$使$f_{N:}$在任意逼近$E$的集合上任意逼近$f$。
  \end{lemma}
  \begin{proof}
    注意由逐点收敛,诸$N$的$A$为升列且并为$E$即可。
  \end{proof}
  \begin{theorem}[Egoroff定理]
    \label{thm:Egoroff}
    有限测度的$E$上定义的逐点收敛可测函数列$\curb{f_n}\to f$在一\epsclo $E$的闭集$F$上一致收敛。
  \end{theorem}
  \begin{proof}
    据上引理,对任意$n$取$A_n$与$E$为\close{\epsilon/2^{n+1}}而$f_{N:}$与$f$为\close{1/n},由是其交$A$与$E$为\close{\epsilon}且一致收敛。再取闭集逼近$A$即可。
  \end{proof}
  \begin{proposition}
    对在$E$上定义的简单函数,存在连续函数在任意逼近$E$的集合上与之相等。
  \end{proposition}
  \begin{proof}
    对诸$E_k$选取闭集逼近之,后调用Urysohn引理。
  \end{proof}
  \begin{theorem}[Lusin定理]
    \label{thm:Lusin}
    对可测函数,前开命题成立。
  \end{theorem}
  \begin{proof}
    由简单函数逼近之,后以连续函数逼近之,再选取一致收敛的闭集。
  \end{proof}


%ContentEnds
 
\ifx\allfiles\undefined %如果位置放错,可能出现意外中断
\end{document}
\fi