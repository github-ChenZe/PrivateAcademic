%Ch10.LpSpace
\ifx\allfiles\undefined
\documentclass{ctexrep}
\usepackage{amsmath}
\usepackage{amssymb}
\usepackage{amsthm}
\usepackage{amsfonts}
\usepackage{mathrsfs}
\usepackage{enumitem}
\usepackage{braket}
\usepackage{hyperref}


\newcommand{\pare}[1]{\left(#1\right)}
\newcommand{\blr}[1]{\left[#1\right)}
\newcommand{\lbr}[1]{\left(#1\right]}
\newcommand{\brac}[1]{\left[#1\right]}
\newcommand{\curb}[1]{\left\{#1\right\}}
\newcommand{\abs}[1]{\left|\, #1 \,\right|}
\newcommand{\rec}[1]{\frac{1}{#1}}
\newcommand{\N}{\mathbb{N}}
\newcommand{\Q}{\mathbb{Q}}
\newcommand{\Z}{\mathbb{Z}}
\newcommand{\R}{\mathbb{R}}
\newcommand{\unk}{\mathcal{X}}
\newcommand{\bu}[3]{#1_{#2}^{\pare{#3}}}
\newcommand{\dref}[1]{定义\ref{def:#1}}
\newcommand{\tref}[1]{定理\ref{thm:#1}}
\newcommand{\lref}[1]{引理\ref{lem:#1}}
\newcommand{\cref}[1]{推论\ref{coll:#1}}
\newcommand{\pref}[1]{命题\ref{prp:#1}}
\newcommand{\func}[3]{#1:\, #2 \rightarrow #3}
\newcommand{\overbar}[1]{\mkern 1.5mu\overline{\mkern-1.5mu#1\mkern-1.5mu}\mkern 1.5mu}
\newcommand{\clo}[1]{\overbar{#1}}
\newcommand{\supi}[2]{\overbar{\int_{#1}^{#2}}}
\newcommand{\infi}[2]{\underbar{\int_{#1}^{#2}}}
\newcommand{\setf}{\mathscr}
\newcommand{\bool}{\mathrm{bool}}
\newcommand{\inc}{++}
\newcommand{\defeq}{:=}
\newcommand{\ntuple}{$n$元组}
\newcommand{\card}[1]{\#\pare{#1}}
\newcommand{\setcond}[2]{\curb{#1 \, \left| \, #2 \right.}}
\newcommand{\setcondl}[2]{\curb{\left. #1 \, \right| \, #2}}
\newcommand{\bv}[1]{\mathbf{#1}}
\newcommand{\bfa}{\bv{a}}
\newcommand{\bfb}{\bv{b}}
\newcommand{\bfx}{\bv{x}}
\newcommand{\bfy}{\bv{y}}
\newcommand{\bfe}{\bv{e}}
\newcommand{\bfF}{\bv{F}}
\newcommand{\bff}{\bv{f}}
\newcommand{\bfG}{\bv{G}}
\newcommand{\bfH}{\bv{H}}
\newcommand{\bfg}{\bv{g}}
\newcommand{\bfh}{\bv{h}}
\newcommand{\bfr}{\bv{r}}
\newcommand{\bfk}{\bv{k}}
\newcommand{\bfu}{\bv{u}}
\newcommand{\bfv}{\bv{v}}
\newcommand{\oo}[1]{o\pare{#1}}
\newcommand{\OO}[1]{O\pare{#1}}
\newcommand{\norm}[1]{\left\| #1 \right\|}
\newcommand{\DD}{\mathbf{D}}
\newcommand{\comp}{\circ}
\newcommand{\const}{\mathrm{const}}
\newcommand{\dist}[2]{d\pare{#1,#2}}
\newcommand{\len}{\ell}
\newcommand{\siga}{$\sigma$-代数}
\newcommand{\cara}{Carath\'{e}odory}
\newcommand{\Gd}{G_\delta}
\newcommand{\Fs}{F_\sigma}
\newcommand{\mmani}{$m$-维流形}
\newcommand{\open}[1]{\mathcal{#1}}
\newcommand{\half}{\frac{1}{2}}
\newcommand{\maxo}[1]{\text{max}\curb{#1}}
\newcommand{\mino}[1]{\text{min}\curb{#1}}
\newcommand{\epsclo}{$\epsilon$-接近}
\newcommand{\close}[1]{$#1$-接近}
\newcommand{\cinf}{$C^\infty$}
\newcommand{\cuno}{$C^1$}
\newcommand{\Int}{\text{Int}\,}
\newcommand{\Ext}{\text{Ext}\,}
\newcommand{\funcf}{\mathcal}
\newcommand{\DDu}{\overbar{\DD}}
\newcommand{\DDl}{\underbar{\DD}}
\newcommand{\Diff}[1]{\mathrm{Diff}_{#1}\,}
\newcommand{\Av}[1]{\mathrm{Av}_{#1}\,}
\newcommand{\Lip}[1]{Lipschitz-$#1$}
\newcommand{\sgn}[1]{\mathrm{sgn}}
\newcommand{\eset}{\varnothing}
\newcommand{\cT}{\mathcal{T}}
\newcommand{\cS}{\mathcal{S}}
\newcommand{\cG}{\mathcal{G}}
\newcommand{\cF}{\mathcal{F}}
\newcommand{\cC}{\mathcal{C}}
\newcommand{\cB}{\mathcal{B}}

\newcommand{\hd}{H\"{o}lder}

\renewcommand{\proofname}{证明}

\newenvironment{cenum}{\begin{enumerate}\itemsep0em}{\end{enumerate}}

\newtheorem{definition}{定义}[section]
\newtheorem{lemma}{引理}[section]
\newtheorem{theorem}{定理}[section]
\newtheorem{collary}{推论}[section]
\newtheorem{proposition}{命题}[section]
\newtheorem{axiom}{公理}[section]
\newtheorem{ex}{例}[section]

\newenvironment{aenum}{\begin{enumerate}[label=\textnormal{(\alph*)}]}{\end{enumerate}}
\begin{document}
\fi

%Content

\chapter{$L^p$空间}
  \section{$L^p$空间:完备性与逼近}
  \subsection{赋范线性空间}
  \begin{definition}
    二函数称为等价,如果其几乎处处相等。
  \end{definition}
  \begin{definition}
    $E$上满足
    \[ \int_E \abs{f}^p < \infty \]
    之函数等价类全体构成一线性空间,谓$L^p$空间。
  \end{definition}
  由
  \[ \abs{a+b}^p \le 2^p \curb{\abs{a}^p+\abs{b^p}} \]
  知其可构成线性空间。
  \begin{definition}
    若$f$几乎处处满足
    \[ \abs{f\pare{x}} \le M, \]
    谓之本质有界。其等价类全体构成$L^\infty$。
  \end{definition}
  \begin{definition}
    线性空间上一泛函$\norm{\cdot}$称为范数,若
    \begin{align*}
      \norm{f+g} &\le \norm{f}+\norm{g},\\
      \norm{\alpha f} &= \abs{\alpha}\norm{f},\\
      \norm{f} &\ge 0.
    \end{align*}
    最后的等号严格成立当且仅当$f=0$。
  \end{definition}
  \begin{ex}
    易知$L^1$构成一赋范线性空间。
  \end{ex}
  \begin{ex}
    易知$L^\infty$关于$\norm{f}=\inf M$构成一赋范线性空间。
  \end{ex}
  \begin{ex}
    易知$\ell_1$与$\ell_\infty$构成一赋范线性空间。
  \end{ex}
  \begin{ex}
    易知紧区间上的连续函数全体关于$\norm{f}=\max f$构成一赋范线性空间。
  \end{ex}
  \subsection{Young不等式,H\"{o}lder不等式,Minkowski不等式}
  \begin{definition}
    对于$1<p<\infty$以及$L^p$中的$f$,定义
    \[ \norm{f}_p = \brac{\int \abs{f}^p}^{1/p}.  \]
  \end{definition}
  \begin{definition}
    对$p\in\pare{1,\infty}$定义其共轭$q=p/\pare{p-1}$,同在$\pare{1,\infty}$内且
    \[ \rec{p}+\rec{q}=1. \]
  \end{definition}
  \begin{theorem}[Young不等式]
    设$p$,$q$共轭,对正数$a$,$b$有
    \[ ab\le \frac{a^p}{p}+\frac{b^q}{q}. \]
  \end{theorem}
  \begin{proof}
    由Jensen不等式,
    \[ \frac{\log a^p}{p} + \frac{\log b^q}{q} \le \log\pare{\frac{a^p}{p} + \frac{b^q}{q}}. \qedhere \]
  \end{proof}
  \begin{theorem}[H\"{o}lder不等式]
    对于$L^p$之$f$与$L^q$之$g$,有
    \[ \int \abs{f\cdot g} \le \norm{f}_p \cdot\norm{g}_q. \]
  \end{theorem}
  \begin{proof}
    不妨设$\norm{f}=\norm{g}=1$,从而由Young不等式易得
    \[ \int\abs{f\cdot g} \le 1. \qedhere \]
  \end{proof}
  \begin{collary}
    $f$之共轭$f^* = \norm{f}^{1-p}_p\cdot\sgn\pare{f}\cdot\abs{f}^{p-1}$为$L^q$,且
    \[ \int f\cdot f^* = \norm{f}_p,\quad \norm{f^*}_q=1. \]
  \end{collary}
  \begin{theorem}[Minkowski不等式]
    若$f$与$g$均为$L^p$,则$f+g$同且
    \[ \norm{f+g}_p \le \norm{f}_p + \norm{g}_p. \]
  \end{theorem}
  \begin{proof}
    借助前开推论与H\"{o}lder不等式,
    \[ \norm{f+g}_p = \int f\cdot\pare{f+g}^* + \int g\cdot\pare{f+g}^* \le \norm{f}_p + \norm{g}_p. \qedhere \]
  \end{proof}
  \begin{collary}[Cauchy-Schwarz不等式]
  对于$L^2$内的$f$与$g$,
    \[ \int\abs{fg} \le \sqrt{\int f^2}\cdot\sqrt{\int g^2}. \]
  \end{collary}
  \begin{collary}
    \label{coll:lpui}
    若$\funcf{F}$内诸$\norm{f}_p\le M$,则$\funcf{F}$一致可积。
  \end{collary}
  \begin{proof}
    由H\"{o}lder不等式,
    \[ \brac{\int_A \abs{f}} \le \brac{\int_E \abs{f}^p}^{1/p} \brac{m\pare{A}}^{1/q}. \qedhere \]
  \end{proof}
  \begin{collary}
    有限测度上若$p_1<p_2$,则$L^{p_2}\subset L^{p_1}$。且
    \[ \norm{f}_{p_1} \le c\norm{f}_{p_2}, \]
    其中$c=\brac{m\pare{E}}^{\frac{p_2-p_1}{p_1p_2}}$。
  \end{collary}
  \begin{proof}
    令$p=p_2/p_1$,则$f^{p_1}\in L^p$。由H\"{o}lder不等式,
    \[ \int_E \abs{f}^{p_1} \le \norm{f}^{p_1}_{p_2}\brac{m\pare{E}}^{1/q}. \qedhere \]
  \end{proof}
  \begin{ex}
    通常,有限测度集如$\lbr{0,1}$上上述包含关系是严格的。取$-1/p_1<\alpha<-1/p_2$,有$x^\alpha \in L^{p_1} - L^{p_2}$。
  \end{ex}
  \begin{ex}
    在$\pare{0,\infty}$上$f=x^{-1/2}/\pare{1+\abs{\log x}}$仅仅属于$L^2$。
  \end{ex}
  \subsection{$L^p$的完备性}
  \begin{definition}
    称一序列收敛于$f$,如果
    \[ \lim_{n\to\infty}\norm{f-f_n}=0. \]
  \end{definition}
  \begin{definition}
    完备的赋范线性空间称为Banach空间。
  \end{definition}
  \begin{proposition}
    完备空间内的收敛序列均为Cauchy序列,且包含收敛子序列的Cauchy序列收敛。
  \end{proposition}
  \begin{proof}
    后一命题注意
    \[ \norm{f_n-f} \le \norm{f_n-f_{n_k}} + \norm{f_{n_k}-f}. \qedhere  \]
  \end{proof}
  \begin{definition}
    一序列称为快速Cauchy的,如果对于一收敛级数$\sum\epsilon_k$,有
    \[ \norm{f_{k+1}-f_k} \le \epsilon_k^2. \]
  \end{definition}
  \begin{proposition}
    快速Cauchy序列都是Cauchy的,而任一Cauchy序列均有快速子列。
  \end{proposition}
  \begin{proof}
    选取子列满足
    \[ \norm{f_{n_{k+1}}-f_{n_k}} \le \pare{\half}^k. \qedhere \]
  \end{proof}
  \begin{theorem}
    $L^p$内的快速Cauchy序列依范数且几乎处处逐点收敛。
  \end{theorem}
  \begin{proof}
    依定义选取$\sum\epsilon_k$后注意
    \[ m\pare{\abs{f_{k+1}-f_k}^p>\epsilon_k^p} \le \rec{\epsilon_k^p}\int\abs{f_{k+1}-f_k}^p \le \epsilon_k^p. \]
    由Borel-Cantelli引理知$f_n$几乎处处逐点收敛。由Fatou引理,
    \[ \int\abs{f-f_n}^p \le \int\abs{f_{n+k}-f_n}^p \le \brac{\sum_{j=n} \epsilon_j^2}^p. \qedhere \]
  \end{proof}
  \begin{theorem}[Riesz-Fischer]
    $L^p$空间为Banach空间。且Cauchy序列存在子列几乎处处逐点收敛。
  \end{theorem}
  \begin{ex}
    $f_n=n^{1/p}\chi_{\lbr{0,1/n}}$逐点收敛于零,但不依范数收敛。
  \end{ex}
  \begin{theorem}
    对$1\le p<\infty$,逐点收敛的序列依范数收敛当且仅当
    \[ \lim_{n\to\infty}\int\abs{f_n}^p=\int\abs{f}^p. \]
  \end{theorem}
  \begin{proof}
    若依范数收敛,由三角不等式即得结论。反之设极限成立,令
    \[ h_n = \frac{\abs{f_n}^p+\abs{f}^p}{2}-\abs{\frac{f_n-f}{2}}^p. \]
    由凸性知$h_n\ge 0$,且$\lim h_n = \abs{f}^p$逐点收敛,由Fatou引理
    \[ \int \abs{f}^p \le \lim \inf \int h_n = \int \abs{f}^p - \lim\sup\int\abs{\frac{f_n-f}{2}}^p. \qedhere \]
  \end{proof}
  \begin{theorem}
    对$1\le p<\infty$,逐点收敛的序列依范数收敛当且仅当$\curb{\abs{f_n}^p}$一致可积且紧密。
  \end{theorem}
  \begin{proof}
    由\cref{mcon2int},要求$\abs{f_n-f}^p$一致可积且紧密。再注意
    \[ \abs{f_n-f}^p \le 2^p\curb{\abs{f_n}^p+\abs{f}^p}, \quad \abs{f_n}^p \le 2^p\curb{\abs{f_n-f}^p+\abs{f}^p}. \qedhere \]
  \end{proof}
  \subsection{逼近与可分性}
  \begin{definition}
    $L^p$下一函数族称为稠密的,如果其依范数可任意逼近$L^p$。
  \end{definition}
  \begin{proposition}
    简单函数在$L^p$内稠密。
  \end{proposition}
  \begin{proof}
    借助简单函数逼近引理,注意$\abs{\varphi_n-g}^p \le 2^{p+1}\abs{g}^p$后控制收敛。
  \end{proof}
  \begin{proposition}
    对$1\le p<\infty$,阶梯函数在紧区间上的$L^p$稠密。
  \end{proposition}
  \begin{proof}
    注意阶梯函数可在任意小的集合外逼近简单函数即可。而对$p=\infty$,再小的非零测集皆会导致范数不得为零,是故于其不成立。
  \end{proof}
  \begin{definition}
    空间谓可分者,其下存在一可数稠密子集。
  \end{definition}
  \begin{theorem}
    对$1\le p < \infty$,$L^p$可分。
  \end{theorem}
  \begin{proof}
    紧区间内有理阶梯函数稠密,积分可由$\brac{-n,n}$上单调收敛逼近。
  \end{proof}
  \begin{ex}
    紧区间上的$L^\infty$不可分。
  \end{ex}
  \begin{proof}
    不可数特征函数族的区间稍变,逼近不复成立,不可以可数族逼近。
  \end{proof}
  \begin{theorem}
    对$1\le p<\infty$,有界支撑的连续函数在$L^p$中稠密。
  \end{theorem}
  \section{$L^p$空间的对偶与弱收敛}
  \subsection{$L^p$的对偶与表示}
  \begin{definition}
    线性泛函是函数上的线性算子。
  \end{definition}
  \begin{ex}
    $T\pare{f}=\int fg$与$T\pare{f}=\int f \, \mathrm{d} g$均为线性泛函。
  \end{ex}
  \begin{definition}
    所有$\abs{T\pare{f}} \le M \norm{f}$的$M$的下确界记作$\norm{T}$。
  \end{definition}
  由三角不等式知线性泛函连续。同时有
  \[ \norm{T} = \sup\setcond{T\pare{f}}{\norm{f}\le 1}. \]
  \begin{proposition}
    赋范线性空间上的线性算子的空间构成一赋范线性空间。
  \end{proposition}
  \begin{proposition}
    $L^p$上的算子
    \[ T\pare{f}=\int g\cdot f \]
    的范数为$\norm{g}$。
  \end{proposition}
  \begin{proposition}
     在一稠密子集上相等的线性算子相等。
  \end{proposition}
  \begin{lemma}
    可测函数$g$若对$L^p$上的简单函数$f$皆满足
    \[ \abs{\int g\cdot f} \le M \norm{f}, \]
    则$g\in L^q$,且$\norm{g}\le M$。
  \end{lemma}
  \begin{proof}
    对于$p>1$,考虑$g$的下逼近,只证$\int\varphi_n^p\le M^p$即可,再注意$\varphi_n^q\le\abs{g}\varphi_n^{q-1}$,以及$p\pare{q-1}=q$并借助题设。
    \par
    对于$p=1$,需证$M$为一本质上界。考虑$f$为诸特征函数即可。
  \end{proof}
  \begin{theorem}
    紧区间上的线性泛函满足$T\pare{f}=\int g\cdot f$的形式。
  \end{theorem}
  \begin{proof}
    令$\Phi\pare{x}=T\chi_{\blr{a,x}}$,其绝对连续,故$\Phi'=g$积分还原。对阶梯函数,
    \[ T\pare{f} = \int g\cdot f. \]
    控制收敛后知对简单函数均成立之。调用前开命题再注意简单函数稠密。
  \end{proof}
  \begin{theorem}[$L^p$的Riesz表示定理]
    $1\le p<\infty$上的线性泛函有$g$满足
    \[ Tf = \Braket{g|f}, \quad \norm{T} = \norm{f}. \]
  \end{theorem}
  \begin{proof}
    考虑$\brac{-n,n}$上的限制后不断扩大$n$,Fatou后知$g\in L^q$。
  \end{proof}
  \subsection{弱收敛性}
  \begin{ex}
    $\brac{0,1}$上的$1/2^n$-方波在$L^p$内不存在收敛子列。
  \end{ex}
  \begin{definition}
    赋范线性空间上的序列$\curb{f_n}$,若$Tf_n\to Tf$对任意$T$成立,则称之弱收敛。
  \end{definition}
  \begin{proposition}
    $\curb{f_n}$弱收敛于$f$当且仅当对任意$g$成立
    \[ \lim_{n\to\infty} g\cdot f_n = \int g \cdot f. \]
  \end{proposition}
  弱收敛具有唯一性。因为
  \[ \int \pare{f_1-f_2}^*\cdot f_2 = \lim_{n\to\infty} \int \pare{f_1-f_2}^*\cdot f_n = \int \pare{f_1-f_2}^* f_2. \]
  \begin{theorem}
    $L^p$上的弱收敛序列有诸$\norm{f_n}$有界且
    \begin{equation}
    \label{eq:flefn}
      \norm{f} \le \lim \inf \norm{f_n}.
    \end{equation}
  \end{theorem}
  \begin{proof}
    注意到
    \[ \int f^*\cdot f_n \le \norm{f^*}_q \cdot \norm{f_n}_p = \norm{f_n}_p \]
    后Fatou即可。为证明有界,假设$\curb{\norm{f_n}}$无界,选取$\curb{f_n}$的子列$\curb{g_n}$满足$\norm{g_n}\ge n\cdot 3^n$,并再度选取子列$\curb{h_n}$满足$\norm{h_n}/\pare{n\cdot 3^n}\rightarrow \alpha\in\brac{1,+\infty}$。
    \par
    于是$\cF_n = n\cdot 3^n/\norm{h_n}\cdot h_n$满足$\cF_n$弱收敛于$f$且$\norm{\cF_n}=n\cdot 3^n$。定义
    \[ \epsilon_{n+1} = \rec{3^{n+1}}\cdot\sgn \int \brac{\sum_k^n \epsilon_k \cdot f_k^*}\cdot f_{n+1}, \]
    则$\norm{\epsilon_k\cdot f_k^*}=1/3^k$,由$L^p$的完备性知$g=\sum^\infty \epsilon_k \cdot f_k^*$收敛于$L^p$内。而
    \[ \abs{\int g\cdot f_n} = \abs{\int \pare{\sum^{k=n} \epsilon_k \cdot f_k^* \cdot f_n}} -\abs{\int{\sum_{k=n+1}\epsilon_k \cdot f_k^* \cdot f_n}} \ge n - \frac{\norm{f_n}}{2\cdot 3^n}. \]
    与$\int g\cdot f_n\rightarrow \int g\cdot f$矛盾。
  \end{proof}
  \begin{collary}
  设$f_n$弱收敛于$f$,$g_n$强收敛于$g$,则
  \[ \int g_n \cdot f_n \rightarrow \int g\cdot f. \]
  \end{collary}
  \begin{proposition}
  设$\cF$张成的空间在$L^q$中稠密,则$L^p$中有界的$\curb{f_n}$弱收敛于$f$当且仅当对任意$g\in\cF$,
  \[ \int f_n\cdot g \rightarrow \int f\cdot g. \]
  \end{proposition}
  \begin{proof}
  注意到
  \[ \int f_n\cdot \cG - \int f\cdot\cG = \int \pare{f_n-f}\cdot\pare{\cG-g_k} + \int \pare{f_n-f}\cdot g_k. \]
  前者由\hd 不等式知可任意小,后者由题设知可任意小。
  \end{proof}
  注意对于任意$q>1$的$L^q$,简单函数稠密。对于$q<\infty$,简单函数均为有界支撑。对任意$1<q<\infty$的$L^p$,阶梯函数在闭区间上稠密。因此
  \begin{theorem}
  \label{thm:wa}
  对$1\le p<\infty$,有界$\curb{f_n}$弱收敛于$f$当且仅当对任意可测集,
  \[ \int_A f_n \rightarrow \int_A f. \]
  若$p>1$只需考虑有限测度的$A$。
  \end{theorem}
  \begin{theorem}
  \label{thm:wax}
  对于$1<p<\infty$与闭区间$\brac{a,b}$上的$L^p$,有界$\curb{f_n}$弱收敛于$f$当且仅当对任意$x$,
  \[ \int_a^x f_n \rightarrow \int_a^x f. \]
  \end{theorem}
  考虑$\brac{0,1}$上的
  \begin{align*}
    f_n = 
    \begin{cases}
    1,& k/2^n+1/2^{2n+1}<x<\pare{k+1}/2^n \\
    1-{2^{n+1},}& \text{otherwise.}
    \end{cases}
  \end{align*}
  知$p=1$时不成立,而上述函数族在$p>1$时无界,故同样不成立。
  \begin{ex}[Riemann-Lebesgue引理]
  令$f_n=\sin nx$,则$f_n$满足\tref{wax}的条件,然而
  \[ \int_{-\pi}^{\pi}\sin^2 nx\rightarrow \pi. \]
  因此$L^2$中$\curb{f_n}$即不强收敛,也不逐点收敛。
  \end{ex}
  \begin{ex}
  取$f_n=n\cdot\chi_{\lbr{0,1/n}}$,在$\brac{0,1}$上逐点收敛至零,但不弱收敛。
  \end{ex}
  \begin{ex}
  取$f_0$为$\pare{-1,0}-\pare{0,1}-\pare{1,0}$,$f_n\pare{x} = f_0\pare{x-n}$,$f=0$,则$p>1$时\tref{wa}的条件满足但$p=1$时不满足,考虑$g=1$便知。
  \end{ex}
  \begin{theorem}
  对于$1<p<\infty$与有界的$\curb{f_n}$几乎处处逐点收敛于$f$,有$\curb{f_n}$弱收敛于$f$。
  \end{theorem}
  \begin{proof}
  由Fatou知$f\in L^p$,再由\cref{lpui}知\tref{wa}条件满足。
  \end{proof}
  \begin{theorem}[Radon-Riesz]
  对于$1<p<\infty$,若$\curb{f_n}$弱收敛于$f$,则$f_n\rightarrow f$当且仅当$\norm{f_n}\rightarrow\norm{f}$。
  \end{theorem}
  \begin{proof}
  对$p=2$,在题设下有
  \[ \norm{f-f_n}^2 = \int \abs{f_n}^2-2\cdot\int f_n\cdot f+\int\abs{f}^2 = 0. \qedhere \]
  \end{proof}
  \begin{collary}
  对$1<p<\infty$,弱收敛于$f$的$\curb{f_n}$存在收敛子列当且仅当
  \[ \norm{f} = \lim\inf\norm{f_n}. \]
  \end{collary}
  \begin{proof}
  只注意对收敛子列存在的情形,\eqref{eq:flefn}与$\lim\inf\norm{f_n}\le\lim\norm{f_{n_k}}$。
  \end{proof}
  \begin{ex}
  令$\brac{-\pi,\pi}$上的$f_n=1+\sin nx$,$\curb{f_n}$弱收敛于$1$且$\norm{f_n}\rightarrow 2\pi$,当$f_n$并不强收敛于$1$。
  \end{ex}
  
  
%ContentEnds
 
\ifx\allfiles\undefined %如果位置放错,可能出现意外中断
\end{document}
\fi