\documentclass{ctexart}

\begin{document}
凡存在碳碳雙鍵者,皆應視為可發生氧化反應。
\par
陽離子的混合溶液之電解池中,應注意氧化方式為Fe3+ + Cu2+->Fe2+ + Cu,之後並不析出Fe。
\par
混合物之元素質量分數推斷,應先求出諸單質中諸元素之質量分數,而後因勢象形。
\par
應當注意,酯基與羧基中的碳氧雙鍵不得被氫氣加成。
\par
試圖計算酯的水解反應消耗氫氧化鈉數量時,應考慮可能生成的酚消耗的氫氧化鈉。
\par
試圖寫出酯的水解產物時,應注意酯基的方向。
\par
胺基酸中的NH2可與酸反應。

\end{document}